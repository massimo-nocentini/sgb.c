\input cwebmac
% This file is part of the Stanford GraphBase (c) Stanford University 1993
% This material goes at the beginning of all Stanford GraphBase CWEB files

\def\topofcontents{
  \leftline{\sc\today\ at \hours}\bigskip\bigskip
  \centerline{\titlefont\title}}

\font\ninett=cmtt9
\def\botofcontents{\vskip 0pt plus 1filll
    \ninerm\baselineskip10pt
    \noindent\copyright\ 1993 Stanford University
    \bigskip\noindent
    This file may be freely copied and distributed, provided that
    no changes whatsoever are made. All users are asked to help keep
    the Stanford GraphBase files consistent and ``uncorrupted,''
    identical everywhere in the world. Changes are permissible only
    if the modified file is given a new name, different from the names of
    existing files in the Stanford GraphBase, and only if the modified file is
    clearly identified as not being part of that GraphBase.
    (The {\ninett CWEB} system has a ``change file'' facility by
    which users can easily make minor alterations without modifying
    the master source files in any way. Everybody is supposed to use
    change files instead of changing the files.)
    The author has tried his best to produce correct and useful programs,
    in order to help promote computer science research,
    but no warranty of any kind should be assumed.
    \smallskip\noindent
    Preliminary work on the Stanford GraphBase project
    was supported in part by National Science
    Foundation grant CCR-86-10181.}

\def\prerequisite#1{\def\startsection{\noindent
    Important: Before reading {\sc\title},
    please read or at least skim the program for {\sc#1}.\bigskip
    \let\startsection=\stsec\stsec}}
\def\prerequisites#1#2{\def\startsection{\noindent
    Important: Before reading {\sc\title}, please read
    or at least skim the programs for {\sc#1} and {\sc#2}.\bigskip
    \let\startsection=\stsec\stsec}}



\def\title{BOOK\_\kern.05emCOMPONENTS}
\def\<#1>{$\langle${\rm#1}$\rangle$}

\prerequisite{GB\_\,BOOKS}

\N{1}{1}Bicomponents. This demonstration program computes the
biconnected components of GraphBase graphs derived from world literature,
using a variant of Hopcroft and Tarjan's algorithm [R. E. Tarjan, ``Depth-first
search and linear graph algorithms,'' {\sl SIAM Journal on Computing\/
\bf1} (1972), 146--160]. Articulation points and ordinary (connected)
components are also obtained as byproducts of the computation.

Two edges belong to the same  biconnected component---or ``bicomponent''
for short---if and only if they are identical or both belong to a
simple cycle. This defines an equivalence relation on edges.
The bicomponents of a connected graph form a
free tree, if we say that two bicomponents are adjacent when they have
a common vertex (i.e., when there is a vertex belonging to at least one edge
in each of the bicomponents). Such a vertex is called an {\sl articulation
point\/}; there is a unique articulation point between any two adjacent
bicomponents. If we choose one bicomponent to be the ``root'' of the
free tree, the other bicomponents can be represented conveniently as
lists of vertices, with the articulation point that leads toward the root
listed last. This program displays the bicomponents in exactly that way.

\fi

\M{2}We permit command-line options in typical \UNIX/ style so that a variety
of
graphs can be studied: The user can say `\.{-t}\<title>',
`\.{-n}\<number>', `\.{-x}\<number>', `\.{-f}\<number>',
`\.{-l}\<number>', `\.{-i}\<number>', `\.{-o}\<number>', and/or
`\.{-s}\<number>' to change the default values of the parameters in
the graph generated by \PB{$\\{book}(\|t,\|n,\|x,\|f,\|l,\|i,\|o,\|s)$}.

When the bicomponents are listed, each character in the book is identified by
a two-letter code, as found in the associated data file.
An explanation of these codes will appear first if the \.{-v} or \.{-V} option
is specified. The \.{-V} option prints a fuller explanation than~\.{-v}; it
also shows each character's weighted number of appearances.

The special command-line option \.{-g}$\langle\,$filename$\,\rangle$
overrides all others. It substitutes an external graph previously saved by
\PB{\\{save\_graph}} for the graphs produced by \PB{\\{book}}.


\Y\B\8\#\&{include} \.{"gb\_graph.h"}\C{ the GraphBase data structures }\6
\8\#\&{include} \.{"gb\_books.h"}\C{ the \PB{\\{book}} routine }\6
\8\#\&{include} \.{"gb\_io.h"}\C{ the \PB{\\{imap\_chr}} routine }\6
\8\#\&{include} \.{"gb\_save.h"}\C{ \PB{\\{restore\_graph}} }\6
\ATH\7
\X7:Global variables\X\6
\X4:Subroutines\X\7
\1\1${}\\{main}(\\{argc},\39\\{argv}){}$\6
\&{int} \\{argc};\C{ the number of command-line arguments }\6
\&{char} ${}{*}\\{argv}[\,]{}$;\C{ an array of strings containing those
arguments }\2\2\6
${}\{{}$\5
\1\&{Graph} ${}{*}\|g{}$;\C{ the graph we will work on }\6
\&{register} \&{Vertex} ${}{*}\|v{}$;\C{ the current vertex of interest }\6
\&{char} ${}{*}\|t\K\.{"anna"}{}$;\C{ the book to use }\6
\&{unsigned} \&{long} \|n${}\K\T{0}{}$;\C{ the desired number of vertices (0
means infinity) }\6
\&{unsigned} \&{long} \|x${}\K\T{0}{}$;\C{ the number of major characters to
exclude }\6
\&{unsigned} \&{long} \|f${}\K\T{0}{}$;\C{ the first chapter to include }\6
\&{unsigned} \&{long} \|l${}\K\T{0}{}$;\C{ the last chapter to include (0 means
infinity) }\6
\&{long} \|i${}\K\T{1}{}$;\C{ the weight for appearances in selected chapters }%
\6
\&{long} \|o${}\K\T{1}{}$;\C{ the weight for appearances in unselected chapters
}\6
\&{long} \|s${}\K\T{0}{}$;\C{ the random number seed }\7
\X3:Scan the command-line options\X;\6
\&{if} (\\{filename})\1\5
${}\|g\K\\{restore\_graph}(\\{filename});{}$\2\6
\&{else}\1\5
${}\|g\K\\{book}(\|t,\39\|n,\39\|x,\39\|f,\39\|l,\39\|i,\39\|o,\39\|s);{}$\2\6
\&{if} ${}(\|g\E\NULL){}$\5
${}\{{}$\1\6
${}\\{fprintf}(\\{stderr},\39\.{"Sorry,\ can't\ create}\)\.{\ the\ graph!\
(error\ c}\)\.{ode\ \%ld)\\n"},\39\\{panic\_code});{}$\6
\&{return} ${}{-}\T{1};{}$\6
\4${}\}{}$\2\6
${}\\{printf}(\.{"Biconnectivity\ anal}\)\.{ysis\ of\ \%s\\n\\n"},\39\|g\MG%
\\{id});{}$\6
\&{if} (\\{verbose})\1\5
\X5:Print the cast of selected characters\X;\2\6
\X12:Perform the Hopcroft-Tarjan algorithm on \PB{\|g}\X;\6
\&{return} \T{0};\C{ normal exit }\6
\4${}\}{}$\2\par
\fi

\M{3}\B\X3:Scan the command-line options\X${}\E{}$\6
\&{while} ${}(\MM\\{argc}){}$\5
${}\{{}$\1\6
\&{if} ${}(\\{strncmp}(\\{argv}[\\{argc}],\39\.{"-t"},\39\T{2})\E\T{0}){}$\1\5
${}\|t\K\\{argv}[\\{argc}]+\T{2};{}$\2\6
\&{else} \&{if} ${}(\\{sscanf}(\\{argv}[\\{argc}],\39\.{"-n\%lu"},\39{\AND}\|n)%
\E\T{1}){}$\1\5
;\2\6
\&{else} \&{if} ${}(\\{sscanf}(\\{argv}[\\{argc}],\39\.{"-x\%lu"},\39{\AND}\|x)%
\E\T{1}){}$\1\5
;\2\6
\&{else} \&{if} ${}(\\{sscanf}(\\{argv}[\\{argc}],\39\.{"-f\%lu"},\39{\AND}\|f)%
\E\T{1}){}$\1\5
;\2\6
\&{else} \&{if} ${}(\\{sscanf}(\\{argv}[\\{argc}],\39\.{"-l\%lu"},\39{\AND}\|l)%
\E\T{1}){}$\1\5
;\2\6
\&{else} \&{if} ${}(\\{sscanf}(\\{argv}[\\{argc}],\39\.{"-i\%ld"},\39{\AND}\|i)%
\E\T{1}){}$\1\5
;\2\6
\&{else} \&{if} ${}(\\{sscanf}(\\{argv}[\\{argc}],\39\.{"-o\%ld"},\39{\AND}\|o)%
\E\T{1}){}$\1\5
;\2\6
\&{else} \&{if} ${}(\\{sscanf}(\\{argv}[\\{argc}],\39\.{"-s\%ld"},\39{\AND}\|s)%
\E\T{1}){}$\1\5
;\2\6
\&{else} \&{if} ${}(\\{strcmp}(\\{argv}[\\{argc}],\39\.{"-v"})\E\T{0}){}$\1\5
${}\\{verbose}\K\T{1};{}$\2\6
\&{else} \&{if} ${}(\\{strcmp}(\\{argv}[\\{argc}],\39\.{"-V"})\E\T{0}){}$\1\5
${}\\{verbose}\K\T{2};{}$\2\6
\&{else} \&{if} ${}(\\{strncmp}(\\{argv}[\\{argc}],\39\.{"-g"},\39\T{2})\E%
\T{0}){}$\1\5
${}\\{filename}\K\\{argv}[\\{argc}]+\T{2};{}$\2\6
\&{else}\5
${}\{{}$\1\6
${}\\{fprintf}(\\{stderr},\39\.{"Usage:\ \%s\ [-ttitle]}\)%
\.{[-nN][-xN][-fN][-lN]}\)\.{[-iN][-oN][-sN][-v][}\)\.{-gfoo]\\n"},\39\\{argv}[%
\T{0}]);{}$\6
\&{return} ${}{-}\T{2};{}$\6
\4${}\}{}$\2\6
\4${}\}{}$\2\6
\&{if} (\\{filename})\1\5
${}\\{verbose}\K\T{0}{}$;\2\par
\U2.\fi

\M{4}\B\X4:Subroutines\X${}\E{}$\6
\&{char} ${}{*}\\{filename}\K\NULL{}$;\C{ external graph to be restored }\6
\&{char} \\{code\_name}[\T{3}][\T{3}];\7
\1\1\&{char} ${}{*}\\{vertex\_name}(\|v,\39\|i{}$)\C{ return (as a string) the
name of vertex \PB{\|v} }\6
\&{Vertex} ${}{*}\|v;{}$\6
\&{char} \|i;\C{ \PB{\|i} should be 0, 1, or 2 to avoid clash in \PB{\\{code%
\_name}} array }\2\2\6
${}\{{}$\1\6
\&{if} (\\{filename})\1\5
\&{return} \|v${}\MG\\{name}{}$;\C{ not a \PB{\\{book}} graph }\2\6
${}\\{code\_name}[\|i][\T{0}]\K\\{imap\_chr}(\|v\MG\\{short\_code}/\T{36});{}$\6
${}\\{code\_name}[\|i][\T{1}]\K\\{imap\_chr}(\|v\MG\\{short\_code}\MOD%
\T{36});{}$\6
\&{return} \\{code\_name}[\|i];\6
\4${}\}{}$\2\par
\U2.\fi

\M{5}\B\X5:Print the cast of selected characters\X${}\E{}$\6
${}\{{}$\1\6
\&{for} ${}(\|v\K\|g\MG\\{vertices};{}$ ${}\|v<\|g\MG\\{vertices}+\|g\MG\|n;{}$
${}\|v\PP){}$\5
${}\{{}$\1\6
\&{if} ${}(\\{verbose}\E\T{1}){}$\1\5
${}\\{printf}(\.{"\%s=\%s\\n"},\39\\{vertex\_name}(\|v,\39\T{0}),\39\|v\MG%
\\{name});{}$\2\6
\&{else}\1\5
${}\\{printf}(\.{"\%s=\%s,\ \%s\ [weight\ \%}\)\.{ld]\\n"},\39\\{vertex\_name}(%
\|v,\39\T{0}),\39\|v\MG\\{name},\39\|v\MG\\{desc},\3{-1}\39\|i*\|v\MG\\{in%
\_count}+\|o*\|v\MG\\{out\_count});{}$\2\6
\4${}\}{}$\2\6
\\{printf}(\.{"\\n"});\6
\4${}\}{}$\2\par
\U2.\fi

\N{1}{6}The algorithm.
The Hopcroft-Tarjan algorithm is inherently recursive. We will
implement the recursion explicitly via linked lists, instead of using
\CEE/'s runtime stack, because some computer systems bog down in the
presence of deeply nested recursion.

Each vertex goes through three stages during the algorithm. First it is
``unseen''; then it is ``active''; finally it becomes ``settled,'' when it
has been assigned to a bicomponent.

The data structures that represent the current state of the algorithm
are implemented by using five of the utility fields in each vertex:
\PB{\\{rank}}, \PB{\\{parent}}, \PB{\\{untagged}}, \PB{\\{link}}, and \PB{%
\\{min}}. We will consider each of
these in turn.

\fi

\M{7}First is the integer \PB{\\{rank}} field, which is zero when a vertex is
unseen.
As soon as the vertex is first examined, it becomes active and its \PB{%
\\{rank}}
becomes and remains nonzero. Indeed, the $k$th vertex to become active
will receive rank~$k$.

It's convenient to think of the Hopcroft-Tarjan algorithm as a simple adventure
game in which we want to explore all the rooms of a cave. Passageways between
the rooms allow two-way travel. When we come
into a room for the first time, we assign a new number to that room;
this is its rank. Later on we might happen to come into the same room
again, and we will notice that it has nonzero rank. Then we'll be able
to make a quick exit, saying ``we've already been here.'' (The extra
complexities of computer games, like dragons that might need to be
vanquished, do not arise.)

\Y\B\4\D$\\{rank}$ \5
$\|z.{}$\|I\C{ the \PB{\\{rank}} of a vertex is stored in utility field \PB{%
\|z} }\par
\Y\B\4\X7:Global variables\X${}\E{}$\6
\&{long} \\{nn};\C{ the number of vertices that have been seen }\par
\As8, 10, 13\ETs20.
\U2.\fi

\M{8}The active vertices will always form an oriented tree, whose arcs are
a subset of the arcs in the original graph. A tree arc from \PB{\|u} to~\PB{%
\|v}
will be represented by \PB{$\|v\MG\\{parent}\E\|u$}. Every active vertex has a
parent, which is usually another active vertex; the only exception is
the root of the tree, whose \PB{\\{parent}} is a dummy vertex called \PB{%
\\{dummy}}.
The dummy vertex has rank zero.

In the cave analogy, the ``parent'' of room \PB{\|v} is the room we were in
immediately before entering \PB{\|v} the first time. By following parent
pointers, we will be able to leave the cave whenever we want.

\Y\B\4\D$\\{parent}$ \5
$\|y.{}$\|V\C{ the \PB{\\{parent}} of a vertex is stored in utility field \PB{%
\|y} }\par
\Y\B\4\X7:Global variables\X${}\mathrel+\E{}$\6
\&{Vertex} \\{dummy};\C{ imaginary parent of the root vertex }\par
\fi

\M{9}All edges in the original undirected graph are explored systematically
during
a depth-first search. Whenever we look at an edge, we tag it so that
we won't need to explore it again. In a cave, for example, we might
mark each passageway between rooms once we've tried to go through~it.

In a GraphBase graph, undirected edges are represented as a pair of directed
arcs. Each of these arcs will be examined and eventually tagged.

The algorithm doesn't actually place a tag on its \PB{\&{Arc}} records;
instead,
each vertex \PB{\|v} has a pointer \PB{$\|v\MG\\{untagged}$} that leads to all
hitherto-unexplored arcs from~\PB{\|v}. The arcs of the list that appear
between \PB{$\|v\MG\\{arcs}$} and \PB{$\|v\MG\\{untagged}$} are the ones
already examined.

\Y\B\4\D$\\{untagged}$ \5
$\|x.{}$\|A\C{ the \PB{\\{untagged}} field points to an \PB{\&{Arc}} record, or
\PB{$\NULL$} }\par
\fi

\M{10}The algorithm maintains a special stack, the \PB{\\{active\_stack}},
which contains
all the currently active vertices. Each vertex has a \PB{\\{link}} field that
points
to the vertex that is next lower on its stack, or to \PB{$\NULL$} if the vertex
is
at the bottom. The vertices on \PB{\\{active\_stack}} always appear in
increasing
order of rank from bottom to top.

\Y\B\4\D$\\{link}$ \5
$\|w.{}$\|V\C{ the \PB{\\{link}} field of a vertex occupies utility field \PB{%
\|w} }\par
\Y\B\4\X7:Global variables\X${}\mathrel+\E{}$\6
\&{Vertex} ${}{*}\\{active\_stack}{}$;\C{ the top of the stack of active
vertices }\par
\fi

\M{11}Finally there's a \PB{\\{min}} field, which is the tricky part that makes
everything work. If vertex~\PB{\|v} is unseen or settled, its \PB{\\{min}}
field is
irrelevant. Otherwise \PB{$\|v\MG\\{min}$} points to the active vertex~\PB{\|u}
of smallest rank having the following property:
There is a directed path from \PB{\|v} to \PB{\|u} consisting of
zero or more mature tree arcs followed by a single non-tree arc.

What is a tree arc, you ask. And what is a mature arc? Good questions. At the
moment when arcs of the graph are tagged, we classify them either as tree
arcs (if they correspond to a new \PB{\\{parent}} link in the tree of active
nodes) or non-tree arcs (otherwise). The tree arcs therefore correspond to
passageways that have led us to new territory. A tree arc becomes mature
when it is no longer on the path from the root to the current vertex being
explored. We also say that a vertex becomes mature when it is
no longer on that path. All arcs from a mature vertex have been tagged.

We said before that every vertex is initially unseen, then active, and
finally settled. With our new definitions, we see further that every arc starts
out untagged, then it becomes either a non-tree arc or a tree arc. In the
latter case, the arc begins as an immature tree arc and eventually matures.

The dummy vertex is considered to be active, and we assume that
there is a non-tree arc from the root vertex back to \PB{\\{dummy}}. Thus
there is a non-tree arc from \PB{\|v} to \PB{$\|v\MG\\{parent}$} for all~\PB{%
\|v}, and \PB{$\|v\MG\\{min}$}
will always point to a vertex whose rank is less than or equal to
\PB{$\|v\MG\\{parent}\MG\\{rank}$}. It will turn out that \PB{$\|v\MG\\{min}$}
is always an ancestor
of~\PB{\|v}.

Just believe these definitions, for now. All will become clear soon.

\Y\B\4\D$\\{min}$ \5
$\|v.{}$\|V\C{ the \PB{\\{min}} field of a vertex occupies utility field \PB{%
\|v} }\par
\fi

\M{12}Depth-first search explores a graph by systematically visiting all
vertices and seeing what they can lead to. In the Hopcroft-Tarjan algorithm, as
we have said, the active vertices form an oriented tree. One of these
vertices is called the current vertex.

If the current vertex still has an arc that hasn't been tagged, we
tag one such arc and there are two cases: Either the arc leads to
an unseen vertex, or it doesn't. If it does, the arc becomes a tree
arc; the previously unseen vertex becomes active, and it becomes the
new current vertex.  On the other hand if the arc leads to a vertex
that has already been seen, the arc becomes a non-tree arc and the
current vertex doesn't change.

Finally there will come a time when the current vertex~\PB{\|v} has no
untagged arcs. At this point, the
algorithm might decide that \PB{\|v} and all its descendants
form a bicomponent, together with \PB{$\|v\MG\\{parent}$}.
Indeed, this condition turns out to be true if and only if
\PB{$\|v\MG\\{min}\E\|v\MG\\{parent}$}; a proof appears below. If so, \PB{\|v}
and all its descendants
become settled, and they leave the tree. If not, the tree arc from
\PB{\|v}'s parent~\PB{\|u} to~\PB{\|v} becomes mature, so the value of \PB{$\|v%
\MG\\{min}$} is
used to update the value of \PB{$\|u\MG\\{min}$}. In both cases, \PB{\|v}
becomes mature
and the new current vertex will be the parent of~\PB{\|v}. Notice that only the
value of \PB{$\|u\MG\\{min}$} needs to be updated, when the arc from \PB{\|u}
to~\PB{\|v}
matures; all other values \PB{$\|w\MG\\{min}$} stay the same, because a newly
mature arc has no mature predecessors.

The cave analogy helps to clarify the situation: Suppose we enter
room~\PB{\|v} from room~\PB{\|u}. If there's no way out of the subcave starting
at~\PB{\|v} unless we come back through~\PB{\|u}, and if we can get to~\PB{\|u}
from
all of \PB{\|v}'s descendants without passing through~\PB{\|v}, then room~\PB{%
\|v}
and its descendants will become a bicomponent together with~\PB{\|u}.  Once
such a bicomponent is identified, we close it off and don't explore
that subcave any further.

If \PB{\|v} is the root of the tree, it always has \PB{$\|v\MG\\{min}\E%
\\{dummy}$}, so it
will always define a new bicomponent at the moment it matures.  Then
the depth-first search will terminate, since \PB{\|v}~has no real parent.
But the Hopcroft-Tarjan algorithm will press on, trying to find a
vertex~\PB{\|u} that is still unseen. If such a vertex exists, a
new depth-first search will begin with \PB{\|u} as the root. This process
keeps on going until at last all vertices are happily settled.

The beauty of this algorithm is that it all works very efficiently
when we organize it as follows:

\Y\B\4\X12:Perform the Hopcroft-Tarjan algorithm on \PB{\|g}\X${}\E{}$\6
\X14:Make all vertices unseen and all arcs untagged\X;\6
\&{for} ${}(\\{vv}\K\|g\MG\\{vertices};{}$ ${}\\{vv}<\|g\MG\\{vertices}+\|g\MG%
\|n;{}$ ${}\\{vv}\PP){}$\1\6
\&{if} ${}(\\{vv}\MG\\{rank}\E\T{0}{}$)\C{ \PB{\\{vv}} is still unseen }\1\6
\X15:Perform a depth-first search with \PB{\\{vv}} as the root, finding the
bicomponents of all unseen vertices reachable from~\PB{\\{vv}}\X;\2\2\par
\U2.\fi

\M{13}\B\X7:Global variables\X${}\mathrel+\E{}$\6
\&{Vertex} ${}{*}\\{vv}{}$;\C{ sweeps over all vertices, making sure none is
left unseen }\par
\fi

\M{14}It's easy to get the data structures started, according to the
conventions stipulated above.

\Y\B\4\X14:Make all vertices unseen and all arcs untagged\X${}\E{}$\6
\&{for} ${}(\|v\K\|g\MG\\{vertices};{}$ ${}\|v<\|g\MG\\{vertices}+\|g\MG\|n;{}$
${}\|v\PP){}$\5
${}\{{}$\1\6
${}\|v\MG\\{rank}\K\T{0};{}$\6
${}\|v\MG\\{untagged}\K\|v\MG\\{arcs};{}$\6
\4${}\}{}$\2\6
${}\\{nn}\K\T{0};{}$\6
${}\\{active\_stack}\K\NULL;{}$\6
${}\\{dummy}.\\{rank}\K\T{0}{}$;\par
\U12.\fi

\M{15}The task of starting a depth-first search isn't too bad either.
Throughout
this part of the algorithm, variable~\PB{\|v} will point to the current vertex.

\Y\B\4\X15:Perform a depth-first search with \PB{\\{vv}} as the root, finding
the bicomponents of all unseen vertices reachable from~\PB{\\{vv}}\X${}\E{}$\6
${}\{{}$\1\6
${}\|v\K\\{vv};{}$\6
${}\|v\MG\\{parent}\K{\AND}\\{dummy};{}$\6
\X16:Make vertex \PB{\|v} active\X;\6
\&{do}\5
\X17:Explore one step from the current vertex~\PB{\|v}, possibly moving to
another current vertex and calling~it~\PB{\|v}\X\5
\&{while} ${}(\|v\I{\AND}\\{dummy});{}$\6
\4${}\}{}$\2\par
\U12.\fi

\M{16}\B\X16:Make vertex \PB{\|v} active\X${}\E{}$\6
$\|v\MG\\{rank}\K\PP\\{nn};{}$\6
${}\|v\MG\\{link}\K\\{active\_stack};{}$\6
${}\\{active\_stack}\K\|v;{}$\6
${}\|v\MG\\{min}\K\|v\MG\\{parent}{}$;\par
\Us15\ET17.\fi

\M{17}Now things get interesting. But we're just doing what any well-organized
spelunker would do when calmly exploring a cave.
There are three main cases,
depending on whether the current vertex stays where it is, moves
to a new child, or backtracks to a parent.

\Y\B\4\X17:Explore one step from the current vertex~\PB{\|v}, possibly moving
to another current vertex and calling~it~\PB{\|v}\X${}\E{}$\6
${}\{{}$\5
\1\&{register} \&{Vertex} ${}{*}\|u{}$;\C{ a vertex adjacent to \PB{\|v} }\6
\&{register} \&{Arc} ${}{*}\|a\K\|v\MG\\{untagged}{}$;\C{ \PB{\|v}'s first
remaining untagged arc, if any }\7
\&{if} (\|a)\5
${}\{{}$\1\6
${}\|u\K\|a\MG\\{tip};{}$\6
${}\|v\MG\\{untagged}\K\|a\MG\\{next}{}$;\C{ tag the arc from \PB{\|v} to \PB{%
\|u} }\6
\&{if} ${}(\|u\MG\\{rank}){}$\5
${}\{{}$\C{ we've seen \PB{\|u} already }\1\6
\&{if} ${}(\|u\MG\\{rank}<\|v\MG\\{min}\MG\\{rank}){}$\1\5
${}\|v\MG\\{min}\K\|u{}$;\C{ non-tree arc, just update \PB{$\|v\MG\\{min}$} }\2%
\6
\4${}\}{}$\5
\2\&{else}\5
${}\{{}$\C{ \PB{\|u} is presently unseen }\1\6
${}\|u\MG\\{parent}\K\|v{}$;\C{ the arc from \PB{\|v} to \PB{\|u} is a new tree
arc }\6
${}\|v\K\|u{}$;\C{ \PB{\|u} will now be the current vertex }\6
\X16:Make vertex \PB{\|v} active\X;\6
\4${}\}{}$\2\6
\4${}\}{}$\5
\2\&{else}\5
${}\{{}$\C{ all arcs from \PB{\|v} are tagged, so \PB{\|v} matures }\1\6
${}\|u\K\|v\MG\\{parent}{}$;\C{ prepare to backtrack in the tree }\6
\&{if} ${}(\|v\MG\\{min}\E\|u){}$\1\5
\X19:Remove \PB{\|v} and all its successors on the active stack from the tree,
and report them as a bicomponent of the graph together with~\PB{\|u}\X\2\6
\&{else}\C{ the arc from \PB{\|u} to \PB{\|v} has just matured,
making \PB{$\|v\MG\\{min}$} visible from \PB{\|u} }\6
\, \&{if} ${}(\|v\MG\\{min}\MG\\{rank}<\|u\MG\\{min}\MG\\{rank}){}$\1\5
${}\|u\MG\\{min}\K\|v\MG\\{min};{}$\2\6
${}\|v\K\|u{}$;\C{ the former parent of \PB{\|v} is the new current vertex \PB{%
\|v} }\6
\4${}\}{}$\2\6
\4${}\}{}$\2\par
\U15.\fi

\M{18}The elements of the active stack are always in order by rank, and
all children of a vertex~\PB{\|v} in the tree have rank higher than~\PB{\|v}.
The Hopcroft-Tarjan algorithm relies on a converse property:
{\sl All active nodes whose rank exceeds that of the current vertex~\PB{\|v}
are descendants of~\PB{\|v}.} (This property holds because the algorithm has
constructed the tree by assigning ranks in preorder, ``the order of
succession to the throne.'' First come \PB{\|v}'s firstborn and descendants,
then the nextborn, and so on.) Therefore the descendants of the
current vertex always appear consecutively at the top of the stack.

Suppose \PB{\|v} is a mature, active vertex with \PB{$\|v\MG\\{min}\E\|v\MG%
\\{parent}$}, and
let \PB{$\|u\K\|v\MG\\{parent}$}. We want to prove that \PB{\|v} and its
descendants,
together with~\PB{\|u} and all edges between these vertices, form a
biconnected graph.  Call this subgraph~$H$. The parent links
define a subtree of~$H$, rooted at~\PB{\|u}, and \PB{\|v} is the only vertex
having \PB{\|u} as a parent (because all other vertices are descendants
of~\PB{\|v}). Let \PB{\|x} be any vertex of~$H$ different from \PB{\|u} and %
\PB{\|v}.
Then there is a path from \PB{\|x} to \PB{$\|x\MG\\{min}$} that does not touch~%
\PB{$\|x\MG\\{parent}$},
and \PB{$\|x\MG\\{min}$} is a proper ancestor of \PB{$\|x\MG\\{parent}$}. This
property
is sufficient to establish the biconnectedness of~$H$. (A proof appears
at the conclusion of this program.) Moreover, we cannot add any
more vertices to~$H$ without losing biconnectivity. If \PB{\|w}~is another
vertex, either \PB{\|w} has been output already as a non-articulation point
of a previous biconnected component, or we can prove that
there is no path from \PB{\|w} to~\PB{\|v} that avoids the vertex~\PB{\|u}.

\fi

\M{19}Therefore we are justified in settling \PB{\|v} and its active
descendants now.
Removing them from the tree of active vertices does not remove any
vertex from which there is a path to a vertex of rank less than \PB{$\|u\MG%
\\{rank}$}.
Hence their removal does not affect the validity of the \PB{$\|w\MG\\{min}$}
value
for any vertex~\PB{\|w} that remains active.

A slight technicality arises with respect to whether or not
the parent of~\PB{\|v}, vertex~\PB{\|u}, is part of the present bicomponent.
When \PB{\|u} is the dummy vertex, we have already printed the final
bicomponent
of a connected component of the original graph, unless \PB{\|v} was
an isolated vertex. Otherwise \PB{\|u} is an
articulation point that will occur in subsequent bicomponents,
unless the new bicomponent is the final bicomponent of a connected component.
(This aspect of the algorithm is probably its most subtle point;
consideration of an example or two should clarify everything.)

We print out enough information for a reader to verify the
biconnectedness of the claimed component easily.

\Y\B\4\X19:Remove \PB{\|v} and all its successors on the active stack from the
tree, and report them as a bicomponent of the graph together with~\PB{\|u}\X${}%
\E{}$\6
\&{if} ${}(\|u\E{\AND}\\{dummy}){}$\5
${}\{{}$\C{ \PB{\\{active\_stack}} contains just \PB{\|v} }\1\6
\&{if} (\\{artic\_pt})\1\5
${}\\{printf}(\.{"\ and\ \%s\ (this\ ends\ }\)\.{a\ connected\ componen}\)\.{t\
of\ the\ graph)\\n"},\39\\{vertex\_name}(\\{artic\_pt},\39\T{0}));{}$\2\6
\&{else}\1\5
${}\\{printf}(\.{"Isolated\ vertex\ \%s\\}\)\.{n"},\39\\{vertex\_name}(\|v,\39%
\T{0}));{}$\2\6
${}\\{active\_stack}\K\\{artic\_pt}\K\NULL;{}$\6
\4${}\}{}$\5
\2\&{else}\5
${}\{{}$\5
\1\&{register} \&{Vertex} ${}{*}\|t{}$;\C{ runs through the vertices of the
                    new bicomponent }\7
\&{if} (\\{artic\_pt})\1\5
${}\\{printf}(\.{"\ and\ articulation\ p}\)\.{oint\ \%s\\n"},\39\\{vertex%
\_name}(\\{artic\_pt},\39\T{0}));{}$\2\6
${}\|t\K\\{active\_stack};{}$\6
${}\\{active\_stack}\K\|v\MG\\{link};{}$\6
${}\\{printf}(\.{"Bicomponent\ \%s"},\39\\{vertex\_name}(\|v,\39\T{0}));{}$\6
\&{if} ${}(\|t\E\|v){}$\1\5
\\{putchar}(\.{'\\n'});\C{ single vertex }\2\6
\&{else}\5
${}\{{}$\1\6
\\{printf}(\.{"\ also\ includes:\\n"});\6
\&{while} ${}(\|t\I\|v){}$\5
${}\{{}$\1\6
${}\\{printf}(\.{"\ \%s\ (from\ \%s;\ ..to\ }\)\.{\%s)\\n"},\39\\{vertex%
\_name}(\|t,\39\T{0}),\39\\{vertex\_name}(\|t\MG\\{parent},\39\T{1}),\39%
\\{vertex\_name}(\|t\MG\\{min},\39\T{2}));{}$\6
${}\|t\K\|t\MG\\{link};{}$\6
\4${}\}{}$\2\6
\4${}\}{}$\2\6
${}\\{artic\_pt}\K\|u{}$;\C{ the printout will be finished later }\6
\4${}\}{}$\2\par
\U17.\fi

\M{20}Like all global variables, \PB{\\{artic\_pt}} is initially zero (\PB{$%
\NULL$}).

\Y\B\4\X7:Global variables\X${}\mathrel+\E{}$\6
\&{Vertex} ${}{*}\\{artic\_pt}{}$;\C{ articulation point to be printed if the
current   bicomponent isn't the last in its connected component }\par
\fi

\N{1}{21}Proofs.
The program is done, but we still should prove that it works.
First we want to clarify the informal definition by verifying that
the cycle relation between edges, as stated in the introduction, is indeed an
equivalence relation.

\def\dash{\mathrel-\joinrel\joinrel\mathrel-}
Suppose $u\dash v$ and $w\dash x$ are edges of a simple cycle~$C$, while
$w\dash x$ and $y\dash z$ are edges of a simple cycle~$D$. We want to show
that there is a simple cycle containing the edges $u\dash v$ and $y\dash z$.
There are vertices $a,b\in C$ such that $a\dash^\ast y\dash z\dash^\ast b$
is a subpath of~$D$ containing no other vertices of $C$ besides $a$ and~$b$.
Join this subpath to the subpath in $C$ that runs from $b$ to~$a$ through
the edge $u\dash v$.

Therefore the stated relation between edges is transitive, and it is
an equivalence relation.
A graph is biconnected if it contains a single vertex, or if each of
its vertices is adjacent to at least one other vertex and any two edges are
equivalent.

\fi

\M{22}Next we prove the well-known fact that a graph is biconnected if and
only if it is connected and, for any three distinct vertices $x$,
$y$,~$z$, it contains a path from $x$ to~$y$ that does not touch~$z$.
Call the latter condition property~P.

Suppose $G$ is biconnected, and let $x,y$ be distinct vertices of~$G$.
Then there exist edges $u\dash x$ and $v\dash y$, which are either
identical (hence $x$ and~$y$ are adjacent) or part of a simple cycle
(hence there are two paths from $x$ to~$y$, having no other vertices in
common). Thus $G$ has property~P.

Suppose, conversely, that $G$ has property~P, and let $u\dash v,
w\dash x$ be distinct edges of~$G$. We want to show that these edges
belong to some simple cycle. The proof is by induction on
$k=\min\bigl(d(u,w),d(u,x),\allowbreak d(v,w),d(v,x)\bigr)$, where $d$~denotes
distance. If $k=0$, property~P gives the result directly. If $k>0$,
we can assume by symmetry that $k=d(u,w)$; so there's a vertex $y$
with $u\dash y$ and $d(y,w)=k-1$. And we have $u\dash v$ equivalent to
$u\dash y$ by property~P, $u\dash y$ equivalent to $w\dash x$ by induction,
hence $u\dash v$ is equivalent to $w\dash x$ by transitivity.

\fi

\M{23}Finally, we prove that $G$ satisfies property~P if it has the
following properties: (1)~There are two distinguished vertices $u$
and~$v$.  (2)~Some of the edges of~$G$ form a subtree rooted at~$u$,
and $v$ is the only vertex whose parent in this tree is~$u$.
(3)~Every vertex~$x$ other than $u$ or $v$ has a path to its
grandparent that does not go through its parent.

If property P doesn't hold, there are distinct vertices $x,y,z$ such
that every path from $x$ to~$y$ goes through~$z$. In particular, $z$ must be
between $x$ and~$y$ in the unique path~$\pi$ that joins them in the subtree.
It follows that $z\ne u$ is the parent of some node $z'$ in that path; hence
$z'\ne u$ and $z'\ne v$. But we can
avoid $z$ by going from $z'$ to the grandparent of $z'$, which is
also part of path~$\pi$ unless $z$ is also the parent of another node
$z''$ in~$\pi$. In the latter case, however,
we can avoid $z$ by going from $z'$ to the grandparent of $z'$ and from there
to $z''$, since $z'$ and $z''$ have the same grandparent.

\fi

\N{1}{24}Index. We close with a list that shows where the identifiers of this
program are defined and used.

\fi


\inx
\fin
\con
