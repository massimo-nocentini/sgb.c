\input cwebmac
% This file is part of the Stanford GraphBase (c) Stanford University 1993
% This material goes at the beginning of all Stanford GraphBase CWEB files

\def\topofcontents{
  \leftline{\sc\today\ at \hours}\bigskip\bigskip
  \centerline{\titlefont\title}}

\font\ninett=cmtt9
\def\botofcontents{\vskip 0pt plus 1filll
    \ninerm\baselineskip10pt
    \noindent\copyright\ 1993 Stanford University
    \bigskip\noindent
    This file may be freely copied and distributed, provided that
    no changes whatsoever are made. All users are asked to help keep
    the Stanford GraphBase files consistent and ``uncorrupted,''
    identical everywhere in the world. Changes are permissible only
    if the modified file is given a new name, different from the names of
    existing files in the Stanford GraphBase, and only if the modified file is
    clearly identified as not being part of that GraphBase.
    (The {\ninett CWEB} system has a ``change file'' facility by
    which users can easily make minor alterations without modifying
    the master source files in any way. Everybody is supposed to use
    change files instead of changing the files.)
    The author has tried his best to produce correct and useful programs,
    in order to help promote computer science research,
    but no warranty of any kind should be assumed.
    \smallskip\noindent
    Preliminary work on the Stanford GraphBase project
    was supported in part by National Science
    Foundation grant CCR-86-10181.}

\def\prerequisite#1{\def\startsection{\noindent
    Important: Before reading {\sc\title},
    please read or at least skim the program for {\sc#1}.\bigskip
    \let\startsection=\stsec\stsec}}
\def\prerequisites#1#2{\def\startsection{\noindent
    Important: Before reading {\sc\title}, please read
    or at least skim the programs for {\sc#1} and {\sc#2}.\bigskip
    \let\startsection=\stsec\stsec}}



\def\title{ECON\_\,ORDER}
\def\<#1>{$\langle${\rm#1}$\rangle$}

\prerequisite{GB\_\,ECON}

\N{1}{1}Near-triangular ordering.
This demonstration program takes a matrix of data
constructed by the {\sc GB\_\,ECON} module and permutes the economic sectors
so that the first sectors of the ordering tend to be producers of
primary materials for other industries, while the last sectors
tend to be final-product
industries that deliver their output mostly to end users.

More precisely, suppose the rows of the matrix represent the outputs
of a sector and the columns represent the inputs. This program attempts
to find a permutation of rows and columns that minimizes the sum of
the elements below the main diagonal. (If this sum were zero, the
matrix would be upper triangular; each supplier of a sector would precede
it in the ordering, while each customer of that sector would follow it.)

The general problem of finding a minimizing permutation is NP-complete;
it includes, as a very special case, the {\sc FEEDBACK ARC SET} problem
discussed in Karp's classic paper [{\sl Complexity of Computer
Computations} (Plenum Press, 1972), 85--103].
But sophisticated ``branch and cut'' methods have been developed that work
well in practice on problems of reasonable size.
Here we use a simple heuristic downhill method
to find a permutation that is locally optimum, in the sense that
the below-diagonal sum does not decrease if any individual
sector is moved to another position while preserving the relative order
of the other sectors. We start with a random permutation and repeatedly
improve it, choosing the improvement that gives the least positive
gain at each step. A primary motive for the present implementation
was to get further experience with this method of cautious descent, which
was proposed by A. M. Gleason in {\sl AMS Proceedings of Symposia in Applied
Mathematics\/ \bf10} (1958), 175--178. (See the comments following
the program below.)

\fi

\M{2}As explained in {\sc GB\_\,ECON}, the subroutine call \PB{$\\{econ}(\|n,%
\T{2},\T{0},\|s)$}
constructs a graph whose \PB{$\|n\Z\T{79}$} vertices represent sectors of the
U.S. economy and whose arcs $u\to v$ are assigned numbers corresponding to the
flow of products from sector~\PB{\|u} to sector~\PB{\|v}. When \PB{$\|n<%
\T{79}$}, the
\PB{\|n} sectors are obtained from a basic set of 79 sectors by
combining related commodities. If \PB{$\|s\K\T{0}$}, the combination is done in
a way that tends to equalize the row sums, while if \PB{$\|s>\T{0}$}, the
combination
is done by choosing a random subtree of a given 79-leaf tree;
the ``randomness'' is fully determined by the value of~\PB{\|s}.

This program uses two random number seeds, one for \PB{\\{econ}} and one
for choosing the random initial permutation. The former is called~\PB{\|s}
and the latter is called~\PB{\|t}. A further parameter, \PB{\|r}, governs the
number of repetitions to be made; the machine will try \PB{\|r}~different
starting permutations
on the same matrix. When \PB{$\|r>\T{1}$}, new solutions are displayed only
when
they improve on the previous best.

By default, \PB{$\|n\K\T{79}$}, \PB{$\|r\K\T{1}$}, and \PB{$\|s\K\|t\K\T{0}$}.
The user can change these
default parameters by specifying options
on the command line, at least in a \UNIX/ implementation, thereby
obtaining a variety of special effects. The relevant
command-line options are \.{-n}\<number>, \.{-r}\<number>,
\.{-s}\<number>, and/or \.{-t}\<number>. Additional options
\.{-v} (verbose), \.{-V} (extreme verbosity), and \.{-g}
(greedy or steepest descent instead of cautious descent) are also provided.

Here is the overall layout of this \CEE/ program:

\Y\B\8\#\&{include} \.{"gb\_graph.h"}\C{ the GraphBase data structures }\6
\8\#\&{include} \.{"gb\_flip.h"}\C{ the random number generator }\6
\8\#\&{include} \.{"gb\_econ.h"}\C{ the \PB{\\{econ}} routine }\6
\ATH\7
\X3:Global variables\X\7
\1\1${}\\{main}(\\{argc},\39\\{argv}){}$\6
\&{int} \\{argc};\C{ the number of command-line arguments }\6
\&{char} ${}{*}\\{argv}[\,]{}$;\C{ an array of strings containing those
arguments }\2\2\6
${}\{{}$\5
\1\&{unsigned} \&{long} \|n${}\K\T{79}{}$;\C{ the desired number of sectors }\6
\&{long} \|s${}\K\T{0}{}$;\C{ random \\{seed} for \PB{\\{econ}} }\6
\&{long} \|t${}\K\T{0}{}$;\C{ random \\{seed} for initial permutation }\6
\&{unsigned} \&{long} \|r${}\K\T{1}{}$;\C{ the number of repetitions }\6
\&{long} \\{greedy}${}\K\T{0}{}$;\C{ should we use steepest descent? }\6
\&{register} \&{long} \|j${},\39\|k{}$;\C{ all-purpose indices }\7
\X4:Scan the command-line options\X;\6
${}\|g\K\\{econ}(\|n,\39\T{2\$L},\39\T{0\$L},\39\|s);{}$\6
\&{if} ${}(\|g\E\NULL){}$\5
${}\{{}$\1\6
${}\\{fprintf}(\\{stderr},\39\.{"Sorry,\ can't\ create}\)\.{\ the\ matrix!\
(error\ }\)\.{code\ \%ld)\\n"},\39\\{panic\_code});{}$\6
\&{return} ${}{-}\T{1};{}$\6
\4${}\}{}$\2\6
${}\\{printf}(\.{"Ordering\ the\ sector}\)\.{s\ of\ \%s,\ using\ seed\ }\)\.{%
\%ld:\\n"},\39\|g\MG\\{id},\39\|t);{}$\6
${}\\{printf}(\.{"\ (\%s\ descent\ method}\)\.{)\\n"},\39\\{greedy}\?%
\.{"Steepest"}:\.{"Cautious"});{}$\6
\X5:Put the graph data into matrix form\X;\6
\X6:Print an obvious lower bound\X;\6
\\{gb\_init\_rand}(\|t);\6
\&{while} ${}(\|r\MM){}$\1\5
\X8:Find a locally optimum permutation and report the below-diagonal sum\X;\2\6
\&{return} \T{0};\C{ normal exit }\6
\4${}\}{}$\2\par
\fi

\M{3}Besides the matrix $M$ of input/output coefficients, we will find it
convenient to use the matrix $\Delta$, where $\Delta_{jk}=M_{jk}-M_{kj}$.

\Y\B\4\D$\.{INF}$ \5
\T{\^7fffffff}\C{ infinity (or darn near) }\par
\Y\B\4\X3:Global variables\X${}\E{}$\6
\&{Graph} ${}{*}\|g{}$;\C{ the graph we will work on }\6
\&{long} \\{mat}[\T{79}][\T{79}];\C{ the corresponding matrix }\6
\&{long} \\{del}[\T{79}][\T{79}];\C{ skew-symmetric differences }\6
\&{long} \\{best\_score}${}\K\.{INF}{}$;\C{ the smallest below-diagonal sum
we've seen so far }\par
\As7\ET12.
\U2.\fi

\M{4}\B\X4:Scan the command-line options\X${}\E{}$\6
\&{while} ${}(\MM\\{argc}){}$\5
${}\{{}$\1\6
\&{if} ${}(\\{sscanf}(\\{argv}[\\{argc}],\39\.{"-n\%lu"},\39{\AND}\|n)\E%
\T{1}){}$\1\5
;\2\6
\&{else} \&{if} ${}(\\{sscanf}(\\{argv}[\\{argc}],\39\.{"-r\%lu"},\39{\AND}\|r)%
\E\T{1}){}$\1\5
;\2\6
\&{else} \&{if} ${}(\\{sscanf}(\\{argv}[\\{argc}],\39\.{"-s\%ld"},\39{\AND}\|s)%
\E\T{1}){}$\1\5
;\2\6
\&{else} \&{if} ${}(\\{sscanf}(\\{argv}[\\{argc}],\39\.{"-t\%ld"},\39{\AND}\|t)%
\E\T{1}){}$\1\5
;\2\6
\&{else} \&{if} ${}(\\{strcmp}(\\{argv}[\\{argc}],\39\.{"-v"})\E\T{0}){}$\1\5
${}\\{verbose}\K\T{1};{}$\2\6
\&{else} \&{if} ${}(\\{strcmp}(\\{argv}[\\{argc}],\39\.{"-V"})\E\T{0}){}$\1\5
${}\\{verbose}\K\T{2};{}$\2\6
\&{else} \&{if} ${}(\\{strcmp}(\\{argv}[\\{argc}],\39\.{"-g"})\E\T{0}){}$\1\5
${}\\{greedy}\K\T{1};{}$\2\6
\&{else}\5
${}\{{}$\1\6
${}\\{fprintf}(\\{stderr},\39\.{"Usage:\ \%s\ [-nN][-rN}\)%
\.{][-sN][-tN][-g][-v][}\)\.{-V]\\n"},\39\\{argv}[\T{0}]);{}$\6
\&{return} ${}{-}\T{2};{}$\6
\4${}\}{}$\2\6
\4${}\}{}$\2\par
\U2.\fi

\M{5}The optimum permutation is a function only of the $\Delta$ matrix, because
we can subtract any constant from both $M_{jk}$ and $M_{kj}$ without changing
the basic problem.

\Y\B\4\X5:Put the graph data into matrix form\X${}\E{}$\6
${}\{{}$\5
\1\&{register} \&{Vertex} ${}{*}\|v;{}$\6
\&{register} \&{Arc} ${}{*}\|a;{}$\7
${}\|n\K\|g\MG\|n;{}$\6
\&{for} ${}(\|v\K\|g\MG\\{vertices};{}$ ${}\|v<\|g\MG\\{vertices}+\|n;{}$ ${}%
\|v\PP){}$\1\6
\&{for} ${}(\|a\K\|v\MG\\{arcs};{}$ \|a; ${}\|a\K\|a\MG\\{next}){}$\1\5
${}\\{mat}[\|v-\|g\MG\\{vertices}][\|a\MG\\{tip}-\|g\MG\\{vertices}]\K\|a\MG%
\\{flow};{}$\2\2\6
\&{for} ${}(\|j\K\T{0};{}$ ${}\|j<\|n;{}$ ${}\|j\PP){}$\1\6
\&{for} ${}(\|k\K\T{0};{}$ ${}\|k<\|n;{}$ ${}\|k\PP){}$\1\5
${}\\{del}[\|j][\|k]\K\\{mat}[\|j][\|k]-\\{mat}[\|k][\|j];{}$\2\2\6
\4${}\}{}$\2\par
\U2.\fi

\M{6}Nontrivial lower bounds that can be made strong enough to find provably
optimum solutions to the ordering problem can be based on linear programming,
as shown for example by Gr\"otschel, J\"unger, and Reinelt
[{\sl Operations Research \bf32} (1984), 1195--1220].
The basic idea is to formulate the problem as
the task of minimizing $\sum M_{jk}x_{jk}$ for integer variables $x_{jk}\ge0$,
subject to the conditions $x_{jk}+x_{kj}=1$ and $x_{ik}\le x_{ij}+x_{jk}$
for all triples $(i,j,k)$ of distinct subscripts; these conditions are
necessary and sufficient. Relaxing the integrality constraints gives a
lower bound, and we can also add additional inequalities such as
$x_{14}+x_{25}+x_{36}+x_{42}+x_{43}+x_{51}+x_{53}+x_{61}+x_{62}\le7$.
The interesting story of inequalities like this has been surveyed by
P. C. Fishburn [{\sl Mathematical Social Sciences\/ \bf23} (1992), 67--80].

However, our goal is more modest---we just want
to study two of the simplest heuristics. So we will be happy with a trivial
bound based only on the constraints $x_{jk}+x_{kj}=1$.

\Y\B\4\X6:Print an obvious lower bound\X${}\E{}$\6
${}\{{}$\5
\1\&{register} \&{long} \\{sum}${}\K\T{0};{}$\7
\&{for} ${}(\|j\K\T{1};{}$ ${}\|j<\|n;{}$ ${}\|j\PP){}$\1\6
\&{for} ${}(\|k\K\T{0};{}$ ${}\|k<\|j;{}$ ${}\|k\PP){}$\1\6
\&{if} ${}(\\{mat}[\|j][\|k]\Z\\{mat}[\|k][\|j]){}$\1\5
${}\\{sum}\MRL{+{\K}}\\{mat}[\|j][\|k];{}$\2\6
\&{else}\1\5
${}\\{sum}\MRL{+{\K}}\\{mat}[\|k][\|j];{}$\2\2\2\6
${}\\{printf}(\.{"(The\ amount\ of\ feed}\)\.{-forward\ must\ be\ at\ }\)%
\.{least\ \%ld.)\\n"},\39\\{sum});{}$\6
\4${}\}{}$\2\par
\U2.\fi

\N{1}{7}Descent.
At each stage in our search, \PB{\\{mapping}} will be the current permutation;
in other words, the sector in row and column~\PB{\|k} will be
\PB{$\|g\MG\\{vertices}+\\{mapping}[\|k]$}. The current below-diagonal sum will
be
the value of \PB{\\{score}}. We will not actually have to permute anything
inside of \PB{\\{mat}}.

\Y\B\4\D$\\{sec\_name}(\|k)$ \5
$(\|g\MG\\{vertices}+\\{mapping}[\|k])\MG{}$\\{name}\par
\Y\B\4\X3:Global variables\X${}\mathrel+\E{}$\6
\&{long} \\{mapping}[\T{79}];\C{ current permutation }\6
\&{long} \\{score};\C{ current sum of elements above main diagonal }\6
\&{long} \\{steps};\C{ the number of iterations so far }\par
\fi

\M{8}\B\X8:Find a locally optimum permutation and report the below-diagonal sum%
\X${}\E{}$\6
${}\{{}$\1\6
\X9:Initialize \PB{\\{mapping}} to a random permutation\X;\6
\&{while} (\T{1})\5
${}\{{}$\1\6
\X10:Figure out the next move to make; \PB{\&{break}} if at local optimum\X;\6
\&{if} (\\{verbose})\1\5
${}\\{printf}(\.{"\%8ld\ after\ step\ \%ld}\)\.{\\n"},\39\\{score},\39%
\\{steps});{}$\2\6
\&{else} \&{if} ${}(\\{steps}\MOD\T{1000}\E\T{0}\W\\{steps}>\T{0}){}$\5
${}\{{}$\1\6
\\{putchar}(\.{'.'});\6
\\{fflush}(\\{stdout});\C{ progress report }\6
\4${}\}{}$\2\6
\X13:Take the next step\X;\6
\4${}\}{}$\2\6
${}\\{printf}(\.{"\\n\%s\ is\ \%ld,\ found\ }\)\.{after\ \%ld\ step\%s.\\n"},%
\3{-1}\39\\{best\_score}\E\.{INF}\?\.{"Local\ minimum\ feed-}\)\.{forward"}:%
\.{"Another\ local\ minim}\)\.{um"},\3{-1}\39\\{score},\39\\{steps},\39%
\\{steps}\E\T{1}\?\.{""}:\.{"s"});{}$\6
\&{if} ${}(\\{verbose}\V\\{score}<\\{best\_score}){}$\5
${}\{{}$\1\6
\\{printf}(\.{"The\ corresponding\ e}\)\.{conomic\ order\ is:\\n"});\6
\&{for} ${}(\|k\K\T{0};{}$ ${}\|k<\|n;{}$ ${}\|k\PP){}$\1\5
${}\\{printf}(\.{"\ \%s\\n"},\39\\{sec\_name}(\|k));{}$\2\6
\&{if} ${}(\\{score}<\\{best\_score}){}$\1\5
${}\\{best\_score}\K\\{score};{}$\2\6
\4${}\}{}$\2\6
\4${}\}{}$\2\par
\U2.\fi

\M{9}\B\X9:Initialize \PB{\\{mapping}} to a random permutation\X${}\E{}$\6
$\\{steps}\K\\{score}\K\T{0};{}$\6
\&{for} ${}(\|k\K\T{0};{}$ ${}\|k<\|n;{}$ ${}\|k\PP){}$\5
${}\{{}$\1\6
${}\|j\K\\{gb\_unif\_rand}(\|k+\T{1});{}$\6
${}\\{mapping}[\|k]\K\\{mapping}[\|j];{}$\6
${}\\{mapping}[\|j]\K\|k;{}$\6
\4${}\}{}$\2\6
\&{for} ${}(\|j\K\T{1};{}$ ${}\|j<\|n;{}$ ${}\|j\PP){}$\1\6
\&{for} ${}(\|k\K\T{0};{}$ ${}\|k<\|j;{}$ ${}\|k\PP){}$\1\5
${}\\{score}\MRL{+{\K}}\\{mat}[\\{mapping}[\|j]][\\{mapping}[\|k]];{}$\2\2\6
\&{if} ${}(\\{verbose}>\T{1}){}$\5
${}\{{}$\1\6
\\{printf}(\.{"\\nInitial\ permutati}\)\.{on:\\n"});\6
\&{for} ${}(\|k\K\T{0};{}$ ${}\|k<\|n;{}$ ${}\|k\PP){}$\1\5
${}\\{printf}(\.{"\ \%s\\n"},\39\\{sec\_name}(\|k));{}$\2\6
\4${}\}{}$\2\par
\U8.\fi

\M{10}If we move, say, \PB{\\{mapping}[\T{5}]} to \PB{\\{mapping}[\T{3}]} and
shift the previous
entries \PB{\\{mapping}[\T{3}]} and \PB{\\{mapping}[\T{4}]} right one, the
score decreases by
$$\hbox{\PB{$\\{del}[\\{mapping}[\T{5}]][\\{mapping}[\T{3}]]+\\{del}[%
\\{mapping}[\T{5}]][\\{mapping}[\T{4}]]$}}\,.$$
Similarly, if we move \PB{\\{mapping}[\T{5}]} to \PB{\\{mapping}[\T{7}]} and
shift the previous
entries \PB{\\{mapping}[\T{6}]} and \PB{\\{mapping}[\T{7}]} left one, the score
decreases by
$$\hbox{\PB{$\\{del}[\\{mapping}[\T{6}]][\\{mapping}[\T{5}]]+\\{del}[%
\\{mapping}[\T{7}]][\\{mapping}[\T{5}]]$}}\,.$$

The number of possible moves is $(n-1)^2$. Our job is to find the
one that makes the score decrease, but by as little as possible (or, if
\PB{$\\{greedy}\I\T{0}$}, to make the score decrease as much as possible).

\Y\B\4\X10:Figure out the next move to make; \PB{\&{break}} if at local optimum%
\X${}\E{}$\6
$\\{best\_d}\K\\{greedy}\?\T{0}:\.{INF};{}$\6
${}\\{best\_k}\K{-}\T{1};{}$\6
\&{for} ${}(\|k\K\T{0};{}$ ${}\|k<\|n;{}$ ${}\|k\PP){}$\5
${}\{{}$\5
\1\&{register} \&{long} \|d${}\K\T{0};{}$\7
\&{for} ${}(\|j\K\|k-\T{1};{}$ ${}\|j\G\T{0};{}$ ${}\|j\MM){}$\5
${}\{{}$\1\6
${}\|d\MRL{+{\K}}\\{del}[\\{mapping}[\|k]][\\{mapping}[\|j]];{}$\6
\X11:Record the move from \PB{\|k} to \PB{\|j}, if \PB{\|d} is better than \PB{%
\\{best\_d}}\X;\6
\4${}\}{}$\2\6
${}\|d\K\T{0};{}$\6
\&{for} ${}(\|j\K\|k+\T{1};{}$ ${}\|j<\|n;{}$ ${}\|j\PP){}$\5
${}\{{}$\1\6
${}\|d\MRL{+{\K}}\\{del}[\\{mapping}[\|j]][\\{mapping}[\|k]];{}$\6
\X11:Record the move from \PB{\|k} to \PB{\|j}, if \PB{\|d} is better than \PB{%
\\{best\_d}}\X;\6
\4${}\}{}$\2\6
\4${}\}{}$\2\6
\&{if} ${}(\\{best\_k}<\T{0}){}$\1\5
\&{break};\2\par
\U8.\fi

\M{11}\B\X11:Record the move from \PB{\|k} to \PB{\|j}, if \PB{\|d} is better
than \PB{\\{best\_d}}\X${}\E{}$\6
\&{if} ${}(\|d>\T{0}\W(\\{greedy}\?\|d>\\{best\_d}:\|d<\\{best\_d})){}$\5
${}\{{}$\1\6
${}\\{best\_k}\K\|k;{}$\6
${}\\{best\_j}\K\|j;{}$\6
${}\\{best\_d}\K\|d;{}$\6
\4${}\}{}$\2\par
\U10.\fi

\M{12}\B\X3:Global variables\X${}\mathrel+\E{}$\6
\&{long} \\{best\_d};\C{ best improvement seen so far on this step }\6
\&{long} \\{best\_k}${},\39\\{best\_j}{}$;\C{ moving \PB{\\{best\_k}} to \PB{%
\\{best\_j}} improves by \PB{\\{best\_d}} }\par
\fi

\M{13}\B\X13:Take the next step\X${}\E{}$\6
\&{if} ${}(\\{verbose}>\T{1}){}$\1\5
${}\\{printf}(\.{"Now\ move\ \%s\ to\ the\ }\)\.{\%s,\ past\\n"},\39\\{sec%
\_name}(\\{best\_k}),\39\\{best\_j}<\\{best\_k}\?\.{"left"}:\.{"right"});{}$\2\6
${}\|j\K\\{best\_k};{}$\6
${}\|k\K\\{mapping}[\|j];{}$\6
\&{do}\5
${}\{{}$\1\6
\&{if} ${}(\\{best\_j}<\\{best\_k}){}$\1\5
${}\\{mapping}[\|j]\K\\{mapping}[\|j-\T{1}],\39\|j\MM;{}$\2\6
\&{else}\1\5
${}\\{mapping}[\|j]\K\\{mapping}[\|j+\T{1}],\39\|j\PP;{}$\2\6
\&{if} ${}(\\{verbose}>\T{1}){}$\1\5
${}\\{printf}(\.{"\ \ \ \ \%s\ (\%ld)\\n"},\39\\{sec\_name}(\|j),\3{-1}\39%
\\{best\_j}<\\{best\_k}\?\\{del}[\\{mapping}[\|j+\T{1}]][\|k]:\\{del}[\|k][%
\\{mapping}[\|j-\T{1}]]);{}$\2\6
\4${}\}{}$\5
\2\5
\&{while} ${}(\|j\I\\{best\_j});{}$\6
${}\\{mapping}[\|j]\K\|k;{}$\6
${}\\{score}\MRL{-{\K}}\\{best\_d};{}$\6
${}\\{steps}\PP{}$;\par
\U8.\fi

\M{14}How well does cautious descent work? In this application, it
is definitely too cautious. For example, after lots of computation with the
default settings, it comes up
with a pretty good value (457342), but only after taking 39,418 steps!
Then (if \PB{$\|r>\T{1}$}) it tries again and stops with 461584 after 47,634
steps.
The greedy algorithm with the same starting permutations obtains the
local minimum 457408 after only 93 steps, then 460411 after 83 steps.
The greedy algorithm tends to find solutions that are a bit inferior,
but it is so much faster that it allows us to run many
more experiments. After 20 trials with the default settings, it finds
a permutation with only 456315 below the diagonal,
and after about 250 more it reduces this upper bound to 456295.
(Gerhard Reinelt has proved, via branch-and-cut,
that 456295 is in fact optimum.)

The method of stratified greed, which is illustrated in the {\sc
FOOTBALL} module, should do better than the ordinary greedy algorithm;
and interesting results can be expected when stratified greed is compared
also to other methods like simulated annealing and genetic breeding.
Comparisons should be made by seeing which method can come up with the
best upper bound after calculating for a given number of mems (see
{\sc MILES\_\,SPAN}). The upper bound obtained in any run is a random
variable, so several independent trials of each method should be~made.

Question: Suppose we divide the vertices into two subsets and prescribe
a fixed permutation on each subset. Is it NP-complete to find the
optimum way to merge these two permutations---i.e., to find a
permutation, extending the given ones, that has the smallest
below-diagonal sum?

\fi

\N{1}{15}Index. We close with a list that shows where the identifiers of this
program are defined and used.

\fi


\inx
\fin
\con
