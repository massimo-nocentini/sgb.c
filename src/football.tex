\input cwebmac
% This file is part of the Stanford GraphBase (c) Stanford University 1993
% This material goes at the beginning of all Stanford GraphBase CWEB files

\def\topofcontents{
  \leftline{\sc\today\ at \hours}\bigskip\bigskip
  \centerline{\titlefont\title}}

\font\ninett=cmtt9
\def\botofcontents{\vskip 0pt plus 1filll
    \ninerm\baselineskip10pt
    \noindent\copyright\ 1993 Stanford University
    \bigskip\noindent
    This file may be freely copied and distributed, provided that
    no changes whatsoever are made. All users are asked to help keep
    the Stanford GraphBase files consistent and ``uncorrupted,''
    identical everywhere in the world. Changes are permissible only
    if the modified file is given a new name, different from the names of
    existing files in the Stanford GraphBase, and only if the modified file is
    clearly identified as not being part of that GraphBase.
    (The {\ninett CWEB} system has a ``change file'' facility by
    which users can easily make minor alterations without modifying
    the master source files in any way. Everybody is supposed to use
    change files instead of changing the files.)
    The author has tried his best to produce correct and useful programs,
    in order to help promote computer science research,
    but no warranty of any kind should be assumed.
    \smallskip\noindent
    Preliminary work on the Stanford GraphBase project
    was supported in part by National Science
    Foundation grant CCR-86-10181.}

\def\prerequisite#1{\def\startsection{\noindent
    Important: Before reading {\sc\title},
    please read or at least skim the program for {\sc#1}.\bigskip
    \let\startsection=\stsec\stsec}}
\def\prerequisites#1#2{\def\startsection{\noindent
    Important: Before reading {\sc\title}, please read
    or at least skim the programs for {\sc#1} and {\sc#2}.\bigskip
    \let\startsection=\stsec\stsec}}



\def\title{FOOTBALL}

\prerequisite{GB\_\,GAMES}

\N{1}{1}Introduction. This demonstration program uses graphs
constructed by the {\sc GB\_\,GAMES} module to produce
an interactive program called \.{football}, which finds preposterously
long chains of scores to ``prove'' that one given team might outrank another
by a huge margin.

\def\<#1>{$\langle${\rm#1}$\rangle$}
The program prompts you for a starting team. If you simply type \<return>,
it exits; otherwise you should enter a team name (e.g., `\.{Stanford}')
before typing \<return>.

Then the program prompts you for another team. If you simply type
\<return> at this point, it will go back and ask for a new starting team;
otherwise you should specify another name (e.g., `\.{Harvard}').

Then the program finds and displays a chain from the starting team
to the other one. For example, you might see the following:
$$\vbox{\halign{\tt#\hfil\cr
Oct 06: Stanford Cardinal 36, Notre Dame Fighting Irish 31 (+5)\cr
Oct 20: Notre Dame Fighting Irish 29, Miami Hurricanes 20 (+14)\cr
Jan 01: Miami Hurricanes 46, Texas Longhorns 3 (+57)\cr
Nov 03: Texas Longhorns 41, Texas Tech Red Raiders 22 (+76)\cr
Nov 17: Texas Tech Red Raiders 62, Southern Methodist Mustangs 7 (+131)\cr
Sep 08: Southern Methodist Mustangs 44, Vanderbilt Commodores 7 (+168)\cr
\omit\qquad\vdots\cr
Nov 10: Cornell Big Red 41, Columbia Lions 0 (+2188)\cr
Sep 15: Columbia Lions 6, Harvard Crimson 9 (+2185)\cr}}$$
The chain isn't necessarily optimal; it's just this particular
program's best guess. Another chain, which establishes a victory margin
of $+2279$ points, can in fact be produced by modifying this
program to search back from Harvard instead of forward from Stanford.
Algorithms that find even better chains should be fun to invent.

Actually this program has two variants. If you invoke it by saying simply
`\.{football}', you get chains found by a simple ``greedy algorithm.''
But if you invoke it by saying `\.{football} \<number>', assuming \UNIX/
command-line conventions, the program works harder. Higher values of
\<number> do more calculation and tend to find better chains. For
example, the simple greedy algorithm favors Stanford over Harvard by
only 781; \.{football}~\.{10} raises this to 1895; the
example above corresponds to \.{football}~\.{4000}.

\fi

\M{2}Here is the general program layout, as seen by the \CEE/ compiler:

\Y\B\8\#\&{include} \.{"gb\_graph.h"}\C{ the standard GraphBase data structures
}\6
\8\#\&{include} \.{"gb\_games.h"}\C{ the routine that sets up the graph of
scores }\6
\8\#\&{include} \.{"gb\_flip.h"}\C{ random number generator }\6
\ATH\7
\X10:Type declarations\X\6
\X4:Global variables\X\6
\X7:Subroutines\X\7
\1\1${}\\{main}(\\{argc},\39\\{argv}){}$\6
\&{int} \\{argc};\C{ the number of command-line arguments }\6
\&{char} ${}{*}\\{argv}[\,]{}$;\C{ an array of strings containing those
arguments }\2\2\6
${}\{{}$\1\6
\X3:Scan the command-line options\X;\6
\X5:Set up the graph\X;\6
\&{while} (\T{1})\5
${}\{{}$\1\6
\X6:Prompt for starting team and goal team; \PB{\&{break}} if none given\X;\6
\X9:Find a chain from \PB{\\{start}} to \PB{\\{goal}}, and print it\X;\6
\4${}\}{}$\2\6
\&{return} \T{0};\C{ normal exit }\6
\4${}\}{}$\2\par
\fi

\M{3}Let's deal with \UNIX/-dependent stuff first. The rest of this program
should work without change on any operating system.

\Y\B\4\X3:Scan the command-line options\X${}\E{}$\6
\&{if} ${}(\\{argc}\E\T{3}\W\\{strcmp}(\\{argv}[\T{2}],\39\.{"-v"})\E\T{0}){}$%
\1\5
${}\\{verbose}\K\\{argc}\K\T{2}{}$;\C{ secret option }\2\6
\&{if} ${}(\\{argc}\E\T{1}){}$\1\5
${}\\{width}\K\T{0};{}$\2\6
\&{else} \&{if} ${}(\\{argc}\E\T{2}\W\\{sscanf}(\\{argv}[\T{1}],\39\.{"\%ld"},%
\39{\AND}\\{width})\E\T{1}){}$\5
${}\{{}$\1\6
\&{if} ${}(\\{width}<\T{0}){}$\1\5
${}\\{width}\K{-}\\{width}{}$;\C{ a \UNIX/ user might have used a hyphen }\2\6
\4${}\}{}$\5
\2\&{else}\5
${}\{{}$\1\6
${}\\{fprintf}(\\{stderr},\39\.{"Usage:\ \%s\ [searchwi}\)\.{dth]\\n"},\39%
\\{argv}[\T{0}]);{}$\6
\&{return} ${}{-}\T{2};{}$\6
\4${}\}{}$\2\par
\U2.\fi

\M{4}\B\X4:Global variables\X${}\E{}$\6
\&{long} \\{width};\C{ number of cases examined per stratum }\6
\&{Graph} ${}{*}\|g{}$;\C{ the graph containing score information }\6
\&{Vertex} ${}{*}\|u,\39{*}\|v{}$;\C{ vertices of current interest }\6
\&{Arc} ${}{*}\|a{}$;\C{ arc of current interest }\6
\&{Vertex} ${}{*}\\{start},\39{*}\\{goal}{}$;\C{ teams specified by the user }\6
\&{long} \\{mm};\C{ counter used only in \PB{\\{verbose}} mode }\par
\As11, 20\ETs29.
\U2.\fi

\M{5}An arc from \PB{\|u} to \PB{\|v} in the graph generated by \PB{\\{games}}
has a \PB{\\{len}} field
equal to the number of points scored by \PB{\|u} against \PB{\|v}.
For our purposes we want also a \PB{\\{del}} field, which gives the difference
between the number of points scored by \PB{\|u} and the number of points
scored by~\PB{\|v} in that game.

\Y\B\4\D$\\{del}$ \5
$\|a.{}$\|I\C{ \PB{\\{del}} info appears in utility field \PB{\|a} of an \PB{%
\&{Arc}} record }\par
\Y\B\4\X5:Set up the graph\X${}\E{}$\6
$\|g\K\\{games}(\T{0\$L},\39\T{0\$L},\39\T{0\$L},\39\T{0\$L},\39\T{0\$L},\39%
\T{0\$L},\39\T{0\$L},\39\T{0\$L}){}$;\C{ this default graph has the data for
the entire 1990 season }\6
\&{if} ${}(\|g\E\NULL){}$\5
${}\{{}$\1\6
${}\\{fprintf}(\\{stderr},\39\.{"Sorry,\ can't\ create}\)\.{\ the\ graph!\
(error\ c}\)\.{ode\ \%ld)\\n"},\39\\{panic\_code});{}$\6
\&{return} ${}{-}\T{1};{}$\6
\4${}\}{}$\2\6
\&{for} ${}(\|v\K\|g\MG\\{vertices};{}$ ${}\|v<\|g\MG\\{vertices}+\|g\MG\|n;{}$
${}\|v\PP){}$\1\6
\&{for} ${}(\|a\K\|v\MG\\{arcs};{}$ \|a; ${}\|a\K\|a\MG\\{next}){}$\1\6
\&{if} ${}(\|a\MG\\{tip}>\|v){}$\5
${}\{{}$\C{ arc \PB{$\|a+\T{1}$} is the mate of arc \PB{\|a} iff \PB{$\|a\MG%
\\{tip}>\|v$} }\1\6
${}\|a\MG\\{del}\K\|a\MG\\{len}-(\|a+\T{1})\MG\\{len};{}$\6
${}(\|a+\T{1})\MG\\{del}\K{-}\|a\MG\\{del};{}$\6
\4${}\}{}$\2\2\2\par
\U2.\fi

\N{1}{6}Terminal interaction. While we're getting trivialities out of the way,
we might as well take care of the simple dialog that transpires
between this program and the user.

\Y\B\4\X6:Prompt for starting team and goal team; \PB{\&{break}} if none given%
\X${}\E{}$\6
\\{putchar}(\.{'\\n'});\C{ make a blank line for visual punctuation }\6
\4\\{restart}:\C{ if we avoid this label, the \PB{\&{break}} command will be
broken }\6
\&{if} ${}((\\{start}\K\\{prompt\_for\_team}(\.{"Starting"}))\E\NULL){}$\1\5
\&{break};\2\6
\&{if} ${}((\\{goal}\K\\{prompt\_for\_team}(\.{"\ \ \ Other"}))\E\NULL){}$\1\5
\&{goto} \\{restart};\2\6
\&{if} ${}(\\{start}\E\\{goal}){}$\5
${}\{{}$\1\6
\\{printf}(\.{"\ (Um,\ please\ give\ m}\)\.{e\ the\ names\ of\ two\ D}\)%
\.{ISTINCT\ teams.)\\n"});\6
\&{goto} \\{restart};\6
\4${}\}{}$\2\par
\U2.\fi

\M{7}The user must spell team names exactly as they appear in the file
\.{games.dat}. Thus, for example, `\.{Berkeley}' and `\.{Cal}' don't
work; it has to be `\.{California}'. Similarly, a person must type
`\.{Pennsylvania}' instead of `\.{Penn}', `\.{Nevada-Las} \.{Vegas}'
instead of `\.{UNLV}'. A backslash is necessary in `\.{Texas} \.{A\\\&M}'.

\Y\B\4\X7:Subroutines\X${}\E{}$\6
\1\1\&{Vertex} ${}{*}\\{prompt\_for\_team}(\|s){}$\6
\&{char} ${}{*}\|s{}$;\C{ string used in prompt message }\2\2\6
${}\{{}$\5
\1\&{register} \&{char} ${}{*}\|q{}$;\C{ current position in \PB{\\{buffer}} }\6
\&{register} \&{Vertex} ${}{*}\|v{}$;\C{ current vertex being examined in
sequential search }\6
\&{char} \\{buffer}[\T{30}];\C{ a line of input }\7
\&{while} (\T{1})\5
${}\{{}$\1\6
${}\\{printf}(\.{"\%s\ team:\ "},\39\|s);{}$\6
\\{fflush}(\\{stdout});\C{ make sure the user sees the prompt }\6
${}\\{fgets}(\\{buffer},\39\T{30},\39\\{stdin});{}$\6
\&{if} ${}(\\{buffer}[\T{0}]\E\.{'\\n'}){}$\1\5
\&{return} ${}\NULL{}$;\C{ the user just hit \<return> }\2\6
${}\\{buffer}[\T{29}]\K\.{'\\n'};{}$\6
\&{for} ${}(\|q\K\\{buffer};{}$ ${}{*}\|q\I\.{'\\n'};{}$ ${}\|q\PP){}$\1\5
;\C{ scan to end of input }\2\6
${}{*}\|q\K\.{'\\0'};{}$\6
\&{for} ${}(\|v\K\|g\MG\\{vertices};{}$ ${}\|v<\|g\MG\\{vertices}+\|g\MG\|n;{}$
${}\|v\PP){}$\1\6
\&{if} ${}(\\{strcmp}(\\{buffer},\39\|v\MG\\{name})\E\T{0}){}$\1\5
\&{return} \|v;\C{ aha, we found it }\2\2\6
\\{printf}(\.{"\ (Sorry,\ I\ don't\ kn}\)\.{ow\ any\ team\ by\ that\ }\)%
\.{name.)\\n"});\6
${}\\{printf}(\.{"\ (One\ team\ I\ do\ kno}\)\.{w\ is\ \%s...)\\n"},\39(\|g\MG%
\\{vertices}+\\{gb\_unif\_rand}(\|g\MG\|n))\MG\\{name});{}$\6
\4${}\}{}$\2\6
\4${}\}{}$\2\par
\A13.
\U2.\fi

\N{1}{8}Greed. This program's primary task is to find the longest possible
simple path from \PB{\\{start}} to \PB{\\{goal}}, using \PB{\\{del}} as the
length of each
arc in the path. This is an NP-complete problem, and the number of
possibilities is pretty huge, so the present program is content to
use heuristics that are reasonably easy to compute. (Researchers are hereby
challenged to come up with better heuristics. Does simulated annealing
give good results? How about genetic algorithms?)

Perhaps the first approach that comes to mind is a simple ``greedy'' approach
in which each step takes the largest possible \PB{\\{del}} that doesn't prevent
us from eventually getting to \PB{\\{goal}}. So that's the method we will
implement first.

\fi

\M{9}\B\X9:Find a chain from \PB{\\{start}} to \PB{\\{goal}}, and print it\X${}%
\E{}$\6
\X12:Initialize the allocation of auxiliary memory\X;\6
\&{if} ${}(\\{width}\E\T{0}){}$\1\5
\X17:Use a simple-minded greedy algorithm to find a chain from \PB{\\{start}}
to \PB{\\{goal}}\X\2\6
\&{else}\1\5
\X19:Use a stratified heuristic to find a chain from \PB{\\{start}} to \PB{%
\\{goal}}\X;\2\6
\X15:Print the solution corresponding to \PB{\\{cur\_node}}\X;\6
\X14:Recycle the auxiliary memory used\X;\par
\U2.\fi

\M{10}We might as well use data structures that are more general than we need,
in anticipation of a more complex heuristic that will be implemented later.
The set of all possible solutions can be viewed as a backtrack tree
in which the branches from each node are the games that can possibly
follow that node. We will examine a small part of that gigantic tree.

\Y\B\4\X10:Type declarations\X${}\E{}$\6
\&{typedef} \&{struct} \&{node\_struct} ${}\{{}$\1\6
\&{Arc} ${}{*}\\{game}{}$;\C{ game from the current team to the next team }\6
\&{long} \\{tot\_len};\C{ accumulated length from \PB{\\{start}} to here }\6
\&{struct} \&{node\_struct} ${}{*}\\{prev}{}$;\C{ node that gave us the current
team }\6
\&{struct} \&{node\_struct} ${}{*}\\{next}{}$;\C{ list pointer to node in same
stratum (see below) }\2\6
${}\}{}$ \&{node};\par
\U2.\fi

\M{11}\B\X4:Global variables\X${}\mathrel+\E{}$\6
\&{Area} \\{node\_storage};\C{ working storage for heuristic calculations }\6
\&{node} ${}{*}\\{next\_node}{}$;\C{ where the next node is slated to go }\6
\&{node} ${}{*}\\{bad\_node}{}$;\C{ end of current allocation block }\6
\&{node} ${}{*}\\{cur\_node}{}$;\C{ current node of particular interest }\par
\fi

\M{12}\B\X12:Initialize the allocation of auxiliary memory\X${}\E{}$\6
$\\{next\_node}\K\\{bad\_node}\K\NULL{}$;\par
\U9.\fi

\M{13}\B\X7:Subroutines\X${}\mathrel+\E{}$\6
\1\1\&{node} ${}{*}\\{new\_node}(\|x,\39\|d){}$\6
\&{node} ${}{*}\|x{}$;\C{ an old node that the new node will call \PB{\\{prev}}
}\6
\&{long} \|d;\C{ incremental change to \PB{\\{tot\_len}} }\2\2\6
${}\{{}$\1\6
\&{if} ${}(\\{next\_node}\E\\{bad\_node}){}$\5
${}\{{}$\1\6
${}\\{next\_node}\K\\{gb\_typed\_alloc}(\T{1000},\39\&{node},\39\\{node%
\_storage});{}$\6
\&{if} ${}(\\{next\_node}\E\NULL){}$\1\5
\&{return} ${}\NULL{}$;\C{ we're out of space }\2\6
${}\\{bad\_node}\K\\{next\_node}+\T{1000};{}$\6
\4${}\}{}$\2\6
${}\\{next\_node}\MG\\{prev}\K\|x;{}$\6
${}\\{next\_node}\MG\\{tot\_len}\K(\|x\?\|x\MG\\{tot\_len}:\T{0})+\|d;{}$\6
\&{return} \\{next\_node}${}\PP;{}$\6
\4${}\}{}$\2\par
\fi

\M{14}\B\X14:Recycle the auxiliary memory used\X${}\E{}$\6
\\{gb\_free}(\\{node\_storage});\par
\U9.\fi

\M{15}When we're done, \PB{$\\{cur\_node}\MG\\{game}\MG\\{tip}$} will be the %
\PB{\\{goal}} vertex, and
we can get back to the \PB{\\{start}} vertex by following \PB{\\{prev}} links
from \PB{\\{cur\_node}}. It looks better to print the answers from \PB{%
\\{start}} to
\PB{\\{goal}}, so maybe we should have changed our algorithm to go the
other way.

But let's not worry over trifles. It's easy to change
the order of a linked list. The secret is simply to think of the list
as a stack, from which we pop all the elements off to another stack;
the new stack has the elements in reverse order.

\Y\B\4\X15:Print the solution corresponding to \PB{\\{cur\_node}}\X${}\E{}$\6
$\\{next\_node}\K\NULL{}$;\C{ now we'll use \PB{\\{next\_node}} as top of
temporary stack }\6
\&{do}\5
${}\{{}$\5
\1\&{register} \&{node} ${}{*}\|t;{}$\7
${}\|t\K\\{cur\_node};{}$\6
${}\\{cur\_node}\K\|t\MG\\{prev}{}$;\C{ pop }\6
${}\|t\MG\\{prev}\K\\{next\_node};{}$\6
${}\\{next\_node}\K\|t{}$;\C{ push }\6
\4${}\}{}$\5
\2\5
\&{while} (\\{cur\_node});\6
\&{for} ${}(\|v\K\\{start};{}$ ${}\|v\I\\{goal};{}$ ${}\|v\K\|u,\39\\{next%
\_node}\K\\{next\_node}\MG\\{prev}){}$\5
${}\{{}$\1\6
${}\|a\K\\{next\_node}\MG\\{game};{}$\6
${}\|u\K\|a\MG\\{tip};{}$\6
\X16:Print the score of game \PB{\|a} between \PB{\|v} and \PB{\|u}\X;\6
${}\\{printf}(\.{"\ (\%+ld)\\n"},\39\\{next\_node}\MG\\{tot\_len});{}$\6
\4${}\}{}$\2\par
\U9.\fi

\M{16}\B\X16:Print the score of game \PB{\|a} between \PB{\|v} and \PB{\|u}%
\X${}\E{}$\6
${}\{{}$\5
\1\&{register} \&{long} \|d${}\K\|a\MG\\{date}{}$;\C{ date of the game, 0 means
Aug 26 }\7
\&{if} ${}(\|d\Z\T{5}){}$\1\5
${}\\{printf}(\.{"\ Aug\ \%02ld"},\39\|d+\T{26});{}$\2\6
\&{else} \&{if} ${}(\|d\Z\T{35}){}$\1\5
${}\\{printf}(\.{"\ Sep\ \%02ld"},\39\|d-\T{5});{}$\2\6
\&{else} \&{if} ${}(\|d\Z\T{66}){}$\1\5
${}\\{printf}(\.{"\ Oct\ \%02ld"},\39\|d-\T{35});{}$\2\6
\&{else} \&{if} ${}(\|d\Z\T{96}){}$\1\5
${}\\{printf}(\.{"\ Nov\ \%02ld"},\39\|d-\T{66});{}$\2\6
\&{else} \&{if} ${}(\|d\Z\T{127}){}$\1\5
${}\\{printf}(\.{"\ Dec\ \%02ld"},\39\|d-\T{96});{}$\2\6
\&{else}\1\5
\\{printf}(\.{"\ Jan\ 01"});\C{ \PB{$\|d\K\T{128}$} }\2\6
${}\\{printf}(\.{":\ \%s\ \%s\ \%ld,\ \%s\ \%s\ }\)\.{\%ld"},\39\|v\MG\\{name},%
\39\|v\MG\\{nickname},\39\|a\MG\\{len},\39\|u\MG\\{name},\39\|u\MG\\{nickname},%
\39\|a\MG\\{len}-\|a\MG\\{del});{}$\6
\4${}\}{}$\2\par
\U15.\fi

\M{17}We can't just move from \PB{\|v} to any adjacent vertex; we can go only
to a vertex from which \PB{\\{goal}} can be reached without touching \PB{\|v}
or any other vertex already used on the path from \PB{\\{start}}.

Furthermore, if the locally best move from \PB{\|v} is directly to \PB{%
\\{goal}},
we don't want to make that move unless it's our last chance; we can
probably do better by making the chain longer. Otherwise, for example,
a chain between a team and its worst opponent would consist of
only a single game.

To keep track of untouchable vertices, we use a utility field
called \PB{\\{blocked}} in each vertex record. Another utility field,
\PB{\\{valid}}, will be set to a validation code in each vertex that
still leads to the goal.

\Y\B\4\D$\\{blocked}$ \5
$\|u.{}$\|I\par
\B\4\D$\\{valid}$ \5
$\|v.{}$\|V\par
\Y\B\4\X17:Use a simple-minded greedy algorithm to find a chain from \PB{%
\\{start}} to \PB{\\{goal}}\X${}\E{}$\6
${}\{{}$\1\6
\&{for} ${}(\|v\K\|g\MG\\{vertices};{}$ ${}\|v<\|g\MG\\{vertices}+\|g\MG\|n;{}$
${}\|v\PP){}$\1\5
${}\|v\MG\\{blocked}\K\T{0},\39\|v\MG\\{valid}\K\NULL;{}$\2\6
${}\\{cur\_node}\K\NULL;{}$\6
\&{for} ${}(\|v\K\\{start};{}$ ${}\|v\I\\{goal};{}$ ${}\|v\K\\{cur\_node}\MG%
\\{game}\MG\\{tip}){}$\5
${}\{{}$\5
\1\&{register} \&{long} \|d${}\K{-}\T{10000};{}$\6
\&{register} \&{Arc} ${}{*}\\{best\_arc}{}$;\C{ arc that achieves \PB{$\\{del}%
\K\|d$} }\6
\&{register} \&{Arc} ${}{*}\\{last\_arc}{}$;\C{ arc that goes directly to \PB{%
\\{goal}} }\7
${}\|v\MG\\{blocked}\K\T{1};{}$\6
${}\\{cur\_node}\K\\{new\_node}(\\{cur\_node},\39\T{0\$L});{}$\6
\&{if} ${}(\\{cur\_node}\E\NULL){}$\5
${}\{{}$\1\6
${}\\{fprintf}(\\{stderr},\39\.{"Oops,\ there\ isn't\ e}\)\.{nough\ memory!%
\\n"}){}$;\5
\&{return} ${}{-}\T{2};{}$\6
\4${}\}{}$\2\6
\X18:Set \PB{$\|u\MG\\{valid}\K\|v$} for all \PB{\|u} to which \PB{\|v} might
now move\X;\6
\&{for} ${}(\|a\K\|v\MG\\{arcs};{}$ \|a; ${}\|a\K\|a\MG\\{next}){}$\1\6
\&{if} ${}(\|a\MG\\{del}>\|d\W\|a\MG\\{tip}\MG\\{valid}\E\|v){}$\1\6
\&{if} ${}(\|a\MG\\{tip}\E\\{goal}){}$\1\5
${}\\{last\_arc}\K\|a;{}$\2\6
\&{else}\1\5
${}\\{best\_arc}\K\|a,\39\|d\K\|a\MG\\{del};{}$\2\2\2\6
${}\\{cur\_node}\MG\\{game}\K(\|d\E{-}\T{10000}\?\\{last\_arc}:\\{best%
\_arc}){}$;\C{ use \PB{\\{last\_arc}} as a last resort }\6
${}\\{cur\_node}\MG\\{tot\_len}\MRL{+{\K}}\\{cur\_node}\MG\\{game}\MG%
\\{del};{}$\6
\4${}\}{}$\2\6
\4${}\}{}$\2\par
\U9.\fi

\M{18}A standard marking algorithm supplies the final missing link in
our algorithm.

\Y\B\4\D$\\{link}$ \5
$\|w.{}$\|V\par
\Y\B\4\X18:Set \PB{$\|u\MG\\{valid}\K\|v$} for all \PB{\|u} to which \PB{\|v}
might now move\X${}\E{}$\6
$\|u\K\\{goal}{}$;\C{ \PB{\|u} will be the top of a stack of nodes to be
explored }\6
${}\|u\MG\\{link}\K\NULL;{}$\6
${}\|u\MG\\{valid}\K\|v;{}$\6
\&{do}\5
${}\{{}$\1\6
\&{for} ${}(\|a\K\|u\MG\\{arcs},\39\|u\K\|u\MG\\{link};{}$ \|a; ${}\|a\K\|a\MG%
\\{next}){}$\1\6
\&{if} ${}(\|a\MG\\{tip}\MG\\{blocked}\E\T{0}\W\|a\MG\\{tip}\MG\\{valid}\I%
\|v){}$\5
${}\{{}$\1\6
${}\|a\MG\\{tip}\MG\\{valid}\K\|v{}$;\C{ mark \PB{$\|a\MG\\{tip}$} reachable
from \PB{\\{goal}} }\6
${}\|a\MG\\{tip}\MG\\{link}\K\|u;{}$\6
${}\|u\K\|a\MG\\{tip}{}$;\C{ push it on the stack, so that its successors
             will be marked too }\6
\4${}\}{}$\2\2\6
\4${}\}{}$\5
\2\5
\&{while} (\|u);\par
\U17.\fi

\N{1}{19}Stratified greed.  One approach to better chains is the following
algorithm, motivated by similar ideas of Pang Chen [Ph.D. thesis,
Stanford University, 1989]:  Suppose the nodes of
a (possibly huge) backtrack tree are classified into a (fairly small)
number of strata, by a function $h$ with the property that $h({\rm
child})<h({\rm parent})$ and $h({\rm goal})=0$. Suppose further that
we want to find a node $x$ that maximizes a given function~$f(x)$,
where it is reasonable to believe that $f$(child) will be relatively
large among nodes in a child's stratum only if $f$(parent) is
relatively large in the parent's stratum. Then it makes sense to
restrict backtracking to, say, the top $w$ nodes of each stratum,
ranked by their $f$ values.

The greedy algorithm already described is a special case of this general
approach, with $w=1$ and with $h(x)=-($length of chain leading to~$x)$.
The refined algorithm we are about to describe uses a general value of $w$
and a somewhat more relevant stratification function: Given a node~$x$
of the backtrack tree for longest paths, corresponding to a path from
\PB{\\{start}} to a certain vertex~$u=u(x)$, we will let $h(x)$ be the number
of
vertices that lie between \PB{\|u} and \PB{\\{goal}} (in the sense that the
simple
path from \PB{\\{start}} to~\PB{\|u} can be extended until it passes through
such
a vertex and then all the way to~\PB{\\{goal}}).

Here is the top level of the stratified greedy algorithm. We maintain
a linked list of nodes for each stratum, that is, for each possible value
of~$h$. The number of nodes required is bounded by $w$ times the
number of strata.

\Y\B\4\X19:Use a stratified heuristic to find a chain from \PB{\\{start}} to %
\PB{\\{goal}}\X${}\E{}$\6
${}\{{}$\1\6
\X21:Make \PB{\\{list}[\T{0}]} through \PB{$\\{list}[\|n-\T{1}]$} empty\X;\6
${}\\{cur\_node}\K\NULL{}$;\C{ \PB{$\NULL$} represents the root of the
backtrack tree }\6
${}\|m\K\|g\MG\|n-\T{1}{}$;\C{ the highest stratum not yet fully explored }\6
\&{do}\5
${}\{{}$\1\6
\X27:Place each child~\PB{\|x} of \PB{\\{cur\_node}} into \PB{\\{list}[\|h(%
\|x)]}, retaining at most \PB{\\{width}} nodes of maximum \PB{\\{tot\_len}} on
each list\X;\6
\&{while} ${}(\\{list}[\|m]\E\NULL){}$\1\5
\X23:Decrease \PB{\|m} and get ready to explore another list\X;\2\6
${}\\{cur\_node}\K\\{list}[\|m];{}$\6
${}\\{list}[\|m]\K\\{cur\_node}\MG\\{next}{}$;\C{ remove a node from highest
remaining stratum }\6
\&{if} (\\{verbose})\1\5
\X24:Print ``verbose'' info about \PB{\\{cur\_node}}\X;\2\6
\4${}\}{}$\5
\2\5
\&{while} ${}(\|m>\T{0}){}$;\C{ exactly one node should be in \PB{\\{list}[%
\T{0}]} (see below) }\6
\4${}\}{}$\2\par
\U9.\fi

\M{20}The calculation of $h(x)$ is somewhat delicate, and we will defer it
for a moment. But the list manipulation is easy, so we can finish it
quickly while it's fresh in our minds.

\Y\B\4\D$\.{MAX\_N}$ \5
\T{120}\C{ the number of teams in \.{games.dat} }\par
\Y\B\4\X4:Global variables\X${}\mathrel+\E{}$\6
\&{node} ${}{*}\\{list}[\.{MAX\_N}]{}$;\C{ the best nodes known in given strata
}\6
\&{long} \\{size}[\.{MAX\_N}];\C{ the number of elements in a given \PB{%
\\{list}} }\6
\&{long} \|m${},\39\|h{}$;\C{ current lists of interest }\6
\&{node} ${}{*}\|x{}$;\C{ a child of \PB{\\{cur\_node}} }\par
\fi

\M{21}\B\X21:Make \PB{\\{list}[\T{0}]} through \PB{$\\{list}[\|n-\T{1}]$} empty%
\X${}\E{}$\6
\&{for} ${}(\|m\K\T{0};{}$ ${}\|m<\|g\MG\|n;{}$ ${}\|m\PP){}$\1\5
${}\\{list}[\|m]\K\NULL,\39\\{size}[\|m]\K\T{0}{}$;\2\par
\U19.\fi

\M{22}The lists are maintained in order by \PB{\\{tot\_len}}, with the largest
\PB{\\{tot\_len}} value at the end so that we can easily delete the smallest.

When \PB{$\|h\K\T{0}$}, we retain only one node instead of~\PB{\\{width}}
different nodes,
because we are interested in only one solution.

\Y\B\4\X22:Place node~\PB{\|x} into \PB{\\{list}[\|h]}, retaining at most \PB{%
\\{width}} nodes of maximum \PB{\\{tot\_len}}\X${}\E{}$\6
\&{if} ${}((\|h>\T{0}\W\\{size}[\|h]\E\\{width})\V(\|h\E\T{0}\W\\{size}[\T{0}]>%
\T{0})){}$\5
${}\{{}$\1\6
\&{if} ${}(\|x\MG\\{tot\_len}\Z\\{list}[\|h]\MG\\{tot\_len}){}$\1\5
\&{goto} \\{done};\C{ drop node \PB{\|x} }\2\6
${}\\{list}[\|h]\K\\{list}[\|h]\MG\\{next}{}$;\C{ drop one node from \PB{%
\\{list}[\|h]} }\6
\4${}\}{}$\5
\2\&{else}\1\5
${}\\{size}[\|h]\PP;{}$\2\6
${}\{{}$\5
\1\&{register} \&{node} ${}{*}\|p,\39{*}\|q{}$;\C{ node in list and its
predecessor }\7
\&{for} ${}(\|p\K\\{list}[\|h],\39\|q\K\NULL;{}$ \|p; ${}\|q\K\|p,\39\|p\K\|p%
\MG\\{next}{}$)\5
\1\&{if} ${}(\|x\MG\\{tot\_len}\Z\|p\MG\\{tot\_len}){}$\1\5
\&{break};\2\2\6
${}\|x\MG\\{next}\K\|p;{}$\6
\&{if} (\|q)\1\5
${}\|q\MG\\{next}\K\|x{}$;\5
\2\&{else}\1\5
${}\\{list}[\|h]\K\|x;{}$\2\6
\4${}\}{}$\2\6
\4\\{done}:\5
;\par
\U27.\fi

\M{23}We reverse the list so that large entries will tend to go in first.

\Y\B\4\X23:Decrease \PB{\|m} and get ready to explore another list\X${}\E{}$\6
${}\{{}$\5
\1\&{register} \&{node} ${}{*}\|r\K\NULL,\39{*}\|s\K\\{list}[\MM\|m],\39{*}%
\|t;{}$\7
\&{while} (\|s)\1\5
${}\|t\K\|s\MG\\{next},\39\|s\MG\\{next}\K\|r,\39\|r\K\|s,\39\|s\K\|t;{}$\2\6
${}\\{list}[\|m]\K\|r;{}$\6
${}\\{mm}\K\T{0}{}$;\C{ \PB{\\{mm}} is an index for ``verbose'' printing }\6
\4${}\}{}$\2\par
\U19.\fi

\M{24}\B\X24:Print ``verbose'' info about \PB{\\{cur\_node}}\X${}\E{}$\6
${}\{{}$\1\6
${}\\{cur\_node}\MG\\{next}\K{}$(\&{node} ${}{*})((\PP\\{mm}\LL\T{8})+\|m){}$;%
\C{ pack an ID for this node }\6
${}\\{printf}(\.{"[\%lu,\%lu]=[\%lu,\%lu]}\)\.{\&\%s\ (\%+ld)\\n"},\39\|m,\39%
\\{mm},\3{-1}\39\\{cur\_node}\MG\\{prev}\?{}$((\&{unsigned} \&{long})\,\\{cur%
\_node}${}\MG\\{prev}\MG\\{next})\AND\T{\^ff}:\T{0\$L},\3{-1}\39\\{cur\_node}%
\MG\\{prev}\?{}$((\&{unsigned} \&{long})\,\\{cur\_node}${}\MG\\{prev}\MG%
\\{next})\GG\T{8}:\T{0\$L},\3{-1}\39\\{cur\_node}\MG\\{game}\MG\\{tip}\MG%
\\{name},\39\\{cur\_node}\MG\\{tot\_len});{}$\6
\4${}\}{}$\2\par
\U19.\fi

\M{25}Incidentally, it is plausible to conjecture that the stratified algorithm
always beats the simple greedy algorithm, but that conjecture is false.
For example, the greedy algorithm is able to rank Harvard over Stanford
by 1529, while the stratified algorithm achieves only 1527 when
\PB{$\\{width}\K\T{1}$}. On the other hand, the greedy algorithm often fails
miserably; when comparing two Ivy League teams, it doesn't find a
way to break out of the Ivy and Patriot Leagues.

\fi

\N{1}{26}Bicomponents revisited.
How difficult is it to compute the function $h$? Given a connected graph~$G$
with two distinguished vertices $u$ and~$v$, we want to count the number
of vertices that might appear on a simple path from $u$ to~$v$.
(This is {\sl not\/} the same as the number of vertices reachable from both
$u$ and~$v$. For example, consider a ``claw'' graph with four vertices
$\{u,v,w,x\}$ and with edges only from $x$ to the other three vertices;
in this graph $w$ is reachable from $u$ and~$v$, but it is not on any simple
path between them.)

The best way to solve this problem is probably to compute the bicomponents
of~$G$, or least to compute some of them. Another demo program,
{\sc BOOK\_\kern.05emCOMPONENTS}, explains the relevant theory in some
detail, and we will assume familiarity with that algorithm in the present
discussion.

Let us imagine extending $G$ to a slightly larger graph $G^+$ by
adding a dummy vertex~$o$ that is adjacent only to $v$. Suppose we determine
the bicomponents of $G^+$ by depth-first search starting at~$o$.
These bicomponents form a tree rooted at the bicomponent that contains
just $o$ and~$v$. The number of vertices on paths between $u$ and~$v$,
not counting $v$ itself, is then the number of vertices in the bicomponent
containing~$u$ and in any other bicomponents between that one and the root.

Strictly speaking, each articulation point belongs
to two or more bicomponents. But we will assign each articulation point
to its bicomponent that is nearest the root of the tree; then the vertices
of each bicomponent are precisely the vertices output in bursts by the
depth-first procedure. The bicomponents we want to enumerate are $B_1$, $B_2$,
\dots,~$B_k$, where $B_1$ is the bicomponent containing~$u$ and
$B_{j+1}$ is the bicomponent containing the articulation point associated
with~$B_j$; we stop at~$B_k$ when its associated articulation point is~$v$.
(Often $k=1$.)

The ``children'' of a given graph~$G$ are obtained by removing vertex~$u$
and by considering paths from $u'$ to~$v$, where $u'$ is a vertex
formerly adjacent to~$u$; thus $u'$ is either in~$B_1$ or it is $B_1$'s
associated articulation point. Removing $u$ will, in general, split
$B_1$ into a tree of smaller bicomponents, but $B_2,\ldots,B_k$ will be
unaffected. The implementation below does not take full advantage of this
observation, because the amount of memory required to avoid recomputation
would probably be prohibitive.

\fi

\M{27}The following program is copied almost verbatim from
{\sc BOOK\_\kern.05emCOMPONENTS}.
Instead of repeating the commentary that appears there, we will mention
only the significant differences. One difference is that we start
the depth-first search at a definite place, the \PB{\\{goal}}.

\Y\B\4\X27:Place each child~\PB{\|x} of \PB{\\{cur\_node}} into \PB{\\{list}[%
\|h(\|x)]}, retaining at most \PB{\\{width}} nodes of maximum \PB{\\{tot\_len}}
on each list\X${}\E{}$\6
\X28:Make all vertices unseen and all arcs untagged, except for vertices that
have already been used in steps leading up to \PB{\\{cur\_node}}\X;\6
\X30:Perform a depth-first search with \PB{\\{goal}} as the root, finding
bicomponents and determining the number of vertices accessible between any
given vertex and \PB{\\{goal}}\X;\6
\&{for} ${}(\|a\K(\\{cur\_node}\?\\{cur\_node}\MG\\{game}\MG\\{tip}:\\{start})%
\MG\\{arcs};{}$ \|a; ${}\|a\K\|a\MG\\{next}){}$\1\6
\&{if} ${}((\|u\K\|a\MG\\{tip})\MG\\{untagged}\E\NULL){}$\5
${}\{{}$\C{ \PB{\\{goal}} is reachable from \PB{\|u} }\1\6
${}\|x\K\\{new\_node}(\\{cur\_node},\39\|a\MG\\{del});{}$\6
\&{if} ${}(\|x\E\NULL){}$\5
${}\{{}$\1\6
${}\\{fprintf}(\\{stderr},\39\.{"Oops,\ there\ isn't\ e}\)\.{nough\ memory!%
\\n"}){}$;\5
\&{return} ${}{-}\T{3};{}$\6
\4${}\}{}$\2\6
${}\|x\MG\\{game}\K\|a;{}$\6
\X35:Set \PB{\|h} to the number of vertices on paths between \PB{\|u} and \PB{%
\\{goal}}\X;\6
\X22:Place node~\PB{\|x} into \PB{\\{list}[\|h]}, retaining at most \PB{%
\\{width}} nodes of maximum \PB{\\{tot\_len}}\X;\6
\4${}\}{}$\2\2\par
\U19.\fi

\M{28}Setting the \PB{\\{rank}} field of a vertex to infinity before beginning
a depth-first search is tantamount to removing that vertex from
the graph, because it tells the algorithm not to look further at
such a vertex.

\Y\B\4\D$\\{rank}$ \5
$\|z.{}$\|I\C{ when was this vertex first seen? }\par
\B\4\D$\\{parent}$ \5
$\|u.{}$\|V\C{ who told me about this vertex? }\par
\B\4\D$\\{untagged}$ \5
$\|x.{}$\|A\C{ what is its first untagged arc? }\par
\B\4\D$\\{min}$ \5
$\|v.{}$\|V\C{ how low in the tree can we jump from its mature descendants? }%
\par
\Y\B\4\X28:Make all vertices unseen and all arcs untagged, except for vertices
that have already been used in steps leading up to \PB{\\{cur\_node}}\X${}\E{}$%
\6
\&{for} ${}(\|v\K\|g\MG\\{vertices};{}$ ${}\|v<\|g\MG\\{vertices}+\|g\MG\|n;{}$
${}\|v\PP){}$\5
${}\{{}$\1\6
${}\|v\MG\\{rank}\K\T{0};{}$\6
${}\|v\MG\\{untagged}\K\|v\MG\\{arcs};{}$\6
\4${}\}{}$\2\6
\&{for} ${}(\|x\K\\{cur\_node};{}$ \|x; ${}\|x\K\|x\MG\\{prev}){}$\1\5
${}\|x\MG\\{game}\MG\\{tip}\MG\\{rank}\K\|g\MG\|n{}$;\C{ ``infinite'' rank (or
close enough) }\2\6
${}\\{start}\MG\\{rank}\K\|g\MG\|n;{}$\6
${}\\{nn}\K\T{0};{}$\6
${}\\{active\_stack}\K\\{settled\_stack}\K\NULL{}$;\par
\U27.\fi

\M{29}\B\X4:Global variables\X${}\mathrel+\E{}$\6
\&{Vertex} ${}{*}\\{active\_stack}{}$;\C{ the top of the stack of active
vertices }\6
\&{Vertex} ${}{*}\\{settled\_stack}{}$;\C{ the top of the stack of bicomponents
found }\6
\&{long} \\{nn};\C{ the number of vertices that have been seen }\6
\&{Vertex} \\{dummy};\C{ imaginary parent of \PB{\\{goal}}; its \PB{\\{rank}}
is zero }\par
\fi

\M{30}The \PB{\\{settled\_stack}} will contain a list of all bicomponents in
the opposite order from which they are discovered. This is the order
we'll need later for computing the \PB{\|h} function in each bicomponent.

\Y\B\4\X30:Perform a depth-first search with \PB{\\{goal}} as the root, finding
bicomponents and determining the number of vertices accessible between any
given vertex and \PB{\\{goal}}\X${}\E{}$\6
${}\{{}$\1\6
${}\|v\K\\{goal};{}$\6
${}\|v\MG\\{parent}\K{\AND}\\{dummy};{}$\6
\X31:Make vertex \PB{\|v} active\X;\6
\&{do}\5
\X32:Explore one step from the current vertex~\PB{\|v}, possibly moving to
another current vertex and calling~it~\PB{\|v}\X\5
\&{while} ${}(\|v\I{\AND}\\{dummy});{}$\6
\X34:Use \PB{\\{settled\_stack}} to put the mutual reachability count for each
vertex \PB{\|u} in \PB{$\|u\MG\\{parent}\MG\\{rank}$}\X;\6
\4${}\}{}$\2\par
\U27.\fi

\M{31}\B\X31:Make vertex \PB{\|v} active\X${}\E{}$\6
$\|v\MG\\{rank}\K\PP\\{nn};{}$\6
${}\|v\MG\\{link}\K\\{active\_stack};{}$\6
${}\\{active\_stack}\K\|v;{}$\6
${}\|v\MG\\{min}\K\|v\MG\\{parent}{}$;\par
\Us30\ET32.\fi

\M{32}\B\X32:Explore one step from the current vertex~\PB{\|v}, possibly moving
to another current vertex and calling~it~\PB{\|v}\X${}\E{}$\6
${}\{{}$\5
\1\&{register} \&{Vertex} ${}{*}\|u{}$;\C{ a vertex adjacent to \PB{\|v} }\6
\&{register} \&{Arc} ${}{*}\|a\K\|v\MG\\{untagged}{}$;\C{ \PB{\|v}'s first
remaining untagged arc, if any }\7
\&{if} (\|a)\5
${}\{{}$\1\6
${}\|u\K\|a\MG\\{tip};{}$\6
${}\|v\MG\\{untagged}\K\|a\MG\\{next}{}$;\C{ tag the arc from \PB{\|v} to \PB{%
\|u} }\6
\&{if} ${}(\|u\MG\\{rank}){}$\5
${}\{{}$\C{ we've seen \PB{\|u} already }\1\6
\&{if} ${}(\|u\MG\\{rank}<\|v\MG\\{min}\MG\\{rank}){}$\1\5
${}\|v\MG\\{min}\K\|u{}$;\C{ non-tree arc, just update \PB{$\|v\MG\\{min}$} }\2%
\6
\4${}\}{}$\5
\2\&{else}\5
${}\{{}$\C{ \PB{\|u} is presently unseen }\1\6
${}\|u\MG\\{parent}\K\|v{}$;\C{ the arc from \PB{\|v} to \PB{\|u} is a new tree
arc }\6
${}\|v\K\|u{}$;\C{ \PB{\|u} will now be the current vertex }\6
\X31:Make vertex \PB{\|v} active\X;\6
\4${}\}{}$\2\6
\4${}\}{}$\5
\2\&{else}\5
${}\{{}$\C{ all arcs from \PB{\|v} are tagged, so \PB{\|v} matures }\1\6
${}\|u\K\|v\MG\\{parent}{}$;\C{ prepare to backtrack in the tree }\6
\&{if} ${}(\|v\MG\\{min}\E\|u){}$\1\5
\X33:Remove \PB{\|v} and all its successors on the active stack from the tree,
and report them as a bicomponent of the graph together with~\PB{\|u}\X\2\6
\&{else}\C{ the arc from \PB{\|u} to \PB{\|v} has just matured,
making \PB{$\|v\MG\\{min}$} visible from \PB{\|u} }\6
\, \&{if} ${}(\|v\MG\\{min}\MG\\{rank}<\|u\MG\\{min}\MG\\{rank}){}$\1\5
${}\|u\MG\\{min}\K\|v\MG\\{min};{}$\2\6
${}\|v\K\|u{}$;\C{ the former parent of \PB{\|v} is the new current vertex \PB{%
\|v} }\6
\4${}\}{}$\2\6
\4${}\}{}$\2\par
\U30.\fi

\M{33}When a bicomponent is found, we reset the \PB{\\{parent}} field of each
vertex
so that, afterwards, two vertices will belong to the same bicomponent
if and only if they have the same \PB{\\{parent}}. (This trick was not used in
{\sc BOOK\_\kern.05emCOMPONENTS}, but it does appear in the similar algorithm
of {\sc ROGET\_\,COMPONENTS}.) The new parent, \PB{\|v}, will represent that
bicomponent in subsequent computation; we put it onto \PB{\\{settled\_stack}}.
We also reset \PB{$\|v\MG\\{rank}$} to be the bicomponent's size, plus a
constant
large enough to keep the algorithm from getting confused. (Vertex~\PB{\|u}
might still have untagged arcs leading into this bicomponent; we need to
keep the ranks at least as big as the rank of \PB{$\|u\MG\\{min}$}.) Notice
that
\PB{$\|v\MG\\{min}$} is \PB{\|u}, the articulation point associated with this
bicomponent.
Later the \PB{\\{rank}} field will
contain the sum of all counts between here and the root.

We don't have to do anything when \PB{$\|v\E\\{goal}$}; the trivial root
bicomponent
always comes out last.

\Y\B\4\X33:Remove \PB{\|v} and all its successors on the active stack from the
tree, and report them as a bicomponent of the graph together with~\PB{\|u}\X${}%
\E{}$\6
${}\{{}$\5
\1\&{if} ${}(\|v\I\\{goal}){}$\5
${}\{{}$\5
\1\&{register} \&{Vertex} ${}{*}\|t{}$;\C{ runs through the vertices of the
                      new bicomponent }\6
\&{long} \|c${}\K\T{0}{}$;\C{ the number of vertices removed }\7
${}\|t\K\\{active\_stack};{}$\6
\&{while} ${}(\|t\I\|v){}$\5
${}\{{}$\1\6
${}\|c\PP;{}$\6
${}\|t\MG\\{parent}\K\|v;{}$\6
${}\|t\K\|t\MG\\{link};{}$\6
\4${}\}{}$\2\6
${}\\{active\_stack}\K\|v\MG\\{link};{}$\6
${}\|v\MG\\{parent}\K\|v;{}$\6
${}\|v\MG\\{rank}\K\|c+\|g\MG\|n{}$;\C{ the true component size is \PB{$\|c+%
\T{1}$} }\6
${}\|v\MG\\{link}\K\\{settled\_stack};{}$\6
${}\\{settled\_stack}\K\|v;{}$\6
\4${}\}{}$\2\6
\4${}\}{}$\2\par
\U32.\fi

\M{34}So here's how we sum the ranks. When we get to this step, the \\{settled}
stack contains all bicomponent representatives except \PB{\\{goal}} itself.

\Y\B\4\X34:Use \PB{\\{settled\_stack}} to put the mutual reachability count for
each vertex \PB{\|u} in \PB{$\|u\MG\\{parent}\MG\\{rank}$}\X${}\E{}$\6
\&{while} (\\{settled\_stack})\5
${}\{{}$\1\6
${}\|v\K\\{settled\_stack};{}$\6
${}\\{settled\_stack}\K\|v\MG\\{link};{}$\6
${}\|v\MG\\{rank}\MRL{+{\K}}\|v\MG\\{min}\MG\\{parent}\MG\\{rank}+\T{1}-\|g\MG%
\|n;{}$\6
\4${}\}{}$\C{ note that \PB{$\\{goal}\MG\\{parent}\MG\\{rank}\K\T{0}$} }\2\par
\U30.\fi

\M{35}And here's the last piece of the puzzle.

\Y\B\4\X35:Set \PB{\|h} to the number of vertices on paths between \PB{\|u} and
\PB{\\{goal}}\X${}\E{}$\6
$\|h\K\|u\MG\\{parent}\MG\\{rank}{}$;\par
\U27.\fi

\N{1}{36}Index. Finally, here's a list that shows where the identifiers of this
program are defined and used.

\fi


\inx
\fin
\con
