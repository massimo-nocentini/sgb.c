\input cwebmac
% This file is part of the Stanford GraphBase (c) Stanford University 1993
% This material goes at the beginning of all Stanford GraphBase CWEB files

\def\topofcontents{
  \leftline{\sc\today\ at \hours}\bigskip\bigskip
  \centerline{\titlefont\title}}

\font\ninett=cmtt9
\def\botofcontents{\vskip 0pt plus 1filll
    \ninerm\baselineskip10pt
    \noindent\copyright\ 1993 Stanford University
    \bigskip\noindent
    This file may be freely copied and distributed, provided that
    no changes whatsoever are made. All users are asked to help keep
    the Stanford GraphBase files consistent and ``uncorrupted,''
    identical everywhere in the world. Changes are permissible only
    if the modified file is given a new name, different from the names of
    existing files in the Stanford GraphBase, and only if the modified file is
    clearly identified as not being part of that GraphBase.
    (The {\ninett CWEB} system has a ``change file'' facility by
    which users can easily make minor alterations without modifying
    the master source files in any way. Everybody is supposed to use
    change files instead of changing the files.)
    The author has tried his best to produce correct and useful programs,
    in order to help promote computer science research,
    but no warranty of any kind should be assumed.
    \smallskip\noindent
    Preliminary work on the Stanford GraphBase project
    was supported in part by National Science
    Foundation grant CCR-86-10181.}

\def\prerequisite#1{\def\startsection{\noindent
    Important: Before reading {\sc\title},
    please read or at least skim the program for {\sc#1}.\bigskip
    \let\startsection=\stsec\stsec}}
\def\prerequisites#1#2{\def\startsection{\noindent
    Important: Before reading {\sc\title}, please read
    or at least skim the programs for {\sc#1} and {\sc#2}.\bigskip
    \let\startsection=\stsec\stsec}}



\def\title{GB\_\,BOOKS}
\def\<#1>{\hbox{$\langle$\rm#1$\rangle$}}

\prerequisites{GB\_\,GRAPH}{GB\_\,IO}

\N{1}{1}Introduction. This GraphBase module contains the \PB{\\{book}}
subroutine, which creates a family of undirected graphs that are based on
classic works of literature.  It also contains the \PB{\\{bi\_book}}
subroutine, which creates a related family of bipartite graphs.
An example of the use of \PB{\\{book}} can be found in the demonstration
program {\sc BOOK\_\kern.05emCOMPONENTS}.

\Y\B\4\X1:\.{gb\_books.h\,}\X${}\E{}$\6
\&{extern} \&{Graph} ${}{*}\\{book}(\,);{}$\6
\&{extern} \&{Graph} ${}{*}\\{bi\_book}(\,){}$;\par
\As6, 18\ETs23.\fi

\M{2}The subroutine call \PB{$\\{book}(\hbox{\<title>},\|n,\|x,\\{first%
\_chapter},\\{last\_chapter},\\{in\_weight},\\{out\_weight},\\{seed})$}
constructs a graph based on the information in \<title>\.{.dat},
where \<title> is either \.{"anna"} (for {\sl Anna Karenina\/}),
\.{"david"} (for {\sl David Copperfield\/}),
\.{"jean"} (for {\sl Les Mis\'erables\/}),
\.{"huck"} (for {\sl Huckleberry Finn\/}), or
\.{"homer"} (for {\sl The Iliad\/}).
Each vertex of the graph corresponds to one of the characters in the
selected book. Edges between vertices correspond to encounters between
those characters. The length of each edge is~1.

Subsets of the book can be selected by specifying that the edge data should be
restricted to chapters between \PB{\\{first\_chapter}} and \PB{\\{last%
\_chapter}},
inclusive. If \PB{$\\{first\_chapter}\K\T{0}$}, the result is the same as if
\PB{$\\{first\_chapter}\K\T{1}$}. If \PB{$\\{last\_chapter}\K\T{0}$} or if \PB{%
\\{last\_chapter}} exceeds
the total number of chapters in the book, the result is the same as
if \PB{\\{last\_chapter}} were the number of the book's final chapter.

The constructed graph will have $\min(n,N)-x$ vertices, where $N$ is the
total number of characters in the selected book.
However, if \PB{\|n} is zero, \PB{\|n} is automatically made equal to the
maximum
possible value,~$N$. If \PB{\|n} is less than~$N$, the \PB{$\|n-\|x$}
characters will be
selected by assigning  a weight to each character and choosing the \PB{\|n}
with
largest weight, then excluding the largest~\PB{\|x} of these,
using random numbers to break ties in case of equal weights.
Weights are computed by the formula
$$ \PB{\\{in\_weight}}\cdot\\{chapters\_in}+\PB{\\{out\_weight}}\cdot%
\\{chapters\_out}, $$
where \\{chapters\_in} is the number of chapters between \PB{\\{first%
\_chapter}}
and \PB{\\{last\_chapter}} in which a particular character appears, and
\\{chapters\_out} is the number of other chapters in which that
character appears. Both \PB{\\{in\_weight}} and \PB{\\{out\_weight}} must be at
most
1,000,000 in absolute value.

Vertices of the graph will appear in order of decreasing weight.
The \PB{\\{seed}} parameter defines the pseudo-random numbers used wherever
a ``random'' choice between equal-weight vertices needs to be made.
As usual with GraphBase routines, different choices of \PB{\\{seed}}
will in general produce different selections,
but in a system-independent manner; identical results will be obtained on
all computers when identical parameters have been specified.
Any \PB{\\{seed}} value between 0 and $2^{31}-1$ is permissible.

\fi

\M{3}Examples: The call \PB{$\\{book}(\.{"anna"},\T{0},\T{0},\T{0},\T{0},\T{0},%
\T{0},\T{0})$} will construct a
graph on 138 vertices that represent all 138 characters of Tolstoy's
{\sl Anna Karenina\/}, as recorded in \.{anna.dat}. Two vertices will
be adjacent if the corresponding characters
encounter each other anywhere in the book. The call
\PB{$\\{book}(\.{"anna"},\T{50},\T{0},\T{0},\T{0},\T{1},\T{1},\T{0})$} is
similar, but it is restricted to
the 50 characters that occur most frequently, i.e., in the most chapters.
The call \PB{$\\{book}(\.{"anna"},\T{50},\T{0},\T{10},\T{120},\T{1},\T{1},%
\T{0})$} has the same vertices, but it
has edges only for encounters that take place between chapter~10
and chapter~120, inclusive. The call \PB{$\\{book}(\.{"anna"},\T{50},\T{0},%
\T{10},\T{120},\T{1},\T{0},\T{0})$} is
similar, but its vertices are the 50 characters that occur most often in
chapters 10 through~120, without regard to how often they occur in
the rest of the book. The call \PB{$\\{book}(\.{"anna"},\T{50},\T{0},\T{10},%
\T{120},\T{0},\T{0},\T{0})$} is
also similar, but it chooses 50 characters completely at random
(possibly from those that don't occur in the selected chapters at all).

Parameter \PB{\|x}, which causes the \PB{\|x} vertices of highest weight to be
excluded, is usually either 0 or~1. It is provided primarily so that
users can set \PB{$\|x\K\T{1}$} with respect to {\sl David Copperfield\/} and {%
\sl
Huckleberry Finn}; those novels are narrated by their principal
character, so they have edges between the principal character and
almost everybody else. (Characters cannot get into the action of a
first-person account unless they encounter the narrator or unless the
narrator is quoting some other person's story.) The corresponding
graphs tend to have more interesting connectivity properties if we
leave the narrator out by setting \PB{$\|x\K\T{1}$}. For example, there are 87
characters in {\sl David Copperfield\/}; the call
\PB{$\\{book}(\.{"david"},\T{0},\T{1},\T{0},\T{0},\T{1},\T{1},\T{0})$} produces
a graph with 86 vertices, one
for every character except David Copperfield himself.

\fi

\M{4}The subroutine call \PB{$\\{bi\_book}(\hbox{\<title>},\|n,\|x,\\{first%
\_chapter},\\{last\_chapter},\\{in\_weight},\\{out\_weight},\\{seed})$}
produces a bipartite graph in which the
vertices of the first part are exactly the same as the vertices of the
graph returned by \PB{\\{book}}, while the vertices of the second part are
the selected chapters. For example,
$\PB{\\{bi\_book}}(\PB{\.{"anna"}},\allowbreak 50,0,10,120,1,1,0)$
creates a bipartite graph with $50+111$ vertices. There is an edge between
each character and the chapters in which that character appears.

\fi

\M{5}Chapter numbering needs further explanation. {\sl Anna Karenina\/}
has 239 chapters, which are numbered 1.1 through 8.19 in the
work itself but renumbered 1 through 239 as far as the \PB{\\{book}} routine
is concerned. Thus, setting \PB{$\\{first\_chapter}\K\T{10}$} and \PB{$\\{last%
\_chapter}\K\T{120}$}
turns out to be equivalent to selecting chapters 1.10 through 4.19
(more precisely, chapter~10 of book~1 through chapter~19 of book~4).
{\sl Les Mis\'erables\/} has an even more involved scheme; its
356 chapters range from 1.1.1 (part~1, book~1, chapter~1) to
5.9.6 (part~5, book~9, chapter~6). After \PB{\\{book}} or \PB{\\{bi\_book}} has
created
a graph, the external integer variable \PB{\\{chapters}} will contain the total
number of chapters, and \PB{\\{chap\_name}} will be an array of strings
containing the structured chapter numbers. For example, after
\PB{$\\{book}(\.{"jean"},\hbox{\dots})$}, we will have \PB{$\\{chapters}\K%
\T{356}$},
\PB{$\\{chap\_name}[\T{1}]\K\.{"1.1.1"}$}, \dots, \PB{$\\{chap\_name}[\T{356}]%
\K\.{"5.9.6"}$};
\PB{\\{chap\_name}[\T{0}]} will be~\PB{\.{""}}.

\Y\B\4\D$\.{MAX\_CHAPS}$ \5
\T{360}\C{ no book will have this many chapters }\par
\Y\B\4\X5:External variables\X${}\E{}$\6
\&{long} \\{chapters};\C{ the total number of chapters in the selected book }\6
\&{char} ${}{*}\\{chap\_name}[\.{MAX\_CHAPS}]\K\{\.{""}\}{}$;\C{ string names
of those chapters }\par
\U8.\fi

\M{6}As usual, we put declarations of the external variables into the header
file
for users to {\bf include}.

\Y\B\4\X1:\.{gb\_books.h\,}\X${}\mathrel+\E{}$\6
\&{extern} \&{long} \\{chapters};\C{ the total number of chapters in the
selected book }\6
\&{extern} \&{char} ${}{*}\\{chap\_name}[\,]{}$;\C{ string names of those
chapters }\par
\fi

\M{7}If the \PB{\\{book}} or \PB{\\{bi\_book}} routine encounters a problem, it
returns \PB{$\NULL$} (\.{NULL}),
after putting a code number into the external variable
\PB{\\{panic\_code}}. This code number identifies the type of failure.
Otherwise \PB{\\{book}} returns a pointer to the newly created graph, which
will be represented with the data structures explained in {\sc GB\_\,GRAPH}.
(The external variable \PB{\\{panic\_code}} is itself defined in {\sc GB\_%
\,GRAPH}.)

\Y\B\4\D$\\{panic}(\|c)$ \5
${}\{{}$\5
\1${}\\{panic\_code}\K\|c{}$;\5
${}\\{gb\_trouble\_code}\K\T{0}{}$;\5
\&{return} ${}\NULL{}$;\5
${}\}{}$\2\Y\par
\B\F\\{node}\5
\\{long}\C{ the \&{node} type is defined below }\par
\fi

\M{8}The \CEE/ file \.{gb\_books.c} has the overall shape shown here.
It makes use of an internal subroutine
called \PB{\\{bgraph}}, which combines the work of \PB{\\{book}} and \PB{\\{bi%
\_book}}.

\Y\B\8\#\&{include} \.{"gb\_io.h"}\C{ we will use the {\sc GB\_\,IO} routines
for input }\6
\8\#\&{include} \.{"gb\_flip.h"}\C{ we will use the {\sc GB\_\,FLIP} routines
                      for random numbers }\6
\8\#\&{include} \.{"gb\_graph.h"}\C{ we will use the {\sc GB\_\,GRAPH} data
structures }\6
\8\#\&{include} \.{"gb\_sort.h"}\C{ and the \PB{\\{gb\_linksort}} routine }\6
\ATH\7
\X13:Type declarations\X\6
\X11:Private variables\X\6
\X5:External variables\X\7
\1\1\&{static} \&{Graph} ${}{*}\\{bgraph}(\\{bipartite},\39\\{title},\39\|n,\39%
\|x,\39\\{first\_chapter},\39\\{last\_chapter},\39\\{in\_weight},\39\\{out%
\_weight},\39\\{seed}){}$\6
\&{long} \\{bipartite};\C{ should we make the graph bipartite? }\6
\&{char} ${}{*}\\{title}{}$;\C{ identification of the selected book }\6
\&{unsigned} \&{long} \|n;\C{ number of vertices desired before exclusion }\6
\&{unsigned} \&{long} \|x;\C{ number of vertices to exclude }\6
\&{unsigned} \&{long} \\{first\_chapter}${},\39\\{last\_chapter}{}$;\C{
interval of chapters leading to edges }\6
\&{long} \\{in\_weight};\C{ weight coefficient pertaining to chapters
                in that interval }\6
\&{long} \\{out\_weight};\C{ weight coefficient pertaining to chapters
                 not in that interval }\6
\&{long} \\{seed};\C{ random number seed }\2\2\6
${}\{{}$\5
\1\X9:Local variables\X\7
\\{gb\_init\_rand}(\\{seed});\6
\X10:Check that the parameters are valid\X;\6
\X15:Skim the data file, recording the characters and computing their
statistics\X;\6
\X27:Choose the vertices and put them into an empty graph\X;\6
\X29:Read the data file more carefully and fill the graph as instructed\X;\6
\&{if} (\\{gb\_trouble\_code})\5
${}\{{}$\1\6
\\{gb\_recycle}(\\{new\_graph});\6
\\{panic}(\\{alloc\_fault});\C{ (expletive deleted)                      we ran
out of memory somewhere back there }\6
\4${}\}{}$\2\6
\&{return} \\{new\_graph};\6
\4${}\}{}$\2\7
\1\1\&{Graph} ${}{*}\\{book}(\\{title},\39\|n,\39\|x,\39\\{first\_chapter},\39%
\\{last\_chapter},\39\\{in\_weight},\39\\{out\_weight},\39\\{seed}){}$\6
\&{char} ${}{*}\\{title};{}$\6
\&{unsigned} \&{long} \|n${},\39\|x,\39\\{first\_chapter},\39\\{last%
\_chapter};{}$\6
\&{long} \\{in\_weight}${},\39\\{out\_weight},\39\\{seed};\2\2{}$\6
${}\{{}$\5
\1\&{return} \\{bgraph}${}(\T{0\$L},\39\\{title},\39\|n,\39\|x,\39\\{first%
\_chapter},\39\\{last\_chapter},\39\\{in\_weight},\39\\{out\_weight},\39%
\\{seed}){}$;\5
${}\}{}$\2\7
\1\1\&{Graph} ${}{*}\\{bi\_book}(\\{title},\39\|n,\39\|x,\39\\{first\_chapter},%
\39\\{last\_chapter},\39\\{in\_weight},\39\\{out\_weight},\39\\{seed}){}$\6
\&{char} ${}{*}\\{title};{}$\6
\&{unsigned} \&{long} \|n${},\39\|x,\39\\{first\_chapter},\39\\{last%
\_chapter};{}$\6
\&{long} \\{in\_weight}${},\39\\{out\_weight},\39\\{seed};\2\2{}$\6
${}\{{}$\5
\1\&{return} \\{bgraph}${}(\T{1\$L},\39\\{title},\39\|n,\39\|x,\39\\{first%
\_chapter},\39\\{last\_chapter},\39\\{in\_weight},\39\\{out\_weight},\39%
\\{seed}){}$;\5
${}\}{}$\2\par
\fi

\M{9}\B\X9:Local variables\X${}\E{}$\6
\&{Graph} ${}{*}\\{new\_graph}{}$;\C{ the graph constructed by \PB{\\{book}} or
\PB{\\{bi\_book}} }\6
\&{register} \&{long} \|j${},\39\|k{}$;\C{ all-purpose indices }\6
\&{long} \\{characters};\C{ the total number of characters in the selected book
}\6
\&{register} \&{node} ${}{*}\|p{}$;\C{ information about the current character
}\par
\A21.
\U8.\fi

\M{10}\B\D$\.{MAX\_CHARS}$ \5
\T{600}\C{ there won't be more characters than this }\par
\Y\B\4\X10:Check that the parameters are valid\X${}\E{}$\6
\&{if} ${}(\|n\E\T{0}){}$\1\5
${}\|n\K\.{MAX\_CHARS};{}$\2\6
\&{if} ${}(\\{first\_chapter}\E\T{0}){}$\1\5
${}\\{first\_chapter}\K\T{1};{}$\2\6
\&{if} ${}(\\{last\_chapter}\E\T{0}){}$\1\5
${}\\{last\_chapter}\K\.{MAX\_CHAPS};{}$\2\6
\&{if} ${}(\\{in\_weight}>\T{1000000}\V\\{in\_weight}<{-}\T{1000000}\V\\{out%
\_weight}>\T{1000000}\V\\{out\_weight}<{-}\T{1000000}){}$\1\5
\\{panic}(\\{bad\_specs});\C{ the magnitude of at least one weight is too big }%
\2\6
${}\\{sprintf}(\\{file\_name},\39\.{"\%.6s.dat"},\39\\{title});{}$\6
\&{if} ${}(\\{gb\_open}(\\{file\_name})\I\T{0}){}$\1\5
\\{panic}(\\{early\_data\_fault});\C{ couldn't open the file; \PB{\\{io%
\_errors}} tells why }\2\par
\U8.\fi

\M{11}\B\X11:Private variables\X${}\E{}$\6
\&{static} \&{char} \\{file\_name}[\,]${}\K\.{"xxxxxx.dat"}{}$;\par
\As14\ET24.
\U8.\fi

\N{1}{12}Vertices.
Each character in a book has been given a two-letter code name for
internal use. The code names are explained at the beginning of each
data file by a number of lines that look like this:
$$\hbox{\tt XX \<name>,\<description>}$$
For example, here's one of the lines near the beginning of \PB{\.{"anna.dat"}}:
$$\hbox{\tt AL Alexey Alexandrovitch Karenin, minister of state}$$
The \<name> does not contain a comma; the \<description> might.

A blank line follows the cast of characters.

Internally, we will think of the two-letter code as a radix-36 integer.
Thus \.{AA} will be the number $10\times36+10$, and \.{ZZ} will be
$35\times36+35$. The \PB{\\{gb\_number}} routine in {\sc GB\_\,IO} is set up to
input radix-36 integers just as it does hexadecimal ones.
In {\sl The Iliad}, many of the minor characters have numeric digits
in their code names because the total number of characters is too
large to permit mnemonic codes for everybody.

\Y\B\4\D$\.{MAX\_CODE}$ \5
\T{1296}\C{ $36\times36$, the number of two-digit codes in radix 36 }\par
\fi

\M{13}In order to choose the vertices, we want to represent each character
as a node whose key corresponds to its weight; then the \PB{\\{gb\_linksort}}
routine of {\sc GB\_\,SORT} will provide the desired rank-ordering. We will
find it convenient to use these nodes for all the data processing that
\PB{\\{bgraph}} has to do.

\Y\B\4\X13:Type declarations\X${}\E{}$\6
\&{typedef} \&{struct} \&{node\_struct} ${}\{{}$\C{ records to be sorted by %
\PB{\\{gb\_linksort}} }\1\6
\&{long} \\{key};\C{ the nonnegative sort key (weight plus $2^{30}$) }\6
\&{struct} \&{node\_struct} ${}{*}\\{link}{}$;\C{ pointer to next record }\6
\&{long} \\{code};\C{ code number of this character }\6
\&{long} \\{in};\C{ number of occurrences in selected chapters }\6
\&{long} \\{out};\C{ number of occurrences in unselected chapters }\6
\&{long} \\{chap};\C{ seen most recently in this chapter }\6
\&{Vertex} ${}{*}\\{vert}{}$;\C{ vertex corresponding to this character }\2\6
${}\}{}$ \&{node};\par
\U8.\fi

\M{14}Not only do nodes point to codes, we also want codes to point to nodes.

\Y\B\4\X11:Private variables\X${}\mathrel+\E{}$\6
\&{static} \&{node} \\{node\_block}[\.{MAX\_CHARS}];\C{ array of nodes for
working storage }\6
\&{static} \&{node} ${}{*}\\{xnode}[\.{MAX\_CODE}]{}$;\C{ the node, if any,
having a given code }\par
\fi

\M{15}We will read the data file twice, once quickly (to collect statistics)
and once more thoroughly (to record detailed information). Here is the
quick version.

\Y\B\4\X15:Skim the data file, recording the characters and computing their
statistics\X${}\E{}$\6
\X16:Read the character codes at the beginning of the data file, and prepare a
node for each one\X;\6
\X19:Skim the chapter information, counting the number of chapters in which
each character appears\X;\6
\&{if} ${}(\\{gb\_close}(\,)\I\T{0}){}$\1\5
\\{panic}(\\{late\_data\_fault});\C{ checksum or other failure in data file;
see \PB{\\{io\_errors}} }\2\par
\U8.\fi

\M{16}\B\X16:Read the character codes at the beginning of the data file, and
prepare a node for each one\X${}\E{}$\6
\&{for} ${}(\|k\K\T{0};{}$ ${}\|k<\.{MAX\_CODE};{}$ ${}\|k\PP){}$\1\5
${}\\{xnode}[\|k]\K\NULL;{}$\2\6
${}\{{}$\5
\1\&{register} \&{long} \|c;\C{ current code entering the system }\7
${}\|p\K\\{node\_block}{}$;\C{ current node entering the system }\6
\&{while} ${}((\|c\K\\{gb\_number}(\T{36}))\I\T{0}){}$\5
${}\{{}$\C{ note that \.{00} is not a legal code }\1\6
\&{if} ${}(\|c\G\.{MAX\_CODE}\V\\{gb\_char}(\,)\I\.{'\ '}){}$\1\5
\\{panic}(\\{syntax\_error});\C{ unreadable line in data file }\2\6
\&{if} ${}(\|p\G{\AND}\\{node\_block}[\.{MAX\_CHARS}]){}$\1\5
${}\\{panic}(\\{syntax\_error}+\T{1}){}$;\C{ data has too many characters }\2\6
${}\|p\MG\\{link}\K(\|p\E\\{node\_block}\?\NULL:\|p-\T{1});{}$\6
${}\|p\MG\\{code}\K\|c;{}$\6
${}\\{xnode}[\|c]\K\|p;{}$\6
${}\|p\MG\\{in}\K\|p\MG\\{out}\K\|p\MG\\{chap}\K\T{0};{}$\6
${}\|p\MG\\{vert}\K\NULL;{}$\6
${}\|p\PP;{}$\6
\\{gb\_newline}(\,);\6
\4${}\}{}$\2\6
${}\\{characters}\K\|p-\\{node\_block};{}$\6
\\{gb\_newline}(\,);\C{ bypass the blank line that terminates the character
data }\6
\4${}\}{}$\2\par
\U15.\fi

\M{17}Later we will read through this part of the file again, extracting
additional information if it turns out to be relevant. The
\<description> string is provided to users in a \PB{\\{desc}} field,
in case anybody cares to look at it. The \PB{\\{in}} and \PB{\\{out}}
statistics
are also made available in utility fields called \PB{\\{in\_count}} and \PB{%
\\{out\_count}}.
The code value is placed in the \PB{\\{short\_code}} field.

\Y\B\4\D$\\{desc}$ \5
$\|z.{}$\|S\C{ utility field \PB{\|z} points to the \<description> string }\par
\B\4\D$\\{in\_count}$ \5
$\|y.{}$\|I\C{ utility field \PB{\|y} counts appearances in selected chapters }%
\par
\B\4\D$\\{out\_count}$ \5
$\|x.{}$\|I\C{ utility field \PB{\|x} counts appearances in other chapters }\par
\B\4\D$\\{short\_code}$ \5
$\|u.{}$\|I\C{ utility field \PB{\|u} contains a radix-36 number }\par
\Y\B\4\X17:Read the data about characters again, noting vertex names and the
associated descriptions\X${}\E{}$\6
${}\{{}$\5
\1\&{register} \&{long} \|c;\C{ current code entering the system a second time
}\7
\&{while} ${}((\|c\K\\{gb\_number}(\T{36}))\I\T{0}){}$\5
${}\{{}$\5
\1\&{register} \&{Vertex} ${}{*}\|v\K\\{xnode}[\|c]\MG\\{vert};{}$\7
\&{if} (\|v)\5
${}\{{}$\1\6
\&{if} ${}(\\{gb\_char}(\,)\I\.{'\ '}){}$\1\5
\\{panic}(\\{impossible});\C{ can't happen }\2\6
${}\\{gb\_string}(\\{str\_buf},\39\.{','}){}$;\C{ scan the \<name> part }\6
${}\|v\MG\\{name}\K\\{gb\_save\_string}(\\{str\_buf});{}$\6
\&{if} ${}(\\{gb\_char}(\,)\I\.{','}){}$\1\5
${}\\{panic}(\\{syntax\_error}+\T{2}){}$;\C{ missing comma after \<name> }\2\6
\&{if} ${}(\\{gb\_char}(\,)\I\.{'\ '}){}$\1\5
${}\\{panic}(\\{syntax\_error}+\T{3}){}$;\C{ missing space after comma }\2\6
${}\\{gb\_string}(\\{str\_buf},\39\.{'\\n'}){}$;\C{ scan the \<description>
part }\6
${}\|v\MG\\{desc}\K\\{gb\_save\_string}(\\{str\_buf});{}$\6
${}\|v\MG\\{in\_count}\K\\{xnode}[\|c]\MG\\{in};{}$\6
${}\|v\MG\\{out\_count}\K\\{xnode}[\|c]\MG\\{out};{}$\6
${}\|v\MG\\{short\_code}\K\|c;{}$\6
\4${}\}{}$\2\6
\\{gb\_newline}(\,);\6
\4${}\}{}$\2\6
\\{gb\_newline}(\,);\C{ bypass the blank line that terminates the character
data }\6
\4${}\}{}$\2\par
\U29.\fi

\M{18}\B\X1:\.{gb\_books.h\,}\X${}\mathrel+\E{}$\6
\8\#\&{define} ${}\\{desc}\hbox{\quad}\|z.{}$\|S\C{ utility field definitions
for the header file }\6
\8\#\&{define} ${}\\{in\_count}\hbox{\quad}\|y.\|I{}$\6
\8\#\&{define} ${}\\{out\_count}\hbox{\quad}\|x.\|I{}$\6
\8\#\&{define} ${}\\{short\_code}\hbox{\quad}\|u.\|I{}$\par
\fi

\N{1}{19}Edges.
The second part of the data file has a line for each chapter, containing
``cliques of encounters.'' For example, the line
$$\hbox{\tt3.22:AA,BB,CC,DD;CC,DD,EE;AA,FF}$$
means that, in chapter 22 of book 3, there were encounters between the pairs
$$\def\\{{\rm,} }
\hbox{\tt AA-BB\\AA-CC\\AA-DD\\BB-CC\\BB-DD\\CC-DD\\CC-EE\\DD-EE\\{\rm and }%
AA-FF\rm.}$$
(The encounter \.{CC-DD} is specified twice, once in the clique
\.{AA,BB,CC,DD} and once in \.{CC,DD,EE}; this does not imply anything about
the actual number of encounters between \.{CC} and \.{DD} in the chapter.)

A clique might involve one character only, when that character is featured
in sort of a soliloquy.

A chapter might contain no references to characters at all. In such a case
the `\.:' following the chapter number is omitted.

There might be more encounters than will fit on a single line. In such cases,
continuation lines begin with `\.{\&:}'. This convention turns out to be
needed only in \.{homer.dat}; chapters in {\sl The Iliad\/} are
substantially more complex than the chapters in other GraphBase books.

On our first pass over the data, we simply want to compute statistics about
who appears in what chapters, so we ignore the distinction between
commas and semicolons.

\Y\B\4\X19:Skim the chapter information, counting the number of chapters in
which each character appears\X${}\E{}$\6
\&{for} ${}(\|k\K\T{1};{}$ ${}\|k<\.{MAX\_CHAPS}\W\R\\{gb\_eof}(\,);{}$ ${}\|k%
\PP){}$\5
${}\{{}$\1\6
${}\\{gb\_string}(\\{str\_buf},\39\.{':'}){}$;\C{ read past the chapter number
}\6
\&{if} ${}(\\{str\_buf}[\T{0}]\E\.{'\&'}){}$\1\5
${}\|k\MM{}$;\C{ continuation of previous chapter }\2\6
\&{while} ${}(\\{gb\_char}(\,)\I\.{'\\n'}){}$\5
${}\{{}$\5
\1\&{register} \&{long} \|c${}\K\\{gb\_number}(\T{36});{}$\7
\&{if} ${}(\|c\G\.{MAX\_CODE}){}$\1\5
${}\\{panic}(\\{syntax\_error}+\T{4}){}$;\C{ missing punctuation between
characters }\2\6
${}\|p\K\\{xnode}[\|c];{}$\6
\&{if} ${}(\|p\E\NULL){}$\1\5
${}\\{panic}(\\{syntax\_error}+\T{5}){}$;\C{ unknown character }\2\6
\&{if} ${}(\|p\MG\\{chap}\I\|k){}$\5
${}\{{}$\1\6
${}\|p\MG\\{chap}\K\|k;{}$\6
\&{if} ${}(\|k\G\\{first\_chapter}\W\|k\Z\\{last\_chapter}){}$\1\5
${}\|p\MG\\{in}\PP;{}$\2\6
\&{else}\1\5
${}\|p\MG\\{out}\PP;{}$\2\6
\4${}\}{}$\2\6
\4${}\}{}$\2\6
\\{gb\_newline}(\,);\6
\4${}\}{}$\2\6
\&{if} ${}(\|k\E\.{MAX\_CHAPS}){}$\1\5
${}\\{panic}(\\{syntax\_error}+\T{6}){}$;\C{ too many chapters }\2\6
${}\\{chapters}\K\|k-\T{1}{}$;\par
\U15.\fi

\M{20}Our second pass over the data is very similar to the first, if we
are simply computing a bipartite graph. In that case we add an edge
to the graph between each selected chapter and each selected character
in that chapter. Local variable \PB{\\{chap\_base}} will point to a
vertex such that \PB{$\\{chap\_base}+\|k$} is the vertex corresponding to
chapter~\PB{\|k}.

The \PB{\\{in\_count}} of a chapter vertex is the degree of that vertex, i.e.,
the
number of selected characters that appear in the corresponding chapter.
The \PB{\\{out\_count}} is the number of characters that appear in the
chapter but were omitted from the graph. Thus the \PB{\\{in\_count}} and
\PB{\\{out\_count}} for chapters are analogous to the \PB{\\{in\_count}} and %
\PB{\\{out\_count}}
for characters.

\Y\B\4\X20:Read the chapter information a second time and create the
appropriate bipartite edges\X${}\E{}$\6
${}\{{}$\1\6
\&{for} ${}(\|p\K\\{node\_block};{}$ ${}\|p<\\{node\_block}+\\{characters};{}$
${}\|p\PP){}$\1\5
${}\|p\MG\\{chap}\K\T{0};{}$\2\6
\&{for} ${}(\|k\K\T{1};{}$ ${}\R\\{gb\_eof}(\,);{}$ ${}\|k\PP){}$\5
${}\{{}$\1\6
${}\\{gb\_string}(\\{str\_buf},\39\.{':'}){}$;\C{ read the chapter number }\6
\&{if} ${}(\\{str\_buf}[\T{0}]\E\.{'\&'}){}$\1\5
${}\|k\MM;{}$\2\6
\&{else}\5
${}\{{}$\1\6
\&{if} ${}(\\{str\_buf}[\\{strlen}(\\{str\_buf})-\T{1}]\E\.{'\\n'}){}$\1\5
${}\\{str\_buf}[\\{strlen}(\\{str\_buf})-\T{1}]\K\.{'\\0'};{}$\2\6
${}\\{chap\_name}[\|k]\K\\{gb\_save\_string}(\\{str\_buf});{}$\6
\4${}\}{}$\2\6
\&{if} ${}(\|k\G\\{first\_chapter}\W\|k\Z\\{last\_chapter}){}$\5
${}\{{}$\5
\1\&{register} \&{Vertex} ${}{*}\|u\K\\{chap\_base}+\|k;{}$\7
\&{if} ${}(\\{str\_buf}[\T{0}]\I\.{'\&'}){}$\5
${}\{{}$\1\6
${}\|u\MG\\{name}\K\\{chap\_name}[\|k];{}$\6
${}\|u\MG\\{desc}\K\\{null\_string};{}$\6
${}\|u\MG\\{in\_count}\K\|u\MG\\{out\_count}\K\T{0};{}$\6
\4${}\}{}$\2\6
\&{while} ${}(\\{gb\_char}(\,)\I\.{'\\n'}){}$\5
${}\{{}$\5
\1\&{register} \&{long} \|c${}\K\\{gb\_number}(\T{36});{}$\7
${}\|p\K\\{xnode}[\|c];{}$\6
\&{if} ${}(\|p\MG\\{chap}\I\|k){}$\5
${}\{{}$\5
\1\&{register} \&{Vertex} ${}{*}\|v\K\|p\MG\\{vert};{}$\7
${}\|p\MG\\{chap}\K\|k;{}$\6
\&{if} (\|v)\5
${}\{{}$\5
\1${}\\{gb\_new\_edge}(\|v,\39\|u,\39\T{1\$L});{}$\6
${}\|u\MG\\{in\_count}\PP;{}$\6
\4${}\}{}$\5
\2\&{else}\1\5
${}\|u\MG\\{out\_count}\PP;{}$\2\6
\4${}\}{}$\2\6
\4${}\}{}$\2\6
\4${}\}{}$\2\6
\\{gb\_newline}(\,);\6
\4${}\}{}$\2\6
\4${}\}{}$\2\par
\U29.\fi

\M{21}\B\X9:Local variables\X${}\mathrel+\E{}$\6
\&{Vertex} ${}{*}\\{chap\_base}{}$;\C{ the bipartite vertex for chapter~\PB{%
\|k} is \PB{$\\{chap\_base}+\|k$} }\par
\fi

\M{22}The second pass has to work a little harder when we are recording
encounters from cliques, but the logic isn't difficult really.
We insert a reference to the first chapter that generated each edge, in
utility field \PB{\\{chap\_no}} of the corresponding \PB{\&{Arc}} record.

\Y\B\4\D$\\{chap\_no}$ \5
$\|a.{}$\|I\C{ utility field \PB{\|a} holds a chapter number }\par
\Y\B\4\X22:Read the chapter information a second time and create the
appropriate edges for encounters\X${}\E{}$\6
\&{for} ${}(\|k\K\T{1};{}$ ${}\R\\{gb\_eof}(\,);{}$ ${}\|k\PP){}$\5
${}\{{}$\5
\1\&{char} ${}{*}\|s;{}$\7
${}\|s\K\\{gb\_string}(\\{str\_buf},\39\.{':'}){}$;\C{ read the chapter number
}\6
\&{if} ${}(\\{str\_buf}[\T{0}]\E\.{'\&'}){}$\1\5
${}\|k\MM;{}$\2\6
\&{else}\5
${}\{{}$\5
\1\&{if} ${}({*}(\|s-\T{2})\E\.{'\\n'}){}$\1\5
${}{*}(\|s-\T{2})\K\.{'\\0'};{}$\2\6
${}\\{chap\_name}[\|k]\K\\{gb\_save\_string}(\\{str\_buf});{}$\6
\4${}\}{}$\2\6
\&{if} ${}(\|k\G\\{first\_chapter}\W\|k\Z\\{last\_chapter}){}$\5
${}\{{}$\5
\1\&{register} \&{long} \|c${}\K\\{gb\_char}(\,);{}$\7
\&{while} ${}(\|c\I\.{'\\n'}){}$\5
${}\{{}$\5
\1\&{register} \&{Vertex} ${}{*}{*}\\{pp}\K\\{clique\_table};{}$\6
\&{register} \&{Vertex} ${}{*}{*}\\{qq},\39{*}{*}\\{rr}{}$;\C{ pointers within
the clique table }\7
\&{do}\5
${}\{{}$\5
\1${}\|c\K\\{gb\_number}(\T{36}){}$;\C{ set \PB{\|c} to code for next character
of clique }\6
\&{if} ${}(\\{xnode}[\|c]\MG\\{vert}{}$)\C{ is that character a selected
vertex? }\1\6
${}{*}\\{pp}\PP\K\\{xnode}[\|c]\MG\\{vert}{}$;\C{ if so, that vertex joins the
current clique }\2\6
${}\|c\K\\{gb\_char}(\,);{}$\6
\4${}\}{}$\5
\2\5
\&{while} ${}(\|c\E\.{','}){}$;\C{ repeat until end of the clique }\6
\&{for} ${}(\\{qq}\K\\{clique\_table};{}$ ${}\\{qq}+\T{1}<\\{pp};{}$ ${}\\{qq}%
\PP){}$\1\6
\&{for} ${}(\\{rr}\K\\{qq}+\T{1};{}$ ${}\\{rr}<\\{pp};{}$ ${}\\{rr}\PP){}$\1\5
\X25:Make the vertices \PB{${*}\\{qq}$} and \PB{${*}\\{rr}$} adjacent, if they
aren't already\X;\2\2\6
\4${}\}{}$\2\6
\4${}\}{}$\2\6
\\{gb\_newline}(\,);\6
\4${}\}{}$\2\par
\U29.\fi

\M{23}\B\X1:\.{gb\_books.h\,}\X${}\mathrel+\E{}$\6
\8\#\&{define} \\{chap\_no}\5${}\|a.\|I{}$\C{ utility field definition in the
header file }\par
\fi

\M{24}\B\X11:Private variables\X${}\mathrel+\E{}$\6
\&{static} \&{Vertex} ${}{*}\\{clique\_table}[\T{30}]{}$;\C{ pointers to
vertices in the current clique }\par
\fi

\M{25}\B\X25:Make the vertices \PB{${*}\\{qq}$} and \PB{${*}\\{rr}$} adjacent,
if they aren't already\X${}\E{}$\6
${}\{{}$\5
\1\&{register} \&{Vertex} ${}{*}\|u\K{*}\\{qq},\39{*}\|v\K{*}\\{rr};{}$\6
\&{register} \&{Arc} ${}{*}\|a;{}$\7
\&{for} ${}(\|a\K\|u\MG\\{arcs};{}$ \|a; ${}\|a\K\|a\MG\\{next}){}$\1\6
\&{if} ${}(\|a\MG\\{tip}\E\|v){}$\1\5
\&{goto} \\{found};\2\2\6
${}\\{gb\_new\_edge}(\|u,\39\|v,\39\T{1\$L}){}$;\C{ not found, so they weren't
already adjacent }\6
\&{if} ${}(\|u<\|v){}$\1\5
${}\|a\K\|u\MG\\{arcs};{}$\2\6
\&{else}\1\5
${}\|a\K\|v\MG\\{arcs}{}$;\C{ the new edge consists of arcs \PB{\|a} and \PB{$%
\|a+\T{1}$} }\2\6
${}\|a\MG\\{chap\_no}\K(\|a+\T{1})\MG\\{chap\_no}\K\|k;{}$\6
\4\\{found}:\5
;\6
\4${}\}{}$\2\par
\U22.\fi

\N{1}{26}Administration.
The program is now complete except for a few missing organizational details.
I will add these after lunch.

\fi

\M{27}OK, I'm back; what needs to be done? The main thing is to create
the graph itself.

\Y\B\4\X27:Choose the vertices and put them into an empty graph\X${}\E{}$\6
\&{if} ${}(\|n>\\{characters}){}$\1\5
${}\|n\K\\{characters};{}$\2\6
\&{if} ${}(\|x>\|n){}$\1\5
${}\|x\K\|n;{}$\2\6
\&{if} ${}(\\{last\_chapter}>\\{chapters}){}$\1\5
${}\\{last\_chapter}\K\\{chapters};{}$\2\6
\&{if} ${}(\\{first\_chapter}>\\{last\_chapter}){}$\1\5
${}\\{first\_chapter}\K\\{last\_chapter}+\T{1};{}$\2\6
${}\\{new\_graph}\K\\{gb\_new\_graph}(\|n-\|x+(\\{bipartite}\?\\{last%
\_chapter}-\\{first\_chapter}+\T{1}:\T{0}));{}$\6
\&{if} ${}(\\{new\_graph}\E\NULL){}$\1\5
\\{panic}(\\{no\_room});\C{ out of memory already }\2\6
${}\\{strcpy}(\\{new\_graph}\MG\\{util\_types},\39\.{"IZZIISIZZZZZZZ"}){}$;\C{
declare the types of utility fields }\6
${}\\{sprintf}(\\{new\_graph}\MG\\{id},\39\.{"\%sbook(\\"\%s\\",\%lu,\%}\)%
\.{lu,\%lu,\%lu,\%ld,\%ld,\%}\)\.{ld)"},\39\\{bipartite}\?\.{"bi\_"}:\.{""},\39%
\\{title},\39\|n,\39\|x,\39\\{first\_chapter},\39\\{last\_chapter},\39\\{in%
\_weight},\39\\{out\_weight},\39\\{seed});{}$\6
\&{if} (\\{bipartite})\5
${}\{{}$\1\6
${}\\{mark\_bipartite}(\\{new\_graph},\39\|n-\|x);{}$\6
${}\\{chap\_base}\K\\{new\_graph}\MG\\{vertices}+(\\{new\_graph}\MG\\{n\_1}-%
\\{first\_chapter});{}$\6
\4${}\}{}$\2\6
\X28:Compute the weights and assign vertices to chosen nodes\X;\par
\U8.\fi

\M{28}\B\X28:Compute the weights and assign vertices to chosen nodes\X${}\E{}$\6
\&{for} ${}(\|p\K\\{node\_block};{}$ ${}\|p<\\{node\_block}+\\{characters};{}$
${}\|p\PP){}$\1\5
${}\|p\MG\\{key}\K\\{in\_weight}*(\|p\MG\\{in})+\\{out\_weight}*(\|p\MG%
\\{out})+\T{\^40000000};{}$\2\6
${}\\{gb\_linksort}(\\{node\_block}+\\{characters}-\T{1});{}$\6
${}\|k\K\|n{}$;\C{ we will look at this many nodes }\6
${}\{{}$\5
\1\&{register} \&{Vertex} ${}{*}\|v\K\\{new\_graph}\MG\\{vertices}{}$;\C{ the
next vertex to define }\7
\&{for} ${}(\|j\K\T{127};{}$ ${}\|j\G\T{0};{}$ ${}\|j\MM){}$\1\6
\&{for} ${}(\|p\K{}$(\&{node} ${}{*})\,\\{gb\_sorted}[\|j];{}$ \|p; ${}\|p\K\|p%
\MG\\{link}){}$\5
${}\{{}$\1\6
\&{if} ${}(\|x>\T{0}){}$\1\5
${}\|x\MM{}$;\C{ ignore this node }\2\6
\&{else}\1\5
${}\|p\MG\\{vert}\K\|v\PP{}$;\C{ choose this node }\2\6
\&{if} ${}(\MM\|k\E\T{0}){}$\1\5
\&{goto} \\{done};\2\6
\4${}\}{}$\2\2\6
\4${}\}{}$\2\6
\4\\{done}:\5
;\par
\U27.\fi

\M{29}Once the graph is there, we're ready to fill it in.

\Y\B\4\X29:Read the data file more carefully and fill the graph as instructed%
\X${}\E{}$\6
\&{if} ${}(\\{gb\_open}(\\{file\_name})\I\T{0}){}$\1\5
${}\\{panic}(\\{impossible}+\T{1}){}$;\C{ this can't happen, because we were
successful before }\2\6
\X17:Read the data about characters again, noting vertex names and the
associated descriptions\X;\6
\&{if} (\\{bipartite})\1\5
\X20:Read the chapter information a second time and create the appropriate
bipartite edges\X\2\6
\&{else}\1\5
\X22:Read the chapter information a second time and create the appropriate
edges for encounters\X;\2\6
\&{if} ${}(\\{gb\_close}(\,)\I\T{0}){}$\1\5
${}\\{panic}(\\{impossible}+\T{2}){}$;\C{ again, can hardly happen the second
time around }\2\par
\U8.\fi

\N{1}{30}Index. As usual, we close with an index that
shows where the identifiers of \\{gb\_books} are defined and used.
\fi

\inx
\fin
\con
