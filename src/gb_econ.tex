\input cwebmac
% This file is part of the Stanford GraphBase (c) Stanford University 1993
% This material goes at the beginning of all Stanford GraphBase CWEB files

\def\topofcontents{
  \leftline{\sc\today\ at \hours}\bigskip\bigskip
  \centerline{\titlefont\title}}

\font\ninett=cmtt9
\def\botofcontents{\vskip 0pt plus 1filll
    \ninerm\baselineskip10pt
    \noindent\copyright\ 1993 Stanford University
    \bigskip\noindent
    This file may be freely copied and distributed, provided that
    no changes whatsoever are made. All users are asked to help keep
    the Stanford GraphBase files consistent and ``uncorrupted,''
    identical everywhere in the world. Changes are permissible only
    if the modified file is given a new name, different from the names of
    existing files in the Stanford GraphBase, and only if the modified file is
    clearly identified as not being part of that GraphBase.
    (The {\ninett CWEB} system has a ``change file'' facility by
    which users can easily make minor alterations without modifying
    the master source files in any way. Everybody is supposed to use
    change files instead of changing the files.)
    The author has tried his best to produce correct and useful programs,
    in order to help promote computer science research,
    but no warranty of any kind should be assumed.
    \smallskip\noindent
    Preliminary work on the Stanford GraphBase project
    was supported in part by National Science
    Foundation grant CCR-86-10181.}

\def\prerequisite#1{\def\startsection{\noindent
    Important: Before reading {\sc\title},
    please read or at least skim the program for {\sc#1}.\bigskip
    \let\startsection=\stsec\stsec}}
\def\prerequisites#1#2{\def\startsection{\noindent
    Important: Before reading {\sc\title}, please read
    or at least skim the programs for {\sc#1} and {\sc#2}.\bigskip
    \let\startsection=\stsec\stsec}}



\def\title{GB\_\,ECON}

\prerequisites{GB\_\,GRAPH}{GB\_\,IO}

\N{1}{1}Introduction. This GraphBase module contains the \PB{\\{econ}}
subroutine,
which creates a family of directed graphs related to the flow of money
between industries.  An example of the use of this procedure can be
found in the demo program {\sc ECON\_\,ORDER}.

\Y\B\4\X1:\.{gb\_econ.h\,}\X${}\E{}$\6
\&{extern} \&{Graph} ${}{*}\\{econ}(\,){}$;\par
\A5.\fi

\M{2}The subroutine call \PB{$\\{econ}(\|n,\\{omit},\\{threshold},\\{seed})$}
constructs a directed graph based on the information in \.{econ.dat}.
Each vertex of the graph corresponds to one of 81 sectors of the U.S.
economy. The data values come from the year 1985; they were derived from
tables published in {\sl Survey of Current Business\/ \bf70} (1990), 41--56.

If \PB{$\\{omit}\K\\{threshold}\K\T{0}$}, the directed graph is a
``circulation'';
that is, each arc has an associated \PB{\\{flow}} value, and
the sum of arc flows leaving each vertex is equal to the
sum of arc flows entering. This sum is called the ``total commodity output''
for the sector in question. The flow in an arc from sector $j$~to
sector~$k$ is the amount of the commodity made by sector~$j$ that was
used by sector~$k$, rounded to millions of dollars at producers' prices.
For example, the total commodity output of the sector called \.{Apparel}
is 54031, meaning that the total cost of making all kinds of apparel in
1985 was about 54 billion dollars. There is an arc from \.{Apparel} to
itself with a flow of 9259, meaning that 9.259 billion dollars' worth
of apparel went from one group within the apparel industry to another.
There also is an arc of flow~44 from \.{Apparel} to \.{Household}
\.{furniture}, indicating that some 44 million dollars' worth of apparel
went into the making of household furniture. By looking at all
arcs that leave the \.{Apparel} vertex, you can see where all that
new apparel went; by looking at all arcs that enter \.{Apparel}, you can
see what ingredients the apparel industry needed to make~it.

One vertex, called \.{Users}, represents people like you and me, the
non-industrial end users of everything. The arc from \.{Apparel} to
\.{Users} has flow 42172; this is the ``total final demand'' for
apparel, the amount that didn't flow into other sectors of the economy
before it reached people like us. The arc from \.{Users} to \.{Apparel}
has flow 19409, which is called the ``value added'' by users; it
represents wages and salaries paid to support the manufacturing
process. The sum of total final demand over all sectors, which also
equals the sum of value added over all sectors, is conventionally
called the Gross National Product (GNP). In 1985 the GNP was 3999362,
nearly 4 trillion dollars, according to \.{econ.dat}. (The sum of all
arc flows coming out of all vertices was 7198680; this sum
overestimates the total economic activity, because it counts some
items more than once---statistics are recorded whenever an item
passes a statistics gatherer. Economists try to adjust the data so that
they avoid double-counting as much as possible.)

Speaking of economists, there is another special vertex called
\.{Adjustments}, included by economists so that GNP is measured
more accurately. This vertex takes account of such things as changes in
the value of inventories, and imported materials that cannot be obtained
within the U.S., as well as work done for the government and for foreign
concerns. In 1985, these adjustments accounted for about 11\% of the GNP.

Incidentally, some of the ``total final demand'' arcs are negative.
For example, the arc from \.{Petroleum} \.{and} \.{natural} \.{gas}
\.{production} to \.{Users} has flow $-27032$. This might seem strange
at first, but it makes sense when imports are considered, because
crude oil and natural gas go more to other industries than to end users.
Total final demand does not mean total user demand.

\Y\B\4\D$\\{flow}$ \5
$\|a.{}$\|I\C{ utility field \PB{\|a} specifies the flow in an arc }\par
\fi

\M{3}If \PB{$\\{omit}\K\T{1}$}, the \.{Users} vertex is omitted from the
digraph; in
particular, this will eliminate all arcs of negative flow. If
\PB{$\\{omit}\K\T{2}$}, the \.{Adjustments} vertex is also omitted, thereby
leaving
79~sectors with arcs showing inter-industry flow. (The graph is no
longer a ``circulation,'' of course, when \PB{$\\{omit}>\T{0}$}.)  If \.{Users}
and
\.{Adjustments} are not omitted, \.{Users} is the last vertex of the
graph, and \.{Adjustments} is next-to-last.

If \PB{$\\{threshold}\K\T{0}$}, the digraph has an arc for every nonzero \PB{%
\\{flow}}.
But if \PB{$\\{threshold}>\T{0}$}, the digraph becomes more sparse;
there is then an arc from $j$ to~$k$ if and
only if the amount of commodity $j$ used by sector~$k$ exceeds
\PB{$\\{threshold}/\T{65536}$} times the total input of sector~$k$.  (The total
input figure always includes value added, even if \PB{$\\{omit}>\T{0}$}.)
Thus the arcs go to each sector from
that sector's main suppliers. When \PB{$\|n\K\T{79}$}, \PB{$\\{omit}\K\T{2}$},
and
\PB{$\\{threshold}\K\T{0}$}, the digraph has 4602 arcs out of a possible
$79\times79=6241$. Raising \PB{\\{threshold}} to 1 decreases the number of
arcs to 4473; raising it to 6000 leaves only~72 arcs.
The \PB{\\{len}} field in each arc is~1.

The constructed graph will have $\min(n,81-\PB{\\{omit}})$ vertices. If \PB{%
\|n} is less
than \PB{$\T{81}-\\{omit}$}, the \PB{\|n} vertices will be selected by
repeatedly combining
related sectors. For example, two of the 81 original sectors are called
`\.{Paper} \.{products,} \.{except} \.{containers}' and
`\.{Paperboard} \.{containers} \.{and} \.{boxes}'; these might be combined
into a sector called `\.{Paper} \.{products}'. There is a binary tree
with 79 leaves, which describes a fixed hierarchical breakdown of the
79 non-special sectors. This tree is
pruned, if necessary, by replacing pairs of leaves by their parent node,
which becomes a new leaf; pruning continues
until just \PB{\|n} leaves remain. Although pruning is a bottom-up process, its
effect can also be obtained from the top down if we imagine ``growing''
the tree, starting out with a whole economy as a single sector and
repeatedly subdividing a sector into two parts. For example,
if \PB{$\\{omit}\K\T{2}$} and \PB{$\|n\K\T{2}$}, the two sectors will
be called \.{Goods} and \.{Services}. If \PB{$\|n\K\T{3}$}, \.{Goods} might be
subdivided into \.{Natural} \.{Resources} and \.{Manufacturing}; or
\.{Services} might be subdivided into \.{Indirect} \.{Services} and
\.{Direct} \.{Services}.

If \PB{$\\{seed}\K\T{0}$}, the binary tree is pruned in such a way that the %
\PB{\|n}
resulting sectors are as equal as possible with respect to total
input and output, while respecting the tree structure. If \PB{$\\{seed}>%
\T{0}$},
the pruning is carried out at random, in such a way that all \PB{\|n}-leaf
subtrees of the original tree are obtained with approximately equal
probability (depending on \PB{\\{seed}} in a machine-independent fashion).
Any \PB{\\{seed}} value from 1 to $2^{31}-1=2147483647$ is permissible.

As usual in GraphBase routines, you can set \PB{$\|n\K\T{0}$} to get the
default
situation where \PB{\|n} has its maximum value. For example, either
\PB{$\\{econ}(\T{0},\T{0},\T{0},\T{0})$} or \PB{$\\{econ}(\T{81},\T{0},\T{0},%
\T{0})$} produces the full graph;
\PB{$\\{econ}(\T{0},\T{2},\T{0},\T{0})$} or \PB{$\\{econ}(\T{79},\T{2},\T{0},%
\T{0})$} produces the full graph except
for the two special vertices.

\Y\B\4\D$\.{MAX\_N}$ \5
\T{81}\C{ maximum number of vertices in constructed graph }\par
\B\4\D$\.{NORM\_N}$ \5
$\.{MAX\_N}-{}$\T{2}\C{ the number of normal SIC sectors }\par
\B\4\D$\.{ADJ\_SEC}$ \5
$\.{MAX\_N}-{}$\T{1}\C{ code number for the \.{Adjustments} sector }\par
\fi

\M{4}The U.S. Bureau of Economic Analysis and the U.S. Bureau of the Census
have
assigned code numbers 1--79 to the individual sectors for which
statistics are given in \.{econ.dat}. These sector numbers are
traditionally called Standard Industrial Classification (SIC) codes.
If for some reason you want to know the SIC codes for
all sectors represented by vertex \PB{\|v} of a graph generated by \PB{%
\\{econ}},
you can access them via a list of \PB{\&{Arc}} nodes starting at the utility
field \PB{$\|v\MG\\{SIC\_codes}$}.
This list is linked by \PB{\\{next}} fields in the usual way, and each
SIC code appears in the \PB{\\{len}} field; the \PB{\\{tip}} field is unused.

The special vertex \.{Adjustments} is given code number~80; it is
actually a composite of six different SIC categories, numbered 80--86 in their
published tables.

For example, if \PB{$\|n\K\T{80}$} and \PB{$\\{omit}\K\T{1}$}, each list will
have length~1.
Hence \PB{$\|v\MG\\{SIC\_codes}\MG\\{next}$} will equal \PB{$\NULL$} for each~%
\PB{\|v}, and
\PB{$\|v\MG\\{SIC\_codes}\MG\\{len}$} will be \PB{\|v}'s SIC code, a number
between 1 and~80.

The special vertex \.{Users} has no SIC code; it is the only vertex
whose \PB{\\{SIC\_codes}} field will be null in the graph returned by \PB{%
\\{econ}}.

\Y\B\4\D$\\{SIC\_codes}$ \5
$\|z.{}$\|A\C{ utility field \PB{\|z} leads to the SIC codes for a vertex }\par
\fi

\M{5}The total output of each sector, which also equals the total input of that
sector, is placed in utility field \PB{\\{sector\_total}} of the corresponding
vertex.

\Y\B\4\D$\\{sector\_total}$ \5
$\|y.{}$\|I\C{ utility field \PB{\|y} holds the total flow in and out }\par
\Y\B\4\X1:\.{gb\_econ.h\,}\X${}\mathrel+\E{}$\6
\8\#\&{define} ${}\\{flow}\hbox{\quad}\|a.\|I{}$\C{ definitions of utility
fields in the header file }\6
\8\#\&{define} ${}\\{SIC\_codes}\hbox{\quad}\|z.\|A{}$\6
\8\#\&{define} ${}\\{sector\_total}\hbox{\quad}\|y.\|I{}$\par
\fi

\M{6}If the \PB{\\{econ}} routine encounters a problem, it returns \PB{$\NULL$}
(\.{NULL}), after putting a nonzero number into the external variable
\PB{\\{panic\_code}}. This code number identifies the type of failure.
Otherwise \PB{\\{econ}} returns a pointer to the newly created graph, which
will be represented with the data structures explained in {\sc GB\_\,GRAPH}.
(The external variable \PB{\\{panic\_code}} is itself defined in
{\sc GB\_\,GRAPH}.)

\Y\B\4\D$\\{panic}(\|c)$ \5
${}\{{}$\5
\1${}\\{panic\_code}\K\|c{}$;\5
${}\\{gb\_trouble\_code}\K\T{0}{}$;\5
\&{return} ${}\NULL{}$;\5
${}\}{}$\2\par
\fi

\M{7}The \CEE/ file \.{gb\_econ.c} has the following overall shape:

\Y\B\8\#\&{include} \.{"gb\_io.h"}\C{ we will use the {\sc GB\_\,IO} routines
for input }\6
\8\#\&{include} \.{"gb\_flip.h"}\C{ we will use the {\sc GB\_\,FLIP} routines
for random numbers }\6
\8\#\&{include} \.{"gb\_graph.h"}\C{ and of course we'll use the {\sc GB\_%
\,GRAPH} data structures }\6
\ATH\7
\X11:Type declarations\X\6
\X12:Private variables\X\7
\1\1\&{Graph} ${}{*}\\{econ}(\|n,\39\\{omit},\39\\{threshold},\39\\{seed}){}$\6
\&{unsigned} \&{long} \|n;\C{ number of vertices desired }\6
\&{unsigned} \&{long} \\{omit};\C{ number of special vertices to omit }\6
\&{unsigned} \&{long} \\{threshold};\C{ minimum per-64K-age in arcs leading in
}\6
\&{long} \\{seed};\C{ random number seed }\2\2\6
${}\{{}$\5
\1\X8:Local variables\X\7
\\{gb\_init\_rand}(\\{seed});\6
\\{init\_area}(\\{working\_storage});\6
\X9:Check the parameters and adjust them for defaults\X;\6
\X10:Set up a graph with \PB{\|n} vertices\X;\6
\X14:Read \.{econ.dat} and note the binary tree structure\X;\6
\X17:Determine the \PB{\|n} sectors to use in the graph\X;\6
\X25:Put the appropriate arcs into the graph\X;\6
\&{if} ${}(\\{gb\_close}(\,)\I\T{0}){}$\1\5
\\{panic}(\\{late\_data\_fault});\C{ something's wrong with \PB{%
\.{"econ.dat"}}; see \PB{\\{io\_errors}} }\2\6
\\{gb\_free}(\\{working\_storage});\6
\&{if} (\\{gb\_trouble\_code})\5
${}\{{}$\1\6
\\{gb\_recycle}(\\{new\_graph});\6
\\{panic}(\\{alloc\_fault});\C{ oops, we ran out of memory somewhere back there
}\6
\4${}\}{}$\2\6
\&{return} \\{new\_graph};\6
\4${}\}{}$\2\par
\fi

\M{8}\B\X8:Local variables\X${}\E{}$\6
\&{Graph} ${}{*}\\{new\_graph}{}$;\C{ the graph constructed by \PB{\\{econ}} }\6
\&{register} \&{long} \|j${},\39\|k{}$;\C{ all-purpose indices }\6
\&{Area} \\{working\_storage};\C{ tables needed while \PB{\\{econ}} does its
thinking }\par
\A13.
\U7.\fi

\M{9}\B\X9:Check the parameters and adjust them for defaults\X${}\E{}$\6
\&{if} ${}(\\{omit}>\T{2}){}$\1\5
${}\\{omit}\K\T{2};{}$\2\6
\&{if} ${}(\|n\E\T{0}\V\|n>\.{MAX\_N}-\\{omit}){}$\1\5
${}\|n\K\.{MAX\_N}-\\{omit};{}$\2\6
\&{else} \&{if} ${}(\|n+\\{omit}<\T{3}){}$\1\5
${}\\{omit}\K\T{3}-\|n{}$;\C{ we need at least one normal sector }\2\6
\&{if} ${}(\\{threshold}>\T{65536}){}$\1\5
${}\\{threshold}\K\T{65536}{}$;\2\par
\U7.\fi

\M{10}\B\X10:Set up a graph with \PB{\|n} vertices\X${}\E{}$\6
$\\{new\_graph}\K\\{gb\_new\_graph}(\|n);{}$\6
\&{if} ${}(\\{new\_graph}\E\NULL){}$\1\5
\\{panic}(\\{no\_room});\C{ out of memory before we're even started }\2\6
${}\\{sprintf}(\\{new\_graph}\MG\\{id},\39\.{"econ(\%lu,\%lu,\%lu,\%l}\)%
\.{d)"},\39\|n,\39\\{omit},\39\\{threshold},\39\\{seed});{}$\6
${}\\{strcpy}(\\{new\_graph}\MG\\{util\_types},\39\.{"ZZZZIAIZZZZZZZ"}){}$;\par
\U7.\fi

\N{1}{11}The economic tree.
As we read in the data, we construct a sequential list of nodes,
each of which represents either a micro-sector of the economy (one of
the basic SIC sectors) or a macro-sector (which is the union of two subnodes).
In more technical terms, the nodes form an extended binary tree,
whose external nodes correspond to micro-sectors and whose internal nodes
correspond to macro-sectors. The nodes of the tree appear in preorder.
Subsequently we will do a variety of operations on this binary tree,
proceeding either top-down (from the beginning of the list to the end)
or bottom-up (from the end to the beginning).

Each node is a rather large record, because we will store a complete
vector of sector output data in each node.

\Y\B\4\X11:Type declarations\X${}\E{}$\6
\&{typedef} \&{struct} \&{node\_struct} ${}\{{}$\C{ records for micro- and
macro-sectors }\1\6
\&{struct} \&{node\_struct} ${}{*}\\{rchild}{}$;\C{ pointer to right child of
macro-sector }\6
\&{char} \\{title}[\T{44}];\C{ \PB{\.{"Sector\ name"}} }\6
\&{long} ${}\\{table}[\.{MAX\_N}+\T{1}]{}$;\C{ outputs from this sector }\6
\&{unsigned} \&{long} \\{total};\C{ total input to this sector ($=$ total
output) }\6
\&{long} \\{thresh};\C{ \PB{\\{flow}} must exceed \PB{\\{thresh}} in arcs to
this sector }\6
\&{long} \.{SIC};\C{ SIC code number; initially zero in macro-sectors }\6
\&{long} \\{tag};\C{ 1 if this node will be a vertex in the graph }\6
\&{struct} \&{node\_struct} ${}{*}\\{link}{}$;\C{ next smallest unexplored
sector }\6
\&{Arc} ${}{*}\\{SIC\_list}{}$;\C{ first item on list of SIC codes }\2\6
${}\}{}$ \&{node};\par
\U7.\fi

\M{12}When we read the given data in preorder, we'll need a stack to remember
what nodes still need to have their \PB{\\{rchild}} pointer filled in.
(There is a no need for an \\{lchild} pointer, because the left child
always follows its parent immediately in preorder.)

\Y\B\4\X12:Private variables\X${}\E{}$\6
\&{static} \&{node} ${}{*}\\{stack}[\.{NORM\_N}+\.{NORM\_N}];{}$\6
\&{static} \&{node} ${}{*}{*}\\{stack\_ptr}{}$;\C{ current position in \PB{%
\\{stack}} }\6
\&{static} \&{node} ${}{*}\\{node\_block}{}$;\C{ array of nodes, specifies the
tree in preorder }\6
\&{static} \&{node} ${}{*}\\{node\_index}[\.{MAX\_N}+\T{1}]{}$;\C{ which node
has a given SIC code }\par
\A26.
\U7.\fi

\M{13}\B\X8:Local variables\X${}\mathrel+\E{}$\6
\&{register} \&{node} ${}{*}\|p,\39{*}\\{pl},\39{*}\\{pr}{}$;\C{ current node
and its children }\6
\&{register} \&{node} ${}{*}\|q{}$;\C{ register for list manipulation }\par
\fi

\M{14}\B\X14:Read \.{econ.dat} and note the binary tree structure\X${}\E{}$\6
$\\{node\_block}\K\\{gb\_typed\_alloc}(\T{2}*\.{MAX\_N}-\T{3},\39\&{node},\39%
\\{working\_storage});{}$\6
\&{if} (\\{gb\_trouble\_code})\1\5
${}\\{panic}(\\{no\_room}+\T{1}){}$;\C{ no room to copy the data }\2\6
\&{if} ${}(\\{gb\_open}(\.{"econ.dat"})\I\T{0}){}$\1\5
\\{panic}(\\{early\_data\_fault});\C{ couldn't open \PB{\.{"econ.dat"}} using
GraphBase conventions }\2\6
\X15:Read and store the sector names and SIC numbers\X;\6
\&{for} ${}(\|k\K\T{1};{}$ ${}\|k\Z\.{MAX\_N};{}$ ${}\|k\PP){}$\1\5
\X16:Read and store the output coefficients for sector \PB{\|k}\X;\2\par
\U7.\fi

\M{15}The first part of \.{econ.dat} specifies the nodes of the binary
tree in preorder. Each line contains a node name
followed by a colon, and the colon is followed by the SIC number if
that node is a leaf.

The tree is uniquely specified in this way,
because of the nature of preorder. (Think of Polish prefix notation,
in which a formula like `${+}x{+}xx$' means `${+}(x,{+}(x,x))$'; the
parentheses in Polish notation are redundant.)

The two special sector names don't appear in the file; we manufacture
them ourselves.

The program here is careful not to clobber itself in the
presence of arbitrarily garbled data.

\Y\B\4\X15:Read and store the sector names and SIC numbers\X${}\E{}$\6
$\\{stack\_ptr}\K\\{stack};{}$\6
\&{for} ${}(\|p\K\\{node\_block};{}$ ${}\|p<\\{node\_block}+\.{NORM\_N}+\.{NORM%
\_N}-\T{1};{}$ ${}\|p\PP){}$\5
${}\{{}$\5
\1\&{register} \&{long} \|c;\7
${}\\{gb\_string}(\|p\MG\\{title},\39\.{':'});{}$\6
\&{if} ${}(\\{strlen}(\|p\MG\\{title})>\T{43}){}$\1\5
\\{panic}(\\{syntax\_error});\C{ sector name too long }\2\6
\&{if} ${}(\\{gb\_char}(\,)\I\.{':'}){}$\1\5
${}\\{panic}(\\{syntax\_error}+\T{1}){}$;\C{ missing colon }\2\6
${}\|p\MG\.{SIC}\K\|c\K\\{gb\_number}(\T{10});{}$\6
\&{if} ${}(\|c\E\T{0}{}$)\C{ macro-sector }\1\6
${}{*}\\{stack\_ptr}\PP\K\|p{}$;\C{ left child is \PB{$\|p+\T{1}$}, we'll know %
\PB{\\{rchild}} later }\2\6
\&{else}\5
${}\{{}$\C{ micro-sector; \PB{$\|p+\T{1}$} will be somebody's right child }\1\6
${}\\{node\_index}[\|c]\K\|p;{}$\6
\&{if} ${}(\\{stack\_ptr}>\\{stack}){}$\1\5
${}({*}\MM\\{stack\_ptr})\MG\\{rchild}\K\|p+\T{1};{}$\2\6
\4${}\}{}$\2\6
\&{if} ${}(\\{gb\_char}(\,)\I\.{'\\n'}){}$\1\5
${}\\{panic}(\\{syntax\_error}+\T{2}){}$;\C{ garbage on the line }\2\6
\\{gb\_newline}(\,);\6
\4${}\}{}$\2\6
\&{if} ${}(\\{stack\_ptr}\I\\{stack}){}$\1\5
${}\\{panic}(\\{syntax\_error}+\T{3}){}$;\C{ tree malformed }\2\6
\&{for} ${}(\|k\K\.{NORM\_N};{}$ \|k; ${}\|k\MM){}$\1\6
\&{if} ${}(\\{node\_index}[\|k]\E\T{0}){}$\1\5
${}\\{panic}(\\{syntax\_error}+\T{4}){}$;\C{ SIC code not mentioned in the tree
}\2\2\6
${}\\{strcpy}(\|p\MG\\{title},\39\.{"Adjustments"}){}$;\5
${}\|p\MG\.{SIC}\K\.{ADJ\_SEC}{}$;\5
${}\\{node\_index}[\.{ADJ\_SEC}]\K\|p;{}$\6
${}\\{strcpy}((\|p+\T{1})\MG\\{title},\39\.{"Users"}){}$;\5
${}\\{node\_index}[\.{MAX\_N}]\K\|p+\T{1}{}$;\par
\U14.\fi

\M{16}The remaining part of \.{econ.dat} is an $81\times80$ matrix in which
the $k$th row contains the outputs of sector~$k$ to all sectors except
\.{Users}. Each row consists of a blank line followed by 8 data lines;
each data line contains 10 numbers separated by commas.
Zeroes are represented by \PB{\.{""}} instead of by \PB{\.{"0"}}.
For example, the data line $$\hbox{\tt
8490,2182,42,467,,,,,,}$$ follows the initial blank line; it means
that sector~1 output 8490 million dollars to itself, \$2182M to
sector~2, \dots, \$0M to sector~10.

\Y\B\4\X16:Read and store the output coefficients for sector \PB{\|k}\X${}\E{}$%
\6
${}\{{}$\5
\1\&{register} \&{long} \|s${}\K\T{0}{}$;\C{ row sum }\6
\&{register} \&{long} \|x;\C{ entry read from \.{econ.dat} }\7
\&{if} ${}(\\{gb\_char}(\,)\I\.{'\\n'}){}$\1\5
${}\\{panic}(\\{syntax\_error}+\T{5}){}$;\C{ blank line missing between rows }%
\2\6
\\{gb\_newline}(\,);\6
${}\|p\K\\{node\_index}[\|k];{}$\6
\&{for} ${}(\|j\K\T{1};{}$ ${}\|j<\.{MAX\_N};{}$ ${}\|j\PP){}$\5
${}\{{}$\1\6
${}\|p\MG\\{table}[\|j]\K\|x\K\\{gb\_number}(\T{10}){}$;\5
${}\|s\MRL{+{\K}}\|x;{}$\6
${}\\{node\_index}[\|j]\MG\\{total}\MRL{+{\K}}\|x;{}$\6
\&{if} ${}((\|j\MOD\T{10})\E\T{0}){}$\5
${}\{{}$\1\6
\&{if} ${}(\\{gb\_char}(\,)\I\.{'\\n'}){}$\1\5
${}\\{panic}(\\{syntax\_error}+\T{6}){}$;\C{ out of synch in input file }\2\6
\\{gb\_newline}(\,);\6
\4${}\}{}$\5
\2\&{else} \&{if} ${}(\\{gb\_char}(\,)\I\.{','}){}$\1\5
${}\\{panic}(\\{syntax\_error}+\T{7}){}$;\C{ missing comma after entry }\2\6
\4${}\}{}$\2\6
${}\|p\MG\\{table}[\.{MAX\_N}]\K\|s{}$;\C{ sum of \PB{\\{table}[\T{1}]} through
\PB{\\{table}[\T{80}]} }\6
\4${}\}{}$\2\par
\U14.\fi

\N{1}{17}Growing a subtree.
Once all the data appears in \PB{\\{node\_block}}, we want to extract from it
and combine~it as specified by parameters \PB{\|n}, \PB{\\{omit}}, and \PB{%
\\{seed}}.
This amalgamation process effectively prunes the tree; it can also be
regarded as a procedure that grows a subtree of the full economic tree.

\Y\B\4\X17:Determine the \PB{\|n} sectors to use in the graph\X${}\E{}$\6
${}\{{}$\5
\1\&{long} \|l${}\K\|n+\\{omit}-\T{2}{}$;\C{ the number of leaves in the
desired subtree }\7
\&{if} ${}(\|l\E\.{NORM\_N}){}$\1\5
\X18:Choose all sectors\X\2\6
\&{else} \&{if} (\\{seed})\1\5
\X21:Grow a random subtree with \PB{\|l} leaves\X\2\6
\&{else}\1\5
\X19:Grow a subtree with \PB{\|l} leaves by subdividing largest sectors first%
\X;\2\6
\4${}\}{}$\2\par
\U7.\fi

\M{18}The chosen leaves of our subtree are identified by having their
\PB{\\{tag}} field set to~1.

\Y\B\4\X18:Choose all sectors\X${}\E{}$\6
\&{for} ${}(\|k\K\.{NORM\_N};{}$ \|k; ${}\|k\MM){}$\1\5
${}\\{node\_index}[\|k]\MG\\{tag}\K\T{1}{}$;\2\par
\U17.\fi

\M{19}To grow the \PB{\|l}-leaf subtree when \PB{$\\{seed}\K\T{0}$}, we first
pass over the
tree bottom-up to compute the total input (and output) of each macro-sector;
then we proceed from the top down to subdivide sectors in decreasing
order of their total input. This process provides a good introduction to the
bottom-up and top-down tree methods we will be using in several other
parts of the program.

The \PB{\\{special}} node is used here for two purposes: It is the head of a
linked list of unexplored nodes, sorted by decreasing order of
their \PB{\\{total}} fields; and it appears at the end of that list, because
\PB{$\\{special}\MG\\{total}\K\T{0}$}.

\Y\B\4\X19:Grow a subtree with \PB{\|l} leaves by subdividing largest sectors
first\X${}\E{}$\6
${}\{{}$\5
\1\&{register} \&{node} ${}{*}\\{special}\K\\{node\_index}[\.{MAX\_N}]{}$;\C{
the \.{Users} node at the end of \PB{\\{node\_block}} }\7
\&{for} ${}(\|p\K\\{node\_index}[\.{ADJ\_SEC}]-\T{1};{}$ ${}\|p\G\\{node%
\_block};{}$ ${}\|p\MM{}$)\C{ bottom up }\1\6
\&{if} ${}(\|p\MG\\{rchild}){}$\1\5
${}\|p\MG\\{total}\K(\|p+\T{1})\MG\\{total}+\|p\MG\\{rchild}\MG\\{total};{}$\2%
\2\6
${}\\{special}\MG\\{link}\K\\{node\_block}{}$;\5
${}\\{node\_block}\MG\\{link}\K\\{special}{}$;\C{ start at the root }\6
${}\|k\K\T{1}{}$;\C{ \PB{\|k} is the number of nodes we have tagged or put onto
the list }\6
\&{while} ${}(\|k<\|l){}$\1\5
\X20:If the first node on the list is a leaf, delete it and tag it; otherwise
replace it by its two children\X;\2\6
\&{for} ${}(\|p\K\\{special}\MG\\{link};{}$ ${}\|p\I\\{special};{}$ ${}\|p\K\|p%
\MG\\{link}){}$\1\5
${}\|p\MG\\{tag}\K\T{1}{}$;\C{ tag everything on the list }\2\6
\4${}\}{}$\2\par
\U17.\fi

\M{20}\B\X20:If the first node on the list is a leaf, delete it and tag it;
otherwise replace it by its two children\X${}\E{}$\6
${}\{{}$\1\6
${}\|p\K\\{special}\MG\\{link}{}$;\C{ remove \PB{\|p}, the node with greatest %
\PB{\\{total}} }\6
${}\\{special}\MG\\{link}\K\|p\MG\\{link};{}$\6
\&{if} ${}(\|p\MG\\{rchild}\E\T{0}){}$\1\5
${}\|p\MG\\{tag}\K\T{1}{}$;\C{ \PB{\|p} is a leaf }\2\6
\&{else}\5
${}\{{}$\1\6
${}\\{pl}\K\|p+\T{1}{}$;\5
${}\\{pr}\K\|p\MG\\{rchild};{}$\6
\&{for} ${}(\|q\K\\{special};{}$ ${}\|q\MG\\{link}\MG\\{total}>\\{pl}\MG%
\\{total};{}$ ${}\|q\K\|q\MG\\{link}){}$\1\5
;\2\6
${}\\{pl}\MG\\{link}\K\|q\MG\\{link}{}$;\5
${}\|q\MG\\{link}\K\\{pl}{}$;\C{ insert left child in its proper place }\6
\&{for} ${}(\|q\K\\{special};{}$ ${}\|q\MG\\{link}\MG\\{total}>\\{pr}\MG%
\\{total};{}$ ${}\|q\K\|q\MG\\{link}){}$\1\5
;\2\6
${}\\{pr}\MG\\{link}\K\|q\MG\\{link}{}$;\5
${}\|q\MG\\{link}\K\\{pr}{}$;\C{ insert right child in its proper place }\6
${}\|k\PP;{}$\6
\4${}\}{}$\2\6
\4${}\}{}$\2\par
\U19.\fi

\M{21}We can obtain a uniformly distributed \PB{\|l}-leaf subtree of a given
tree
by choosing the root when \PB{$\|l\K\T{1}$} or by using the following idea when
\PB{$\|l>\T{1}$}:
Suppose the given tree~$T$ has subtrees $T_0$ and $T_1$. Then it has
$T(l)$ subtrees with \PB{\|l}~leaves, where $T(l)=\sum_k T_0(k)T_1(l-k)$.
We choose a random number $r$ between 0 and $T(l)-1$, and we find the
smallest $m$ such that $\sum_{k\le m}T_0(k)T_1(l-k)>r$. Then we
proceed recursively to
compute a random $m$-leaf subtree of~$T_0$ and a random $(l-m)$-leaf
subtree of~$T_1$.

A difficulty arises when $T(l)$ is $2^{31}$ or more. But then we can replace
$T_0(k)$ and $T_1(l-k)$ in the formulas above by $\lceil T_0(k)/d_0\rceil$
and $\lceil T_1(k)/d_1\rceil$, respectively, where $d_0$ and $d_1$ are
arbitrary constants; this yields smaller values
$T(l)$ that define approximately the same distribution of~$k$.

The program here computes the $T(l)$ values bottom-up, then grows a
random tree top-down. If node~\PB{\|p} is not a leaf, its \PB{\\{table}[\T{0}]}
field
will be set to the number of leaves below it; and its \PB{\\{table}[\|l]} field
will be set to $T(l)$, for \PB{$\T{1}\Z\|l\Z\\{table}[\T{0}]$}.

The data in \.{econ.dat} is sufficiently simple that most of the $T(l)$
values are less than $2^{31}$. We need to scale them
down to avoid overflow only at the root node of the tree; this
case is handled separately.

We set the \PB{\\{tag}} field of a node equal to the number of leaves to be
grown in the subtree rooted at that node. This convention is consistent
with our previous stipulation that \PB{$\\{tag}\K\T{1}$} should characterize
the
nodes that are chosen to be vertices.

\Y\B\4\X21:Grow a random subtree with \PB{\|l} leaves\X${}\E{}$\6
${}\{{}$\1\6
${}\\{node\_block}\MG\\{tag}\K\|l;{}$\6
\&{for} ${}(\|p\K\\{node\_index}[\.{ADJ\_SEC}]-\T{1};{}$ ${}\|p>\\{node%
\_block};{}$ ${}\|p\MM{}$)\C{ bottom up, except root }\1\6
\&{if} ${}(\|p\MG\\{rchild}){}$\1\5
\X22:Compute the $T(l)$ values for subtree \PB{\|p}\X;\2\2\6
\&{for} ${}(\|p\K\\{node\_block};{}$ ${}\|p<\\{node\_index}[\.{ADJ\_SEC}];{}$
${}\|p\PP{}$)\C{ top down, from root }\1\6
\&{if} ${}(\|p\MG\\{tag}>\T{1}){}$\5
${}\{{}$\1\6
${}\|l\K\|p\MG\\{tag};{}$\6
${}\\{pl}\K\|p+\T{1}{}$;\5
${}\\{pr}\K\|p\MG\\{rchild};{}$\6
\&{if} ${}(\\{pl}\MG\\{rchild}\E\NULL){}$\5
${}\{{}$\1\6
${}\\{pl}\MG\\{tag}\K\T{1}{}$;\5
${}\\{pr}\MG\\{tag}\K\|l-\T{1};{}$\6
\4${}\}{}$\5
\2\&{else} \&{if} ${}(\\{pr}\MG\\{rchild}\E\NULL){}$\5
${}\{{}$\1\6
${}\\{pl}\MG\\{tag}\K\|l-\T{1}{}$;\5
${}\\{pr}\MG\\{tag}\K\T{1};{}$\6
\4${}\}{}$\5
\2\&{else}\1\5
\X24:Stochastically determine the number of leaves to grow in each of \PB{%
\|p}'s children\X;\2\6
\4${}\}{}$\2\2\6
\4${}\}{}$\2\par
\U17.\fi

\M{22}Here we are essentially multiplying two generating functions.
Suppose $f(z)=\sum_l T(l)z^l$; then we are computing $f_p(z)=
z+f_{pl}(z)f_{pr}(z)$.

\Y\B\4\X22:Compute the $T(l)$ values for subtree \PB{\|p}\X${}\E{}$\6
${}\{{}$\1\6
${}\\{pl}\K\|p+\T{1}{}$;\5
${}\\{pr}\K\|p\MG\\{rchild};{}$\6
${}\|p\MG\\{table}[\T{1}]\K\|p\MG\\{table}[\T{2}]\K\T{1}{}$;\C{ $T(1)$ and
$T(2)$ are always 1 }\6
\&{if} ${}(\\{pl}\MG\\{rchild}\E\T{0}){}$\5
${}\{{}$\C{ left child is a leaf }\1\6
\&{if} ${}(\\{pr}\MG\\{rchild}\E\T{0}){}$\1\5
${}\|p\MG\\{table}[\T{0}]\K\T{2}{}$;\C{ and so is the right child }\2\6
\&{else}\5
${}\{{}$\C{ no, it isn't }\1\6
\&{for} ${}(\|k\K\T{2};{}$ ${}\|k\Z\\{pr}\MG\\{table}[\T{0}];{}$ ${}\|k\PP){}$%
\1\5
${}\|p\MG\\{table}[\T{1}+\|k]\K\\{pr}\MG\\{table}[\|k];{}$\2\6
${}\|p\MG\\{table}[\T{0}]\K\\{pr}\MG\\{table}[\T{0}]+\T{1};{}$\6
\4${}\}{}$\2\6
\4${}\}{}$\5
\2\&{else} \&{if} ${}(\\{pr}\MG\\{rchild}\E\T{0}){}$\5
${}\{{}$\C{ right child is a leaf }\1\6
\&{for} ${}(\|k\K\T{2};{}$ ${}\|k\Z\\{pl}\MG\\{table}[\T{0}];{}$ ${}\|k\PP){}$%
\1\5
${}\|p\MG\\{table}[\T{1}+\|k]\K\\{pl}\MG\\{table}[\|k];{}$\2\6
${}\|p\MG\\{table}[\T{0}]\K\\{pl}\MG\\{table}[\T{0}]+\T{1};{}$\6
\4${}\}{}$\5
\2\&{else}\5
${}\{{}$\C{ neither child is a leaf }\1\6
\X23:Set \PB{$\|p\MG\\{table}[\T{2}]$}, \PB{$\|p\MG\\{table}[\T{3}]$}, \dots\
to convolution of \PB{\\{pl}} and \PB{\\{pr}} table entries\X;\6
${}\|p\MG\\{table}[\T{0}]\K\\{pl}\MG\\{table}[\T{0}]+\\{pr}\MG\\{table}[%
\T{0}];{}$\6
\4${}\}{}$\2\6
\4${}\}{}$\2\par
\U21.\fi

\M{23}\B\X23:Set \PB{$\|p\MG\\{table}[\T{2}]$}, \PB{$\|p\MG\\{table}[\T{3}]$}, %
\dots\ to convolution of \PB{\\{pl}} and \PB{\\{pr}} table entries\X${}\E{}$\6
$\|p\MG\\{table}[\T{2}]\K\T{0};{}$\6
\&{for} ${}(\|j\K\\{pl}\MG\\{table}[\T{0}];{}$ \|j; ${}\|j\MM){}$\5
${}\{{}$\5
\1\&{register} \&{long} \|t${}\K\\{pl}\MG\\{table}[\|j];{}$\7
\&{for} ${}(\|k\K\\{pr}\MG\\{table}[\T{0}];{}$ \|k; ${}\|k\MM){}$\1\5
${}\|p\MG\\{table}[\|j+\|k]\MRL{+{\K}}\|t*\\{pr}\MG\\{table}[\|k];{}$\2\6
\4${}\}{}$\2\par
\U22.\fi

\M{24}\B\X24:Stochastically determine the number of leaves to grow in each of %
\PB{\|p}'s children\X${}\E{}$\6
${}\{{}$\5
\1\&{register} \&{long} \\{ss}${},\39\\{rr};{}$\7
${}\|j\K\T{0}{}$;\C{ we will set \PB{$\|j\K\T{1}$} if scaling is necessary at
the root }\6
\&{if} ${}(\|p\E\\{node\_block}){}$\5
${}\{{}$\1\6
${}\\{ss}\K\T{0};{}$\6
\&{if} ${}(\|l>\T{29}\W\|l<\T{67}){}$\5
${}\{{}$\1\6
${}\|j\K\T{1}{}$;\C{ more than $2^{31}$ possibilities exist }\6
\&{for} ${}(\|k\K(\|l>\\{pr}\MG\\{table}[\T{0}]\?\|l-\\{pr}\MG\\{table}[\T{0}]:%
\T{1});{}$ ${}\|k\Z\\{pl}\MG\\{table}[\T{0}]\W\|k<\|l;{}$ ${}\|k\PP){}$\1\5
${}\\{ss}\MRL{+{\K}}((\\{pl}\MG\\{table}[\|k]+\T{\^3ff})\GG\T{10})*\\{pr}\MG%
\\{table}[\|l-\|k]{}$;\C{ scale with $d_0=1024$, $d_1=1$ }\2\6
\4${}\}{}$\5
\2\&{else}\1\6
\&{for} ${}(\|k\K(\|l>\\{pr}\MG\\{table}[\T{0}]\?\|l-\\{pr}\MG\\{table}[\T{0}]:%
\T{1});{}$ ${}\|k\Z\\{pl}\MG\\{table}[\T{0}]\W\|k<\|l;{}$ ${}\|k\PP){}$\1\5
${}\\{ss}\MRL{+{\K}}\\{pl}\MG\\{table}[\|k]*\\{pr}\MG\\{table}[\|l-\|k];{}$\2\2%
\6
\4${}\}{}$\5
\2\&{else}\1\5
${}\\{ss}\K\|p\MG\\{table}[\|l];{}$\2\6
${}\\{rr}\K\\{gb\_unif\_rand}(\\{ss});{}$\6
\&{if} (\|j)\1\6
\&{for} ${}(\\{ss}\K\T{0},\39\|k\K(\|l>\\{pr}\MG\\{table}[\T{0}]\?\|l-\\{pr}\MG%
\\{table}[\T{0}]:\T{1});{}$ ${}\\{ss}\Z\\{rr};{}$ ${}\|k\PP){}$\1\5
${}\\{ss}\MRL{+{\K}}((\\{pl}\MG\\{table}[\|k]+\T{\^3ff})\GG\T{10})*\\{pr}\MG%
\\{table}[\|l-\|k];{}$\2\2\6
\&{else}\1\6
\&{for} ${}(\\{ss}\K\T{0},\39\|k\K(\|l>\\{pr}\MG\\{table}[\T{0}]\?\|l-\\{pr}\MG%
\\{table}[\T{0}]:\T{1});{}$ ${}\\{ss}\Z\\{rr};{}$ ${}\|k\PP){}$\1\5
${}\\{ss}\MRL{+{\K}}\\{pl}\MG\\{table}[\|k]*\\{pr}\MG\\{table}[\|l-\|k];{}$\2\2%
\6
${}\\{pl}\MG\\{tag}\K\|k-\T{1}{}$;\5
${}\\{pr}\MG\\{tag}\K\|l-\|k+\T{1};{}$\6
\4${}\}{}$\2\par
\U21.\fi

\N{1}{25}Arcs.
In the general case, we have to combine some of the basic micro-sectors
into macro-sectors by adding together the appropriate input/output
coefficients. This is a bottom-up pruning process.

Suppose \PB{\|p} is being formed as the union of \PB{\\{pl}} and~\PB{\\{pr}}.
Then the arcs leading out of \PB{\|p} are obtained by summing the numbers
on arcs leading out of \PB{\\{pl}} and~\PB{\\{pr}}; the arcs leading into \PB{%
\|p} are
obtained by summing the numbers on arcs leading into \PB{\\{pl}} and~\PB{%
\\{pr}};
the arcs from \PB{\|p} to itself are obtained by summing the four numbers
on arcs leading from \PB{\\{pl}} or~\PB{\\{pr}} to \PB{\\{pl}} or~\PB{\\{pr}}.

We maintain the \PB{\\{node\_index}} table so that its non-\PB{$\NULL$} entries
contain all the currently active nodes. When \PB{\\{pl}} and~\PB{\\{pr}} are
being pruned in favor of~\PB{\|p}, node \PB{\|p}~inherits \PB{\\{pl}}'s place
in
\PB{\\{node\_index}}; \PB{\\{pr}}'s former place becomes~\PB{$\NULL$}.

\Y\B\4\X25:Put the appropriate arcs into the graph\X${}\E{}$\6
\X28:Prune the sectors that are used in macro-sectors, and form the lists of
SIC sector codes\X;\6
\X30:Make the special nodes invisible if they are omitted, visible otherwise\X;%
\6
\X27:Compute individual thresholds for each chosen sector\X;\6
${}\{{}$\5
\1\&{register} \&{Vertex} ${}{*}\|v\K\\{new\_graph}\MG\\{vertices}+\|n;{}$\7
\&{for} ${}(\|k\K\.{MAX\_N};{}$ \|k; ${}\|k\MM){}$\1\6
\&{if} ${}((\|p\K\\{node\_index}[\|k])\I\NULL){}$\5
${}\{{}$\1\6
${}\\{vert\_index}[\|k]\K\MM\|v;{}$\6
${}\|v\MG\\{name}\K\\{gb\_save\_string}(\|p\MG\\{title});{}$\6
${}\|v\MG\\{SIC\_codes}\K\|p\MG\\{SIC\_list};{}$\6
${}\|v\MG\\{sector\_total}\K\|p\MG\\{total};{}$\6
\4${}\}{}$\5
\2\&{else}\1\5
${}\\{vert\_index}[\|k]\K\NULL;{}$\2\2\6
\&{if} ${}(\|v\I\\{new\_graph}\MG\\{vertices}){}$\1\5
\\{panic}(\\{impossible});\C{ bug in algorithm; this can't happen }\2\6
\&{for} ${}(\|j\K\.{MAX\_N};{}$ \|j; ${}\|j\MM){}$\1\6
\&{if} ${}((\|p\K\\{node\_index}[\|j])\I\NULL){}$\5
${}\{{}$\5
\1\&{register} \&{Vertex} ${}{*}\|u\K\\{vert\_index}[\|j];{}$\7
\&{for} ${}(\|k\K\.{MAX\_N};{}$ \|k; ${}\|k\MM){}$\1\6
\&{if} ${}((\|v\K\\{vert\_index}[\|k])\I\NULL){}$\1\6
\&{if} ${}(\|p\MG\\{table}[\|k]\I\T{0}\W\|p\MG\\{table}[\|k]>\\{node\_index}[%
\|k]\MG\\{thresh}){}$\5
${}\{{}$\1\6
${}\\{gb\_new\_arc}(\|u,\39\|v,\39\T{1\$L});{}$\6
${}\|u\MG\\{arcs}\MG\\{flow}\K\|p\MG\\{table}[\|k];{}$\6
\4${}\}{}$\2\2\2\6
\4${}\}{}$\2\2\6
\4${}\}{}$\2\par
\U7.\fi

\M{26}\B\X12:Private variables\X${}\mathrel+\E{}$\6
\&{static} \&{Vertex} ${}{*}\\{vert\_index}[\.{MAX\_N}+\T{1}]{}$;\C{ the vertex
assigned to an SIC code }\par
\fi

\M{27}The theory underlying this step is the following, for integers
$a,b,c,d$ with $b,d>0$:
$$ {a\over b}>{c\over d} \qquad\iff\qquad
a>\biggl\lfloor{b\over d}\biggr\rfloor\,c +
\biggl\lfloor{(b\bmod d)c\over d}\biggr\rfloor\,.$$
In our case, $b=\hbox{\PB{$\|p\MG\\{total}$}}$ and $c=threshold\le
d=65536=2^{16}$, hence
the multiplications cannot overflow. (But they can come awfully darn close.)

\Y\B\4\X27:Compute individual thresholds for each chosen sector\X${}\E{}$\6
\&{for} ${}(\|k\K\.{MAX\_N};{}$ \|k; ${}\|k\MM){}$\1\6
\&{if} ${}((\|p\K\\{node\_index}[\|k])\I\NULL){}$\5
${}\{{}$\1\6
\&{if} ${}(\\{threshold}\E\T{0}){}$\1\5
${}\|p\MG\\{thresh}\K{-}\T{99999999};{}$\2\6
\&{else}\1\5
${}\|p\MG\\{thresh}\K((\|p\MG\\{total}\GG\T{16})*\\{threshold})+(((\|p\MG%
\\{total}\AND\T{\^ffff})*\\{threshold})\GG\T{16});{}$\2\6
\4${}\}{}$\2\2\par
\U25.\fi

\M{28}\B\X28:Prune the sectors that are used in macro-sectors, and form the
lists of SIC sector codes\X${}\E{}$\6
\&{for} ${}(\|p\K\\{node\_index}[\.{ADJ\_SEC}];{}$ ${}\|p\G\\{node\_block};{}$
${}\|p\MM){}$\5
${}\{{}$\C{ bottom up }\1\6
\&{if} ${}(\|p\MG\.{SIC}){}$\5
${}\{{}$\C{ original leaf }\1\6
${}\|p\MG\\{SIC\_list}\K\\{gb\_virgin\_arc}(\,);{}$\6
${}\|p\MG\\{SIC\_list}\MG\\{len}\K\|p\MG\.{SIC};{}$\6
\4${}\}{}$\5
\2\&{else}\5
${}\{{}$\1\6
${}\\{pl}\K\|p+\T{1}{}$;\5
${}\\{pr}\K\|p\MG\\{rchild};{}$\6
\&{if} ${}(\|p\MG\\{tag}\E\T{0}){}$\1\5
${}\|p\MG\\{tag}\K\\{pl}\MG\\{tag}+\\{pr}\MG\\{tag};{}$\2\6
\&{if} ${}(\|p\MG\\{tag}\Z\T{1}){}$\1\5
\X29:Replace \PB{\\{pl}} and \PB{\\{pr}} by their union, \PB{\|p}\X;\2\6
\4${}\}{}$\2\6
\4${}\}{}$\2\par
\U25.\fi

\M{29}\B\X29:Replace \PB{\\{pl}} and \PB{\\{pr}} by their union, \PB{\|p}\X${}%
\E{}$\6
${}\{{}$\5
\1\&{register} \&{Arc} ${}{*}\|a\K\\{pl}\MG\\{SIC\_list};{}$\6
\&{register} \&{long} \\{jj}${}\K\\{pl}\MG\.{SIC},\39\\{kk}\K\\{pr}\MG%
\.{SIC};{}$\7
${}\|p\MG\\{SIC\_list}\K\|a;{}$\6
\&{while} ${}(\|a\MG\\{next}){}$\1\5
${}\|a\K\|a\MG\\{next};{}$\2\6
${}\|a\MG\\{next}\K\\{pr}\MG\\{SIC\_list};{}$\6
\&{for} ${}(\|k\K\.{MAX\_N};{}$ \|k; ${}\|k\MM){}$\1\6
\&{if} ${}((\|q\K\\{node\_index}[\|k])\I\NULL){}$\5
${}\{{}$\1\6
\&{if} ${}(\|q\I\\{pl}\W\|q\I\\{pr}){}$\1\5
${}\|q\MG\\{table}[\\{jj}]\MRL{+{\K}}\|q\MG\\{table}[\\{kk}];{}$\2\6
${}\|p\MG\\{table}[\|k]\K\\{pl}\MG\\{table}[\|k]+\\{pr}\MG\\{table}[\|k];{}$\6
\4${}\}{}$\2\2\6
${}\|p\MG\\{total}\K\\{pl}\MG\\{total}+\\{pr}\MG\\{total};{}$\6
${}\|p\MG\.{SIC}\K\\{jj};{}$\6
${}\|p\MG\\{table}[\\{jj}]\MRL{+{\K}}\|p\MG\\{table}[\\{kk}];{}$\6
${}\\{node\_index}[\\{jj}]\K\|p;{}$\6
${}\\{node\_index}[\\{kk}]\K\NULL;{}$\6
\4${}\}{}$\2\par
\U28.\fi

\M{30}If the \.{Users} vertex is not omitted, we need to compute each
sector's total final demand, which is calculated so that the row sums
and column sums of the input/output coefficients come out equal. We've
already computed the column sum, \PB{$\|p\MG\\{total}$}; we've also computed
\PB{$\|p\MG\\{table}[\T{1}]+\hbox{\hbox{$\cdots$}}+\|p\MG\\{table}[\.{ADJ%
\_SEC}]$}, and put it into
\PB{$\|p\MG\\{table}[\.{MAX\_N}]$}. So now we want to replace \PB{$\|p\MG%
\\{table}[\.{MAX\_N}]$} by
\PB{$\|p\MG\\{total}-\|p\MG\\{table}[\.{MAX\_N}]$}. As remarked earlier, this
quantity might
be negative.

In the special node \PB{\|p} for the \.{Users} vertex, the preliminary
processing has made \PB{$\|p\MG\\{total}\K\T{0}$}; moreover, \PB{$\|p\MG%
\\{table}[\.{MAX\_N}]$} is the
sum of value added, or GNP.  We want to switch those fields.

We don't have to set the \PB{\\{tag}} fields to 1 in the special nodes, because
the remaining parts of the arc-generation algorithm don't look at those fields.

\Y\B\4\X30:Make the special nodes invisible if they are omitted, visible
otherwise\X${}\E{}$\6
\&{if} ${}(\\{omit}\E\T{2}){}$\1\5
${}\\{node\_index}[\.{ADJ\_SEC}]\K\\{node\_index}[\.{MAX\_N}]\K\NULL;{}$\2\6
\&{else} \&{if} ${}(\\{omit}\E\T{1}){}$\1\5
${}\\{node\_index}[\.{MAX\_N}]\K\NULL;{}$\2\6
\&{else}\5
${}\{{}$\1\6
\&{for} ${}(\|k\K\.{ADJ\_SEC};{}$ \|k; ${}\|k\MM){}$\1\6
\&{if} ${}((\|p\K\\{node\_index}[\|k])\I\NULL){}$\1\5
${}\|p\MG\\{table}[\.{MAX\_N}]\K\|p\MG\\{total}-\|p\MG\\{table}[\.{MAX\_N}];{}$%
\2\2\6
${}\|p\K\\{node\_index}[\.{MAX\_N}]{}$;\C{ the special node }\6
${}\|p\MG\\{total}\K\|p\MG\\{table}[\.{MAX\_N}];{}$\6
${}\|p\MG\\{table}[\.{MAX\_N}]\K\T{0};{}$\6
\4${}\}{}$\2\par
\U25.\fi

\N{1}{31}Index. As usual, we close with an index that
shows where the identifiers of \\{gb\_econ} are defined and used.
\fi

\inx
\fin
\con
