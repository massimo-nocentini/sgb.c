\input cwebmac
% This file is part of the Stanford GraphBase (c) Stanford University 1993
% This material goes at the beginning of all Stanford GraphBase CWEB files

\def\topofcontents{
  \leftline{\sc\today\ at \hours}\bigskip\bigskip
  \centerline{\titlefont\title}}

\font\ninett=cmtt9
\def\botofcontents{\vskip 0pt plus 1filll
    \ninerm\baselineskip10pt
    \noindent\copyright\ 1993 Stanford University
    \bigskip\noindent
    This file may be freely copied and distributed, provided that
    no changes whatsoever are made. All users are asked to help keep
    the Stanford GraphBase files consistent and ``uncorrupted,''
    identical everywhere in the world. Changes are permissible only
    if the modified file is given a new name, different from the names of
    existing files in the Stanford GraphBase, and only if the modified file is
    clearly identified as not being part of that GraphBase.
    (The {\ninett CWEB} system has a ``change file'' facility by
    which users can easily make minor alterations without modifying
    the master source files in any way. Everybody is supposed to use
    change files instead of changing the files.)
    The author has tried his best to produce correct and useful programs,
    in order to help promote computer science research,
    but no warranty of any kind should be assumed.
    \smallskip\noindent
    Preliminary work on the Stanford GraphBase project
    was supported in part by National Science
    Foundation grant CCR-86-10181.}

\def\prerequisite#1{\def\startsection{\noindent
    Important: Before reading {\sc\title},
    please read or at least skim the program for {\sc#1}.\bigskip
    \let\startsection=\stsec\stsec}}
\def\prerequisites#1#2{\def\startsection{\noindent
    Important: Before reading {\sc\title}, please read
    or at least skim the programs for {\sc#1} and {\sc#2}.\bigskip
    \let\startsection=\stsec\stsec}}



\def\title{GB\_\,GAMES}

\prerequisites{GB\_\,GRAPH}{GB\_\,IO}

\N{1}{1}Introduction. This GraphBase module contains the \PB{\\{games}}
subroutine,
which creates a family of undirected graphs based on college football
scores. An example of the use of this procedure can be
found in the demo program {\sc FOOTBALL}.

\Y\B\4\X1:\.{gb\_games.h\,}\X${}\E{}$\6
\&{extern} \&{Graph} ${}{*}\\{games}(\,){}$;\par
\A5.\fi

\M{2}The subroutine call \PB{\\{games}}(\PB{\|n}, \PB{\\{ap0\_weight}}, \PB{%
\\{upi0\_weight}}, \PB{\\{ap1\_weight}},
\PB{\\{upi1\_weight}}, \PB{\\{first\_day}}, \PB{\\{last\_day}}, \PB{\\{seed}})
constructs a graph based on the information in \.{games.dat}.
Each vertex of the graph corresponds to one of 120 football teams
at American colleges and universities (more precisely, to the 106 college
football teams of division I-A together with the 14 division I-AA teams
of the Ivy League and the Patriot League).
Each edge of the graph corresponds to one of the 638 games played
between those teams during the 1990 season.

An arc from vertex~\PB{\|u} to vertex~\PB{\|v} is assigned a length
representing
the number of points scored by \PB{\|u} when playing~\PB{\|v}. Thus the graph
isn't really ``undirected,'' although it is true that its arcs are
paired (i.e., that \PB{\|u} played~\PB{\|v} if and only if \PB{\|v} played~\PB{%
\|u}).
A truly undirected graph with the same vertices and edges can be obtained
by applying the \PB{\\{complement}} routine of {\sc GB\_\,BASIC}.

The constructed graph will have $\min(n,120)$ vertices. If \PB{\|n} is less
than 120, the \PB{\|n} teams will be selected by assigning a weight to
each team and choosing the \PB{\|n} with largest weight, using random
numbers to break ties in case of equal weights. Weights are computed
by the formula
$$ \PB{\\{ap0\_weight}}\cdot\PB{\\{ap0}}+\PB{\\{upi0\_weight}}\cdot\PB{%
\\{upi0}}
+\PB{\\{ap1\_weight}}\cdot\PB{\\{ap1}}+\PB{\\{upi1\_weight}}\cdot\PB{\\{upi1}},
$$
where \PB{\\{ap0}} and \PB{\\{upi0}} are the point scores given to a team in
the
Associated Press and United Press International polls at the beginning
of the season, and \PB{\\{ap1}} and \PB{\\{upi1}} are the similar scores given
at
the end of the season. (The \\{ap} scores were obtained by asking 60
sportswriters to choose and rank the top 25 teams, assigning 25 points
to a team ranked 1st and 1 point to a team ranked 25th; thus the
total of each of the \\{ap} scores, summed over all teams,
is 19500. The \\{upi} scores were
obtained by asking football coaches to choose and rank the top 15
teams, assigning 15 points to a team ranked 1st and 1 point to a team
ranked 15th. In the case of \\{upi0}, there were 48 coaches voting,
making 5760 points altogether; but in the case of \\{upi1}, 59 coaches
were polled, yielding a total of 7080 points. The coaches agreed not
to vote for any team that was on probation for violating NCAA rules,
but the sportswriters had no such policy.)

Parameters \PB{\\{first\_day}} and \PB{\\{last\_day}} can be used to vary the
number of
edges; only games played between \PB{\\{first\_day}} and \PB{\\{last\_day}},
inclusive,
will be included in the constructed graph. Day~0 was August~26, 1990,
when Colorado and Tennessee competed in the Disneyland Pigskin Classic.
Day~128 was January~1, 1991, when the final end-of-season bowl games
were played. About half of each team's games were played between day~0 and
day~50. If \PB{$\\{last\_day}\K\T{0}$}, the value of \PB{\\{last\_day}} is
automatically
increased to~128.

As usual in GraphBase routines, you can set \PB{$\|n\K\T{0}$} to get the
default
situation where \PB{\|n} has its maximum value. For example, either
\PB{$\\{games}(\T{0},\T{0},\T{0},\T{0},\T{0},\T{0},\T{0},\T{0})$} or \PB{$%
\\{games}(\T{120},\T{0},\T{0},\T{0},\T{0},\T{0},\T{0},\T{0})$} produces the
full graph;
\PB{$\\{games}(\T{0},\T{0},\T{0},\T{0},\T{0},\T{50},\T{0},\T{0})$} or \PB{$%
\\{games}(\T{120},\T{0},\T{0},\T{0},\T{0},\T{50},\T{0},\T{0})$}
or \PB{$\\{games}(\T{120},\T{0},\T{0},\T{0},\T{0},\T{50},\T{128},\T{0})$}
produces the graph for the last half
of the season. One way to select a subgraph containing the
30 ``best'' teams is to ask for \PB{$\\{games}(\T{30},\T{0},\T{0},\T{1},\T{2},%
\T{0},\T{0},\T{0})$}, which adds
the votes of the sportswriters to the votes of the coaches
(considering that a coach's first choice is worth 30 points
while a sportswriter's first choice is worth only 25). It turns out
that 67 of the teams did not receive votes in any of the four polls;
the subroutine call \PB{$\\{games}(\T{53},\T{1},\T{1},\T{1},\T{1},\T{0},\T{0},%
\T{0})$} will pick out the 53 teams
that were selected at least once by some sportswriter or coach, and
\PB{$\\{games}(\T{67},{-}\T{1},{-}\T{1},{-}\T{1},{-}\T{1},\T{0},\T{0},\T{0})$}
will pick out the 67 that were not.
A~random selection of 60 teams can be obtained by calling
\PB{$\\{games}(\T{60},\T{0},\T{0},\T{0},\T{0},\T{0},\T{0},\|s)$}. Different
choices of the seed number~\PB{\|s}
will produce different selections in a system-independent manner;
any value of \PB{\|s} between 0 and $2^{31}-1$ is permissible.
If you ask for \PB{$\\{games}(\T{120},\T{0},\T{0},\T{0},\T{0},\T{0},\T{0},%
\|s)$} with different choices of~\PB{\|s},
you always get the full graph, but the vertices will appear in different
(random) orderings depending on~\PB{\|s}.

Parameters \PB{\\{ap0\_weight}}, \PB{\\{upi0\_weight}}, \PB{\\{ap1\_weight}},
and \PB{\\{upi1\_weight}} must be
at most $2^{17}=131072$ in absolute value.

\Y\B\4\D$\.{MAX\_N}$ \5
\T{120}\par
\B\4\D$\.{MAX\_DAY}$ \5
\T{128}\par
\B\4\D$\.{MAX\_WEIGHT}$ \5
\T{131072}\par
\B\4\D$\\{ap}$ \5
$\|u.{}$\|I\C{ Associated Press scores: \PB{$(\\{ap0}\LL\T{16})+\\{ap1}$} }\par
\B\4\D$\\{upi}$ \5
$\|v.{}$\|I\C{ United Press International scores \PB{$(\\{upi0}\LL\T{16})+%
\\{upi1}$} }\par
\fi

\M{3}Most of the teams belong to a ``conference,'' and they play against
almost every other team that belongs to the same conference. For
example, Stanford and nine other teams belong to the
Pacific Ten conference. Eight of Stanford's eleven games were against
other teams of the Pacific Ten; the other three were played against
Colorado (from the Big Eight), San Jos\'e State (from the Big West)
and Notre Dame (which is independent). The graphs produced by \PB{\\{games}}
therefore illustrate ``cliquey'' patterns of social interaction.

Eleven different conferences are included in \.{games.dat}. Utility
field \PB{$\|z.\|S$} of a vertex is set to the name of a team's conference, or
to \PB{$\NULL$}
if that team is independent. (Exactly 24 of the I-A football teams
were independent in 1990.) Two teams \PB{\|u} and \PB{\|v} belong to the same
conference if and only if \PB{$\|u\MG\\{conference}\E\|v\MG\\{conference}$} and
\PB{$\|u\MG\\{conference}\I\NULL$}.

\Y\B\4\D$\\{conference}$ \5
$\|z.{}$\|S\par
\fi

\M{4}Each team has a nickname, which is recorded in utility field \PB{$\|y.%
\|S$}.
For example, Georgia Tech's team is called the Yellow Jackets.
Six teams (Auburn, Clemson, Memphis State, Missouri, Pacific, and
Princeton) are called the Tigers, and five teams
(Fresno State, Georgia, Louisiana Tech, Mississippi State,
Yale) are called the Bulldogs. But most of the teams have a unique
nickname, and 94 distinct nicknames exist.

A shorthand code for team names is also provided, in the \PB{\\{abbr}} field.

\Y\B\4\D$\\{nickname}$ \5
$\|y.{}$\|S\par
\B\4\D$\\{abbr}$ \5
$\|x.{}$\|S\par
\fi

\M{5}If \PB{\|a} points to an arc from \PB{\|u} to \PB{\|v}, utility field %
\PB{$\|a\MG\|a.\|I$} contains
the value 3 if \PB{\|u} was the home team, 1 if \PB{\|v} was the home team, and
2 if both
teams played on neutral territory. The date of that game, represented
as a integer number of days after August~26, 1990, appears in utility
field \PB{$\|a\MG\|b.\|I$}. The arcs in each vertex list \PB{$\|v\MG\\{arcs}$}
appear in reverse order
of their dates: last game first and first game last.

\Y\B\4\D$\.{HOME}$ \5
\T{1}\par
\B\4\D$\.{NEUTRAL}$ \5
\T{2}\C{ this value is halfway between \PB{\.{HOME}} and \PB{\.{AWAY}} }\par
\B\4\D$\.{AWAY}$ \5
\T{3}\par
\B\4\D$\\{venue}$ \5
$\|a.{}$\|I\par
\B\4\D$\\{date}$ \5
$\|b.{}$\|I\par
\Y\B\4\X1:\.{gb\_games.h\,}\X${}\mathrel+\E{}$\6
\8\#\&{define} \\{ap}\5${}\|u.\|I{}$\C{ repeat the definitions in the header
file }\6
\8\#\&{define} \\{upi}\5${}\|v.\|I{}$\6
\8\#\&{define} \\{abbr}\5${}\|x.\|S{}$\6
\8\#\&{define} \\{nickname}\5${}\|y.\|S{}$\6
\8\#\&{define} \\{conference}\5${}\|z.\|S{}$\6
\8\#\&{define} \.{HOME}\5\T{1}\6
\8\#\&{define} \.{NEUTRAL}\5\T{2}\6
\8\#\&{define} \.{AWAY}\5\T{3}\6
\8\#\&{define} \\{venue}\5${}\|a.\|I{}$\6
\8\#\&{define} \\{date}\5${}\|b.\|I{}$\par
\fi

\M{6}If the \PB{\\{games}} routine encounters a problem, it returns \PB{$%
\NULL$}
(\.{NULL}), after putting a code number into the external variable
\PB{\\{panic\_code}}. This code number identifies the type of failure.
Otherwise \PB{\\{games}} returns a pointer to the newly created graph, which
will be represented with the data structures explained in {\sc GB\_\,GRAPH}.
(The external variable \PB{\\{panic\_code}} is itself defined in {\sc GB\_%
\,GRAPH}.)

\Y\B\4\D$\\{panic}(\|c)$ \5
${}\{{}$\5
\1${}\\{panic\_code}\K\|c{}$;\5
${}\\{gb\_trouble\_code}\K\T{0}{}$;\5
\&{return} ${}\NULL{}$;\5
${}\}{}$\2\par
\fi

\M{7}The \CEE/ file \.{gb\_games.c} has the following overall shape:

\Y\B\8\#\&{include} \.{"gb\_io.h"}\C{ we will use the {\sc GB\_\,IO} routines
for input }\6
\8\#\&{include} \.{"gb\_flip.h"}\C{ we will use the {\sc GB\_\,FLIP} routines
for random numbers }\6
\8\#\&{include} \.{"gb\_graph.h"}\C{ we will use the {\sc GB\_\,GRAPH} data
structures }\6
\8\#\&{include} \.{"gb\_sort.h"}\C{ and \PB{\\{gb\_linksort}} for sorting }\6
\ATH\7
\X11:Type declarations\X\6
\X13:Private variables\X\6
\X23:Private functions\X\7
\1\1\&{Graph} ${}{*}\\{games}(\|n,\39\\{ap0\_weight},\39\\{upi0\_weight},\39%
\\{ap1\_weight},\39\\{upi1\_weight},\39\\{first\_day},\39\\{last\_day},\39%
\\{seed}){}$\6
\&{unsigned} \&{long} \|n;\C{ number of vertices desired }\6
\&{long} \\{ap0\_weight};\C{ coefficient of \PB{\\{ap0}} in the weight function
}\6
\&{long} \\{ap1\_weight};\C{ coefficient of \PB{\\{ap1}} in the weight function
}\6
\&{long} \\{upi0\_weight};\C{ coefficient of \PB{\\{upi0}} in the weight
function }\6
\&{long} \\{upi1\_weight};\C{ coefficient of \PB{\\{upi1}} in the weight
function }\6
\&{long} \\{first\_day};\C{ lower cutoff for games to be considered }\6
\&{long} \\{last\_day};\C{ upper cutoff for games to be considered }\6
\&{long} \\{seed};\C{ random number seed }\2\2\6
${}\{{}$\5
\1\X8:Local variables\X\7
\\{gb\_init\_rand}(\\{seed});\6
\X9:Check that the parameters are valid\X;\6
\X10:Set up a graph with \PB{\|n} vertices\X;\6
\X14:Read the first part of \.{games.dat} and compute team weights\X;\6
\X19:Determine the \PB{\|n} teams to use in the graph\X;\6
\X21:Put the appropriate edges into the graph\X;\6
\&{if} ${}(\\{gb\_close}(\,)\I\T{0}){}$\1\5
\\{panic}(\\{late\_data\_fault});\C{ something's wrong with \PB{%
\.{"games.dat"}}; see \PB{\\{io\_errors}} }\2\6
\\{gb\_free}(\\{working\_storage});\6
\&{if} (\\{gb\_trouble\_code})\5
${}\{{}$\1\6
\\{gb\_recycle}(\\{new\_graph});\6
\\{panic}(\\{alloc\_fault});\C{ oops, we ran out of memory somewhere back there
}\6
\4${}\}{}$\2\6
\&{return} \\{new\_graph};\6
\4${}\}{}$\2\par
\fi

\M{8}\B\X8:Local variables\X${}\E{}$\6
\&{Graph} ${}{*}\\{new\_graph}{}$;\C{ the graph constructed by \PB{\\{games}} }%
\6
\&{register} \&{long} \|j${},\39\|k{}$;\C{ all-purpose indices }\par
\U7.\fi

\M{9}\B\X9:Check that the parameters are valid\X${}\E{}$\6
\&{if} ${}(\|n\E\T{0}\V\|n>\.{MAX\_N}){}$\1\5
${}\|n\K\.{MAX\_N};{}$\2\6
\&{if} ${}(\\{ap0\_weight}>\.{MAX\_WEIGHT}\V\\{ap0\_weight}<{-}\.{MAX\_WEIGHT}%
\V\\{upi0\_weight}>\.{MAX\_WEIGHT}\V\\{upi0\_weight}<{-}\.{MAX\_WEIGHT}\V\3{-1}%
\\{ap1\_weight}>\.{MAX\_WEIGHT}\V\\{ap1\_weight}<{-}\.{MAX\_WEIGHT}\V\\{upi1%
\_weight}>\.{MAX\_WEIGHT}\V\\{upi1\_weight}<{-}\.{MAX\_WEIGHT}){}$\1\5
\\{panic}(\\{bad\_specs});\C{ the magnitude of at least one weight is too big }%
\2\6
\&{if} ${}(\\{first\_day}<\T{0}){}$\1\5
${}\\{first\_day}\K\T{0};{}$\2\6
\&{if} ${}(\\{last\_day}\E\T{0}\V\\{last\_day}>\.{MAX\_DAY}){}$\1\5
${}\\{last\_day}\K\.{MAX\_DAY}{}$;\2\par
\U7.\fi

\M{10}\B\X10:Set up a graph with \PB{\|n} vertices\X${}\E{}$\6
$\\{new\_graph}\K\\{gb\_new\_graph}(\|n);{}$\6
\&{if} ${}(\\{new\_graph}\E\NULL){}$\1\5
\\{panic}(\\{no\_room});\C{ out of memory before we're even started }\2\6
${}\\{sprintf}(\\{new\_graph}\MG\\{id},\39\.{"games(\%lu,\%ld,\%ld,\%}\)\.{ld,%
\%ld,\%ld,\%ld,\%ld)"},\39\|n,\39\\{ap0\_weight},\39\\{upi0\_weight},\39\\{ap1%
\_weight},\39\\{upi1\_weight},\39\\{first\_day},\39\\{last\_day},\39%
\\{seed});{}$\6
${}\\{strcpy}(\\{new\_graph}\MG\\{util\_types},\39\.{"IIZSSSIIZZZZZZ"}){}$;\par
\U7.\fi

\N{1}{11}Vertices.
As we read in the data, we construct a list of nodes, each of which contains
a team's name, nickname, conference, and weight. After this list
has been sorted by weight, the top \PB{\|n} entries will be the vertices of the
new graph.

\Y\B\4\X11:Type declarations\X${}\E{}$\6
\&{typedef} \&{struct} \&{node\_struct} ${}\{{}$\C{ records to be sorted by %
\PB{\\{gb\_linksort}} }\1\6
\&{long} \\{key};\C{ the nonnegative sort key (weight plus $2^{30}$) }\6
\&{struct} \&{node\_struct} ${}{*}\\{link}{}$;\C{ pointer to next record }\6
\&{char} \\{name}[\T{24}];\C{ \PB{\.{"College\ Name"}} }\6
\&{char} \\{nick}[\T{22}];\C{ \PB{\.{"Team\ Nickname"}} }\6
\&{char} \\{abb}[\T{6}];\C{ \PB{\.{"ABBR"}} }\6
\&{long} \\{a0}${},\39\\{u0},\39\\{a1},\39\\{u1}{}$;\C{ team scores in press
polls }\6
\&{char} ${}{*}\\{conf}{}$;\C{ pointer to conference name }\6
\&{struct} \&{node\_struct} ${}{*}\\{hash\_link}{}$;\C{ pointer to next %
\.{ABBR} in hash list }\6
\&{Vertex} ${}{*}\\{vert}{}$;\C{ vertex corresponding to this team }\2\6
${}\}{}$ \&{node};\par
\U7.\fi

\M{12}The data in \.{games.dat} appears in two parts. The first 120 lines
have the form
$$\hbox{\tt ABBR College Name(Team Nickname)Conference;a0,u0;a1,u1}$$
and they give basic information about the teams. An internal abbreviation code
\.{ABBR} is used to identify each team in the second part of the data.

The second part presents scores of the games, and it
contains two kinds of lines. If the first character of a line is
`\.>', it means ``change the current date,'' and the remaining
characters specify a date as a one-letter month code followed by the day
of the month. Otherwise the line gives scores of a game, using the
\.{ABBR} codes for two teams. The scores are separated by `\.@' if
the second team was the home team and by `\.,' if both teams were on
neutral territory.

For example, two games were played on December 8, namely the annual Army-Navy
game and the California Raisin Bowl game. These are recorded in three lines
of \.{games.dat} as follows:
$$\vbox{\halign{\tt#\hfil\cr
>D8\cr
NAVY20@ARMY30\cr
SJSU48,CMICH24\cr}}$$
We deduce that Navy played at Army's home stadium, losing 20 to~30;
moreover, San Jos\'e State played Central Michigan on neutral territory and
won, 48 to~24. (The California Raisin Bowl is traditionally a playoff between
the champions of the Big West and Mid-American conferences.)

\fi

\M{13}In order to map \.{ABBR} codes to team names, we use a simple
hash coding scheme. Two abbreviations with the same hash address are
linked together via the \PB{\\{hash\_link}} address in their node.

The constants defined here are taken from the specific data in \.{games.dat},
because this routine is not intended to be perfectly general.

\Y\B\4\D$\.{HASH\_PRIME}$ \5
\T{1009}\par
\Y\B\4\X13:Private variables\X${}\E{}$\6
\&{static} \&{long} \\{ma0}${}\K\T{1451},\39\\{mu0}\K\T{666},\39\\{ma1}\K%
\T{1475},\39\\{mu1}\K\T{847}{}$;\C{ maximum poll values in the data }\6
\&{static} \&{node} ${}{*}\\{node\_block}{}$;\C{ array of nodes holding team
info }\6
\&{static} \&{node} ${}{*}{*}\\{hash\_block}{}$;\C{ array of heads of hash code
lists }\6
\&{static} \&{Area} \\{working\_storage};\C{ memory needed only while \PB{%
\\{games}} is working }\6
\&{static} \&{char} ${}{*}{*}\\{conf\_block}{}$;\C{ array of conference names }%
\6
\&{static} \&{long} \|m;\C{ the number of conference names known so far }\par
\U7.\fi

\M{14}\B\X14:Read the first part of \.{games.dat} and compute team weights\X${}%
\E{}$\6
$\\{node\_block}\K\\{gb\_typed\_alloc}(\.{MAX\_N}+\T{2},\39\&{node},\39%
\\{working\_storage}){}$;\C{ leave room for string overflow }\6
${}\\{hash\_block}\K\\{gb\_typed\_alloc}(\.{HASH\_PRIME},\39{}$\&{node} ${}{*},%
\39\\{working\_storage});{}$\6
${}\\{conf\_block}\K\\{gb\_typed\_alloc}(\.{MAX\_N},\39{}$\&{char} ${}{*},\39%
\\{working\_storage});{}$\6
${}\|m\K\T{0};{}$\6
\&{if} (\\{gb\_trouble\_code})\5
${}\{{}$\1\6
\\{gb\_free}(\\{working\_storage});\6
${}\\{panic}(\\{no\_room}+\T{1}){}$;\C{ nowhere to copy the data }\6
\4${}\}{}$\2\6
\&{if} ${}(\\{gb\_open}(\.{"games.dat"})\I\T{0}){}$\1\5
\\{panic}(\\{early\_data\_fault});\C{ couldn't open \PB{\.{"games.dat"}} using
       GraphBase conventions; \PB{\\{io\_errors}} tells why }\2\6
\&{for} ${}(\|k\K\T{0};{}$ ${}\|k<\.{MAX\_N};{}$ ${}\|k\PP){}$\1\5
\X15:Read and store data for team \PB{\|k}\X;\2\par
\U7.\fi

\M{15}\B\X15:Read and store data for team \PB{\|k}\X${}\E{}$\6
${}\{{}$\5
\1\&{register} \&{node} ${}{*}\|p;{}$\6
\&{register} \&{char} ${}{*}\|q;{}$\7
${}\|p\K\\{node\_block}+\|k;{}$\6
\&{if} (\|k)\1\5
${}\|p\MG\\{link}\K\|p-\T{1};{}$\2\6
${}\|q\K\\{gb\_string}(\|p\MG\\{abb},\39\.{'\ '});{}$\6
\&{if} ${}(\|q>{\AND}\|p\MG\\{abb}[\T{6}]\V\\{gb\_char}(\,)\I\.{'\ '}){}$\1\5
\\{panic}(\\{syntax\_error});\C{ out of sync in \.{games.dat} }\2\6
\X16:Enter \PB{$\|p\MG\\{abb}$} in the hash table\X;\6
${}\|q\K\\{gb\_string}(\|p\MG\\{name},\39\.{'('});{}$\6
\&{if} ${}(\|q>{\AND}\|p\MG\\{name}[\T{24}]\V\\{gb\_char}(\,)\I\.{'('}){}$\1\5
${}\\{panic}(\\{syntax\_error}+\T{1}){}$;\C{ team name too long }\2\6
${}\|q\K\\{gb\_string}(\|p\MG\\{nick},\39\.{')'});{}$\6
\&{if} ${}(\|q>{\AND}\|p\MG\\{nick}[\T{22}]\V\\{gb\_char}(\,)\I\.{')'}){}$\1\5
${}\\{panic}(\\{syntax\_error}+\T{2}){}$;\C{ team nickname too long }\2\6
\X17:Read the conference name for \PB{\|p}\X;\6
\X18:Read the press poll scores for \PB{\|p} and compute \PB{$\|p\MG\\{key}$}%
\X;\6
\\{gb\_newline}(\,);\6
\4${}\}{}$\2\par
\U14.\fi

\M{16}\B\X16:Enter \PB{$\|p\MG\\{abb}$} in the hash table\X${}\E{}$\6
${}\{{}$\5
\1\&{long} \|h${}\K\T{0}{}$;\C{ the hash code }\7
\&{for} ${}(\|q\K\|p\MG\\{abb};{}$ ${}{*}\|q;{}$ ${}\|q\PP){}$\1\5
${}\|h\K(\|h+\|h+{*}\|q)\MOD\.{HASH\_PRIME};{}$\2\6
${}\|p\MG\\{hash\_link}\K\\{hash\_block}[\|h];{}$\6
${}\\{hash\_block}[\|h]\K\|p;{}$\6
\4${}\}{}$\2\par
\U15.\fi

\M{17}\B\X17:Read the conference name for \PB{\|p}\X${}\E{}$\6
${}\{{}$\1\6
${}\\{gb\_string}(\\{str\_buf},\39\.{';'});{}$\6
\&{if} ${}(\\{gb\_char}(\,)\I\.{';'}){}$\1\5
${}\\{panic}(\\{syntax\_error}+\T{3}){}$;\C{ conference name clobbered }\2\6
\&{if} ${}(\\{strcmp}(\\{str\_buf},\39\.{"Independent"})\I\T{0}){}$\5
${}\{{}$\1\6
\&{for} ${}(\|j\K\T{0};{}$ ${}\|j<\|m;{}$ ${}\|j\PP){}$\1\6
\&{if} ${}(\\{strcmp}(\\{str\_buf},\39\\{conf\_block}[\|j])\E\T{0}){}$\1\5
\&{goto} \\{found};\2\2\6
${}\\{conf\_block}[\|m\PP]\K\\{gb\_save\_string}(\\{str\_buf});{}$\6
\4\\{found}:\5
${}\|p\MG\\{conf}\K\\{conf\_block}[\|j];{}$\6
\4${}\}{}$\2\6
\4${}\}{}$\2\par
\U15.\fi

\M{18}The key value computed here will be between 0 and~$2^{31}$, because of
the bound we've imposed on the weight parameters.

\Y\B\4\X18:Read the press poll scores for \PB{\|p} and compute \PB{$\|p\MG%
\\{key}$}\X${}\E{}$\6
$\|p\MG\\{a0}\K\\{gb\_number}(\T{10});{}$\6
\&{if} ${}(\|p\MG\\{a0}>\\{ma0}\V\\{gb\_char}(\,)\I\.{','}){}$\1\5
${}\\{panic}(\\{syntax\_error}+\T{4}){}$;\C{ first AP score clobbered }\2\6
${}\|p\MG\\{u0}\K\\{gb\_number}(\T{10});{}$\6
\&{if} ${}(\|p\MG\\{u0}>\\{mu0}\V\\{gb\_char}(\,)\I\.{';'}){}$\1\5
${}\\{panic}(\\{syntax\_error}+\T{5}){}$;\C{ first UPI score clobbered }\2\6
${}\|p\MG\\{a1}\K\\{gb\_number}(\T{10});{}$\6
\&{if} ${}(\|p\MG\\{a1}>\\{ma1}\V\\{gb\_char}(\,)\I\.{','}){}$\1\5
${}\\{panic}(\\{syntax\_error}+\T{6}){}$;\C{ second AP score clobbered }\2\6
${}\|p\MG\\{u1}\K\\{gb\_number}(\T{10});{}$\6
\&{if} ${}(\|p\MG\\{u1}>\\{mu1}\V\\{gb\_char}(\,)\I\.{'\\n'}){}$\1\5
${}\\{panic}(\\{syntax\_error}+\T{7}){}$;\C{ second UPI score clobbered }\2\6
${}\|p\MG\\{key}\K\\{ap0\_weight}*(\|p\MG\\{a0})+\\{upi0\_weight}*(\|p\MG%
\\{u0})+\\{ap1\_weight}*(\|p\MG\\{a1})+\\{upi1\_weight}*(\|p\MG\\{u1})+\T{%
\^40000000}{}$;\par
\U15.\fi

\M{19}Once all the nodes have been set up, we can use the \PB{\\{gb\_linksort}}
routine to sort them into the desired order. It builds 128
lists from which the desired nodes are readily accessed in decreasing
order of weight, using random numbers to break ties.

We set the abbreviation code to zero in every team that isn't chosen. Then
games involving that team will be excluded when edges are generated below.

\Y\B\4\X19:Determine the \PB{\|n} teams to use in the graph\X${}\E{}$\6
${}\{{}$\5
\1\&{register} \&{node} ${}{*}\|p{}$;\C{ the current node being considered }\6
\&{register} \&{Vertex} ${}{*}\|v\K\\{new\_graph}\MG\\{vertices}{}$;\C{ the
next vertex to use }\7
${}\\{gb\_linksort}(\\{node\_block}+\.{MAX\_N}-\T{1});{}$\6
\&{for} ${}(\|j\K\T{127};{}$ ${}\|j\G\T{0};{}$ ${}\|j\MM){}$\1\6
\&{for} ${}(\|p\K{}$(\&{node} ${}{*})\,\\{gb\_sorted}[\|j];{}$ \|p; ${}\|p\K\|p%
\MG\\{link}){}$\5
${}\{{}$\1\6
\&{if} ${}(\|v<\\{new\_graph}\MG\\{vertices}+\|n){}$\1\5
\X20:Add team \PB{\|p} to the graph\X\2\6
\&{else}\1\5
${}\|p\MG\\{abb}[\T{0}]\K\.{'\\0'}{}$;\C{ this team is not being used }\2\6
\4${}\}{}$\2\2\6
\4${}\}{}$\2\par
\U7.\fi

\M{20}\B\X20:Add team \PB{\|p} to the graph\X${}\E{}$\6
${}\{{}$\1\6
${}\|v\MG\\{ap}\K((\&{long})(\|p\MG\\{a0})\LL\T{16})+\|p\MG\\{a1};{}$\6
${}\|v\MG\\{upi}\K((\&{long})(\|p\MG\\{u0})\LL\T{16})+\|p\MG\\{u1};{}$\6
${}\|v\MG\\{abbr}\K\\{gb\_save\_string}(\|p\MG\\{abb});{}$\6
${}\|v\MG\\{nickname}\K\\{gb\_save\_string}(\|p\MG\\{nick});{}$\6
${}\|v\MG\\{conference}\K\|p\MG\\{conf};{}$\6
${}\|v\MG\\{name}\K\\{gb\_save\_string}(\|p\MG\\{name});{}$\6
${}\|p\MG\\{vert}\K\|v\PP;{}$\6
\4${}\}{}$\2\par
\U19.\fi

\N{1}{21}Arcs.
Finally, we read through the rest of \.{games.dat}, adding a pair of
arcs for each game that belongs to the selected time interval
and was played by two of the selected teams.

\Y\B\4\X21:Put the appropriate edges into the graph\X${}\E{}$\6
${}\{{}$\5
\1\&{register} \&{Vertex} ${}{*}\|u,\39{*}\|v;{}$\6
\&{register} \&{long} \\{today}${}\K\T{0}{}$;\C{ current day of play }\6
\&{long} \\{su}${},\39\\{sv}{}$;\C{ points scored by each team }\6
\&{long} \\{ven};\C{ \PB{\.{HOME}} if \PB{\|v} is home team, \PB{\.{NEUTRAL}}
if on neutral ground }\7
\&{while} ${}(\R\\{gb\_eof}(\,)){}$\5
${}\{{}$\1\6
\&{if} ${}(\\{gb\_char}(\,)\E\.{'>'}){}$\1\5
\X22:Change the current date\X\2\6
\&{else}\1\5
\\{gb\_backup}(\,);\2\6
${}\|u\K\\{team\_lookup}(\,);{}$\6
${}\\{su}\K\\{gb\_number}(\T{10});{}$\6
${}\\{ven}\K\\{gb\_char}(\,);{}$\6
\&{if} ${}(\\{ven}\E\.{'@'}){}$\1\5
${}\\{ven}\K\.{HOME};{}$\2\6
\&{else} \&{if} ${}(\\{ven}\E\.{','}){}$\1\5
${}\\{ven}\K\.{NEUTRAL};{}$\2\6
\&{else}\1\5
${}\\{panic}(\\{syntax\_error}+\T{8}){}$;\C{ bad syntax in game score line }\2\6
${}\|v\K\\{team\_lookup}(\,);{}$\6
${}\\{sv}\K\\{gb\_number}(\T{10});{}$\6
\&{if} ${}(\\{gb\_char}(\,)\I\.{'\\n'}){}$\1\5
${}\\{panic}(\\{syntax\_error}+\T{9}){}$;\C{ bad syntax in game score line }\2\6
\&{if} ${}(\|u\I\NULL\W\|v\I\NULL\W\\{today}\G\\{first\_day}\W\\{today}\Z%
\\{last\_day}){}$\1\5
\X24:Enter a new edge\X;\2\6
\\{gb\_newline}(\,);\6
\4${}\}{}$\2\6
\4${}\}{}$\2\par
\U7.\fi

\M{22}\B\X22:Change the current date\X${}\E{}$\6
${}\{{}$\5
\1\&{register} \&{char} \|c${}\K\\{gb\_char}(\,){}$;\C{ month code }\6
\&{register} \&{long} \|d;\C{ day of football season }\7
\&{switch} (\|c)\5
${}\{{}$\1\6
\4\&{case} \.{'A'}:\5
${}\|d\K{-}\T{26}{}$;\5
\&{break};\C{ August }\6
\4\&{case} \.{'S'}:\5
${}\|d\K\T{5}{}$;\5
\&{break};\C{ thirty days hath September }\6
\4\&{case} \.{'O'}:\5
${}\|d\K\T{35}{}$;\5
\&{break};\C{ October }\6
\4\&{case} \.{'N'}:\5
${}\|d\K\T{66}{}$;\5
\&{break};\C{ November }\6
\4\&{case} \.{'D'}:\5
${}\|d\K\T{96}{}$;\5
\&{break};\C{ December }\6
\4\&{case} \.{'J'}:\5
${}\|d\K\T{127}{}$;\5
\&{break};\C{ January }\6
\4\&{default}:\5
${}\|d\K\T{1000};{}$\6
\4${}\}{}$\2\6
${}\|d\MRL{+{\K}}\\{gb\_number}(\T{10});{}$\6
\&{if} ${}(\|d<\T{0}\V\|d>\.{MAX\_DAY}){}$\1\5
${}\\{panic}(\\{syntax\_error}-\T{1}){}$;\C{ date was clobbered }\2\6
${}\\{today}\K\|d;{}$\6
\\{gb\_newline}(\,);\C{ now ready to read a non-date line }\6
\4${}\}{}$\2\par
\U21.\fi

\M{23}\B\X23:Private functions\X${}\E{}$\6
\1\1\&{static} \&{Vertex} ${}{*}{}$\\{team\_lookup}(\,)\C{ read and decode an
abbreviation }\2\2\6
${}\{{}$\5
\1\&{register} \&{char} ${}{*}\|q\K\\{str\_buf}{}$;\C{ position in \PB{\\{str%
\_buf}} }\6
\&{register} \&{long} \|h${}\K\T{0}{}$;\C{ hash code }\6
\&{register} \&{node} ${}{*}\|p{}$;\C{ position in hash list }\7
\&{while} ${}(\\{gb\_digit}(\T{10})<\T{0}){}$\5
${}\{{}$\1\6
${}{*}\|q\K\\{gb\_char}(\,);{}$\6
${}\|h\K(\|h+\|h+{*}\|q)\MOD\.{HASH\_PRIME};{}$\6
${}\|q\PP;{}$\6
\4${}\}{}$\2\6
\\{gb\_backup}(\,);\C{ prepare to re-scan the digit following the abbreviation
}\6
${}{*}\|q\K\.{'\\0'}{}$;\C{ null-terminate the abbreviation just scanned }\6
\&{for} ${}(\|p\K\\{hash\_block}[\|h];{}$ \|p; ${}\|p\K\|p\MG\\{hash\_link}){}$%
\1\6
\&{if} ${}(\\{strcmp}(\|p\MG\\{abb},\39\\{str\_buf})\E\T{0}){}$\1\5
\&{return} \|p${}\MG\\{vert};{}$\2\2\6
\&{return} ${}\NULL{}$;\C{ not found }\6
\4${}\}{}$\2\par
\U7.\fi

\M{24}We retain the convention of {\sc GB\_\,GRAPH} that the arc from \PB{\|v}
to \PB{\|u}
appears immediately after a matching arc from \PB{\|u} to \PB{\|v} when \PB{$%
\|u<\|v$}.

\Y\B\4\X24:Enter a new edge\X${}\E{}$\6
${}\{{}$\5
\1\&{register} \&{Arc} ${}{*}\|a;{}$\7
\&{if} ${}(\|u>\|v){}$\5
${}\{{}$\5
\1\&{register} \&{Vertex} ${}{*}\|w;{}$\6
\&{register} \&{long} \\{sw};\7
${}\|w\K\|u{}$;\5
${}\|u\K\|v{}$;\5
${}\|v\K\|w;{}$\6
${}\\{sw}\K\\{su}{}$;\5
${}\\{su}\K\\{sv}{}$;\5
${}\\{sv}\K\\{sw};{}$\6
${}\\{ven}\K\.{HOME}+\.{AWAY}-\\{ven};{}$\6
\4${}\}{}$\2\6
${}\\{gb\_new\_arc}(\|u,\39\|v,\39\\{su});{}$\6
${}\\{gb\_new\_arc}(\|v,\39\|u,\39\\{sv});{}$\6
${}\|a\K\|u\MG\\{arcs}{}$;\C{ a pointer to the new arc }\6
\&{if} ${}(\|v\MG\\{arcs}\I\|a+\T{1}){}$\1\5
${}\\{panic}(\\{impossible}+\T{9}){}$;\C{ can't happen }\2\6
${}\|a\MG\\{venue}\K\\{ven}{}$;\5
${}(\|a+\T{1})\MG\\{venue}\K\.{HOME}+\.{AWAY}-\\{ven};{}$\6
${}\|a\MG\\{date}\K(\|a+\T{1})\MG\\{date}\K\\{today};{}$\6
\4${}\}{}$\2\par
\U21.\fi

\N{1}{25}Index. As usual, we close with an index that
shows where the identifiers of \\{gb\_games} are defined and used.
\fi

\inx
\fin
\con
