\input cwebmac
% This file is part of the Stanford GraphBase (c) Stanford University 1993
% This material goes at the beginning of all Stanford GraphBase CWEB files

\def\topofcontents{
  \leftline{\sc\today\ at \hours}\bigskip\bigskip
  \centerline{\titlefont\title}}

\font\ninett=cmtt9
\def\botofcontents{\vskip 0pt plus 1filll
    \ninerm\baselineskip10pt
    \noindent\copyright\ 1993 Stanford University
    \bigskip\noindent
    This file may be freely copied and distributed, provided that
    no changes whatsoever are made. All users are asked to help keep
    the Stanford GraphBase files consistent and ``uncorrupted,''
    identical everywhere in the world. Changes are permissible only
    if the modified file is given a new name, different from the names of
    existing files in the Stanford GraphBase, and only if the modified file is
    clearly identified as not being part of that GraphBase.
    (The {\ninett CWEB} system has a ``change file'' facility by
    which users can easily make minor alterations without modifying
    the master source files in any way. Everybody is supposed to use
    change files instead of changing the files.)
    The author has tried his best to produce correct and useful programs,
    in order to help promote computer science research,
    but no warranty of any kind should be assumed.
    \smallskip\noindent
    Preliminary work on the Stanford GraphBase project
    was supported in part by National Science
    Foundation grant CCR-86-10181.}

\def\prerequisite#1{\def\startsection{\noindent
    Important: Before reading {\sc\title},
    please read or at least skim the program for {\sc#1}.\bigskip
    \let\startsection=\stsec\stsec}}
\def\prerequisites#1#2{\def\startsection{\noindent
    Important: Before reading {\sc\title}, please read
    or at least skim the programs for {\sc#1} and {\sc#2}.\bigskip
    \let\startsection=\stsec\stsec}}



\def\title{GB\_\,PLANE}

\prerequisite{GB\_\,MILES}

\N{1}{1}Introduction. This GraphBase module contains the \PB{\\{plane}}
subroutine,
which constructs undirected planar graphs from vertices located randomly
in a rectangle,
as well as the \PB{\\{plane\_miles}} routine, which constructs planar graphs
based on the mileage and coordinate data in \.{miles.dat}. Both
routines use a general-purpose \PB{\\{delaunay}} subroutine,
which computes the Delaunay triangulation of a given set of points.

\Y\B\4\D$\\{plane\_miles}$ \5
\\{p\_miles}\C{ abbreviation for Procrustean external linkage }\par
\Y\B\4\X1:\.{gb\_plane.h\,}\X${}\E{}$\6
\8\#\&{define} \\{plane\_miles}\5\\{p\_miles}\6
\&{extern} \&{Graph} ${}{*}\\{plane}(\,);{}$\6
\&{extern} \&{Graph} ${}{*}\\{plane\_miles}(\,);{}$\6
\&{extern} \&{void} \\{delaunay}(\,);\par
\As2\ET7.\fi

\M{2}The subroutine call \PB{$\\{plane}(\|n,\\{x\_range},\\{y\_range},%
\\{extend},\\{prob},\\{seed})$} constructs
a planar graph whose vertices have integer coordinates
uniformly distributed in the rectangle
$$\{\,(x,y)\;\mid\;0\le x<\PB{\\{x\_range}}, \;0\le y<\PB{\\{y\_range}}\,\}%
\,.$$
The values of \PB{\\{x\_range}} and \PB{\\{y\_range}} must be at most
$2^{14}=16384$; the
latter value is the default, which is substituted if \PB{\\{x\_range}} or \PB{%
\\{y\_range}}
is given as zero. If \PB{$\\{extend}\E\T{0}$}, the graph will have \PB{\|n}
vertices; otherwise
it will have \PB{$\|n+\T{1}$} vertices, where the \PB{$(\|n+\T{1})$}st is
assigned the coordinates
$(-1,-1)$ and may be regarded as a point at~$\infty$.
Some of the \PB{\|n}~finite vertices might have identical coordinates,
particularly
if the point density \PB{$\|n/(\\{x\_range}*\\{y\_range})$} is not very small.

The subroutine works by first constructing the Delaunay triangulation
of the points, then discarding
each edge of the resulting graph with probability \PB{$\\{prob}/\T{65536}$}.
Thus,
for example, if \PB{\\{prob}} is zero the full Delaunay triangulation will be
returned; if \PB{$\\{prob}\E\T{32768}$}, about half of the Delaunay edges will
remain.
Each finite edge is assigned a length equal to the Euclidean distance between
points, multiplied by $2^{10}$ and
rounded to the nearest integer. If \PB{$\\{extend}\I\T{0}$}, the
Delaunay triangulation will also contain edges between $\infty$ and
all points of the convex hull; such edges, if not discarded, are
assigned length $2^{28}$, otherwise known as \PB{\.{INFTY}}.

If \PB{$\\{extend}\I\T{0}$} and \PB{$\\{prob}\E\T{0}$}, the graph will have
$n+1$ vertices and
$3(n-1)$ edges; this is the maximum number of edges that a planar graph
on $n+1$ vertices can have. In such a case the average degree of a vertex will
be $6(n-1)/(n+1)$, slightly less than~6; hence, if \PB{$\\{prob}\E\T{32768}$},
the average degree of a vertex will usually be near~3.

As with all other GraphBase routines that rely on random numbers,
different values of \PB{\\{seed}} will produce different graphs, in a
machine-independent fashion that is reproducible on many different
computers. Any \PB{\\{seed}} value between 0 and $2^{31}-1$ is permissible.

\Y\B\4\D$\.{INFTY}$ \5
\T{\^10000000\$L}\C{ ``infinite'' length }\par
\Y\B\4\X1:\.{gb\_plane.h\,}\X${}\mathrel+\E{}$\6
\8\#\&{define} \.{INFTY}\5\hbox{\quad}\T{\^10000000\$L}\par
\fi

\M{3}If the \PB{\\{plane}} routine encounters a problem, it returns \PB{$%
\NULL$}
(\.{NULL}), after putting a code number into the external variable
\PB{\\{panic\_code}}. This code number identifies the type of failure.
Otherwise \PB{\\{plane}} returns a pointer to the newly created graph, which
will be represented with the data structures explained in {\sc GB\_\,GRAPH}.
(The external variable \PB{\\{panic\_code}} is itself defined in {\sc GB\_%
\,GRAPH}.)

\Y\B\4\D$\\{panic}(\|c)$ \5
${}\{{}$\5
\1${}\\{panic\_code}\K\|c{}$;\5
${}\\{gb\_trouble\_code}\K\T{0}{}$;\5
\&{return} ${}\NULL{}$;\5
${}\}{}$\2\par
\fi

\M{4}Here is the overall shape of the \CEE/ file \.{gb\_plane.c}\kern.2em:

\Y\B\8\#\&{include} \.{"gb\_flip.h"}\C{ we will use the {\sc GB\_\,FLIP}
routines for random numbers }\6
\8\#\&{include} \.{"gb\_graph.h"}\C{ we will use the {\sc GB\_\,GRAPH} data
structures }\6
\8\#\&{include} \.{"gb\_miles.h"}\C{ and we might use {\sc GB\_\,MILES} for
mileage data }\6
\8\#\&{include} \.{"gb\_io.h"}\C{ and {\sc GB\_\,MILES} uses {\sc GB\_\,IO},
which has \PB{\\{str\_buf}} }\6
\ATH\7
\X25:Type declarations\X\6
\X10:Global variables\X\6
\X13:Subroutines for arithmetic\X\6
\X12:Other subroutines\X\6
\X9:The \PB{\\{delaunay}} routine\X\6
\X5:The \PB{\\{plane}} routine\X\6
\X41:The \PB{\\{plane\_miles}} routine\X\par
\fi

\M{5}\B\X5:The \PB{\\{plane}} routine\X${}\E{}$\6
\1\1\&{Graph} ${}{*}\\{plane}(\|n,\39\\{x\_range},\39\\{y\_range},\39%
\\{extend},\39\\{prob},\39\\{seed}){}$\6
\&{unsigned} \&{long} \|n;\C{ number of vertices desired }\6
\&{unsigned} \&{long} \\{x\_range}${},\39\\{y\_range}{}$;\C{ upper bounds on
rectangular coordinates }\6
\&{unsigned} \&{long} \\{extend};\C{ should a point at infinity be included? }\6
\&{unsigned} \&{long} \\{prob};\C{ probability of rejecting a Delaunay edge }\6
\&{long} \\{seed};\C{ random number seed }\2\2\6
${}\{{}$\5
\1\&{Graph} ${}{*}\\{new\_graph}{}$;\C{ the graph constructed by \PB{\\{plane}}
}\6
\&{register} \&{Vertex} ${}{*}\|v{}$;\C{ the current vertex of interest }\6
\&{register} \&{long} \|k;\C{ the canonical all-purpose index }\7
\\{gb\_init\_rand}(\\{seed});\6
\&{if} ${}(\\{x\_range}>\T{16384}\V\\{y\_range}>\T{16384}){}$\1\5
\\{panic}(\\{bad\_specs});\C{ range too large }\2\6
\&{if} ${}(\|n<\T{2}){}$\1\5
\\{panic}(\\{very\_bad\_specs});\C{ don't make \PB{\|n} so small, you fool }\2\6
\&{if} ${}(\\{x\_range}\E\T{0}){}$\1\5
${}\\{x\_range}\K\T{16384}{}$;\C{ default }\2\6
\&{if} ${}(\\{y\_range}\E\T{0}){}$\1\5
${}\\{y\_range}\K\T{16384}{}$;\C{ default }\2\6
\X6:Set up a graph with \PB{\|n} uniformly distributed vertices\X;\6
\X11:Compute the Delaunay triangulation and run through the Delaunay edges;
reject them with probability \PB{$\\{prob}/\T{65536}$}, otherwise append them
with their Euclidean length\X;\6
\&{if} (\\{gb\_trouble\_code})\5
${}\{{}$\1\6
\\{gb\_recycle}(\\{new\_graph});\6
\\{panic}(\\{alloc\_fault});\C{ oops, we ran out of memory somewhere back there
}\6
\4${}\}{}$\2\6
\&{if} (\\{extend})\1\5
${}\\{new\_graph}\MG\|n\PP{}$;\C{ make the ``infinite'' vertex legitimate }\2\6
\&{return} \\{new\_graph};\6
\4${}\}{}$\2\par
\U4.\fi

\M{6}The coordinates are placed into utility fields \PB{\\{x\_coord}} and \PB{%
\\{y\_coord}}.
A random ID number is also stored in utility field~\PB{\\{z\_coord}}; this
number is
used by the \PB{\\{delaunay}} subroutine to break ties when points are equal or
collinear or cocircular. No two vertices have the same ID number.
(The header file \.{gb\_miles.h} defines \PB{\\{x\_coord}}, \PB{\\{y\_coord}},
and
\PB{\\{index\_no}} to be \PB{$\|x.\|I$}, \PB{$\|y.\|I$}, and \PB{$\|z.\|I$}
respectively.)

\Y\B\4\D$\\{z\_coord}$ \5
$\|z.{}$\|I\par
\Y\B\4\X6:Set up a graph with \PB{\|n} uniformly distributed vertices\X${}\E{}$%
\6
\&{if} (\\{extend})\1\5
${}\\{extra\_n}\PP{}$;\C{ allocate one more vertex than usual }\2\6
${}\\{new\_graph}\K\\{gb\_new\_graph}(\|n);{}$\6
\&{if} ${}(\\{new\_graph}\E\NULL){}$\1\5
\\{panic}(\\{no\_room});\C{ out of memory before we're even started }\2\6
${}\\{sprintf}(\\{new\_graph}\MG\\{id},\39\.{"plane(\%lu,\%lu,\%lu,\%}\)\.{lu,%
\%lu,\%ld)"},\39\|n,\39\\{x\_range},\39\\{y\_range},\39\\{extend},\39\\{prob},%
\39\\{seed});{}$\6
${}\\{strcpy}(\\{new\_graph}\MG\\{util\_types},\39\.{"ZZZIIIZZZZZZZZ"});{}$\6
\&{for} ${}(\|k\K\T{0},\39\|v\K\\{new\_graph}\MG\\{vertices};{}$ ${}\|k<\|n;{}$
${}\|k\PP,\39\|v\PP){}$\5
${}\{{}$\1\6
${}\|v\MG\\{x\_coord}\K\\{gb\_unif\_rand}(\\{x\_range});{}$\6
${}\|v\MG\\{y\_coord}\K\\{gb\_unif\_rand}(\\{y\_range});{}$\6
${}\|v\MG\\{z\_coord}\K((\&{long})(\\{gb\_next\_rand}(\,)/\|n))*\|n+\|k;{}$\6
${}\\{sprintf}(\\{str\_buf},\39\.{"\%ld"},\39\|k){}$;\5
${}\|v\MG\\{name}\K\\{gb\_save\_string}(\\{str\_buf});{}$\6
\4${}\}{}$\2\6
\&{if} (\\{extend})\5
${}\{{}$\1\6
${}\|v\MG\\{name}\K\\{gb\_save\_string}(\.{"INF"});{}$\6
${}\|v\MG\\{x\_coord}\K\|v\MG\\{y\_coord}\K\|v\MG\\{z\_coord}\K{-}\T{1};{}$\6
${}\\{extra\_n}\MM;{}$\6
\4${}\}{}$\2\par
\U5.\fi

\M{7}\B\X1:\.{gb\_plane.h\,}\X${}\mathrel+\E{}$\6
\8\#\&{define} ${}\\{x\_coord}\hbox{\quad}\|x.\|I{}$\6
\8\#\&{define} ${}\\{y\_coord}\hbox{\quad}\|y.\|I{}$\6
\8\#\&{define} ${}\\{z\_coord}\hbox{\quad}\|z.\|I{}$\par
\fi

\N{1}{8}Delaunay triangulation. The Delaunay triangulation of a set of
vertices in the plane consists of all line segments $uv$ such that
there exists a circle passing through $u$ and~$v$ containing no other
vertices. Equivalently, $uv$ is a Delaunay edge if and only if the
Voronoi regions for $u$ and $v$ are adjacent; the Voronoi region of a
vertex~$u$ is the polygon with the property that all points inside it
are closer to $u$ than to any other vertex. In this sense, we can say
that Delaunay edges connect vertices with their ``neighbors.''

The definitions in the previous paragraph assume that no two vertices are
equal, that no three vertices lie on a straight line, and that no four vertices
lie on a circle. If those nondegeneracy conditions aren't satisfied, we can
perturb the points very slightly so that the assumptions do hold.

Another way to characterize the Delaunay triangulation is to consider
what happens when we map a given set of
points onto the unit sphere via stereographic projection: Point $(x,y)$ is
mapped to
$$\bigl(2x/(r^2+1),2y/(r^2+1),(r^2-1)/(r^2+1)\bigr)\,,$$ where $r^2=x^2+y^2$.
If we now extend the configuration by adding $(0,0,1)$,
which is the limiting point on the sphere when $r$ approaches infinity,
the Delaunay edges of the original points
turn out to be edges of the polytope defined by the mapped
points. This polytope, which is the 3-dimensional convex hull of $n+1$ points
on the sphere, also has edges from $(0,0,1)$ to the mapped points
that correspond to the 2-dimensional convex hull of the original points. Under
our assumption of nondegeneracy, the faces of this polytope are all
triangles; hence its edges are said to form a triangulation.

A self-contained presentation of all the relevant theory, together with
an exposition and proof of correctness of the algorithm below, can be found
in the author's monograph {\sl Axioms and Hulls}, Lecture Notes in
Computer Science {\bf606} (Springer-Verlag, 1992).
For further references, see Franz Aurenhammer, {\sl ACM Computing Surveys\/
\bf23} (1991), 345--405.

\fi

\M{9}The \PB{\\{delaunay}} procedure, which finds the Delaunay triangulation of
a given set of vertices, is the key ingredient in \\{gb\_plane}'s
algorithms for generating planar graphs. The given vertices should
appear in a GraphBase graph~\PB{\|g} whose edges, if any, are ignored by
\PB{\\{delaunay}}. The coordinates of each vertex appear in utility fields
\PB{\\{x\_coord}} and~\PB{\\{y\_coord}}, which must be nonnegative and less
than
$2^{14}=16384$. The utility field~\PB{\\{z\_coord}} must contain a unique ID
number, distinct for every vertex, so that the algorithm can break
ties in cases of degeneracy. (Note: These assumptions about the input
data are the responsibility of the calling procedure; \PB{\\{delaunay}} does
not
double-check them. If they are violated, catastrophic failure is possible.)

Instead of returning the Delaunay triangulation as a graph, \PB{\\{delaunay}}
communicates its answer implicitly by performing the procedure call
\PB{$\|f(\|u,\|v)$} on every pair of vertices \PB{\|u} and~\PB{\|v} joined by a
Delaunay edge.
Here \PB{\|f}~is a procedure supplied as a parameter; \PB{\|u} and~\PB{\|v} are
either
pointers to vertices or \PB{$\NULL$} (i.e., \.{NULL}), where \PB{$\NULL$}
denotes the
vertex ``$\infty$.'' As remarked above, edges run between $\infty$ and all
vertices on the convex hull of the given points. The graph of all edges,
including the infinite edges, is planar.

For example, if the vertex at infinity is being ignored, the user can
declare
$$\vcenter{\halign{#\hfil\cr
\PB{\&{void} ${}\\{ins\_finite}(\|u,\|v)$}\cr
\qquad\PB{\&{Vertex} ${}{*}\|u,{*}\|v;$}\cr
\PB{ ${}\{{}$ \1\&{if} ${}(\|u\W\|v{}$) \1${}\\{gb\_new\_edge}(\|u,\|v,\T{1%
\$L}){}$; \2${}\}{}$}\cr}}$$
Then the procedure call \PB{$\\{delaunay}(\|g,\\{ins\_finite})$} will add all
the finite
Delaunay edges to the current graph~\PB{\|g}, giving them all length~1.

If \PB{\\{delaunay}} is unable to allocate enough storage to do its work, it
will set \PB{\\{gb\_trouble\_code}} nonzero and there will be no edges in
the triangulation.

\Y\B\4\X9:The \PB{\\{delaunay}} routine\X${}\E{}$\6
\1\1\&{void} ${}\\{delaunay}(\|g,\39\|f){}$\6
\&{Graph} ${}{*}\|g{}$;\C{ vertices in the plane }\6
\&{void} ${}({*}\|f)(\,){}$;\C{ procedure that absorbs the triangulated edges }%
\2\2\6
${}\{{}$\5
\1\X26:Local variables for \PB{\\{delaunay}}\X;\7
\X34:Find the Delaunay triangulation of \PB{\|g}, or return with \PB{\\{gb%
\_trouble\_code}} nonzero if out of memory\X;\6
\X28:Call \PB{$\|f(\|u,\|v)$} for each Delaunay edge \\{uv}\X;\6
\\{gb\_free}(\\{working\_storage});\6
\4${}\}{}$\2\par
\U4.\fi

\M{10}The procedure passed to \PB{\\{delaunay}} will communicate with \PB{%
\\{plane}} via
global variables called \PB{\\{gprob}} and \PB{\\{inf\_vertex}}.

\Y\B\4\X10:Global variables\X${}\E{}$\6
\&{static} \&{unsigned} \&{long} \\{gprob};\C{ copy of the \PB{\\{prob}}
parameter }\6
\&{static} \&{Vertex} ${}{*}\\{inf\_vertex}{}$;\C{ pointer to the vertex $%
\infty$, or \PB{$\NULL$} }\par
\U4.\fi

\M{11}\B\X11:Compute the Delaunay triangulation and run through the Delaunay
edges; reject them with probability \PB{$\\{prob}/\T{65536}$}, otherwise append
them with their Euclidean length\X${}\E{}$\6
$\\{gprob}\K\\{prob};{}$\6
\&{if} (\\{extend})\1\5
${}\\{inf\_vertex}\K\\{new\_graph}\MG\\{vertices}+\|n;{}$\2\6
\&{else}\1\5
${}\\{inf\_vertex}\K\NULL;{}$\2\6
${}\\{delaunay}(\\{new\_graph},\39\\{new\_euclid\_edge}){}$;\par
\U5.\fi

\M{12}\B\X12:Other subroutines\X${}\E{}$\6
\1\1\&{static} \&{void} ${}\\{new\_euclid\_edge}(\|u,\39\|v){}$\6
\&{Vertex} ${}{*}\|u,\39{*}\|v;\2\2{}$\6
${}\{{}$\5
\1\&{register} \&{long} \\{dx}${},\39\\{dy};{}$\7
\&{if} ${}((\\{gb\_next\_rand}(\,)\GG\T{15})\G\\{gprob}){}$\5
${}\{{}$\1\6
\&{if} (\|u)\5
${}\{{}$\1\6
\&{if} (\|v)\5
${}\{{}$\1\6
${}\\{dx}\K\|u\MG\\{x\_coord}-\|v\MG\\{x\_coord};{}$\6
${}\\{dy}\K\|u\MG\\{y\_coord}-\|v\MG\\{y\_coord};{}$\6
${}\\{gb\_new\_edge}(\|u,\39\|v,\39\\{int\_sqrt}(\\{dx}*\\{dx}+\\{dy}*%
\\{dy}));{}$\6
\4${}\}{}$\5
\2\&{else} \&{if} (\\{inf\_vertex})\1\5
${}\\{gb\_new\_edge}(\|u,\39\\{inf\_vertex},\39\.{INFTY});{}$\2\6
\4${}\}{}$\5
\2\&{else} \&{if} (\\{inf\_vertex})\1\5
${}\\{gb\_new\_edge}(\\{inf\_vertex},\39\|v,\39\.{INFTY});{}$\2\6
\4${}\}{}$\2\6
\4${}\}{}$\2\par
\As20, 21, 40\ETs44.
\U4.\fi

\N{1}{13}Arithmetic. Before we lunge into the world of geometric algorithms,
let's build up some confidence by polishing off some subroutines that
will be needed to ensure correct results. We assume that \PB{\&{long}} integers
are less than $2^{31}$.

First is a routine to calculate $s=\lfloor2^{10}\sqrt x+{1\over2}\rfloor$,
the nearest integer to $2^{10}$ times the square root of a given nonnegative
integer~\PB{\|x}. If \PB{$\|x>\T{0}$}, this
is the unique integer such that $2^{20}x-s\le s^2<2^{20}x+s$.

The following routine appears to work by magic, but the mystery goes
away when one considers the invariant relations
$$ m=\lfloor 2^{2k-21}\rfloor,\qquad
0<y=\lfloor 2^{20-2k}x\rfloor-s^2+s\le q=2s.$$
(Exception: We might actually have $y=0$ for a short time when \PB{$\|q\K%
\T{2}$}.)

\Y\B\4\X13:Subroutines for arithmetic\X${}\E{}$\6
\1\1\&{static} \&{long} \\{int\_sqrt}(\|x)\6
\&{long} \|x;\2\2\6
${}\{{}$\5
\1\&{register} \&{long} \|y${},\39\|m,\39\|q\K\T{2};{}$\6
\&{long} \|k;\7
\&{if} ${}(\|x\Z\T{0}){}$\1\5
\&{return} \T{0};\2\6
\&{for} ${}(\|k\K\T{25},\39\|m\K\T{\^20000000};{}$ ${}\|x<\|m;{}$ ${}\|k\MM,\39%
\|m\MRL{{\GG}{\K}}\T{2}){}$\1\5
;\C{ find the range }\2\6
\&{if} ${}(\|x\G\|m+\|m){}$\1\5
${}\|y\K\T{1};{}$\2\6
\&{else}\1\5
${}\|y\K\T{0};{}$\2\6
\&{do}\5
\X14:Decrease \PB{\|k} by 1, maintaining the invariant relations between \PB{%
\|x}, \PB{\|y}, \PB{\|m}, and \PB{\|q}\X\5
\&{while} (\|k);\6
\&{return} \|q${}\GG\T{1};{}$\6
\4${}\}{}$\2\par
\As15\ET24.
\U4.\fi

\M{14}\B\X14:Decrease \PB{\|k} by 1, maintaining the invariant relations
between \PB{\|x}, \PB{\|y}, \PB{\|m}, and \PB{\|q}\X${}\E{}$\6
${}\{{}$\1\6
\&{if} ${}(\|x\AND\|m){}$\1\5
${}\|y\MRL{+{\K}}\|y+\T{1};{}$\2\6
\&{else}\1\5
${}\|y\MRL{+{\K}}\|y;{}$\2\6
${}\|m\MRL{{\GG}{\K}}\T{1};{}$\6
\&{if} ${}(\|x\AND\|m){}$\1\5
${}\|y\MRL{+{\K}}\|y-\|q+\T{1};{}$\2\6
\&{else}\1\5
${}\|y\MRL{+{\K}}\|y-\|q;{}$\2\6
${}\|q\MRL{+{\K}}\|q;{}$\6
\&{if} ${}(\|y>\|q){}$\1\5
${}\|y\MRL{-{\K}}\|q,\39\|q\MRL{+{\K}}\T{2};{}$\2\6
\&{else} \&{if} ${}(\|y\Z\T{0}){}$\1\5
${}\|q\MRL{-{\K}}\T{2},\39\|y\MRL{+{\K}}\|q;{}$\2\6
${}\|m\MRL{{\GG}{\K}}\T{1};{}$\6
${}\|k\MM;{}$\6
\4${}\}{}$\2\par
\U13.\fi

\M{15}We are going to need multiple-precision arithmetic in order to
calculate certain geometric predicates properly, but it turns out
that we do not need to implement general-purpose subroutines for bignums.
It suffices to have a single special-purpose routine called
$\PB{\\{sign\_test}}(\PB{\\{x1}},\PB{\\{x2}},\PB{\\{x3}},\allowbreak\PB{%
\\{y1}},\PB{\\{y2}},\PB{\\{y3}})$, which
computes a single-precision integer having the same sign as the dot product
$$\hbox{\PB{$\\{x1}*\\{y1}+\\{x2}*\\{y2}+\\{x3}*\\{y3}$}}$$
when we have $-2^{29}<\PB{\\{x1}},\PB{\\{x2}},\PB{\\{x3}}<2^{29}$ and $0\le\PB{%
\\{y1}},\PB{\\{y2}},\PB{\\{y3}}<2^{29}$.

\Y\B\4\X13:Subroutines for arithmetic\X${}\mathrel+\E{}$\6
\1\1\&{static} \&{long} ${}\\{sign\_test}(\\{x1},\39\\{x2},\39\\{x3},\39\\{y1},%
\39\\{y2},\39\\{y3}){}$\6
\&{long} \\{x1}${},\39\\{x2},\39\\{x3},\39\\{y1},\39\\{y2},\39\\{y3};\2\2{}$\6
${}\{{}$\5
\1\&{long} \\{s1}${},\39\\{s2},\39\\{s3}{}$;\C{ signs of individual terms }\6
\&{long} \|a${},\39\|b,\39\|c{}$;\C{ components of a redundant representation
of the dot product }\6
\&{register} \&{long} \|t;\C{ temporary register for swapping }\7
\X16:Determine the signs of the terms\X;\6
\X17:If the answer is obvious, return it without further ado; otherwise,
arrange things so that \PB{$\\{x3}*\\{y3}$} has the opposite sign to \PB{$%
\\{x1}*\\{y1}+\\{x2}*\\{y2}$}\X;\6
\X18:Compute a redundant representation of \PB{$\\{x1}*\\{y1}+\\{x2}*\\{y2}+%
\\{x3}*\\{y3}$}\X;\6
\X19:Return the sign of the redundant representation\X;\6
\4${}\}{}$\2\par
\fi

\M{16}\B\X16:Determine the signs of the terms\X${}\E{}$\6
\&{if} ${}(\\{x1}\E\T{0}\V\\{y1}\E\T{0}){}$\1\5
${}\\{s1}\K\T{0};{}$\2\6
\&{else}\5
${}\{{}$\1\6
\&{if} ${}(\\{x1}>\T{0}){}$\1\5
${}\\{s1}\K\T{1};{}$\2\6
\&{else}\1\5
${}\\{x1}\K{-}\\{x1},\39\\{s1}\K{-}\T{1};{}$\2\6
\4${}\}{}$\2\6
\&{if} ${}(\\{x2}\E\T{0}\V\\{y2}\E\T{0}){}$\1\5
${}\\{s2}\K\T{0};{}$\2\6
\&{else}\5
${}\{{}$\1\6
\&{if} ${}(\\{x2}>\T{0}){}$\1\5
${}\\{s2}\K\T{1};{}$\2\6
\&{else}\1\5
${}\\{x2}\K{-}\\{x2},\39\\{s2}\K{-}\T{1};{}$\2\6
\4${}\}{}$\2\6
\&{if} ${}(\\{x3}\E\T{0}\V\\{y3}\E\T{0}){}$\1\5
${}\\{s3}\K\T{0};{}$\2\6
\&{else}\5
${}\{{}$\1\6
\&{if} ${}(\\{x3}>\T{0}){}$\1\5
${}\\{s3}\K\T{1};{}$\2\6
\&{else}\1\5
${}\\{x3}\K{-}\\{x3},\39\\{s3}\K{-}\T{1};{}$\2\6
\4${}\}{}$\2\par
\U15.\fi

\M{17}The answer is obvious unless one of the terms is positive and one
of the terms is negative.

\Y\B\4\X17:If the answer is obvious, return it without further ado; otherwise,
arrange things so that \PB{$\\{x3}*\\{y3}$} has the opposite sign to \PB{$%
\\{x1}*\\{y1}+\\{x2}*\\{y2}$}\X${}\E{}$\6
\&{if} ${}((\\{s1}\G\T{0}\W\\{s2}\G\T{0}\W\\{s3}\G\T{0})\V(\\{s1}\Z\T{0}\W%
\\{s2}\Z\T{0}\W\\{s3}\Z\T{0})){}$\1\5
\&{return} ${}(\\{s1}+\\{s2}+\\{s3});{}$\2\6
\&{if} ${}(\\{s3}\E\T{0}\V\\{s3}\E\\{s1}){}$\5
${}\{{}$\1\6
${}\|t\K\\{s3}{}$;\5
${}\\{s3}\K\\{s2}{}$;\5
${}\\{s2}\K\|t;{}$\6
${}\|t\K\\{x3}{}$;\5
${}\\{x3}\K\\{x2}{}$;\5
${}\\{x2}\K\|t;{}$\6
${}\|t\K\\{y3}{}$;\5
${}\\{y3}\K\\{y2}{}$;\5
${}\\{y2}\K\|t;{}$\6
\4${}\}{}$\5
\2\&{else} \&{if} ${}(\\{s3}\E\\{s2}){}$\5
${}\{{}$\1\6
${}\|t\K\\{s3}{}$;\5
${}\\{s3}\K\\{s1}{}$;\5
${}\\{s1}\K\|t;{}$\6
${}\|t\K\\{x3}{}$;\5
${}\\{x3}\K\\{x1}{}$;\5
${}\\{x1}\K\|t;{}$\6
${}\|t\K\\{y3}{}$;\5
${}\\{y3}\K\\{y1}{}$;\5
${}\\{y1}\K\|t;{}$\6
\4${}\}{}$\2\par
\U15.\fi

\M{18}We make use of a redundant representation $2^{28}a+2^{14}b+c$, which
can be computed by brute force. (Everything is understood to be multiplied
by \PB{${-}\\{s3}$}.)

\Y\B\4\X18:Compute a redundant representation of \PB{$\\{x1}*\\{y1}+\\{x2}*%
\\{y2}+\\{x3}*\\{y3}$}\X${}\E{}$\6
${}\{{}$\5
\1\&{register} \&{long} \\{lx}${},\39\\{rx},\39\\{ly},\39\\{ry};{}$\7
${}\\{lx}\K\\{x1}/\T{\^4000}{}$;\5
${}\\{rx}\K\\{x1}\MOD\T{\^4000}{}$;\C{ split off the least significant 14 bits
}\6
${}\\{ly}\K\\{y1}/\T{\^4000}{}$;\5
${}\\{ry}\K\\{y1}\MOD\T{\^4000};{}$\6
${}\|a\K\\{lx}*\\{ly}{}$;\5
${}\|b\K\\{lx}*\\{ry}+\\{ly}*\\{rx}{}$;\5
${}\|c\K\\{rx}*\\{ry};{}$\6
${}\\{lx}\K\\{x2}/\T{\^4000}{}$;\5
${}\\{rx}\K\\{x2}\MOD\T{\^4000};{}$\6
${}\\{ly}\K\\{y2}/\T{\^4000}{}$;\5
${}\\{ry}\K\\{y2}\MOD\T{\^4000};{}$\6
${}\|a\MRL{+{\K}}\\{lx}*\\{ly}{}$;\5
${}\|b\MRL{+{\K}}\\{lx}*\\{ry}+\\{ly}*\\{rx}{}$;\5
${}\|c\MRL{+{\K}}\\{rx}*\\{ry};{}$\6
${}\\{lx}\K\\{x3}/\T{\^4000}{}$;\5
${}\\{rx}\K\\{x3}\MOD\T{\^4000};{}$\6
${}\\{ly}\K\\{y3}/\T{\^4000}{}$;\5
${}\\{ry}\K\\{y3}\MOD\T{\^4000};{}$\6
${}\|a\MRL{-{\K}}\\{lx}*\\{ly}{}$;\5
${}\|b\MRL{-{\K}}\\{lx}*\\{ry}+\\{ly}*\\{rx}{}$;\5
${}\|c\MRL{-{\K}}\\{rx}*\\{ry};{}$\6
\4${}\}{}$\2\par
\U15.\fi

\M{19}Here we use the fact that $\vert c\vert<2^{29}$.

\Y\B\4\X19:Return the sign of the redundant representation\X${}\E{}$\6
\&{if} ${}(\|a\E\T{0}){}$\1\5
\&{goto} \\{ez};\2\6
\&{if} ${}(\|a<\T{0}){}$\1\5
${}\|a\K{-}\|a,\39\|b\K{-}\|b,\39\|c\K{-}\|c,\39\\{s3}\K{-}\\{s3};{}$\2\6
\&{while} ${}(\|c<\T{0}){}$\5
${}\{{}$\1\6
${}\|a\MM{}$;\5
${}\|c\MRL{+{\K}}\T{\^10000000};{}$\6
\&{if} ${}(\|a\E\T{0}){}$\1\5
\&{goto} \\{ez};\2\6
\4${}\}{}$\2\6
\&{if} ${}(\|b\G\T{0}){}$\1\5
\&{return} ${}{-}\\{s3}{}$;\C{ the answer is clear when \PB{$\|a>\T{0}\W\|b\G%
\T{0}\W\|c\G\T{0}$} }\2\6
${}\|b\K{-}\|b;{}$\6
${}\|a\MRL{-{\K}}\|b/\T{\^4000};{}$\6
\&{if} ${}(\|a>\T{0}){}$\1\5
\&{return} ${}{-}\\{s3};{}$\2\6
\&{if} ${}(\|a\Z{-}\T{2}){}$\1\5
\&{return} \\{s3};\2\6
\&{return} ${}{-}\\{s3}*((\|a*\T{\^4000}-\|b\MOD\T{\^4000})*\T{\^4000}+\|c);{}$%
\6
\4\\{ez}:\5
\&{if} ${}(\|b\G\T{\^8000}){}$\1\5
\&{return} ${}{-}\\{s3};{}$\2\6
\&{if} ${}(\|b\Z{-}\T{\^8000}){}$\1\5
\&{return} \\{s3};\2\6
\&{return} ${}{-}\\{s3}*(\|b*\T{\^4000}+\|c){}$;\par
\U15.\fi

\N{1}{20}Determinants. The \PB{\\{delaunay}} routine bases all of its decisions
on
two geometric predicates, which depend on whether certain determinants
are positive or negative.

The first predicate, \PB{$\\{ccw}(\|u,\|v,\|w)$}, is true if and only if the
three points
$(u,v,w)$ have a counterclockwise orientation. This means that if we draw the
unique circle through those points, and if we travel along that circle
in the counterclockwise direction starting at~\PB{\|u}, we will encounter
\PB{\|v} before~\PB{\|w}.

It turns out that \PB{$\\{ccw}(\|u,\|v,\|w)$} holds if and only if the
determinant
$$\left\vert\matrix{x_u&y_u&1\cr x_v&y_v&1\cr x_w&y_w&1\cr}
\right\vert=\left\vert\matrix{x_u-x_w&y_u-y_w\cr x_v-x_w&y_v-y_w\cr}
\right\vert$$
is positive. The evaluation must be exact; if the answer is zero, a special
tie-breaking rule must be used because the three points were collinear.
The tie-breaking rule is tricky (and necessarily so, according to the
theory in {\sl Axioms and Hulls\/}).

Integer evaluation of that determinant will not cause \PB{\&{long}} integer
overflow, because we have assumed that all \PB{\|x} and \PB{\|y} coordinates
lie
between 0 and~$2^{14}-1$, inclusive. In fact, we could go up to
$2^{15}-1$ without risking overflow; but the limitation to 14 bits will
be helpful when we consider a more complicated determinant below.

\Y\B\4\X12:Other subroutines\X${}\mathrel+\E{}$\6
\1\1\&{static} \&{long} ${}\\{ccw}(\|u,\39\|v,\39\|w){}$\6
\&{Vertex} ${}{*}\|u,\39{*}\|v,\39{*}\|w;\2\2{}$\6
${}\{{}$\5
\1\&{register} \&{long} \\{wx}${}\K\|w\MG\\{x\_coord},\39\\{wy}\K\|w\MG\\{y%
\_coord}{}$;\C{ $x_w$, $y_w$ }\6
\&{register} \&{long} \\{det}${}\K(\|u\MG\\{x\_coord}-\\{wx})*(\|v\MG\\{y%
\_coord}-\\{wy})-(\|u\MG\\{y\_coord}-\\{wy})*(\|v\MG\\{x\_coord}-\\{wx});{}$\6
\&{Vertex} ${}{*}\|t;{}$\7
\&{if} ${}(\\{det}\E\T{0}){}$\5
${}\{{}$\1\6
${}\\{det}\K\T{1};{}$\6
\&{if} ${}(\|u\MG\\{z\_coord}>\|v\MG\\{z\_coord}){}$\5
${}\{{}$\1\6
${}\|t\K\|u{}$;\5
${}\|u\K\|v{}$;\5
${}\|v\K\|t{}$;\5
${}\\{det}\K{-}\\{det};{}$\6
\4${}\}{}$\2\6
\&{if} ${}(\|v\MG\\{z\_coord}>\|w\MG\\{z\_coord}){}$\5
${}\{{}$\1\6
${}\|t\K\|v{}$;\5
${}\|v\K\|w{}$;\5
${}\|w\K\|t{}$;\5
${}\\{det}\K{-}\\{det};{}$\6
\4${}\}{}$\2\6
\&{if} ${}(\|u\MG\\{z\_coord}>\|v\MG\\{z\_coord}){}$\5
${}\{{}$\1\6
${}\|t\K\|u{}$;\5
${}\|u\K\|v{}$;\5
${}\|v\K\|t{}$;\5
${}\\{det}\K{-}\\{det};{}$\6
\4${}\}{}$\2\6
\&{if} ${}(\|u\MG\\{x\_coord}>\|v\MG\\{x\_coord}\V(\|u\MG\\{x\_coord}\E\|v\MG%
\\{x\_coord}\W\3{-1}(\|u\MG\\{y\_coord}>\|v\MG\\{y\_coord}\V(\|u\MG\\{y\_coord}%
\E\|v\MG\\{y\_coord}\W\3{-1}(\|w\MG\\{x\_coord}>\|u\MG\\{x\_coord}\V(\|w\MG\\{x%
\_coord}\E\|u\MG\\{x\_coord}\W\|w\MG\\{y\_coord}\G\|u\MG\\{y\_coord})))))){}$\1%
\5
${}\\{det}\K{-}\\{det};{}$\2\6
\4${}\}{}$\2\6
\&{return} ${}(\\{det}>\T{0});{}$\6
\4${}\}{}$\2\par
\fi

\M{21}The other geometric predicate, \PB{$\\{incircle}(\|t,\|u,\|v,\|w)$}, is
true if and only
if point \PB{\|t} lies outside the circle passing through \PB{\|u}, \PB{\|v},
and~\PB{\|w},
assuming that \PB{$\\{ccw}(\|u,\|v,\|w)$} holds. This predicate makes us work
harder, because
it is equivalent to the sign of a $4\times4$ determinant that requires
twice as much precision:
$$\left\vert\matrix{x_t&y_t&x_t^2+y_t^2&1\cr
x_u&y_u&x_u^2+y_u^2&1\cr
x_v&y_v&x_v^2+y_v^2&1\cr
x_w&y_w&x_w^2+y_w^2&1\cr}\right\vert=
\left\vert\matrix{x_t-x_w&y_t-y_w&(x_t-x_w)^2+(y_t-y_w)^2\cr
x_u-x_w&y_u-y_w&(x_u-x_w)^2+(y_u-y_w)^2\cr
x_v-x_w&y_v-y_w&(x_v-x_w)^2+(y_v-y_w)^2\cr}
\right\vert\,.$$
The sign can, however, be deduced by the \PB{\\{sign\_test}} subroutine we had
the foresight to provide earlier.

\Y\B\4\X12:Other subroutines\X${}\mathrel+\E{}$\6
\1\1\&{static} \&{long} ${}\\{incircle}(\|t,\39\|u,\39\|v,\39\|w){}$\6
\&{Vertex} ${}{*}\|t,\39{*}\|u,\39{*}\|v,\39{*}\|w;\2\2{}$\6
${}\{{}$\5
\1\&{register} \&{long} \\{wx}${}\K\|w\MG\\{x\_coord},\39\\{wy}\K\|w\MG\\{y%
\_coord}{}$;\C{ $x_w$, $y_w$ }\6
\&{long} \\{tx}${}\K\|t\MG\\{x\_coord}-\\{wx},\39\\{ty}\K\|t\MG\\{y\_coord}-%
\\{wy}{}$;\C{ $x_t-x_w$, $y_t-y_w$ }\6
\&{long} \\{ux}${}\K\|u\MG\\{x\_coord}-\\{wx},\39\\{uy}\K\|u\MG\\{y\_coord}-%
\\{wy}{}$;\C{ $x_u-x_w$, $y_u-y_w$ }\6
\&{long} \\{vx}${}\K\|v\MG\\{x\_coord}-\\{wx},\39\\{vy}\K\|v\MG\\{y\_coord}-%
\\{wy}{}$;\C{ $x_v-x_w$, $y_v-y_w$ }\6
\&{register} \&{long} \\{det}${}\K\\{sign\_test}(\\{tx}*\\{uy}-\\{ty}*\\{ux},%
\39\\{ux}*\\{vy}-\\{uy}*\\{vx},\39\\{vx}*\\{ty}-\\{vy}*\\{tx},\3{-1}\39\\{vx}*%
\\{vx}+\\{vy}*\\{vy},\39\\{tx}*\\{tx}+\\{ty}*\\{ty},\39\\{ux}*\\{ux}+\\{uy}*%
\\{uy});{}$\6
\&{Vertex} ${}{*}\|s;{}$\7
\&{if} ${}(\\{det}\E\T{0}){}$\5
${}\{{}$\1\6
\X22:Sort \PB{$(\|t,\|u,\|v,\|w)$} by ID number\X;\6
\X23:Remove incircle degeneracy\X;\6
\4${}\}{}$\2\6
\&{return} ${}(\\{det}>\T{0});{}$\6
\4${}\}{}$\2\par
\fi

\M{22}\B\X22:Sort \PB{$(\|t,\|u,\|v,\|w)$} by ID number\X${}\E{}$\6
$\\{det}\K\T{1};{}$\6
\&{if} ${}(\|t\MG\\{z\_coord}>\|u\MG\\{z\_coord}){}$\5
${}\{{}$\1\6
${}\|s\K\|t{}$;\5
${}\|t\K\|u{}$;\5
${}\|u\K\|s{}$;\5
${}\\{det}\K{-}\\{det};{}$\6
\4${}\}{}$\2\6
\&{if} ${}(\|v\MG\\{z\_coord}>\|w\MG\\{z\_coord}){}$\5
${}\{{}$\1\6
${}\|s\K\|v{}$;\5
${}\|v\K\|w{}$;\5
${}\|w\K\|s{}$;\5
${}\\{det}\K{-}\\{det};{}$\6
\4${}\}{}$\2\6
\&{if} ${}(\|t\MG\\{z\_coord}>\|v\MG\\{z\_coord}){}$\5
${}\{{}$\1\6
${}\|s\K\|t{}$;\5
${}\|t\K\|v{}$;\5
${}\|v\K\|s{}$;\5
${}\\{det}\K{-}\\{det};{}$\6
\4${}\}{}$\2\6
\&{if} ${}(\|u\MG\\{z\_coord}>\|w\MG\\{z\_coord}){}$\5
${}\{{}$\1\6
${}\|s\K\|u{}$;\5
${}\|u\K\|w{}$;\5
${}\|w\K\|s{}$;\5
${}\\{det}\K{-}\\{det};{}$\6
\4${}\}{}$\2\6
\&{if} ${}(\|u\MG\\{z\_coord}>\|v\MG\\{z\_coord}){}$\5
${}\{{}$\1\6
${}\|s\K\|u{}$;\5
${}\|u\K\|v{}$;\5
${}\|v\K\|s{}$;\5
${}\\{det}\K{-}\\{det};{}$\6
\4${}\}{}$\2\par
\U21.\fi

\M{23}By slightly perturbing the points, we can always make them nondegenerate,
although the details are complicated. A sequence of 12 steps, involving
up to four auxiliary functions
$$\openup3\jot
\eqalign{\PB{\\{ff}}(t,u,v,w)&=\left\vert
\matrix{x_t-x_v&(x_t-x_w)^2+(y_t-y_w)^2-(x_v-x_w)^2-(y_v-y_w)^2\cr
x_u-x_v&(x_u-x_w)^2+(y_u-y_w)^2-(x_v-x_w)^2-(y_v-y_w)^2\cr}
\right\vert\,,\cr
\PB{\\{gg}}(t,u,v,w)&=\left\vert
\matrix{y_t-y_v&(x_t-x_w)^2+(y_t-y_w)^2-(x_v-x_w)^2-(y_v-y_w)^2\cr
y_u-y_v&(x_u-x_w)^2+(y_u-y_w)^2-(x_v-x_w)^2-(y_v-y_w)^2\cr}
\right\vert\,,\cr
\PB{\\{hh}}(t,u,v,w)&=(x_u-x_t)(y_v-y_w)\,,\cr
\PB{\\{jj}}(t,u,v,w)&=(x_u-x_v)^2+(y_u-y_w)^2-(x_t-x_v)^2-(y_t-y_w)^2\,,\cr}
$$
does the trick, as explained in {\sl Axioms and Hulls}.

\Y\B\4\X23:Remove incircle degeneracy\X${}\E{}$\6
${}\{{}$\5
\1\&{long} \\{dd};\7
\&{if} ${}((\\{dd}\K\\{ff}(\|t,\39\|u,\39\|v,\39\|w))<\T{0}\V(\\{dd}\E\T{0}\W%
\3{-1}((\\{dd}\K\\{gg}(\|t,\39\|u,\39\|v,\39\|w))<\T{0}\V(\\{dd}\E\T{0}\W%
\3{-1}((\\{dd}\K\\{ff}(\|u,\39\|t,\39\|w,\39\|v))<\T{0}\V(\\{dd}\E\T{0}\W%
\3{-1}((\\{dd}\K\\{gg}(\|u,\39\|t,\39\|w,\39\|v))<\T{0}\V(\\{dd}\E\T{0}\W%
\3{-1}((\\{dd}\K\\{ff}(\|v,\39\|w,\39\|t,\39\|u))<\T{0}\V(\\{dd}\E\T{0}\W%
\3{-1}((\\{dd}\K\\{gg}(\|v,\39\|w,\39\|t,\39\|u))<\T{0}\V(\\{dd}\E\T{0}\W%
\3{-1}((\\{dd}\K\\{hh}(\|t,\39\|u,\39\|v,\39\|w))<\T{0}\V(\\{dd}\E\T{0}\W%
\3{-1}((\\{dd}\K\\{jj}(\|t,\39\|u,\39\|v,\39\|w))<\T{0}\V(\\{dd}\E\T{0}\W%
\3{-1}((\\{dd}\K\\{hh}(\|v,\39\|t,\39\|u,\39\|w))<\T{0}\V(\\{dd}\E\T{0}\W%
\3{-1}((\\{dd}\K\\{jj}(\|v,\39\|t,\39\|u,\39\|w))<\T{0}\V(\\{dd}\E\T{0}\W%
\\{jj}(\|t,\39\|w,\39\|u,\39\|v)<\T{0})))))))))))))))))))){}$\1\5
${}\\{det}\K{-}\\{det};{}$\2\6
\4${}\}{}$\2\par
\U21.\fi

\M{24}\B\X13:Subroutines for arithmetic\X${}\mathrel+\E{}$\6
\1\1\&{static} \&{long} ${}\\{ff}(\|t,\39\|u,\39\|v,\39\|w){}$\6
\&{Vertex} ${}{*}\|t,\39{*}\|u,\39{*}\|v,\39{*}\|w;\2\2{}$\6
${}\{{}$\5
\1\&{register} \&{long} \\{wx}${}\K\|w\MG\\{x\_coord},\39\\{wy}\K\|w\MG\\{y%
\_coord}{}$;\C{ $x_w$, $y_w$ }\6
\&{long} \\{tx}${}\K\|t\MG\\{x\_coord}-\\{wx},\39\\{ty}\K\|t\MG\\{y\_coord}-%
\\{wy}{}$;\C{ $x_t-x_w$, $y_t-y_w$ }\6
\&{long} \\{ux}${}\K\|u\MG\\{x\_coord}-\\{wx},\39\\{uy}\K\|u\MG\\{y\_coord}-%
\\{wy}{}$;\C{ $x_u-x_w$, $y_u-y_w$ }\6
\&{long} \\{vx}${}\K\|v\MG\\{x\_coord}-\\{wx},\39\\{vy}\K\|v\MG\\{y\_coord}-%
\\{wy}{}$;\C{ $x_v-x_w$, $y_v-y_w$ }\7
\&{return} \\{sign\_test}${}(\\{ux}-\\{tx},\39\\{vx}-\\{ux},\39\\{tx}-\\{vx},%
\39\\{vx}*\\{vx}+\\{vy}*\\{vy},\39\\{tx}*\\{tx}+\\{ty}*\\{ty},\39\\{ux}*\\{ux}+%
\\{uy}*\\{uy});{}$\6
\4${}\}{}$\2\7
\1\1\&{static} \&{long} ${}\\{gg}(\|t,\39\|u,\39\|v,\39\|w){}$\6
\&{Vertex} ${}{*}\|t,\39{*}\|u,\39{*}\|v,\39{*}\|w;\2\2{}$\6
${}\{{}$\5
\1\&{register} \&{long} \\{wx}${}\K\|w\MG\\{x\_coord},\39\\{wy}\K\|w\MG\\{y%
\_coord}{}$;\C{ $x_w$, $y_w$ }\6
\&{long} \\{tx}${}\K\|t\MG\\{x\_coord}-\\{wx},\39\\{ty}\K\|t\MG\\{y\_coord}-%
\\{wy}{}$;\C{ $x_t-x_w$, $y_t-y_w$ }\6
\&{long} \\{ux}${}\K\|u\MG\\{x\_coord}-\\{wx},\39\\{uy}\K\|u\MG\\{y\_coord}-%
\\{wy}{}$;\C{ $x_u-x_w$, $y_u-y_w$ }\6
\&{long} \\{vx}${}\K\|v\MG\\{x\_coord}-\\{wx},\39\\{vy}\K\|v\MG\\{y\_coord}-%
\\{wy}{}$;\C{ $x_v-x_w$, $y_v-y_w$ }\7
\&{return} \\{sign\_test}${}(\\{uy}-\\{ty},\39\\{vy}-\\{uy},\39\\{ty}-\\{vy},%
\39\\{vx}*\\{vx}+\\{vy}*\\{vy},\39\\{tx}*\\{tx}+\\{ty}*\\{ty},\39\\{ux}*\\{ux}+%
\\{uy}*\\{uy});{}$\6
\4${}\}{}$\2\7
\1\1\&{static} \&{long} ${}\\{hh}(\|t,\39\|u,\39\|v,\39\|w){}$\6
\&{Vertex} ${}{*}\|t,\39{*}\|u,\39{*}\|v,\39{*}\|w;\2\2{}$\6
${}\{{}$\1\6
\&{return} ${}(\|u\MG\\{x\_coord}-\|t\MG\\{x\_coord})*(\|v\MG\\{y\_coord}-\|w%
\MG\\{y\_coord});{}$\6
\4${}\}{}$\2\7
\1\1\&{static} \&{long} ${}\\{jj}(\|t,\39\|u,\39\|v,\39\|w){}$\6
\&{Vertex} ${}{*}\|t,\39{*}\|u,\39{*}\|v,\39{*}\|w;\2\2{}$\6
${}\{{}$\5
\1\&{register} \&{long} \\{vx}${}\K\|v\MG\\{x\_coord},\39\\{wy}\K\|w\MG\\{y%
\_coord};{}$\7
\&{return} ${}(\|u\MG\\{x\_coord}-\\{vx})*(\|u\MG\\{x\_coord}-\\{vx})+(\|u\MG%
\\{y\_coord}-\\{wy})*(\|u\MG\\{y\_coord}-\\{wy})\3{-1}-(\|t\MG\\{x\_coord}-%
\\{vx})*(\|t\MG\\{x\_coord}-\\{vx})-(\|t\MG\\{y\_coord}-\\{wy})*(\|t\MG\\{y%
\_coord}-\\{wy});{}$\6
\4${}\}{}$\2\par
\fi

\N{1}{25}Delaunay data structures. Now we have the primitive predicates
we need, and we can get on with the geometric aspects of \PB{\\{delaunay}}.
As mentioned above, each vertex is represented by two coordinates and an
ID number, stored in the utility fields \PB{\\{x\_coord}}, \PB{\\{y\_coord}},
and~\PB{\\{z\_coord}}.

Each edge of the current triangulation is represented by two arcs
pointing in opposite directions; the two arcs are called {\sl mates}. Each
arc conceptually has a triangle on its left and a mate on its right.

An \&{arc} record differs from an \PB{\&{Arc}}; it has three fields:
\smallskip
\PB{\\{vert}} is the vertex this arc leads to, or \PB{$\NULL$} if that vertex
is $\infty$;
\smallskip
\PB{\\{next}} is the next arc having the same triangle at the left;
\smallskip
\PB{\\{inst}} is the branch node that points to the triangle at the left, as
explained below.

\smallskip\noindent
If \PB{\|p} points to an arc, then \PB{$\|p\MG\\{next}\MG\\{next}\MG\\{next}\E%
\|p$}, because a triangle
is bounded by three arcs. We also have \PB{$\|p\MG\\{next}\MG\\{inst}\E\|p\MG%
\\{inst}$}, for
all arcs~\PB{\|p}.

\Y\B\4\X25:Type declarations\X${}\E{}$\6
\&{typedef} \&{struct} \&{a\_struct} ${}\{{}$\1\6
\&{Vertex} ${}{*}\\{vert}{}$;\C{ \PB{\|v}, if this arc goes from \PB{\|u} to %
\PB{\|v} }\6
\&{struct} \&{a\_struct} ${}{*}\\{next}{}$;\C{ the arc from \PB{\|v} that
shares          a triangle with this one }\6
\&{struct} \\{n\_struct} ${}{*}\\{inst}{}$;\C{ instruction to change
when the triangle is modified }\2\6
${}\}{}$ \&{arc};\par
\A29.
\U4.\fi

\M{26}Storage is allocated in such a way that, if \PB{\|p} and \PB{\|q} point
respectively
to an arc and its mate, then \PB{$\|p+\|q\K{\AND}\\{arc\_block}[\T{0}]+{\AND}%
\\{arc\_block}[\|m-\T{1}]$}, where \PB{\|m} is
the total number of arc records allocated in the \PB{\\{arc\_block}} array.
This
convention saves us one pointer field in each arc.

When setting \PB{\|q} to the mate of \PB{\|p}, we need to do the calculation
cautiously using an auxiliary register, because the constant
\PB{${\AND}\\{arc\_block}[\T{0}]+{\AND}\\{arc\_block}[\|m-\T{1}]$} might be too
large to evaluate without
integer overflow on some systems.

\Y\B\4\D$\\{mate}(\|a,\|b)$ \6
${}\{{}$\C{ given \PB{\|a}, set \PB{\|b} to its mate }\1\6
${}\\{reg}\K\\{max\_arc}-(\&{siz\_t})\,\|a;{}$\6
${}\|b\K{}$(\&{arc} ${}{*})(\\{reg}+\\{min\_arc});{}$\6
\4${}\}{}$\2\par
\Y\B\4\X26:Local variables for \PB{\\{delaunay}}\X${}\E{}$\6
\&{register} \&{siz\_t} \\{reg};\C{ used while computing mates }\6
\&{siz\_t} \\{min\_arc}${},\39\\{max\_arc}{}$;\C{ \PB{${\AND}\\{arc\_block}[%
\T{0}]$}, \PB{${\AND}\\{arc\_block}[\|m-\T{1}]$} }\6
\&{arc} ${}{*}\\{next\_arc}{}$;\C{ the first arc record that hasn't yet been
used }\par
\As30\ET32.
\U9.\fi

\M{27}\B\X27:Initialize the array of arcs\X${}\E{}$\6
$\\{next\_arc}\K\\{gb\_typed\_alloc}(\T{6}*\|g\MG\|n-\T{6},\39\&{arc},\39%
\\{working\_storage});{}$\6
\&{if} ${}(\\{next\_arc}\E\NULL){}$\1\5
\&{return};\C{ \PB{\\{gb\_trouble\_code}} is nonzero }\2\6
${}\\{min\_arc}\K(\&{siz\_t})\,\\{next\_arc};{}$\6
${}\\{max\_arc}\K(\&{siz\_t})(\\{next\_arc}+(\T{6}*\|g\MG\|n-\T{7})){}$;\par
\U31.\fi

\M{28}\B\X28:Call \PB{$\|f(\|u,\|v)$} for each Delaunay edge \\{uv}\X${}\E{}$\6
$\|a\K{}$(\&{arc} ${}{*})\,\\{min\_arc};{}$\6
${}\|b\K{}$(\&{arc} ${}{*})\,\\{max\_arc};{}$\6
\&{for} ( ; ${}\|a<\\{next\_arc};{}$ ${}\|a\PP,\39\|b\MM){}$\1\5
${}({*}\|f)(\|a\MG\\{vert},\39\|b\MG\\{vert}){}$;\2\par
\U9.\fi

\M{29}The last and probably most crucial component of the data structure
is the collection of {\sl branch nodes}, which will be linked together
into a binary tree.  Given a new vertex \PB{\|w}, we will ascertain what
triangle it belongs to by starting at the root of this tree and
executing a sequence of instructions, each of which has the form `if
\PB{\|w} lies to the right of the straight line from \PB{\|u} to~\PB{\|v} then
go to
$\alpha$ else go to~$\beta$', where $\alpha$ and~$\beta$ are nodes
that continue the search. This process continues until we reach a
terminal node, which says `congratulations, you're done, \PB{\|w}~is in
triangle such-and-such'. The terminal node points to one of the three
arcs bounding that triangle. If a vertex of the triangle is~$\infty$,
the terminal node points to the arc whose \PB{\\{vert}} pointer is~\PB{$%
\NULL$}.

\Y\B\4\X25:Type declarations\X${}\mathrel+\E{}$\6
\&{typedef} \&{struct} \&{n\_struct} ${}\{{}$\1\6
\&{Vertex} ${}{*}\|u{}$;\C{ first vertex, or \PB{$\NULL$} if this is a terminal
node }\6
\&{Vertex} ${}{*}\|v{}$;\C{ second vertex, or pointer to the triangle
      corresponding to a terminal node }\6
\&{struct} \&{n\_struct} ${}{*}\|l{}$;\C{ go here if \PB{\|w} lies to the left
of $uv$ }\6
\&{struct} \&{n\_struct} ${}{*}\|r{}$;\C{ go here if \PB{\|w} lies to the right
of $uv$ }\2\6
${}\}{}$ \&{node};\par
\fi

\M{30}The search tree just described is actually a dag (a directed acyclic
graph), because it has overlapping subtrees. As the algorithm proceeds,
the dag gets bigger and bigger, since the number of triangles keeps
growing. Instructions are never deleted; we just extend the dag by
substituting new branches for nodes that once were terminal.

The expected number of nodes in this dag is $O(n)$ when there are $n$~vertices,
if we input the vertices in random order. But it can be as high as order~$n^2$
in the worst case. So our program will allocate blocks of nodes dynamically
instead of assuming a maximum size.

\Y\B\4\D$\\{nodes\_per\_block}$ \5
\T{127}\C{ on most computers we want it $\equiv 15$ (mod 16) }\par
\B\4\D$\\{new\_node}(\|x)$ \6
\&{if} ${}(\\{next\_node}\E\\{max\_node}){}$\5
${}\{{}$\1\6
${}\|x\K\\{gb\_typed\_alloc}(\\{nodes\_per\_block},\39\&{node},\39\\{working%
\_storage});{}$\6
\&{if} ${}(\|x\E\NULL){}$\5
${}\{{}$\1\6
\\{gb\_free}(\\{working\_storage});\C{ release \PB{\\{delaunay}}'s auxiliary
memory }\6
\&{return};\C{ \PB{\\{gb\_trouble\_code}} is nonzero }\6
\4${}\}{}$\2\6
${}\\{next\_node}\K\|x+\T{1};{}$\6
${}\\{max\_node}\K\|x+\\{nodes\_per\_block};{}$\6
\4${}\}{}$\5
\2\&{else}\1\5
${}\|x\K\\{next\_node}\PP{}$;\2\Y\par
\B\4\D$\\{terminal\_node}(\|x,\|p)$ \6
${}\{{}$\5
\1\\{new\_node}(\|x);\C{ allocate a new node }\6
${}\|x\MG\|v\K{}$(\&{Vertex} ${}{*})(\|p){}$;\C{ make it point to a given arc
from the triangle }\6
\4${}\}{}$\C{ note that \PB{$\|x\MG\|u\E\NULL$}, representing a terminal node }%
\2\par
\Y\B\4\X26:Local variables for \PB{\\{delaunay}}\X${}\mathrel+\E{}$\6
\&{node} ${}{*}\\{next\_node}{}$;\C{ the first yet-unused node slot    in the
current block of nodes }\6
\&{node} ${}{*}\\{max\_node}{}$;\C{ address of nonexistent node following the
current    block of nodes }\6
\&{node} \\{root\_node};\C{ start here to locate a vertex in its triangle }\6
\&{Area} \\{working\_storage};\C{ where \PB{\\{delaunay}} builds its
triangulation }\par
\fi

\M{31}The algorithm begins with a trivial triangulation that contains
only the first two vertices, together with two ``triangles'' extending
to infinity at their left and right.

\Y\B\4\X31:Initialize the data structures\X${}\E{}$\6
$\\{next\_node}\K\\{max\_node}\K\NULL;{}$\6
\\{init\_area}(\\{working\_storage});\6
\X27:Initialize the array of arcs\X;\6
${}\|u\K\|g\MG\\{vertices};{}$\6
${}\|v\K\|u+\T{1};{}$\6
\X33:Make two ``triangles'' for \PB{\|u}, \PB{\|v}, and $\infty$\X;\par
\U34.\fi

\M{32}We'll need a bunch of local variables to do elementary operations on
data structures.

\Y\B\4\X26:Local variables for \PB{\\{delaunay}}\X${}\mathrel+\E{}$\6
\&{Vertex} ${}{*}\|p,\39{*}\|q,\39{*}\|r,\39{*}\|s,\39{*}\|t,\39{*}\\{tp},%
\39{*}\\{tpp},\39{*}\|u,\39{*}\|v;{}$\6
\&{arc} ${}{*}\|a,\39{*}\\{aa},\39{*}\|b,\39{*}\|c,\39{*}\|d,\39{*}\|e;{}$\6
\&{node} ${}{*}\|x,\39{*}\|y,\39{*}\\{yp},\39{*}\\{ypp}{}$;\par
\fi

\M{33}\B\X33:Make two ``triangles'' for \PB{\|u}, \PB{\|v}, and $\infty$\X${}%
\E{}$\6
$\\{root\_node}.\|u\K\|u;{}$\6
${}\\{root\_node}.\|v\K\|v;{}$\6
${}\|a\K\\{next\_arc};{}$\6
${}\\{terminal\_node}(\|x,\39\|a+\T{1});{}$\6
${}\\{root\_node}.\|l\K\|x;{}$\6
${}\|a\MG\\{vert}\K\|v{}$;\5
${}\|a\MG\\{next}\K\|a+\T{1}{}$;\5
${}\|a\MG\\{inst}\K\|x;{}$\6
${}(\|a+\T{1})\MG\\{next}\K\|a+\T{2}{}$;\5
${}(\|a+\T{1})\MG\\{inst}\K\|x{}$;\C{ \PB{$(\|a+\T{1})\MG\\{vert}\K\NULL$},
representing $\infty$ }\6
${}(\|a+\T{2})\MG\\{vert}\K\|u{}$;\5
${}(\|a+\T{2})\MG\\{next}\K\|a{}$;\5
${}(\|a+\T{2})\MG\\{inst}\K\|x;{}$\6
${}\\{mate}(\|a,\39\|b);{}$\6
${}\\{terminal\_node}(\|x,\39\|b-\T{2});{}$\6
${}\\{root\_node}.\|r\K\|x;{}$\6
${}\|b\MG\\{vert}\K\|u{}$;\5
${}\|b\MG\\{next}\K\|b-\T{2}{}$;\5
${}\|b\MG\\{inst}\K\|x;{}$\6
${}(\|b-\T{2})\MG\\{next}\K\|b-\T{1}{}$;\5
${}(\|b-\T{2})\MG\\{inst}\K\|x{}$;\C{ \PB{$(\|b-\T{2})\MG\\{vert}\K\NULL$},
representing $\infty$ }\6
${}(\|b-\T{1})\MG\\{vert}\K\|v{}$;\5
${}(\|b-\T{1})\MG\\{next}\K\|b{}$;\5
${}(\|b-\T{1})\MG\\{inst}\K\|x;{}$\6
${}\\{next\_arc}\MRL{+{\K}}\T{3}{}$;\par
\U31.\fi

\N{1}{34}Delaunay updating.
The main loop of the algorithm updates the data structure incrementally
by adding one new vertex at a time. The new vertex will always be connected
by an edge (i.e., by two arcs) to each of the vertices of the triangle that
previously enclosed it. It might also deserve to be connected to other
nearby vertices.

\Y\B\4\X34:Find the Delaunay triangulation of \PB{\|g}, or return with \PB{%
\\{gb\_trouble\_code}} nonzero if out of memory\X${}\E{}$\6
\&{if} ${}(\|g\MG\|n<\T{2}){}$\1\5
\&{return};\C{ no edges unless there are at least 2 vertices }\2\6
\X31:Initialize the data structures\X;\6
\&{for} ${}(\|p\K\|g\MG\\{vertices}+\T{2};{}$ ${}\|p<\|g\MG\\{vertices}+\|g\MG%
\|n;{}$ ${}\|p\PP){}$\5
${}\{{}$\1\6
\X35:Find an arc \PB{\|a} on the boundary of the triangle containing \PB{\|p}%
\X;\6
\X36:Divide the triangle left of \PB{\|a} into three triangles surrounding \PB{%
\|p}\X;\6
\X39:Explore the triangles surrounding \PB{\|p}, ``flipping'' their neighbors
until all triangles that should touch \PB{\|p} are found\X;\6
\4${}\}{}$\2\par
\U9.\fi

\M{35}We have set up the branch nodes so that they solve the triangle location
problem.

\Y\B\4\X35:Find an arc \PB{\|a} on the boundary of the triangle containing \PB{%
\|p}\X${}\E{}$\6
$\|x\K{\AND}\\{root\_node};{}$\6
\&{do}\5
${}\{{}$\1\6
\&{if} ${}(\\{ccw}(\|x\MG\|u,\39\|x\MG\|v,\39\|p)){}$\1\5
${}\|x\K\|x\MG\|l;{}$\2\6
\&{else}\1\5
${}\|x\K\|x\MG\|r;{}$\2\6
\4${}\}{}$\5
\2\5
\&{while} ${}(\|x\MG\|u);{}$\6
${}\|a\K{}$(\&{arc} ${}{*})\,\|x\MG\|v{}$;\C{ terminal node points to the arc
we want }\par
\U34.\fi

\M{36}Subdividing a triangle is an easy exercise in data structure
manipulation,
except that we must do something special when one of the vertices is
infinite. Let's look carefully at what needs to be done.

Suppose the triangle containing \PB{\|p} has the vertices \PB{\|q}, \PB{\|r},
and \PB{\|s}
in counterclockwise order. Let \PB{\|x} be the terminal node that points to
the triangle~$\Delta qrs$. We want to change \PB{\|x} so that we will be
able to locate a future point of $\Delta qrs$ within either $\Delta pqr$,
$\Delta prs$, or $\Delta psq$.

If \PB{\|q}, \PB{\|r}, and \PB{\|s} are finite, we will change \PB{\|x} and add
five new nodes
as follows:
$$\vbox{\halign{\hfil#:\enspace&#\hfil\cr
$x$&if left of $rp$, go to $x''$, else go to $x'$;\cr
$x'$&if left of $sp$, go to $y$, else go to $y'$;\cr
$x''$&if left of $qp$, go to $y'$, else go to $y''$;\cr
$y$&you're in $\Delta prs$;\cr
$y'$&you're in $\Delta psq$;\cr
$y''$&you're in $\Delta pqr$.\cr}}$$

But if, say, $q=\infty$, such instructions make no sense,
because there are lines in all directions that run from $\infty$ to any point.
In such a case we use ``wedges'' instead of triangles, as explained below.

At the beginning of the following code, we have \PB{$\|x\E\|a\MG\\{inst}$}.

\Y\B\4\X36:Divide the triangle left of \PB{\|a} into three triangles
surrounding \PB{\|p}\X${}\E{}$\6
$\|b\K\|a\MG\\{next}{}$;\5
${}\|c\K\|b\MG\\{next};{}$\6
${}\|q\K\|a\MG\\{vert}{}$;\5
${}\|r\K\|b\MG\\{vert}{}$;\5
${}\|s\K\|c\MG\\{vert};{}$\6
\X37:Create new terminal nodes \PB{\|y}, \PB{\\{yp}}, \PB{\\{ypp}}, and new
arcs pointing to them\X;\6
\&{if} ${}(\|q\E\NULL)$ \X38:Compile instructions to update convex hull\X \6
\&{else}\5
${}\{{}$\5
\1\&{register} \&{node} ${}{*}\\{xp};{}$\7
${}\|x\MG\|u\K\|r{}$;\5
${}\|x\MG\|v\K\|p;{}$\6
\\{new\_node}(\\{xp});\6
${}\\{xp}\MG\|u\K\|q{}$;\5
${}\\{xp}\MG\|v\K\|p{}$;\5
${}\\{xp}\MG\|l\K\\{yp}{}$;\5
${}\\{xp}\MG\|r\K\\{ypp}{}$;\C{ instruction $x''$ above }\6
${}\|x\MG\|l\K\\{xp};{}$\6
\\{new\_node}(\\{xp});\6
${}\\{xp}\MG\|u\K\|s{}$;\5
${}\\{xp}\MG\|v\K\|p{}$;\5
${}\\{xp}\MG\|l\K\|y{}$;\5
${}\\{xp}\MG\|r\K\\{yp}{}$;\C{ instruction $x'$ above }\6
${}\|x\MG\|r\K\\{xp};{}$\6
\4${}\}{}$\2\par
\U34.\fi

\M{37}The only subtle point here is that \PB{$\|q\K\|a\MG\\{vert}$} might be %
\PB{$\NULL$}. A terminal
node must point to the proper arc of an infinite triangle.

\Y\B\4\X37:Create new terminal nodes \PB{\|y}, \PB{\\{yp}}, \PB{\\{ypp}}, and
new arcs pointing to them\X${}\E{}$\6
$\\{terminal\_node}(\\{yp},\39\|a){}$;\5
${}\\{terminal\_node}(\\{ypp},\39\\{next\_arc}){}$;\5
${}\\{terminal\_node}(\|y,\39\|c);{}$\6
${}\|c\MG\\{inst}\K\|y{}$;\5
${}\|a\MG\\{inst}\K\\{yp}{}$;\5
${}\|b\MG\\{inst}\K\\{ypp};{}$\6
${}\\{mate}(\\{next\_arc},\39\|e);{}$\6
${}\|a\MG\\{next}\K\|e{}$;\5
${}\|b\MG\\{next}\K\|e-\T{1}{}$;\5
${}\|c\MG\\{next}\K\|e-\T{2};{}$\6
${}\\{next\_arc}\MG\\{vert}\K\|q{}$;\5
${}\\{next\_arc}\MG\\{next}\K\|b{}$;\5
${}\\{next\_arc}\MG\\{inst}\K\\{ypp};{}$\6
${}(\\{next\_arc}+\T{1})\MG\\{vert}\K\|r{}$;\5
${}(\\{next\_arc}+\T{1})\MG\\{next}\K\|c{}$;\5
${}(\\{next\_arc}+\T{1})\MG\\{inst}\K\|y;{}$\6
${}(\\{next\_arc}+\T{2})\MG\\{vert}\K\|s{}$;\5
${}(\\{next\_arc}+\T{2})\MG\\{next}\K\|a{}$;\5
${}(\\{next\_arc}+\T{2})\MG\\{inst}\K\\{yp};{}$\6
${}\|e\MG\\{vert}\K(\|e-\T{1})\MG\\{vert}\K(\|e-\T{2})\MG\\{vert}\K\|p;{}$\6
${}\|e\MG\\{next}\K\\{next\_arc}+\T{2}{}$;\5
${}(\|e-\T{1})\MG\\{next}\K\\{next\_arc}{}$;\5
${}(\|e-\T{2})\MG\\{next}\K\\{next\_arc}+\T{1};{}$\6
${}\|e\MG\\{inst}\K\\{yp}{}$;\5
${}(\|e-\T{1})\MG\\{inst}\K\\{ypp}{}$;\5
${}(\|e-\T{2})\MG\\{inst}\K\|y;{}$\6
${}\\{next\_arc}\MRL{+{\K}}\T{3}{}$;\par
\U36.\fi

\M{38}Outside of the current convex hull, we have ``wedges'' instead of
triangles. Wedges are exterior angles whose points lie outside an
edge $rs$ of the convex hull, but not outside the next edge on the other
side of point \PB{\|r}. When a new point lies in such a wedge, we have to
see if it also lies outside the edges $st$, $tu$, etc., in the
clockwise direction, in which case the convex hull loses points
$s$, $t$, etc., and we must update the new wedges accordingly.

This was the hardest part of the program to prove correct; a complete
proof can be found in {\sl Axioms and Hulls}.

\Y\B\4\X38:Compile instructions to update convex hull\X${}\E{}$\6
${}\{{}$\5
\1\&{register} \&{node} ${}{*}\\{xp};{}$\7
${}\|x\MG\|u\K\|r{}$;\5
${}\|x\MG\|v\K\|p{}$;\5
${}\|x\MG\|l\K\\{ypp};{}$\6
\\{new\_node}(\\{xp});\6
${}\\{xp}\MG\|u\K\|s{}$;\5
${}\\{xp}\MG\|v\K\|p{}$;\5
${}\\{xp}\MG\|l\K\|y{}$;\5
${}\\{xp}\MG\|r\K\\{yp};{}$\6
${}\|x\MG\|r\K\\{xp};{}$\6
${}\\{mate}(\|a,\39\\{aa}){}$;\5
${}\|d\K\\{aa}\MG\\{next}{}$;\5
${}\|t\K\|d\MG\\{vert};{}$\6
\&{while} ${}(\|t\I\|r\W(\\{ccw}(\|p,\39\|s,\39\|t))){}$\5
${}\{{}$\5
\1\&{register} \&{node} ${}{*}\\{xpp};{}$\7
${}\\{terminal\_node}(\\{xpp},\39\|d);{}$\6
${}\\{xp}\MG\|r\K\|d\MG\\{inst};{}$\6
${}\\{xp}\K\|d\MG\\{inst};{}$\6
${}\\{xp}\MG\|u\K\|t{}$;\5
${}\\{xp}\MG\|v\K\|p{}$;\5
${}\\{xp}\MG\|l\K\\{xpp}{}$;\5
${}\\{xp}\MG\|r\K\\{yp};{}$\6
${}\\{flip}(\|a,\39\\{aa},\39\|d,\39\|s,\39\NULL,\39\|t,\39\|p,\39\\{xpp},\39%
\\{yp});{}$\6
${}\|a\K\\{aa}\MG\\{next}{}$;\5
${}\\{mate}(\|a,\39\\{aa}){}$;\5
${}\|d\K\\{aa}\MG\\{next};{}$\6
${}\|s\K\|t{}$;\5
${}\|t\K\|d\MG\\{vert};{}$\6
${}\\{yp}\MG\|v\K{}$(\&{Vertex} ${}{*})\,\|a;{}$\6
\4${}\}{}$\2\6
${}\\{terminal\_node}(\\{xp},\39\|d\MG\\{next});{}$\6
${}\|x\K\|d\MG\\{inst}{}$;\5
${}\|x\MG\|u\K\|s{}$;\5
${}\|x\MG\|v\K\|p{}$;\5
${}\|x\MG\|l\K\\{xp}{}$;\5
${}\|x\MG\|r\K\\{yp};{}$\6
${}\|d\MG\\{inst}\K\\{xp}{}$;\5
${}\|d\MG\\{next}\MG\\{inst}\K\\{xp}{}$;\5
${}\|d\MG\\{next}\MG\\{next}\MG\\{inst}\K\\{xp};{}$\6
${}\|r\K\|s{}$;\C{ this value of \PB{\|r} shortens the exploration step that
follows }\6
\4${}\}{}$\2\par
\U36.\fi

\M{39}The updating process finishes by walking around the triangles
that surround \PB{\|p}, making sure that none of them are adjacent to
triangles containing \PB{\|p} in their circumcircle. (Such triangles are
no longer in the Delaunay triangulation, by definition.)

\Y\B\4\X39:Explore the triangles surrounding \PB{\|p}, ``flipping'' their
neighbors until all triangles that should touch \PB{\|p} are found\X${}\E{}$\6
\&{while} (\T{1})\5
${}\{{}$\1\6
${}\\{mate}(\|c,\39\|d){}$;\5
${}\|e\K\|d\MG\\{next};{}$\6
${}\|t\K\|d\MG\\{vert}{}$;\5
${}\\{tp}\K\|c\MG\\{vert}{}$;\5
${}\\{tpp}\K\|e\MG\\{vert};{}$\6
\&{if} ${}(\\{tpp}\W\\{incircle}(\\{tpp},\39\\{tp},\39\|t,\39\|p)){}$\5
${}\{{}$\C{ triangle $tt''t'$ no longer Delaunay }\1\6
\&{register} \&{node} ${}{*}\\{xp},\39{*}\\{xpp};{}$\7
${}\\{terminal\_node}(\\{xp},\39\|e);{}$\6
${}\\{terminal\_node}(\\{xpp},\39\|d);{}$\6
${}\|x\K\|c\MG\\{inst}{}$;\5
${}\|x\MG\|u\K\\{tpp}{}$;\5
${}\|x\MG\|v\K\|p{}$;\5
${}\|x\MG\|l\K\\{xp}{}$;\5
${}\|x\MG\|r\K\\{xpp};{}$\6
${}\|x\K\|d\MG\\{inst}{}$;\5
${}\|x\MG\|u\K\\{tpp}{}$;\5
${}\|x\MG\|v\K\|p{}$;\5
${}\|x\MG\|l\K\\{xp}{}$;\5
${}\|x\MG\|r\K\\{xpp};{}$\6
${}\\{flip}(\|c,\39\|d,\39\|e,\39\|t,\39\\{tp},\39\\{tpp},\39\|p,\39\\{xp},\39%
\\{xpp});{}$\6
${}\|c\K\|e;{}$\6
\4${}\}{}$\2\6
\&{else} \&{if} ${}(\\{tp}\E\|r){}$\1\5
\&{break};\2\6
\&{else}\5
${}\{{}$\1\6
${}\\{mate}(\|c\MG\\{next},\39\\{aa});{}$\6
${}\|c\K\\{aa}\MG\\{next};{}$\6
\4${}\}{}$\2\6
\4${}\}{}$\2\par
\U34.\fi

\M{40}Here \PB{\|d} is the mate of \PB{\|c}, \PB{$\|e\K\|d\MG\\{next}$}, \PB{$%
\|t\K\|d\MG\\{vert}$}, \PB{$\\{tp}\K\|c\MG\\{vert}$},
and \PB{$\\{tpp}\K\|e\MG\\{vert}$}. The triangles $\Delta tt'p$ and $\Delta
t'tt''$ to the
left and right of arc~\PB{\|c} are being replaced in the current triangulation
by $\Delta ptt''$ and $\Delta t''t'p$, corresponding to terminal nodes
\PB{\\{xp}} and \PB{\\{xpp}}. (The values of \PB{\|t} and \PB{\\{tp}} are not
actually used, so
some optimization is possible.)

\Y\B\4\X12:Other subroutines\X${}\mathrel+\E{}$\6
\1\1\&{static} \&{void} ${}\\{flip}(\|c,\39\|d,\39\|e,\39\|t,\39\\{tp},\39%
\\{tpp},\39\|p,\39\\{xp},\39\\{xpp}){}$\6
\&{arc} ${}{*}\|c,\39{*}\|d,\39{*}\|e;{}$\6
\&{Vertex} ${}{*}\|t,\39{*}\\{tp},\39{*}\\{tpp},\39{*}\|p;{}$\6
\&{node} ${}{*}\\{xp},\39{*}\\{xpp};\2\2{}$\6
${}\{{}$\5
\1\&{register} \&{arc} ${}{*}\\{ep}\K\|e\MG\\{next},\39{*}\\{cp}\K\|c\MG%
\\{next},\39{*}\\{cpp}\K\\{cp}\MG\\{next};{}$\7
${}\|e\MG\\{next}\K\|c{}$;\5
${}\|c\MG\\{next}\K\\{cpp}{}$;\5
${}\\{cpp}\MG\\{next}\K\|e;{}$\6
${}\|e\MG\\{inst}\K\|c\MG\\{inst}\K\\{cpp}\MG\\{inst}\K\\{xp};{}$\6
${}\|c\MG\\{vert}\K\|p;{}$\6
${}\|d\MG\\{next}\K\\{ep}{}$;\5
${}\\{ep}\MG\\{next}\K\\{cp}{}$;\5
${}\\{cp}\MG\\{next}\K\|d;{}$\6
${}\|d\MG\\{inst}\K\\{ep}\MG\\{inst}\K\\{cp}\MG\\{inst}\K\\{xpp};{}$\6
${}\|d\MG\\{vert}\K\\{tpp};{}$\6
\4${}\}{}$\2\par
\fi

\N{1}{41}Use of mileage data. The \PB{\\{delaunay}} routine is now complete,
and the
only missing piece of code is the promised routine that generates
planar graphs based on data from the real world.

The subroutine call {\advance\thinmuskip 0mu plus 2mu
\PB{$\\{plane\_miles}(\|n,\\{north\_weight},\\{west\_weight},\\{pop\_weight},%
\\{extend},\\{prob},\\{seed})$}}
will construct a planar graph with min$(128,n)$ vertices, where the
vertices are exactly the same as the cities produced by the subroutine call
\PB{$\\{miles}(\|n,\\{north\_weight},\\{west\_weight},\\{pop\_weight},\T{0},%
\T{0},\\{seed})$}. (As
explained in module {\sc GB\_\,MILES}, the weight parameters \PB{\\{north%
\_weight}},
\PB{\\{west\_weight}}, and \PB{\\{pop\_weight}} are used to rank the cities by
location and/or population.)  The edges of the new graph are obtained
by first constructing the Delaunay triangulation of those cities,
based on a simple projection onto the plane using their latitude and
longitude, then discarding each Delaunay edge with probability
\PB{$\\{prob}/\T{65536}$}. The length of each surviving edge is the same as the
mileage between cities that would appear in the complete graph
produced by \PB{\\{miles}}.

If \PB{$\\{extend}\I\T{0}$}, an additional vertex representing $\infty$ is also
included. The Delaunay triangulation includes edges of length \PB{\.{INFTY}}
connecting this vertex with all cities on the convex hull; these edges,
like the others, are subject to being discarded with probability \PB{$\\{prob}/%
\T{65536}$}.
(See the description of \PB{\\{plane}} for further comments about using
\PB{\\{prob}} to control the sparseness of the graph.)

The weight parameters must satisfy
$$ \vert\PB{\\{north\_weight}}\vert\le100{,}000,\quad
\vert\PB{\\{west\_weight}}\vert\le100{,}000,\quad
\vert\PB{\\{pop\_weight}}\vert\le100.$$
Vertices of the graph will appear in order of decreasing weight.
The \PB{\\{seed}} parameter defines the pseudo-random numbers used wherever
a ``random'' choice between equal-weight vertices needs to be made,
or when deciding whether to discard a Delaunay edge.

\Y\B\4\X41:The \PB{\\{plane\_miles}} routine\X${}\E{}$\6
\1\1\&{Graph} ${}{*}\\{plane\_miles}(\|n,\39\\{north\_weight},\39\\{west%
\_weight},\39\\{pop\_weight},\39\\{extend},\39\\{prob},\39\\{seed}){}$\6
\&{unsigned} \&{long} \|n;\C{ number of vertices desired }\6
\&{long} \\{north\_weight};\C{ coefficient of latitude in the weight function }%
\6
\&{long} \\{west\_weight};\C{ coefficient of longitude in the weight function }%
\6
\&{long} \\{pop\_weight};\C{ coefficient of population in the weight function }%
\6
\&{unsigned} \&{long} \\{extend};\C{ should a point at infinity be included? }\6
\&{unsigned} \&{long} \\{prob};\C{ probability of rejecting a Delaunay edge }\6
\&{long} \\{seed};\C{ random number seed }\2\2\6
${}\{{}$\5
\1\&{Graph} ${}{*}\\{new\_graph}{}$;\C{ the graph constructed by \PB{\\{plane%
\_miles}} }\7
\X42:Use \PB{\\{miles}} to set up the vertices of a graph\X;\6
\X43:Compute the Delaunay triangulation and run through the Delaunay edges;
reject them with probability \PB{$\\{prob}/\T{65536}$}, otherwise append them
with the road length in miles\X;\6
\&{if} (\\{gb\_trouble\_code})\5
${}\{{}$\1\6
\\{gb\_recycle}(\\{new\_graph});\6
\\{panic}(\\{alloc\_fault});\C{ oops, we ran out of memory somewhere back there
}\6
\4${}\}{}$\2\6
${}\\{gb\_free}(\\{new\_graph}\MG\\{aux\_data}){}$;\C{ recycle special memory
used by \PB{\\{miles}} }\6
\&{if} (\\{extend})\1\5
${}\\{new\_graph}\MG\|n\PP{}$;\C{ make the ``infinite'' vertex legitimate }\2\6
\&{return} \\{new\_graph};\6
\4${}\}{}$\2\par
\U4.\fi

\M{42}By setting the \PB{\\{max\_distance}} parameter to~1, we cause \PB{%
\\{miles}}
to produce a graph having the desired vertices but no edges.
The vertices of this graph will have appropriate coordinate fields
\PB{\\{x\_coord}}, \PB{\\{y\_coord}}, and~\PB{\\{z\_coord}}.

\Y\B\4\X42:Use \PB{\\{miles}} to set up the vertices of a graph\X${}\E{}$\6
\&{if} (\\{extend})\1\5
${}\\{extra\_n}\PP{}$;\C{ allocate one more vertex than usual }\2\6
\&{if} ${}(\|n\E\T{0}\V\|n>\.{MAX\_N}){}$\1\5
${}\|n\K\.{MAX\_N}{}$;\C{ compute true number of vertices }\2\6
${}\\{new\_graph}\K\\{miles}(\|n,\39\\{north\_weight},\39\\{west\_weight},\39%
\\{pop\_weight},\39\T{1\$L},\39\T{0\$L},\39\\{seed});{}$\6
\&{if} ${}(\\{new\_graph}\E\NULL){}$\1\5
\&{return} ${}\NULL{}$;\C{ \PB{\\{panic\_code}} has been set by \PB{\\{miles}}
}\2\6
${}\\{sprintf}(\\{new\_graph}\MG\\{id},\39\.{"plane\_miles(\%lu,\%ld}\)\.{,%
\%ld,\%ld,\%lu,\%lu,\%ld}\)\.{)"},\39\|n,\39\\{north\_weight},\39\\{west%
\_weight},\39\\{pop\_weight},\39\\{extend},\39\\{prob},\39\\{seed});{}$\6
\&{if} (\\{extend})\1\5
${}\\{extra\_n}\MM{}$;\C{ restore \PB{\\{extra\_n}} to its previous value }\2%
\par
\U41.\fi

\M{43}\B\X43:Compute the Delaunay triangulation and run through the Delaunay
edges; reject them with probability \PB{$\\{prob}/\T{65536}$}, otherwise append
them with the road length in miles\X${}\E{}$\6
$\\{gprob}\K\\{prob};{}$\6
\&{if} (\\{extend})\5
${}\{{}$\1\6
${}\\{inf\_vertex}\K\\{new\_graph}\MG\\{vertices}+\\{new\_graph}\MG\|n;{}$\6
${}\\{inf\_vertex}\MG\\{name}\K\\{gb\_save\_string}(\.{"INF"});{}$\6
${}\\{inf\_vertex}\MG\\{x\_coord}\K\\{inf\_vertex}\MG\\{y\_coord}\K\\{inf%
\_vertex}\MG\\{z\_coord}\K{-}\T{1};{}$\6
\4${}\}{}$\5
\2\&{else}\1\5
${}\\{inf\_vertex}\K\NULL;{}$\2\6
${}\\{delaunay}(\\{new\_graph},\39\\{new\_mile\_edge}){}$;\par
\U41.\fi

\M{44}The mileages will all have been negated by \PB{\\{miles}}, so we make
them
positive again.

\Y\B\4\X12:Other subroutines\X${}\mathrel+\E{}$\6
\1\1\&{static} \&{void} ${}\\{new\_mile\_edge}(\|u,\39\|v){}$\6
\&{Vertex} ${}{*}\|u,\39{*}\|v;\2\2{}$\6
${}\{{}$\1\6
\&{if} ${}((\\{gb\_next\_rand}(\,)\GG\T{15})\G\\{gprob}){}$\5
${}\{{}$\1\6
\&{if} (\|u)\5
${}\{{}$\1\6
\&{if} (\|v)\1\5
${}\\{gb\_new\_edge}(\|u,\39\|v,\39{-}\\{miles\_distance}(\|u,\39\|v));{}$\2\6
\&{else} \&{if} (\\{inf\_vertex})\1\5
${}\\{gb\_new\_edge}(\|u,\39\\{inf\_vertex},\39\.{INFTY});{}$\2\6
\4${}\}{}$\5
\2\&{else} \&{if} (\\{inf\_vertex})\1\5
${}\\{gb\_new\_edge}(\\{inf\_vertex},\39\|v,\39\.{INFTY});{}$\2\6
\4${}\}{}$\2\6
\4${}\}{}$\2\par
\fi

\N{1}{45}Index. As usual, we close with an index that
shows where the identifiers of \\{gb\_plane} are defined and used.
\fi

\inx
\fin
\con
