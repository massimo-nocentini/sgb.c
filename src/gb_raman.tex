\input cwebmac
% This file is part of the Stanford GraphBase (c) Stanford University 1993
% This material goes at the beginning of all Stanford GraphBase CWEB files

\def\topofcontents{
  \leftline{\sc\today\ at \hours}\bigskip\bigskip
  \centerline{\titlefont\title}}

\font\ninett=cmtt9
\def\botofcontents{\vskip 0pt plus 1filll
    \ninerm\baselineskip10pt
    \noindent\copyright\ 1993 Stanford University
    \bigskip\noindent
    This file may be freely copied and distributed, provided that
    no changes whatsoever are made. All users are asked to help keep
    the Stanford GraphBase files consistent and ``uncorrupted,''
    identical everywhere in the world. Changes are permissible only
    if the modified file is given a new name, different from the names of
    existing files in the Stanford GraphBase, and only if the modified file is
    clearly identified as not being part of that GraphBase.
    (The {\ninett CWEB} system has a ``change file'' facility by
    which users can easily make minor alterations without modifying
    the master source files in any way. Everybody is supposed to use
    change files instead of changing the files.)
    The author has tried his best to produce correct and useful programs,
    in order to help promote computer science research,
    but no warranty of any kind should be assumed.
    \smallskip\noindent
    Preliminary work on the Stanford GraphBase project
    was supported in part by National Science
    Foundation grant CCR-86-10181.}

\def\prerequisite#1{\def\startsection{\noindent
    Important: Before reading {\sc\title},
    please read or at least skim the program for {\sc#1}.\bigskip
    \let\startsection=\stsec\stsec}}
\def\prerequisites#1#2{\def\startsection{\noindent
    Important: Before reading {\sc\title}, please read
    or at least skim the programs for {\sc#1} and {\sc#2}.\bigskip
    \let\startsection=\stsec\stsec}}



\def\title{GB\_\,RAMAN}
\let\==\equiv % congruence sign

\prerequisite{GB\_\,GRAPH}

\N{1}{1}Introduction. This GraphBase module contains the \PB{\\{raman}}
subroutine,
which creates a family of ``Ramanujan graphs'' based on a theory
developed by Alexander Lubotzky, Ralph Phillips, and Peter Sarnak
[see {\sl Combinatorica \bf8} (1988), 261--277].

Ramanujan graphs are defined by the following properties:
They are connected, undirected graphs in which every vertex has
degree~$k$, and every eigenvalue of the adjacency matrix
is either $\pm k$ or has absolute value $\le2\sqrt{\mathstrut k-1}$.
Such graphs are known to have good expansion properties, small diameter,
and relatively small independent sets; they cannot be colored with
fewer than $k/\bigl(2\sqrt{\mathstrut k-1}\,\bigr)$ colors unless they are
bipartite. The particular examples of Ramanujan graphs constructed here
are based on interesting properties of quaternions with integer coefficients.

An example of the use of this procedure can be found in the demo program
called {\sc GIRTH}.

\Y\B\4\X1:\.{gb\_raman.h\,}\X${}\E{}$\6
\&{extern} \&{Graph} ${}{*}\\{raman}(\,){}$;\par
\fi

\M{2}The subroutine call \PB{$\\{raman}(\|p,\|q,\\{type},\\{reduce})$}
constructs an undirected graph in which each vertex has degree~\PB{$\|p+%
\T{1}$}.
The number of vertices is~\PB{$\|q+\T{1}$} if \PB{$\\{type}\K\T{1}$}, or~${1%
\over2}q(q+1)$ if \PB{$\\{type}\K\T{2}$},
or ${1\over2}(q-1)q(q+1)$ if \PB{$\\{type}\K\T{3}$}, or \PB{$(\|q-\T{1})\|q(%
\|q+\T{1})$} if
\PB{$\\{type}\K\T{4}$}. The graph will be bipartite if and only if it has
type~4.
Parameters \PB{\|p} and \PB{\|q} must be distinct prime numbers,
and \PB{\|q}~must be odd. Furthermore there are additional restrictions:
If \PB{$\|p\K\T{2}$}, the other parameter \PB{\|q} must satisfy $q\bmod8\in%
\{1,3\}$
and $q\bmod13\in\{1,3,4,9,10,12\}$; this rules out about one fourth of
all primes. Moreover, if \PB{$\\{type}\K\T{3}$} the value of \PB{\|p} must be a
quadratic residue modulo~$q$; in other words, there must be an
integer~$x$ such that $x^2\=p$ (mod~$q$). If \PB{$\\{type}\K\T{4}$}, the value
of \PB{\|p}
must not be a quadratic residue.

If you specify \PB{$\\{type}\K\T{0}$}, the procedure
will choose the largest permissible type (either 3 or~4);
the value of the type selected will
appear as part of the string placed in the resulting graph's \PB{\\{id}} field.
For example, if \PB{$\\{type}\K\T{0}$}, \PB{$\|p\K\T{2}$}, and \PB{$\|q\K%
\T{43}$}, a type~4 graph will be
generated, because 2 is not a quadratic residue modulo~43. This
graph will have $44\times43\times42=79464$ vertices, each of degree~3.
(Notice that graphs of types 3 and~4 can be quite large even when
\PB{\|q} is rather small.)

The largest permissible value of \PB{\|q} is 46337; this is the largest
prime whose square is less than $2^{31}$. Of course you would use
it only for a graph of type~1.

If \PB{\\{reduce}} is nonzero, loops and multiple edges will be suppressed.
In this case the degrees of some vertices might turn out to be less than~\PB{$%
\|p+\T{1}$},
in spite of what was said above.

Although type 4 graphs are bipartite, the vertices
are not separated into two blocks as in other bipartite
graphs produced by GraphBase routines.

All edges of the graphs have length 1.

\fi

\M{3}If the \PB{\\{raman}} routine encounters a problem, it returns \PB{$%
\NULL$}
(\.{NULL}), after putting a code number into the external variable
\PB{\\{panic\_code}}. This code number identifies the type of failure.
Otherwise \PB{\\{raman}} returns a pointer to the newly created graph, which
will be represented with the data structures explained in {\sc GB\_\,GRAPH}.
(The external variable \PB{\\{panic\_code}} is itself defined in
{\sc GB\_\,GRAPH}.)

\Y\B\4\D$\\{panic}(\|c)$ \5
${}\{{}$\5
\1${}\\{panic\_code}\K\|c{}$;\5
${}\\{gb\_trouble\_code}\K\T{0}{}$;\5
\&{return} ${}\NULL{}$;\5
${}\}{}$\2\par
\B\4\D$\\{dead\_panic}(\|c)$ \6
${}\{{}$\5
\1\\{gb\_free}(\\{working\_storage});\5
\\{panic}(\|c);\5
${}\}{}$\2\par
\B\4\D$\\{late\_panic}(\|c)$ \6
${}\{{}$\5
\1\\{gb\_recycle}(\\{new\_graph});\5
\\{dead\_panic}(\|c);\5
${}\}{}$\2\par
\fi

\M{4}The \CEE/ file \.{gb\_raman.c} has the following general shape:

\Y\B\8\#\&{include} \.{"gb\_graph.h"}\C{ we will use the {\sc GB\_\,GRAPH} data
structures }\6
\ATH\7
\X18:Type declarations\X\6
\X6:Private variables and routines\X\7
\1\1\&{Graph} ${}{*}\\{raman}(\|p,\39\|q,\39\\{type},\39\\{reduce}){}$\6
\&{long} \|p;\C{ one less than the desired degree; must be prime }\6
\&{long} \|q;\C{ size parameter; must be prime and properly related to \PB{%
\\{type}} }\6
\&{unsigned} \&{long} \\{type};\C{ selector between different possible
constructions }\6
\&{unsigned} \&{long} \\{reduce};\C{ if nonzero, multiple edges and self-loops
won't occur }\2\2\6
${}\{{}$\5
\1\X5:Local variables\X\7
\X7:Prepare tables for doing arithmetic modulo~\PB{\|q}\X;\6
\X12:Choose or verify the \PB{\\{type}}, and determine the number \PB{\|n} of
vertices\X;\6
\X13:Set up a graph with \PB{\|n} vertices, and assign vertex labels\X;\6
\X19:Compute \PB{$\|p+\T{1}$} generators that will define the graph's edges\X;\6
\X26:Append the edges\X;\6
\&{if} (\\{gb\_trouble\_code})\1\5
\\{late\_panic}(\\{alloc\_fault});\C{ oops, we ran out of memory somewhere back
there }\2\6
\\{gb\_free}(\\{working\_storage});\6
\&{return} \\{new\_graph};\6
\4${}\}{}$\2\par
\fi

\M{5}\B\X5:Local variables\X${}\E{}$\6
\&{Graph} ${}{*}\\{new\_graph}{}$;\C{ the graph constructed by \PB{\\{raman}} }%
\6
\&{Area} \\{working\_storage};\C{ place for auxiliary tables }\par
\A9.
\U4.\fi

\N{1}{6}Brute force number theory. Instead of using routines like Euclid's
algorithm to compute inverses and square roots modulo~\PB{\|q}, we have
plenty of time to build complete tables, since \PB{\|q} is smaller than
the number of vertices we will be generating.

We will make three tables: \PB{\\{q\_sqr}[\|k]} will contain $k^2$ modulo~\PB{%
\|q};
\PB{\\{q\_sqrt}[\|k]} will contain one of the values of $\sqrt{\mathstrut k}$
if $k$ is a quadratic residue; and \PB{\\{q\_inv}[\|k]} will contain the
multiplicative
inverse of~\PB{\|k}.

\Y\B\4\X6:Private variables and routines\X${}\E{}$\6
\&{static} \&{long} ${}{*}\\{q\_sqr}{}$;\C{ squares }\6
\&{static} \&{long} ${}{*}\\{q\_sqrt}{}$;\C{ square roots (or $-1$ if not a
quadratic residue) }\6
\&{static} \&{long} ${}{*}\\{q\_inv}{}$;\C{ reciprocals }\par
\As15, 20, 22\ETs30.
\U4.\fi

\M{7}\B\X7:Prepare tables for doing arithmetic modulo~\PB{\|q}\X${}\E{}$\6
\&{if} ${}(\|q<\T{3}\V\|q>\T{46337}){}$\1\5
\\{panic}(\\{very\_bad\_specs});\C{ \PB{\|q} is way too small or way too big }%
\2\6
\&{if} ${}(\|p<\T{2}){}$\1\5
${}\\{panic}(\\{very\_bad\_specs}+\T{1}){}$;\C{ \PB{\|p} is way too small }\2\6
\\{init\_area}(\\{working\_storage});\6
${}\\{q\_sqr}\K\\{gb\_typed\_alloc}(\T{3}*\|q,\39\&{long},\39\\{working%
\_storage});{}$\6
\&{if} ${}(\\{q\_sqr}\E\T{0}){}$\1\5
${}\\{panic}(\\{no\_room}+\T{1});{}$\2\6
${}\\{q\_sqrt}\K\\{q\_sqr}+\|q;{}$\6
${}\\{q\_inv}\K\\{q\_sqrt}+\|q{}$;\C{ note that \PB{\\{gb\_alloc}} has
initialized everything to zero }\6
\X8:Compute the \PB{\\{q\_sqr}} and \PB{\\{q\_sqrt}} tables\X;\6
\X10:Find a primitive root \PB{\|a}, modulo \PB{\|q}, and its inverse \PB{%
\\{aa}}\X;\6
\X11:Compute the \PB{\\{q\_inv}} table\X;\par
\U4.\fi

\M{8}\B\X8:Compute the \PB{\\{q\_sqr}} and \PB{\\{q\_sqrt}} tables\X${}\E{}$\6
\&{for} ${}(\|a\K\T{1};{}$ ${}\|a<\|q;{}$ ${}\|a\PP){}$\1\5
${}\\{q\_sqrt}[\|a]\K{-}\T{1};{}$\2\6
\&{for} ${}(\|a\K\T{1},\39\\{aa}\K\T{1};{}$ ${}\|a<\|q;{}$ ${}\\{aa}\K(\\{aa}+%
\|a+\|a+\T{1})\MOD\|q,\39\|a\PP){}$\5
${}\{{}$\1\6
${}\\{q\_sqr}[\|a]\K\\{aa};{}$\6
${}\\{q\_sqrt}[\\{aa}]\K\|q-\|a{}$;\C{ the smaller square root will survive }\6
${}\\{q\_inv}[\\{aa}]\K{-}\T{1}{}$;\C{ we make \PB{\\{q\_inv}[\\{aa}]} nonzero
when \PB{\\{aa}} can't be a primitive root }\6
\4${}\}{}$\2\par
\U7.\fi

\M{9}\B\X5:Local variables\X${}\mathrel+\E{}$\6
\&{register} \&{long} \|a${},\39\\{aa},\39\|k{}$;\C{ primary indices in loops }%
\6
\&{long} \|b${},\39\\{bb},\39\|c,\39\\{cc},\39\|d,\39\\{dd}{}$;\C{ secondary
indices }\6
\&{long} \|n;\C{ the number of vertices }\6
\&{long} \\{n\_factor};\C{ either ${1\over2}(q-1)$ (type~3) or $q-1$ (type 4) }%
\6
\&{register} \&{Vertex} ${}{*}\|v{}$;\C{ the current vertex of interest }\par
\fi

\M{10}Here we implicitly test that \PB{\|q} is prime, by finding a primitive
root whose powers generate everything. If \PB{\|q} is not prime, its smallest
divisor will cause the inner loop in this step to terminate with \PB{$\|k\G%
\|q$},
because no power of that divisor will be congruent to~1.

\Y\B\4\X10:Find a primitive root \PB{\|a}, modulo \PB{\|q}, and its inverse %
\PB{\\{aa}}\X${}\E{}$\6
\&{for} ${}(\|a\K\T{2};{}$  ; ${}\|a\PP){}$\1\6
\&{if} ${}(\\{q\_inv}[\|a]\E\T{0}){}$\5
${}\{{}$\1\6
\&{for} ${}(\|b\K\|a,\39\|k\K\T{1};{}$ ${}\|b\I\T{1}\W\|k<\|q;{}$ ${}\\{aa}\K%
\|b,\39\|b\K(\|a*\|b)\MOD\|q,\39\|k\PP){}$\1\5
${}\\{q\_inv}[\|b]\K{-}\T{1};{}$\2\6
\&{if} ${}(\|k\G\|q){}$\1\5
${}\\{dead\_panic}(\\{bad\_specs}+\T{1}){}$;\C{ \PB{\|q} is not prime }\2\6
\&{if} ${}(\|k\E\|q-\T{1}){}$\1\5
\&{break};\C{ good, \PB{\|a} is the primitive root we seek }\2\6
\4${}\}{}$\2\2\par
\U7.\fi

\M{11}As soon as we have discovered
a primitive root, it is easy to generate all the inverses. (We
could also generate the discrete logarithms if we had a need for them.)

We set \PB{$\\{q\_inv}[\T{0}]\K\|q$}; this will be our internal representation
of $\infty$.

\Y\B\4\X11:Compute the \PB{\\{q\_inv}} table\X${}\E{}$\6
\&{for} ${}(\|b\K\|a,\39\\{bb}\K\\{aa};{}$ ${}\|b\I\\{bb};{}$ ${}\|b\K(\|a*\|b)%
\MOD\|q,\39\\{bb}\K(\\{aa}*\\{bb})\MOD\|q){}$\1\5
${}\\{q\_inv}[\|b]\K\\{bb},\39\\{q\_inv}[\\{bb}]\K\|b;{}$\2\6
${}\\{q\_inv}[\T{1}]\K\T{1};{}$\6
${}\\{q\_inv}[\|b]\K\|b{}$;\C{ at this point \PB{\|b} must equal \PB{$\|q-%
\T{1}$} }\6
${}\\{q\_inv}[\T{0}]\K\|q{}$;\par
\U7.\fi

\M{12}The conditions we stated for validity of \PB{\|q} when \PB{$\|p\K\T{2}$}
are equivalent
to the existence of $\sqrt{-2}$ and $\sqrt{13}$ modulo~\PB{\|q}, according
to the law of quadratic reciprocity (see, for example, {\sl Fundamental
Algorithms}, exercise 1.2.4--47).

\Y\B\4\X12:Choose or verify the \PB{\\{type}}, and determine the number \PB{%
\|n} of vertices\X${}\E{}$\6
\&{if} ${}(\|p\E\T{2}){}$\5
${}\{{}$\1\6
\&{if} ${}(\\{q\_sqrt}[\T{13}\MOD\|q]<\T{0}\V\\{q\_sqrt}[\|q-\T{2}]<\T{0}){}$\1%
\5
${}\\{dead\_panic}(\\{bad\_specs}+\T{2}){}$;\C{ improper prime to go with \PB{$%
\|p\K\T{2}$} }\2\6
\4${}\}{}$\2\6
\&{if} ${}((\|a\K\|p\MOD\|q)\E\T{0}){}$\1\5
${}\\{dead\_panic}(\\{bad\_specs}+\T{3}){}$;\C{ \PB{\|p} divisible by \PB{\|q}
}\2\6
\&{if} ${}(\\{type}\E\T{0}){}$\1\5
${}\\{type}\K(\\{q\_sqrt}[\|a]>\T{0}\?\T{3}:\T{4});{}$\2\6
${}\\{n\_factor}\K(\\{type}\E\T{3}\?(\|q-\T{1})/\T{2}:\|q-\T{1});{}$\6
\&{switch} (\\{type})\5
${}\{{}$\1\6
\4\&{case} \T{1}:\5
${}\|n\K\|q+\T{1}{}$;\5
\&{break};\6
\4\&{case} \T{2}:\5
${}\|n\K\|q*(\|q+\T{1})/\T{2}{}$;\5
\&{break};\6
\4\&{default}:\6
\&{if} ${}((\\{q\_sqrt}[\|a]>\T{0}\W\\{type}\I\T{3})\V(\\{q\_sqrt}[\|a]<\T{0}\W%
\\{type}\I\T{4}){}$)\1\6
${}\\{dead\_panic}(\\{bad\_specs}+\T{4}){}$;\C{ wrong type for \PB{\|p} modulo %
\PB{\|q} }\2\6
\&{if} ${}(\|q>\T{1289}){}$\1\5
${}\\{dead\_panic}(\\{bad\_specs}+\T{5}){}$;\C{ way too big for types 3, 4 }\2\6
${}\|n\K\\{n\_factor}*\|q*(\|q+\T{1});{}$\6
\&{break};\6
\4${}\}{}$\2\6
\&{if} ${}(\|p\G(\&{long})(\T{\^3fffffff}/\|n)){}$\1\5
${}\\{dead\_panic}(\\{bad\_specs}+\T{6}){}$;\C{ $(p+1)n\ge2^{30}$ }\2\par
\U4.\fi

\N{1}{13}The vertices. Graphs of type 1 have vertices from the
set $\{0,1,\ldots,q-1,\infty\}$, namely the integers modulo~\PB{\|q} with
an additional ``infinite'' element thrown in. The idea is to
operate on these quantities by adding constants, and/or multiplying by
constants, and/or taking reciprocals, modulo~\PB{\|q}.

Graphs of type 2 have vertices that are unordered pairs of
distinct elements from that same $(q+1)$-element set.

Graphs of types 3 and 4 have vertices that are $2\times2$ matrices
having nonzero determinants modulo~\PB{\|q}. The determinants of type~3
matrices
are, in fact, nonzero quadratic residues. We consider two matrices to be
equivalent if one is obtained from the other by multiplying all entries
by a constant (modulo~\PB{\|q}); therefore we will normalize all the matrices
so that the second row is either $(0,1)$ or has the form $(1,x)$ for
some~$x$. The total number of equivalence classes of type~4 matrices obtainable
in this way is $(q+1)q(q-1)$, because we can choose the second row in
$q+1$ ways, after which there are two cases: Either the second row is
$(0,1)$, and we can select the upper right corner element arbitrarily
and choose the upper left corner element nonzero; or the second row is $(1,x)$,
and we can select the upper left corner element arbitrarily and then choose
an upper right corner element to make the determinant nonzero. For type~3
the counting is similar, except that ``nonzero'' becomes ``nonzero
quadratic residue,'' hence there are exactly half as many choices.

It is easy to verify that the equivalence classes of matrices that
correspond to vertices in these graphs of types 3 and~4 are closed
under matrix multiplication. Therefore the vertices can be regarded as the
elements of finite groups. The type~3 group for a given \PB{\|q} is often
called the linear fractional group $LF(2,{\bf F}_q)$, or the
projective special linear group $PSL(2,{\bf F}_q)$, or the linear
simple group $L_2(q)$; it can also be regarded as the group of
$2\times2$ matrices with determinant~1 (mod~$q$), when the matrix $A$
is considered equivalent to $-A$. (This group is a simple group for
all primes \PB{$\|q>\T{2}$}.) The type~4 group is officially known as the
projective general linear group of degree~2 over the field of \PB{%
\|q}~elements,
$PGL(2,{\bf F}_q)$.

\Y\B\4\X13:Set up a graph with \PB{\|n} vertices, and assign vertex labels\X${}%
\E{}$\6
$\\{new\_graph}\K\\{gb\_new\_graph}(\|n);{}$\6
\&{if} ${}(\\{new\_graph}\E\NULL){}$\1\5
\\{dead\_panic}(\\{no\_room});\C{ out of memory before we try to add edges }\2\6
${}\\{sprintf}(\\{new\_graph}\MG\\{id},\39\.{"raman(\%ld,\%ld,\%lu,\%}\)%
\.{lu)"},\39\|p,\39\|q,\39\\{type},\39\\{reduce});{}$\6
${}\\{strcpy}(\\{new\_graph}\MG\\{util\_types},\39\.{"ZZZIIZIZZZZZZZ"});{}$\6
${}\|v\K\\{new\_graph}\MG\\{vertices};{}$\6
\&{switch} (\\{type})\5
${}\{{}$\1\6
\4\&{case} \T{1}:\5
\X14:Assign labels from the set $\{0,1,\ldots,q-1,\infty\}$\X;\5
\&{break};\6
\4\&{case} \T{2}:\5
\X16:Assign labels for pairs of distinct elements\X;\5
\&{break};\6
\4\&{default}:\5
\X17:Assign projective matrix labels\X;\5
\&{break};\6
\4${}\}{}$\2\par
\U4.\fi

\M{14}Type 1 graphs are the easiest to label. We store a serial number
in utility field \PB{$\|x.\|I$}, using $q$ to represent~$\infty$.

\Y\B\4\X14:Assign labels from the set $\{0,1,\ldots,q-1,\infty\}$\X${}\E{}$\6
$\\{new\_graph}\MG\\{util\_types}[\T{4}]\K\.{'Z'};{}$\6
\&{for} ${}(\|a\K\T{0};{}$ ${}\|a<\|q;{}$ ${}\|a\PP){}$\5
${}\{{}$\1\6
${}\\{sprintf}(\\{name\_buf},\39\.{"\%ld"},\39\|a);{}$\6
${}\|v\MG\\{name}\K\\{gb\_save\_string}(\\{name\_buf});{}$\6
${}\|v\MG\|x.\|I\K\|a;{}$\6
${}\|v\PP;{}$\6
\4${}\}{}$\2\6
${}\|v\MG\\{name}\K\\{gb\_save\_string}(\.{"INF"});{}$\6
${}\|v\MG\|x.\|I\K\|q;{}$\6
${}\|v\PP{}$;\par
\U13.\fi

\M{15}\B\X6:Private variables and routines\X${}\mathrel+\E{}$\6
\&{static} \&{char} \\{name\_buf}[\,]${}\K\.{"(1111,1111;1,1111)"}{}$;\C{ place
to form vertex names }\par
\fi

\M{16}The type 2 labels run from $\{0,1\}$ to $\{q-1,\infty\}$; we put the
coefficients into \PB{$\|x.\|I$} and \PB{$\|y.\|I$}, where they might prove
useful in
some applications.

\Y\B\4\X16:Assign labels for pairs of distinct elements\X${}\E{}$\6
\&{for} ${}(\|a\K\T{0};{}$ ${}\|a<\|q;{}$ ${}\|a\PP){}$\1\6
\&{for} ${}(\\{aa}\K\|a+\T{1};{}$ ${}\\{aa}\Z\|q;{}$ ${}\\{aa}\PP){}$\5
${}\{{}$\1\6
\&{if} ${}(\\{aa}\E\|q){}$\1\5
${}\\{sprintf}(\\{name\_buf},\39\.{"\{\%ld,INF\}"},\39\|a);{}$\2\6
\&{else}\1\5
${}\\{sprintf}(\\{name\_buf},\39\.{"\{\%ld,\%ld\}"},\39\|a,\39\\{aa});{}$\2\6
${}\|v\MG\\{name}\K\\{gb\_save\_string}(\\{name\_buf});{}$\6
${}\|v\MG\|x.\|I\K\|a{}$;\5
${}\|v\MG\|y.\|I\K\\{aa};{}$\6
${}\|v\PP;{}$\6
\4${}\}{}$\2\2\par
\U13.\fi

\M{17}For graphs of types 3 and 4, we set the \PB{$\|x.\|I$} and \PB{$\|y.\|I$}
fields to
the elements of the first row of the matrix, and we set the \PB{$\|z.\|I$}
field equal to the ratio of the elements of the second row (again with $q$
representing~$\infty$).

The vertices in this case consist of \PB{$\|q(\|q+\T{1})$} blocks of vertices
having a given second row and a given element in the upper left or upper right
position. Within each block of vertices, the determinants are
respectively congruent modulo~\PB{\|q} to $1^2$, $2^2$, \dots,~$({q-1%
\over2})^2$
in the case of type~3 graphs, or to 1,~2, \dots,~$q-1$ in the case of type~4.

\Y\B\4\X17:Assign projective matrix labels\X${}\E{}$\6
$\\{new\_graph}\MG\\{util\_types}[\T{5}]\K\.{'I'};{}$\6
\&{for} ${}(\|c\K\T{0};{}$ ${}\|c\Z\|q;{}$ ${}\|c\PP){}$\1\6
\&{for} ${}(\|b\K\T{0};{}$ ${}\|b<\|q;{}$ ${}\|b\PP){}$\1\6
\&{for} ${}(\|a\K\T{1};{}$ ${}\|a\Z\\{n\_factor};{}$ ${}\|a\PP){}$\5
${}\{{}$\1\6
${}\|v\MG\|z.\|I\K\|c;{}$\6
\&{if} ${}(\|c\E\|q){}$\5
${}\{{}$\C{ second row of matrix is $(0,1)$ }\1\6
${}\|v\MG\|y.\|I\K\|b;{}$\6
${}\|v\MG\|x.\|I\K(\\{type}\E\T{3}\?\\{q\_sqr}[\|a]:\|a){}$;\C{ determinant is
$a^2$ or $a$ }\6
${}\\{sprintf}(\\{name\_buf},\39\.{"(\%ld,\%ld;0,1)"},\39\|v\MG\|x.\|I,\39%
\|b);{}$\6
\4${}\}{}$\5
\2\&{else}\5
${}\{{}$\C{ second row of matrix is $(1,c)$ }\1\6
${}\|v\MG\|x.\|I\K\|b;{}$\6
${}\|v\MG\|y.\|I\K(\|b*\|c+\|q-(\\{type}\E\T{3}\?\\{q\_sqr}[\|a]:\|a))\MOD%
\|q{}$;\C{ determinant is $a^2$ or $a$ }\6
${}\\{sprintf}(\\{name\_buf},\39\.{"(\%ld,\%ld;1,\%ld)"},\39\|b,\39\|v\MG\|y.%
\|I,\39\|c);{}$\6
\4${}\}{}$\2\6
${}\|v\MG\\{name}\K\\{gb\_save\_string}(\\{name\_buf});{}$\6
${}\|v\PP;{}$\6
\4${}\}{}$\2\2\2\par
\U13.\fi

\N{1}{18}Group generators. We will define a set of \PB{$\|p+\T{1}$}
permutations $\{\pi_0,
\pi_1,\ldots,\pi_p\}$ of the vertices, such that the arcs of our graph will
go from $v$ to $v\pi_k$ for \PB{$\T{0}\Z\|k\Z\|p$}. Thus, each path in the
graph will be
defined by a product of permutations; the cycles of the graph will correspond
to vertices that are left fixed by a product of permutations.
The graph will be undirected, because the inverse of each $\pi_k$ will
also be one of the permutations of the generating set.

In fact, each permutation $\pi_k$ will be defined by a $2\times2$ matrix.
For graphs of types 3 and~4, the permutations will therefore correspond to
certain vertices, and the vertex $v\pi_k$ will simply be the product of matrix
$v$ by matrix $\pi_k$.

For graphs of type 1, the permutations will be defined by linear fractional
transformations, which are mappings of the form
$$v\;\longmapsto\; {av+b\over
cv+d}\bmod q\,.$$
This transformation applies
to all $v\in\{0,1,\ldots,q-1,\infty\}$, under the usual conventions
that $x/0=\infty$ when $x\ne0$ and $(x\infty+x')/(y\infty+y')=x/y$.
The composition of two such transformations is again a linear fractional
transformation, corresponding to the product of the two associated
matrices $\bigl({a\,b\atop c\,d}\bigr)$.

Graphs of type 2 will be handled just like graphs of type 1,
except that we will compute the images of two distinct points
$v=\{v_1,v_2\}$ under the linear fractional transformation. The two
images will be distinct, because the transformation is invertible.

When \PB{$\|p\K\T{2}$}, a special set of three generating matrices $\pi_0$, $%
\pi_1$,
$\pi_2$ can be shown to define Ramanujan graphs; these matrices are
described below. Otherwise \PB{\|p} is odd, and the generators are based on the
theory of integral quaternions. Integral quaternions, for our purposes,
are quadruples of the form
$\alpha=a_0+a_1i+a_2j+a_3k$, where $a_0$, $a_1$, $a_2$, and~$a_3$ are
integers; we multiply them by using the associative but
noncommutative multiplication rules $i^2=j^2=k^2=ijk=-1$. If we write
$\alpha=a+A$, where $a$ is the ``scalar'' $a_0$ and $A$ is the ``vector''
$a_1i+a_2j+a_3k$, the product of quaternions $\alpha=a+A$ and $\beta=b+B$
can be expressed as
$$(a+A)(b+B)=ab-A\cdot B+aB+bA+A\times B\,,$$
where $A\cdot B$ and $A\times B$ are the usual dot product and cross
product of vectors. The conjugate of $\alpha=a+A$ is $\overline\alpha=a-A$,
and we have $\alpha\overline\alpha=a_0^2+a_1^2+a_2^2+a_3^2$. This
important quantity is called $N(\alpha)$, the norm of $\alpha$. It
is not difficult to verify that $N(\alpha\beta)=N(\alpha)N(\beta)$,
because of the basic identity
$\overline{\mathstrut\alpha\beta}=\overline{\mathstrut\beta}
\,\overline{\mathstrut\alpha}$ and the fact that
$\alpha x=x\alpha$ when $x$ is scalar.

Integral quaternions have a beautiful theory; for example, there is a
nice variant of Euclid's algorithm by which we can compute the greatest common
left divisor of any two integral quaternions having odd norm. This algorithm
makes it possible to prove that integral quaternions whose coefficients are
relatively prime can be canonically factored into quaternions whose norm is
prime. However, the details of that theory are beyond the scope of this
documentation. It will suffice for our purposes
to observe that we can use quaternions to define the finite groups
$PSL(2,{\bf F}_q)$ and $PGL(2,{\bf F}_q)$ in a different way from the
definitions given earlier: Suppose
we consider two quaternions to be equivalent if
one is a nonzero scalar multiple of the other,
modulo~$q$. Thus, for example, if $q=3$ we consider $1+4i-j$ to
be equivalent to $1+i+2j$, and also equivalent to $2+2i+j$.
It turns out that there are exactly $(q+1)q(q-1)$ such equivalence classes,
when we omit quaternions whose norm is a multiple of~$q$;
and they form a group under quaternion multiplication that is the same as the
projective group of $2\times2$ matrices under matrix multiplication,
modulo~\PB{\|q}. One way to prove this
is by means of the one-to-one correspondence
$$a_0+a_1i+a_2j+a_3k\;\longleftrightarrow\;
\left(\matrix{a_0+a_1g+a_3h&a_2+a_3g-a_1h\cr
-a_2+a_3g-a_1h&a_0-a_1g-a_3h\cr}\right)\,,$$
where $g$ and $h$ are integers with $g^2+h^2\=-1$ (mod~\PB{\|q}).

Jacobi proved that the number of ways to represent
any odd number \PB{\|p} as a sum of four squares $a_0^2+a_1^2+a_2^2+a_3^2$
is 8 times the sum of divisors of~\PB{\|p}. [This fact appears in the
concluding sentence of his monumental work {\sl Fundamenta Nova
Theori\ae\ Functionum Ellipticorum}, K\"onigsberg, 1829.]
In particular, when \PB{\|p} is prime,
the number of such representations is $8(p+1)$; in other words, there are
exactly $8(p+1)$ quaternions $\alpha=a_0+a_1i+a_2j+a_3k$ with $N(\alpha)=p$.
These quaternions form \PB{$\|p+\T{1}$} equivalence classes under
multiplication
by the eight ``unit quaternions'' $\{\pm1,\pm i,\pm j,\pm k\}$. We will
select one element from each equivalence class, and the resulting \PB{$\|p+%
\T{1}$}
quaternions will correspond to \PB{$\|p+\T{1}$} matrices, which will generate
the \PB{$\|p+\T{1}$}
arcs leading from each vertex in the graphs to be constructed.

\Y\B\4\X18:Type declarations\X${}\E{}$\6
\&{typedef} \&{struct} ${}\{{}$\1\6
\&{long} \\{a0}${},\39\\{a1},\39\\{a2},\39\\{a3}{}$;\C{ coefficients of a
quaternion }\6
\&{unsigned} \&{long} \\{bar};\C{ the index of the inverse (conjugate)
quaternion }\2\6
${}\}{}$ \&{quaternion};\par
\U4.\fi

\M{19}A global variable \PB{\\{gen\_count}} will be declared below,
indicating the number of generators found so far. When \PB{\|p} isn't prime,
we will find more than \PB{$\|p+\T{1}$} solutions, so we allocate an extra slot
in
the \PB{\\{gen}} table to hold a possible overflow entry.

\Y\B\4\X19:Compute \PB{$\|p+\T{1}$} generators that will define the graph's
edges\X${}\E{}$\6
$\\{gen}\K\\{gb\_typed\_alloc}(\|p+\T{2},\39\&{quaternion},\39\\{working%
\_storage});{}$\6
\&{if} ${}(\\{gen}\E\NULL){}$\1\5
${}\\{late\_panic}(\\{no\_room}+\T{2}){}$;\C{ not enough memory }\2\6
${}\\{gen\_count}\K\T{0}{}$;\5
${}\\{max\_gen\_count}\K\|p+\T{1};{}$\6
\&{if} ${}(\|p\E\T{2}){}$\1\5
\X25:Fill the \PB{\\{gen}} table with special generators\X\2\6
\&{else}\1\5
\X21:Fill the \PB{\\{gen}} table with representatives of all quaternions having
norm~\PB{\|p}\X;\2\6
\&{if} ${}(\\{gen\_count}\I\\{max\_gen\_count}){}$\1\5
${}\\{late\_panic}(\\{bad\_specs}+\T{7}){}$;\C{ \PB{\|p} is not prime }\2\par
\U4.\fi

\M{20}\B\X6:Private variables and routines\X${}\mathrel+\E{}$\6
\&{static} \&{quaternion} ${}{*}\\{gen}{}$;\C{ table of the \PB{$\|p+\T{1}$}
generators }\par
\fi

\M{21}As mentioned above, quaternions of norm \PB{\|p} come in sets of 8,
differing from each other only by unit multiples; we need to choose one
of the~8. Suppose $a_0^2+a_1^2+a_2^2+a_3^2=p$.
If $p\bmod4=1$, exactly one of the $a$'s will be odd;
so we call it $a_0$ and assign it a positive sign. When $p\bmod4=3$, exactly
one of the $a$'s will be even; we call it $a_0$, and if it is nonzero we
make it positive. If $a_0=0$, we make sure that one of the
others---say the rightmost appearance of the largest one---is positive.
In this way we obtain a unique representative from each set of 8 equivalent
quaternions.

For example, the four quaternions of norm 3 are $\null\pm i\pm j+k$; the six
of norm~5 are $1\pm2i$, $1\pm2j$, $1\pm2k$.

In the program here we generate solutions to $a^2+b^2+c^2+d^2=p$ when
$a\not\=b\=c\=d$ (mod~2) and $b\le c\le d$. The variables \PB{\\{aa}}, \PB{%
\\{bb}}, and \PB{\\{cc}}
hold the respective values $p-a^2-b^2-c^2-d^2$, $p-a^2-3b^2$, and
$p-a^2-2c^2$. The \PB{\&{for}} statements use the fact that $a^2$ increases
by $4(a+1)$ when $a$ increases by~2.

\Y\B\4\X21:Fill the \PB{\\{gen}} table with representatives of all quaternions
having norm~\PB{\|p}\X${}\E{}$\6
${}\{{}$\5
\1\&{long} \\{sa}${},\39\\{sb}{}$;\C{ $p-a^2$, $p-a^2-b^2$ }\6
\&{long} \\{pp}${}\K(\|p\GG\T{1})\AND\T{1}{}$;\C{ 0 if $p\bmod4=1$, \ 1 if $p%
\bmod4=3$ }\7
\&{for} ${}(\|a\K\T{1}-\\{pp},\39\\{sa}\K\|p-\|a;{}$ ${}\\{sa}>\T{0};{}$ ${}%
\\{sa}\MRL{-{\K}}(\|a+\T{1})\LL\T{2},\39\|a\MRL{+{\K}}\T{2}){}$\1\6
\&{for} ${}(\|b\K\\{pp},\39\\{sb}\K\\{sa}-\|b,\39\\{bb}\K\\{sb}-\|b-\|b;{}$ ${}%
\\{bb}\G\T{0};{}$ ${}\\{bb}\MRL{-{\K}}\T{12}*(\|b+\T{1}),\39\\{sb}\MRL{-{\K}}(%
\|b+\T{1})\LL\T{2},\39\|b\MRL{+{\K}}\T{2}){}$\1\6
\&{for} ${}(\|c\K\|b,\39\\{cc}\K\\{bb};{}$ ${}\\{cc}\G\T{0};{}$ ${}\\{cc}%
\MRL{-{\K}}(\|c+\T{1})\LL\T{3},\39\|c\MRL{+{\K}}\T{2}){}$\1\6
\&{for} ${}(\|d\K\|c,\39\\{aa}\K\\{cc};{}$ ${}\\{aa}\G\T{0};{}$ ${}\\{aa}%
\MRL{-{\K}}(\|d+\T{1})\LL\T{2},\39\|d\MRL{+{\K}}\T{2}){}$\1\6
\&{if} ${}(\\{aa}\E\T{0}){}$\1\5
\X23:Deposit the quaternions associated with $a+bi+cj+dk$\X;\2\2\2\2\2\6
\X24:Change the \PB{\\{gen}} table to matrix format\X;\6
\4${}\}{}$\2\par
\U19.\fi

\M{22}If \PB{$\|a>\T{0}$} and \PB{$\T{0}<\|b<\|c<\|d$}, we obtain 48 different
classes of quaternions
having the same norm by permuting $\{b,c,d\}$ in six ways and attaching
signs to each permutation in eight ways. This happens, for example,
when $p=71$ and $(a,b,c,d)=(6,1,3,5)$. Fewer quaternions arise when
\PB{$\|a\K\T{0}$} or \PB{$\T{0}\K\|b$} or \PB{$\|b\K\|c$} or \PB{$\|c\K\|d$}.

The inverse of the matrix corresponding to a quaternion is the matrix
corresponding to the conjugate quaternion. Therefore a generating
matrix $\pi_k$ will be its own inverse if and only if it comes from
a quaternion with \PB{$\|a\K\T{0}$}.

It is convenient to have a subroutine that deposits a new quaternion
and its conjugate into the table of generators.

\Y\B\4\X6:Private variables and routines\X${}\mathrel+\E{}$\6
\&{static} \&{unsigned} \&{long} \\{gen\_count};\C{ the next available
quaternion slot }\6
\&{static} \&{unsigned} \&{long} \\{max\_gen\_count};\C{ $p+1$, stored as a
global variable }\7
\1\1\&{static} \&{void} ${}\\{deposit}(\|a,\39\|b,\39\|c,\39\|d){}$\6
\&{long} \|a${},\39\|b,\39\|c,\39\|d{}$;\C{ a solution to $a^2+b^2+c^2+d^2=p$ }%
\2\2\6
${}\{{}$\1\6
\&{if} ${}(\\{gen\_count}\G\\{max\_gen\_count}{}$)\C{ oops, we already found %
\PB{$\|p+\T{1}$} solutions }\1\6
${}\\{gen\_count}\K\\{max\_gen\_count}+\T{1}{}$;\C{ this will happen only if %
\PB{\|p} isn't prime }\2\6
\&{else}\5
${}\{{}$\1\6
${}\\{gen}[\\{gen\_count}].\\{a0}\K\\{gen}[\\{gen\_count}+\T{1}].\\{a0}\K%
\|a;{}$\6
${}\\{gen}[\\{gen\_count}].\\{a1}\K\|b{}$;\5
${}\\{gen}[\\{gen\_count}+\T{1}].\\{a1}\K{-}\|b;{}$\6
${}\\{gen}[\\{gen\_count}].\\{a2}\K\|c{}$;\5
${}\\{gen}[\\{gen\_count}+\T{1}].\\{a2}\K{-}\|c;{}$\6
${}\\{gen}[\\{gen\_count}].\\{a3}\K\|d{}$;\5
${}\\{gen}[\\{gen\_count}+\T{1}].\\{a3}\K{-}\|d;{}$\6
\&{if} (\|a)\5
${}\{{}$\1\6
${}\\{gen}[\\{gen\_count}].\\{bar}\K\\{gen\_count}+\T{1};{}$\6
${}\\{gen}[\\{gen\_count}+\T{1}].\\{bar}\K\\{gen\_count};{}$\6
${}\\{gen\_count}\MRL{+{\K}}\T{2};{}$\6
\4${}\}{}$\5
\2\&{else}\5
${}\{{}$\1\6
${}\\{gen}[\\{gen\_count}].\\{bar}\K\\{gen\_count};{}$\6
${}\\{gen\_count}\PP;{}$\6
\4${}\}{}$\2\6
\4${}\}{}$\2\6
\4${}\}{}$\2\par
\fi

\M{23}\B\X23:Deposit the quaternions associated with $a+bi+cj+dk$\X${}\E{}$\6
${}\{{}$\1\6
${}\\{deposit}(\|a,\39\|b,\39\|c,\39\|d);{}$\6
\&{if} (\|b)\5
${}\{{}$\1\6
${}\\{deposit}(\|a,\39{-}\|b,\39\|c,\39\|d){}$;\5
${}\\{deposit}(\|a,\39{-}\|b,\39{-}\|c,\39\|d);{}$\6
\4${}\}{}$\2\6
\&{if} (\|c)\1\5
${}\\{deposit}(\|a,\39\|b,\39{-}\|c,\39\|d);{}$\2\6
\&{if} ${}(\|b<\|c){}$\5
${}\{{}$\1\6
${}\\{deposit}(\|a,\39\|c,\39\|b,\39\|d){}$;\5
${}\\{deposit}(\|a,\39{-}\|c,\39\|b,\39\|d){}$;\5
${}\\{deposit}(\|a,\39\|c,\39\|d,\39\|b){}$;\5
${}\\{deposit}(\|a,\39{-}\|c,\39\|d,\39\|b);{}$\6
\&{if} (\|b)\5
${}\{{}$\1\6
${}\\{deposit}(\|a,\39\|c,\39{-}\|b,\39\|d){}$;\5
${}\\{deposit}(\|a,\39{-}\|c,\39{-}\|b,\39\|d){}$;\5
${}\\{deposit}(\|a,\39\|c,\39\|d,\39{-}\|b){}$;\5
${}\\{deposit}(\|a,\39{-}\|c,\39\|d,\39{-}\|b);{}$\6
\4${}\}{}$\2\6
\4${}\}{}$\2\6
\&{if} ${}(\|c<\|d){}$\5
${}\{{}$\1\6
${}\\{deposit}(\|a,\39\|b,\39\|d,\39\|c){}$;\5
${}\\{deposit}(\|a,\39\|d,\39\|b,\39\|c);{}$\6
\&{if} (\|b)\5
${}\{{}$\1\6
${}\\{deposit}(\|a,\39{-}\|b,\39\|d,\39\|c){}$;\5
${}\\{deposit}(\|a,\39{-}\|b,\39\|d,\39{-}\|c){}$;\5
${}\\{deposit}(\|a,\39\|d,\39{-}\|b,\39\|c){}$;\5
${}\\{deposit}(\|a,\39\|d,\39{-}\|b,\39{-}\|c);{}$\6
\4${}\}{}$\2\6
\&{if} (\|c)\5
${}\{{}$\1\6
${}\\{deposit}(\|a,\39\|b,\39\|d,\39{-}\|c){}$;\5
${}\\{deposit}(\|a,\39\|d,\39\|b,\39{-}\|c);{}$\6
\4${}\}{}$\2\6
\&{if} ${}(\|b<\|c){}$\5
${}\{{}$\1\6
${}\\{deposit}(\|a,\39\|d,\39\|c,\39\|b){}$;\5
${}\\{deposit}(\|a,\39\|d,\39{-}\|c,\39\|b);{}$\6
\&{if} (\|b)\5
${}\{{}$\1\6
${}\\{deposit}(\|a,\39\|d,\39\|c,\39{-}\|b){}$;\5
${}\\{deposit}(\|a,\39\|d,\39{-}\|c,\39{-}\|b);{}$\6
\4${}\}{}$\2\6
\4${}\}{}$\2\6
\4${}\}{}$\2\6
\4${}\}{}$\2\par
\U21.\fi

\M{24}Once we've found the generators in quaternion form, we want to
convert them to $2\times2$ matrices, using the correspondence mentioned
earlier:
$$a_0+a_1i+a_2j+a_3k\;\longleftrightarrow\;
\left(\matrix{a_0+a_1g+a_3h&a_2+a_3g-a_1h\cr
-a_2+a_3g-a_1h&a_0-a_1g-a_3h\cr}\right)\,,$$
where $g$ and $h$ are integers with $g^2+h^2\=-1$ (mod~\PB{\|q}).
Appropriate values for $g$ and~$h$ can always be found by the formulas
$$g=\sqrt{\mathstrut k}\qquad\hbox{and}\qquad h=\sqrt{\mathstrut q-1-k},$$
where $k$ is the largest quadratic residue modulo~\PB{\|q}. For if $q-1$ is
not a quadratic residue, and if $k+1$ isn't a residue either, then
$q-1-k$ must be a quadratic residue because it is congruent to the
product $(q-1)(k+1)$ of nonresidues. (We will have \PB{$\|h\K\T{0}$} if and
only if $q\bmod4=1$; \PB{$\|h\K\T{1}$} if and only if $q\bmod8=3$; $h=\sqrt{%
\mathstrut2}$
if and only if $q\bmod24=7$ or 15; etc.)

\Y\B\4\X24:Change the \PB{\\{gen}} table to matrix format\X${}\E{}$\6
${}\{{}$\5
\1\&{register} \&{long} \|g${},\39\|h;{}$\6
\&{long} \\{a00}${},\39\\{a01},\39\\{a10},\39\\{a11}{}$;\C{ entries of $2%
\times2$ matrix }\7
\&{for} ${}(\|k\K\|q-\T{1};{}$ ${}\\{q\_sqrt}[\|k]<\T{0};{}$ ${}\|k\MM){}$\1\5
;\C{ find the largest quadratic residue, \PB{\|k} }\2\6
${}\|g\K\\{q\_sqrt}[\|k]{}$;\5
${}\|h\K\\{q\_sqrt}[\|q-\T{1}-\|k];{}$\6
\&{for} ${}(\|k\K\|p;{}$ ${}\|k\G\T{0};{}$ ${}\|k\MM){}$\5
${}\{{}$\1\6
${}\\{a00}\K(\\{gen}[\|k].\\{a0}+\|g*\\{gen}[\|k].\\{a1}+\|h*\\{gen}[\|k].%
\\{a3})\MOD\|q;{}$\6
\&{if} ${}(\\{a00}<\T{0}){}$\1\5
${}\\{a00}\MRL{+{\K}}\|q;{}$\2\6
${}\\{a11}\K(\\{gen}[\|k].\\{a0}-\|g*\\{gen}[\|k].\\{a1}-\|h*\\{gen}[\|k].%
\\{a3})\MOD\|q;{}$\6
\&{if} ${}(\\{a11}<\T{0}){}$\1\5
${}\\{a11}\MRL{+{\K}}\|q;{}$\2\6
${}\\{a01}\K(\\{gen}[\|k].\\{a2}+\|g*\\{gen}[\|k].\\{a3}-\|h*\\{gen}[\|k].%
\\{a1})\MOD\|q;{}$\6
\&{if} ${}(\\{a01}<\T{0}){}$\1\5
${}\\{a01}\MRL{+{\K}}\|q;{}$\2\6
${}\\{a10}\K({-}\\{gen}[\|k].\\{a2}+\|g*\\{gen}[\|k].\\{a3}-\|h*\\{gen}[\|k].%
\\{a1})\MOD\|q;{}$\6
\&{if} ${}(\\{a10}<\T{0}){}$\1\5
${}\\{a10}\MRL{+{\K}}\|q;{}$\2\6
${}\\{gen}[\|k].\\{a0}\K\\{a00}{}$;\5
${}\\{gen}[\|k].\\{a1}\K\\{a01}{}$;\5
${}\\{gen}[\|k].\\{a2}\K\\{a10}{}$;\5
${}\\{gen}[\|k].\\{a3}\K\\{a11};{}$\6
\4${}\}{}$\2\6
\4${}\}{}$\2\par
\U21.\fi

\M{25}When \PB{$\|p\K\T{2}$}, the following three appropriate generating
matrices
have been found by Patrick~Chiu:
$$\left(\matrix{1&0\cr 0&-1\cr}\right)\,,\qquad
\left(\matrix{2+s&t\cr t&2-s\cr}\right)\,,\qquad\hbox{and}\qquad
\left(\matrix{2-s&-t\cr-t&2+s\cr}\right)\,,$$
where $s^2\=-2$ and $t^2\=-26$ (mod~$q$). The determinants of
these matrices are respectively $-1$, $32$, and~$32$; the product of
the second and third matrices is 32 times the identity matrix. Notice that when
2 is a quadratic residue (this happens when $q=8k+1$), the determinants
are all quadratic residues, so we get a graph of type~3.
When 2 is a quadratic nonresidue (which happens when $q=8k+3$),
the determinants are all nonresidues, so we get a graph of type~4.

\Y\B\4\X25:Fill the \PB{\\{gen}} table with special generators\X${}\E{}$\6
${}\{{}$\5
\1\&{long} \|s${}\K\\{q\_sqrt}[\|q-\T{2}],\39\|t\K(\\{q\_sqrt}[\T{13}\MOD\|q]*%
\|s)\MOD\|q;{}$\7
${}\\{gen}[\T{0}].\\{a0}\K\T{1}{}$;\5
${}\\{gen}[\T{0}].\\{a1}\K\\{gen}[\T{0}].\\{a2}\K\T{0}{}$;\5
${}\\{gen}[\T{0}].\\{a3}\K\|q-\T{1}{}$;\5
${}\\{gen}[\T{0}].\\{bar}\K\T{0};{}$\6
${}\\{gen}[\T{1}].\\{a0}\K\\{gen}[\T{2}].\\{a3}\K(\T{2}+\|s)\MOD\|q;{}$\6
${}\\{gen}[\T{1}].\\{a1}\K\\{gen}[\T{1}].\\{a2}\K\|t;{}$\6
${}\\{gen}[\T{2}].\\{a1}\K\\{gen}[\T{2}].\\{a2}\K\|q-\|t;{}$\6
${}\\{gen}[\T{1}].\\{a3}\K\\{gen}[\T{2}].\\{a0}\K(\|q+\T{2}-\|s)\MOD\|q;{}$\6
${}\\{gen}[\T{1}].\\{bar}\K\T{2}{}$;\5
${}\\{gen}[\T{2}].\\{bar}\K\T{1};{}$\6
${}\\{gen\_count}\K\T{3};{}$\6
\4${}\}{}$\2\par
\U19.\fi

\N{1}{26}Constructing the edges. The remaining task is to use the permutations
defined by the \PB{\\{gen}} table to create the arcs of the graph and
their inverses.

The \PB{\\{ref}} fields in each arc will refer to the permutation leading to
the
arc. In most cases each vertex \PB{\|v} will have degree exactly \PB{$\|p+%
\T{1}$}, and the
edges emanating from it will appear in a linked list having
the respective \PB{\\{ref}} fields 0,~1, \dots,~\PB{\|p} in order. However,
if \PB{\\{reduce}} is nonzero, self-loops and multiple edges will be
eliminated,
so the degree might be less than \PB{$\|p+\T{1}$}; in this case the \PB{%
\\{ref}} fields
will still be in ascending order, but some generators won't be referenced.

There is a subtle case where \PB{$\\{reduce}\K\T{0}$} but the degree of a
vertex might
actually be greater than \PB{$\|p+\T{1}$}.
We want the graph \PB{\|g} generated by \PB{\\{raman}} to satisfy the
conventions for undirected graphs stated in {\sc GB\_\,GRAPH}; therefore,
if any of the generating permutations has a fixed point, we will create
two arcs for that fixed point, and the corresponding vertex \PB{\|v} will
have an edge running to itself. Since each edge consists of two arcs, such
an edge will produce two consecutive entries in the list \PB{$\|v\MG\\{arcs}$}.
If the generating permutation happens to be its own inverse,
there will be two consecutive entries with the same \PB{\\{ref}} field;
this means there will be more than \PB{$\|p+\T{1}$} entries in \PB{$\|v\MG%
\\{arcs}$},
and the total number of arcs \PB{$\|g\MG\|m$} will exceed \PB{$(\|p+\T{1})%
\|n$}.
Self-inverse generating permutations arise only when \PB{$\|p\K\T{2}$} or
when $p$ is expressible as a sum of three odd squares (hence
$p\bmod8=3$); and such permutations will have fixed points only when
\PB{$\\{type}<\T{3}$}. Therefore this anomaly does not arise often. But it does
occur, for example, in the smallest graph generated by \PB{\\{raman}}, namely
when \PB{$\|p\K\T{2}$}, \PB{$\|q\K\T{3}$}, and \PB{$\\{type}\K\T{1}$}, when
there are 4~vertices and 14 (not~12)
arcs.

\Y\B\4\D$\\{ref}$ \5
$\|a.{}$\|I\C{ the \PB{\\{ref}} field of an arc refers to its permutation
number }\par
\Y\B\4\X26:Append the edges\X${}\E{}$\6
\&{for} ${}(\|k\K\|p;{}$ ${}\|k\G\T{0};{}$ ${}\|k\MM){}$\5
${}\{{}$\5
\1\&{long} \\{kk};\7
\&{if} ${}((\\{kk}\K\\{gen}[\|k].\\{bar})\Z\|k{}$)\C{ we assume that \PB{$%
\\{kk}\K\|k$} or \PB{$\\{kk}\K\|k-\T{1}$} }\1\6
\&{for} ${}(\|v\K\\{new\_graph}\MG\\{vertices};{}$ ${}\|v<\\{new\_graph}\MG%
\\{vertices}+\|n;{}$ ${}\|v\PP){}$\5
${}\{{}$\1\6
\&{register} \&{Vertex} ${}{*}\|u;{}$\7
\X27:Compute the image, \PB{\|u}, of \PB{\|v} under the permutation defined by %
\PB{\\{gen}[\|k]}\X;\6
\&{if} ${}(\|u\E\|v){}$\5
${}\{{}$\1\6
\&{if} ${}(\R\\{reduce}){}$\5
${}\{{}$\1\6
${}\\{gb\_new\_edge}(\|v,\39\|v,\39\T{1\$L});{}$\6
${}\|v\MG\\{arcs}\MG\\{ref}\K\\{kk}{}$;\5
${}(\|v\MG\\{arcs}+\T{1})\MG\\{ref}\K\|k{}$;\C{ see the remarks above regarding
the case \PB{$\\{kk}\K\|k$} }\6
\4${}\}{}$\2\6
\4${}\}{}$\5
\2\&{else}\5
${}\{{}$\5
\1\&{register} \&{Arc} ${}{*}\\{ap};{}$\7
\&{if} ${}(\|u\MG\\{arcs}\W\|u\MG\\{arcs}\MG\\{ref}\E\\{kk}){}$\1\5
\&{continue};\C{ \PB{$\\{kk}\K\|k$} and we've already done this two-cycle }\2\6
\&{else} \&{if} (\\{reduce})\1\6
\&{for} ${}(\\{ap}\K\|v\MG\\{arcs};{}$ \\{ap}; ${}\\{ap}\K\\{ap}\MG\\{next}){}$%
\1\6
\&{if} ${}(\\{ap}\MG\\{tip}\E\|u){}$\1\5
\&{goto} \\{done};\C{ there's already an edge between \PB{\|u} and \PB{\|v} }\2%
\2\2\6
${}\\{gb\_new\_edge}(\|v,\39\|u,\39\T{1\$L});{}$\6
${}\|v\MG\\{arcs}\MG\\{ref}\K\|k{}$;\5
${}\|u\MG\\{arcs}\MG\\{ref}\K\\{kk};{}$\6
\&{if} ${}((\\{ap}\K\|v\MG\\{arcs}\MG\\{next})\I\NULL\W\\{ap}\MG\\{ref}\E%
\\{kk}){}$\5
${}\{{}$\1\6
${}\|v\MG\\{arcs}\MG\\{next}\K\\{ap}\MG\\{next}{}$;\5
${}\\{ap}\MG\\{next}\K\|v\MG\\{arcs}{}$;\5
${}\|v\MG\\{arcs}\K\\{ap};{}$\6
\4${}\}{}$\C{ now the \PB{$\|v\MG\\{arcs}$} list has \PB{\\{ref}} fields in
order again }\2\6
\4\\{done}:\5
;\6
\4${}\}{}$\2\6
\4${}\}{}$\2\2\6
\4${}\}{}$\2\par
\U4.\fi

\M{27}For graphs of types 3 and 4, our job is to compute a $2\times2$ matrix
product, reduce it modulo~\PB{\|q}, and find the appropriate
equivalence class~\PB{\|u}.

\Y\B\4\X27:Compute the image, \PB{\|u}, of \PB{\|v} under the permutation
defined by \PB{\\{gen}[\|k]}\X${}\E{}$\6
\&{if} ${}(\\{type}<\T{3}){}$\1\5
\X31:Compute the image, \PB{\|u}, of \PB{\|v} under the linear fractional
transformation defined by \PB{\\{gen}[\|k]}\X\2\6
\&{else}\5
${}\{{}$\5
\1\&{long} \\{a00}${}\K\\{gen}[\|k].\\{a0},\39\\{a01}\K\\{gen}[\|k].\\{a1},\39%
\\{a10}\K\\{gen}[\|k].\\{a2},\39\\{a11}\K\\{gen}[\|k].\\{a3};{}$\7
${}\|a\K\|v\MG\|x.\|I{}$;\5
${}\|b\K\|v\MG\|y.\|I;{}$\6
\&{if} ${}(\|v\MG\|z.\|I\E\|q){}$\1\5
${}\|c\K\T{0},\39\|d\K\T{1};{}$\2\6
\&{else}\1\5
${}\|c\K\T{1},\39\|d\K\|v\MG\|z.\|I;{}$\2\6
\X28:Compute the matrix product \PB{$(\\{aa},\\{bb};{}$ \\{cc}${},\\{dd})\K(%
\|a,\|b;{}$ \|c${},\|d)*(\\{a00},\\{a01};{}$ \\{a10}${},\\{a11})$}\X;\6
${}\|a\K(\\{cc}\?\\{q\_inv}[\\{cc}]:\\{q\_inv}[\\{dd}]){}$;\C{ now \PB{\|a} is
a normalization factor }\6
${}\|d\K(\|a*\\{dd})\MOD\|q{}$;\5
${}\|c\K(\|a*\\{cc})\MOD\|q{}$;\5
${}\|b\K(\|a*\\{bb})\MOD\|q{}$;\5
${}\|a\K(\|a*\\{aa})\MOD\|q;{}$\6
\X29:Set \PB{\|u} to the vertex whose label is \PB{$(\|a,\|b;{}$ \|c${},\|d)$}%
\X;\6
\4${}\}{}$\2\par
\U26.\fi

\M{28}\B\X28:Compute the matrix product \PB{$(\\{aa},\\{bb};{}$ \\{cc}${},%
\\{dd})\K(\|a,\|b;{}$ \|c${},\|d)*(\\{a00},\\{a01};{}$ \\{a10}${},\\{a11})$}%
\X${}\E{}$\6
$\\{aa}\K(\|a*\\{a00}+\|b*\\{a10})\MOD\|q;{}$\6
${}\\{bb}\K(\|a*\\{a01}+\|b*\\{a11})\MOD\|q;{}$\6
${}\\{cc}\K(\|c*\\{a00}+\|d*\\{a10})\MOD\|q;{}$\6
${}\\{dd}\K(\|c*\\{a01}+\|d*\\{a11})\MOD\|q{}$;\par
\U27.\fi

\M{29}\B\X29:Set \PB{\|u} to the vertex whose label is \PB{$(\|a,\|b;{}$ %
\|c${},\|d)$}\X${}\E{}$\6
\&{if} ${}(\|c\E\T{0}){}$\1\5
${}\|d\K\|q,\39\\{aa}\K\|a;{}$\2\6
\&{else}\5
${}\{{}$\1\6
${}\\{aa}\K(\|a*\|d-\|b)\MOD\|q;{}$\6
\&{if} ${}(\\{aa}<\T{0}){}$\1\5
${}\\{aa}\MRL{+{\K}}\|q;{}$\2\6
${}\|b\K\|a;{}$\6
\4${}\}{}$\C{ now \PB{\\{aa}} is the determinant of the matrix }\2\6
${}\|u\K\\{new\_graph}\MG\\{vertices}+((\|d*\|q+\|b)*\\{n\_factor}+(\\{type}\E%
\T{3}\?\\{q\_sqrt}[\\{aa}]:\\{aa})-\T{1}){}$;\par
\U27.\fi

\N{1}{30}Linear fractional transformations. Given a nonsingular $2\times2$
matrix
$\bigl({a\,b\atop c\,d}\bigr)$, the linear fractional transformation
$z\mapsto(az+b)/(cz+d)$ is defined modulo~$q$ by the
following subroutine. We assume that the matrix $\bigl({a\,b\atop c\,d}\bigr)$
appears in row \PB{\|k} of the \PB{\\{gen}} table.

\Y\B\4\X6:Private variables and routines\X${}\mathrel+\E{}$\6
\1\1\&{static} \&{long} ${}\\{lin\_frac}(\|a,\39\|k){}$\6
\&{long} \|a;\C{ the number being transformed; $q$ represents $\infty$ }\6
\&{long} \|k;\C{ index into \PB{\\{gen}} table }\2\2\6
${}\{{}$\5
\1\&{register} \&{long} \|q${}\K\\{q\_inv}[\T{0}]{}$;\C{ the modulus }\6
\&{long} \\{a00}${}\K\\{gen}[\|k].\\{a0},\39\\{a01}\K\\{gen}[\|k].\\{a1},\39%
\\{a10}\K\\{gen}[\|k].\\{a2},\39\\{a11}\K\\{gen}[\|k].\\{a3}{}$;\C{ the
coefficients }\6
\&{register} \&{long} \\{num}${},\39\\{den}{}$;\C{ numerator and denominator }\7
\&{if} ${}(\|a\E\|q){}$\1\5
${}\\{num}\K\\{a00},\39\\{den}\K\\{a10};{}$\2\6
\&{else}\1\5
${}\\{num}\K(\\{a00}*\|a+\\{a01})\MOD\|q,\39\\{den}\K(\\{a10}*\|a+\\{a11})\MOD%
\|q;{}$\2\6
\&{if} ${}(\\{den}\E\T{0}){}$\1\5
\&{return} \|q;\2\6
\&{else}\1\5
\&{return} ${}(\\{num}*\\{q\_inv}[\\{den}])\MOD\|q;{}$\2\6
\4${}\}{}$\2\par
\fi

\M{31}We are computing the same values of \PB{\\{lin\_frac}} over and over
again in type~2
graphs, but the author was too lazy to optimize this.

\Y\B\4\X31:Compute the image, \PB{\|u}, of \PB{\|v} under the linear fractional
transformation defined by \PB{\\{gen}[\|k]}\X${}\E{}$\6
\&{if} ${}(\\{type}\E\T{1}){}$\1\5
${}\|u\K\\{new\_graph}\MG\\{vertices}+\\{lin\_frac}(\|v\MG\|x.\|I,\39\|k);{}$\2%
\6
\&{else}\5
${}\{{}$\1\6
${}\|a\K\\{lin\_frac}(\|v\MG\|x.\|I,\39\|k){}$;\5
${}\\{aa}\K\\{lin\_frac}(\|v\MG\|y.\|I,\39\|k);{}$\6
${}\|u\K\\{new\_graph}\MG\\{vertices}+(\|a<\\{aa}\?(\|a*(\T{2}*\|q-\T{1}-\|a))/%
\T{2}+\\{aa}-\T{1}:(\\{aa}*(\T{2}*\|q-\T{1}-\\{aa}))/\T{2}+\|a-\T{1});{}$\6
\4${}\}{}$\2\par
\U27.\fi

\N{1}{32}Index. Here is a list that shows where the identifiers of this program
are
defined and used.
\fi

\inx
\fin
\con
