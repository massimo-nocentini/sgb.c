\input cwebmac
% This file is part of the Stanford GraphBase (c) Stanford University 1993
% This material goes at the beginning of all Stanford GraphBase CWEB files

\def\topofcontents{
  \leftline{\sc\today\ at \hours}\bigskip\bigskip
  \centerline{\titlefont\title}}

\font\ninett=cmtt9
\def\botofcontents{\vskip 0pt plus 1filll
    \ninerm\baselineskip10pt
    \noindent\copyright\ 1993 Stanford University
    \bigskip\noindent
    This file may be freely copied and distributed, provided that
    no changes whatsoever are made. All users are asked to help keep
    the Stanford GraphBase files consistent and ``uncorrupted,''
    identical everywhere in the world. Changes are permissible only
    if the modified file is given a new name, different from the names of
    existing files in the Stanford GraphBase, and only if the modified file is
    clearly identified as not being part of that GraphBase.
    (The {\ninett CWEB} system has a ``change file'' facility by
    which users can easily make minor alterations without modifying
    the master source files in any way. Everybody is supposed to use
    change files instead of changing the files.)
    The author has tried his best to produce correct and useful programs,
    in order to help promote computer science research,
    but no warranty of any kind should be assumed.
    \smallskip\noindent
    Preliminary work on the Stanford GraphBase project
    was supported in part by National Science
    Foundation grant CCR-86-10181.}

\def\prerequisite#1{\def\startsection{\noindent
    Important: Before reading {\sc\title},
    please read or at least skim the program for {\sc#1}.\bigskip
    \let\startsection=\stsec\stsec}}
\def\prerequisites#1#2{\def\startsection{\noindent
    Important: Before reading {\sc\title}, please read
    or at least skim the programs for {\sc#1} and {\sc#2}.\bigskip
    \let\startsection=\stsec\stsec}}



\def\title{GIRTH}
\let\==\equiv % congruence sign

\prerequisite{GB\_\,RAMAN}

\N{1}{1}Introduction. This demonstration program uses graphs
constructed by the \PB{\\{raman}} procedure in the {\sc GB\_\,RAMAN} module to
produce
an interactive program called \.{girth}, which computes the girth and
diameter of a class of Ramanujan graphs.

The girth of a graph is the length of its shortest cycle; the diameter
is the maximum length of a shortest path between two vertices.
A Ramanujan graph is a connected, undirected graph in which every vertex
has degree~\PB{$\|p+\T{1}$}, with the property that every eigenvalue of its
adjacency
matrix is either $\pm(p+1)$ or has absolute value $\le2\sqrt{\mathstrut p}$.

Exact values for the girth are of interest because the bipartite graphs
produced by \PB{\\{raman}} apparently have larger girth than any other known
family of regular graphs, even if we consider graphs whose existence
is known only by nonconstructive methods, except for the cubic ``sextet''
graphs of Biggs, Hoare, and Weiss [{\sl Combinatorica\/ \bf3} (1983),
153--165; {\bf4} (1984), 241--245].

Exact values for the diameter are of interest because the diameter of
any Ramanujan graph is at most twice the minimum possible diameter
of any regular graph.

The program will prompt you for two numbers, \PB{\|p} and \PB{\|q}. These
should
be distinct prime numbers, not too large, with \PB{$\|q>\T{2}$}.  A graph is
constructed in which each vertex has degree~\PB{$\|p+\T{1}$}. The number of
vertices is $(q^3-q)/2$ if \PB{\|p} is a quadratic residue modulo~\PB{\|q}, or
$q^3-q$ if \PB{\|p} is not a quadratic residue. In the latter case, the
graph is bipartite and it is known to have rather large girth.

If \PB{$\|p\K\T{2}$}, the value of \PB{\|q} is further restricted to be of the
form
$104k+(1,3,9,17,25,27,35,43,49,\allowbreak51,75,81)$. This means that the only
feasible values of \PB{\|q} to go with \PB{$\|p\K\T{2}$} are probably 3, 17,
and 43;
the next case, \PB{$\|q\K\T{107}$}, would generate a bipartite graph with
1,224,936 vertices and 3,674,808 arcs, thus requiring approximately
113 megabytes of memory (not to mention a nontrivial amount of
computer time). If you want to compute the girth and diameter
of Ramanujan graphs for large \PB{\|p} and/or~\PB{\|q}, much better methods are
available based on number theory; the present program is merely a
demonstration of how to interface with the output of \PB{\\{raman}}.
Incidentally, the graph for \PB{$\|p\K\T{2}$} and \PB{$\|q\K\T{43}$} turns
out to have 79464 vertices, girth 20, and diameter~22.

The program will examine the graph and compute its girth and its diameter,
then will prompt you for another choice of \PB{\|p} and \PB{\|q}.

\fi

\M{2}Here is the general layout of this program, as seen by the \CEE/ compiler:

\Y\B\8\#\&{include} \.{"gb\_graph.h"}\C{ the standard GraphBase data structures
}\6
\8\#\&{include} \.{"gb\_raman.h"}\C{ Ramanujan graph generator }\6
\ATH\7
\X3:Global variables\X\7
\1\1\\{main}(\,)\2\2\6
${}\{{}$\1\6
\\{printf}(\.{"This\ program\ explor}\)\.{es\ the\ girth\ and\ dia}\)\.{meter\
of\ Ramanujan\ g}\)\.{raphs.\\n"});\6
\\{printf}(\.{"The\ bipartite\ graph}\)\.{s\ have\ q\^3-q\ vertice}\)\.{s,\ and%
\ the\ non-bipar}\)\.{tite\\n"});\6
\\{printf}(\.{"graphs\ have\ half\ th}\)\.{at\ number.\ Each\ vert}\)\.{ex\ has%
\ degree\ p+1.\\n}\)\.{"});\6
\\{printf}(\.{"Both\ p\ and\ q\ should}\)\.{\ be\ odd\ prime\ number}\)\.{s;%
\\n"});\6
\\{printf}(\.{"\ \ or\ you\ can\ try\ p\ }\)\.{=\ 2\ with\ q\ =\ 17\ or\ 4}\)%
\.{3.\\n"});\6
\&{while} (\T{1})\5
${}\{{}$\1\6
\X4:Prompt the user for \PB{\|p} and \PB{\|q}; \PB{\&{break}} if unsuccessful%
\X;\6
${}\|g\K\\{raman}(\|p,\39\|q,\39\T{0\$L},\39\T{0\$L});{}$\6
\&{if} ${}(\|g\E\NULL){}$\1\5
\X5:Explain that the graph could not be constructed\X\2\6
\&{else}\5
${}\{{}$\1\6
\X10:Print the theoretical bounds on girth and diameter of \PB{\|g}\X;\6
\X12:Compute and print the true girth and diameter of \PB{\|g}\X;\6
\\{gb\_recycle}(\|g);\6
\4${}\}{}$\2\6
\4${}\}{}$\2\6
\&{return} \T{0};\C{ normal exit }\6
\4${}\}{}$\2\par
\fi

\M{3}\B\X3:Global variables\X${}\E{}$\6
\&{Graph} ${}{*}\|g{}$;\C{ the current Ramanujan graph }\6
\&{long} \|p;\C{ the branching factor (degree minus one) }\6
\&{long} \|q;\C{ cube root of the graph size }\6
\&{char} \\{buffer}[\T{16}];\C{ place to collect what the user types }\par
\A11.
\U2.\fi

\M{4}\B\D$\\{prompt}(\|s)$ \6
${}\{{}$\5
\1\\{printf}(\|s);\5
\\{fflush}(\\{stdout});\C{ make sure the user sees the prompt }\6
\&{if} ${}(\\{fgets}(\\{buffer},\39\T{15},\39\\{stdin})\E\NULL){}$\1\5
\&{break};\5
\2${}\}{}$\2\par
\Y\B\4\X4:Prompt the user for \PB{\|p} and \PB{\|q}; \PB{\&{break}} if
unsuccessful\X${}\E{}$\6
\\{prompt}(\.{"\\nChoose\ a\ branchin}\)\.{g\ factor,\ p:\ "});\6
\&{if} ${}(\\{sscanf}(\\{buffer},\39\.{"\%ld"},\39{\AND}\|p)\I\T{1}){}$\1\5
\&{break};\2\6
\\{prompt}(\.{"OK,\ now\ choose\ the\ }\)\.{cube\ root\ of\ graph\ s}\)\.{ize,\
q:\ "});\6
\&{if} ${}(\\{sscanf}(\\{buffer},\39\.{"\%ld"},\39{\AND}\|q)\I\T{1}){}$\1\5
\&{break};\2\par
\U2.\fi

\M{5}\B\X5:Explain that the graph could not be constructed\X${}\E{}$\6
$\\{printf}(\.{"\ Sorry,\ I\ couldn't\ }\)\.{make\ that\ graph\ (\%s)}\)\.{.%
\\n"},\3{-1}\39\\{panic\_code}\E\\{very\_bad\_specs}\?\.{"q\ is\ out\ of\
range"}:\3{-1}\\{panic\_code}\E\\{very\_bad\_specs}+\T{1}\?\.{"p\ is\ out\ of\
range"}:\3{-1}\\{panic\_code}\E\\{bad\_specs}+\T{5}\?\.{"q\ is\ too\ big"}:%
\3{-1}\\{panic\_code}\E\\{bad\_specs}+\T{6}\?\.{"p\ is\ too\ big"}:\3{-1}%
\\{panic\_code}\E\\{bad\_specs}+\T{1}\?\.{"q\ isn't\ prime"}:\3{-1}\\{panic%
\_code}\E\\{bad\_specs}+\T{7}\?\.{"p\ isn't\ prime"}:\3{-1}\\{panic\_code}\E%
\\{bad\_specs}+\T{3}\?\.{"p\ is\ a\ multiple\ of\ }\)\.{q"}:\3{-1}\\{panic%
\_code}\E\\{bad\_specs}+\T{2}\?\.{"q\ isn't\ compatible\ }\)\.{with\ p=2"}:%
\3{-1}\.{"not\ enough\ memory"}){}$;\par
\U2.\fi

\N{1}{6}Bounds. The theory of Ramanujan graphs allows us to predict the
girth and diameter to within a factor of 2~or~so.

In the first place, we can easily derive an upper bound on the girth
and a lower bound on the diameter, valid for any $n$-vertex regular graph
of degree~$p+1$. Such a graph has at most $(p+1)p^{k-1}$ points at
distance~$k$ from any given vertex; this implies a lower bound
on the diameter~$d$:
$$1+(p+1)+(p+1)p+(p+1)p^2+\cdots+(p+1)p^{d-1}\;\ge\;n.$$
Similarly, if the girth $g$ is odd, say $g=2k+1$, the points at
distance~$\le k$ from any vertex must be distinct, so we have
$$1+(p+1)+(p+1)p+(p+1)p^2+\cdots+(p+1)p^{k-1}\;\le\;n;$$
and if $g=2k+2$, at least $p^k$ further points must exist at distance
$k+1$, because the $(p+1)p^k$ paths of length $k+1$ can end at
a particular vertex at most $p+1$ times. Thus
$$1+(p+1)+(p+1)p+(p+1)p^2+\cdots+(p+1)p^{k-1}+p^k\;\le\;n$$
when the girth is even.

In the following code we let $\PB{\\{pp}}=p^{dl}$ and
$s=1+(p+1)+\cdots+(p+1)p^{dl}$.

\Y\B\4\X6:Compute the ``trivial'' bounds \PB{\\{gu}} and \PB{\\{dl}} on girth
and diameter\X${}\E{}$\6
$\|s\K\|p+\T{2}{}$;\5
${}\\{dl}\K\T{1}{}$;\5
${}\\{pp}\K\|p{}$;\5
${}\\{gu}\K\T{3};{}$\6
\&{while} ${}(\|s<\|n){}$\5
${}\{{}$\1\6
${}\|s\MRL{+{\K}}\\{pp};{}$\6
\&{if} ${}(\|s\Z\|n){}$\1\5
${}\\{gu}\PP;{}$\2\6
${}\\{dl}\PP;{}$\6
${}\\{pp}\MRL{*{\K}}\|p;{}$\6
${}\|s\MRL{+{\K}}\\{pp};{}$\6
\&{if} ${}(\|s\Z\|n){}$\1\5
${}\\{gu}\PP;{}$\2\6
\4${}\}{}$\2\par
\U10.\fi

\M{7}When \PB{$\|p>\T{2}$}, we can use the theory of integral quaternions to
derive a lower
bound on the girth of the graphs produced by \PB{\\{raman}}. A path of
length~$g$
from a vertex to itself exists if and only if there is an integral
quaternion $\alpha=a_0+a_1i+a_2j+a_3k$ of norm $p^g$ such that
the $a$'s are not all multiples of~$p$, while
$a_1$, $a_2$, and $a_3$ are multiples of~$q$ and $a_0\not\=a_1\=a_2\=a_3$
(mod~2). This means we have integers $(a_0,a_1,a_2,a_3)$ with
$$a_0^2+a_1^2+a_2^2+a_3^2=p^{\,g},$$ satisfying the stated properties
mod~$q$ and mod~2.
If $a_1$, $a_2$, and $a_3$ are even, they cannot all be zero so
we must have $p^{\,g}\ge1+4q^2$; if they are odd, we must have
$p^{\,g}\ge4+3q^2$. (The latter is possible only when $g$ is odd and
$p\bmod4=3$.) Since $n$ is roughly proportional to~$q^3$, this means
$g$ must be at least about ${2\over3}\log_p n$. Thus $g$~isn't
too much less than the maximum girth possible in any regular graph,
which we have shown is at most about $2\log_p n$.

When the graph is bipartite we can, in fact, prove that $g$ is
approximately ${4\over3}\log_p n$. The bipartite case occurs if and
only if $p$ is not a quadratic residue modulo~\PB{\|q}; hence the
number~$g$ in the previous paragraph must be even, say $g=2r$. Then
$p^{\,g}\bmod4=1$, and $a_0$ must be odd.  The congruence $a_0^2\=p^{2r}$
(mod~$q^2$) implies that $a_0\=\pm p^r$, because all numbers
relatively prime to $q^2$ are powers of a primitive root. We can
assume without loss of generality that $a_0=p^r-2mq^2$, where
$0<m<p^r/q^2$; it follows in particular that $p^r>q^2$.  Conversely,
if $p^r-q^2$ can be written as a sum of three squares
$b_1^2+b_2^2+b_3^2$, then
$p^{2r}=(p^r-2q^2)^2+(2b_1q)^2+(2b_2q)^2+(2b_3q)^2$ is a
representation of the required type. If $p^r-q^2$ is a positive
integer that cannot be represented as a sum of three squares, a
well-known theorem of Legendre tells us that $p^r-q^2=4^ts$, where
$s\=7$ (mod~8).  Since $p$ and $q$ are odd, we have $t\ge1$; hence
$p^r-2q^2$ is odd. If $p^r-2q^2$ is a positive odd integer, Legendre's
theorem tells us that we can write $2p^r-4q^2=b_1^2+b_2^2+b_3^2$;
hence $p^{2r}=(p^r-4q^2)^2+ (2b_1q)^2+(2b_2q)^2+(2b_3q)^2$. We
conclude that the girth is either $2\lceil\log_pq^2\rceil$ or
$2\lceil\log_p2q^2\rceil$. (This explicit calculation, which makes our
program for calculating the girth unnecessary or at best redundant in
the bipartite case, is due to G. A. Margulis and, independently, to
Biggs and Boshier [{\sl Journal of Combinatorial Theory\/ \bf B49}
(1990), 190--194].)

A girth of 1 or 2 can occur, since these graphs might have self-loops
or multiple edges if \PB{\|p} is sufficiently large.

\Y\B\4\X7:Compute a lower bound \PB{\\{gl}} on the girth\X${}\E{}$\6
\&{if} (\\{bipartite})\5
${}\{{}$\5
\1\&{long} \|b${}\K\|q*\|q;{}$\7
\&{for} ${}(\\{gl}\K\T{1},\39\\{pp}\K\|p;{}$ ${}\\{pp}\Z\|b;{}$ ${}\\{gl}\PP,%
\39\\{pp}\MRL{*{\K}}\|p){}$\1\5
;\C{ iterate until $p^{\,g}>q^2$ }\2\6
${}\\{gl}\MRL{+{\K}}\\{gl};{}$\6
\4${}\}{}$\5
\2\&{else}\5
${}\{{}$\5
\1\&{long} \\{b1}${}\K\T{1}+\T{4}*\|q*\|q,\39\\{b2}\K\T{4}+\T{3}*\|q*\|q{}$;\C{
bounds on $p^{\,g}$ }\7
\&{for} ${}(\\{gl}\K\T{1},\39\\{pp}\K\|p;{}$ ${}\\{pp}<\\{b1};{}$ ${}\\{gl}\PP,%
\39\\{pp}\MRL{*{\K}}\|p){}$\5
${}\{{}$\1\6
\&{if} ${}(\\{pp}\G\\{b2}\W(\\{gl}\AND\T{1})\W(\|p\AND\T{2})){}$\1\5
\&{break};\2\6
\4${}\}{}$\2\6
\4${}\}{}$\2\par
\U10.\fi

\M{8}Upper bounds on the diameter of any Ramanujan graph can be derived
as shown in the paper by Lubotzky, Phillips, and Sarnak in
{\sl Combinatorica \bf8} (1988), page~275. (However, a slight correction
to their proof is necessary---their parameter~$l$ should be~odd
when $x$ and~$y$ lie in different parts of a bipartite graph.)
Their argument demonstrates that $p^{(d-1)/2}<2n$ in the
nonbipartite case and $p^{(d-2)/2}<n$ in the bipartite case; therefore
we obtain the upper bound $d\le 2\log_p n+O(1)$, which is about twice the lower
bound that holds in an arbitrary regular graph.

\Y\B\4\X8:Compute an upper bound \PB{\\{du}} on the diameter\X${}\E{}$\6
${}\{{}$\5
\1\&{long} \\{nn}${}\K(\\{bipartite}\?\|n:\T{2}*\|n);{}$\7
\&{for} ${}(\\{du}\K\T{0},\39\\{pp}\K\T{1};{}$ ${}\\{pp}<\\{nn};{}$ ${}\\{du}%
\MRL{+{\K}}\T{2},\39\\{pp}\MRL{*{\K}}\|p){}$\1\5
;\2\6
\X9:Decrease \PB{\\{du}} by 1, if $\PB{\\{pp}}/\PB{\\{nn}}\ge\sqrt p$\X;\6
\&{if} (\\{bipartite})\1\5
${}\\{du}\PP;{}$\2\6
\4${}\}{}$\2\par
\U10.\fi

\M{9}Floating point arithmetic might not be accurate enough for the test
required in this section. We avoid it by using an all-integer method
analogous to Euclid's algorithm, based on the continued fraction for
$\sqrt p$ [{\sl Seminumerical Algorithms}, exercise 4.5.3--12]. In the
loop here we want to compare \PB{$\\{nn}/\\{pp}$} to $(\sqrt p+a)/b$, where $%
\sqrt p+a
>b>0$ and $p-a^2$ is a multiple of~$b$.

\Y\B\4\X9:Decrease \PB{\\{du}} by 1, if $\PB{\\{pp}}/\PB{\\{nn}}\ge\sqrt p$%
\X${}\E{}$\6
${}\{{}$\5
\1\&{long} \\{qq}${}\K\\{pp}/\\{nn};{}$\7
\&{if} ${}(\\{qq}*\\{qq}>\|p){}$\1\5
${}\\{du}\MM;{}$\2\6
\&{else} \&{if} ${}((\\{qq}+\T{1})*(\\{qq}+\T{1})>\|p){}$\5
${}\{{}$\C{ $\PB{\\{qq}}=\lfloor\sqrt p\,\rfloor$ }\1\6
\&{long} \\{aa}${}\K\\{qq},\39\\{bb}\K\|p-\\{aa}*\\{aa},\39\\{parity}\K%
\T{0};{}$\7
${}\\{pp}\MRL{-{\K}}\\{qq}*\\{nn};{}$\6
\&{while} (\T{1})\5
${}\{{}$\1\6
\&{long} \|x${}\K(\\{aa}+\\{qq})/\\{bb},\39\|y\K\\{nn}-\|x*\\{pp};{}$\7
\&{if} ${}(\|y\Z\T{0}){}$\1\5
\&{break};\2\6
${}\\{aa}\K\\{bb}*\|x-\\{aa}{}$;\C{ now $0<\PB{\\{aa}}<\sqrt p$ }\6
${}\\{bb}\K(\|p-\\{aa}*\\{aa})/\\{bb};{}$\6
${}\\{nn}\K\\{pp}{}$;\5
${}\\{pp}\K\|y;{}$\6
${}\\{parity}\MRL{{\XOR}{\K}}\T{1};{}$\6
\4${}\}{}$\2\6
\&{if} ${}(\R\\{parity}){}$\1\5
${}\\{du}\MM;{}$\2\6
\4${}\}{}$\2\6
\4${}\}{}$\2\par
\U8.\fi

\M{10}\B\X10:Print the theoretical bounds on girth and diameter of \PB{\|g}%
\X${}\E{}$\6
$\|n\K\|g\MG\|n;{}$\6
\&{if} ${}(\|n\E(\|q+\T{1})*\|q*(\|q-\T{1})){}$\1\5
${}\\{bipartite}\K\T{1};{}$\2\6
\&{else}\1\5
${}\\{bipartite}\K\T{0};{}$\2\6
${}\\{printf}(\.{"The\ graph\ has\ \%ld\ v}\)\.{ertices,\ each\ of\ deg}\)%
\.{ree\ \%ld,\ and\ it\ is\ \%}\)\.{sbipartite.\\n"},\39\|n,\39\|p+\T{1},\39%
\\{bipartite}\?\.{""}:\.{"not\ "});{}$\6
\X6:Compute the ``trivial'' bounds \PB{\\{gu}} and \PB{\\{dl}} on girth and
diameter\X;\6
${}\\{printf}(\.{"Any\ such\ graph\ must}\)\.{\ have\ diameter\ >=\ \%l}\)\.{d\
and\ girth\ <=\ \%ld;\\}\)\.{n"},\39\\{dl},\39\\{gu});{}$\6
\X8:Compute an upper bound \PB{\\{du}} on the diameter\X;\6
${}\\{printf}(\.{"theoretical\ conside}\)\.{rations\ tell\ us\ that}\)\.{\ this%
\ one's\ diameter}\)\.{\ is\ <=\ \%ld"},\39\\{du});{}$\6
\&{if} ${}(\|p\E\T{2}){}$\1\5
\\{printf}(\.{".\\n"});\2\6
\&{else}\5
${}\{{}$\1\6
\X7:Compute a lower bound \PB{\\{gl}} on the girth\X;\6
${}\\{printf}(\.{",\\nand\ its\ girth\ is}\)\.{\ >=\ \%ld.\\n"},\39\\{gl});{}$\6
\4${}\}{}$\2\par
\U2.\fi

\M{11}We had better declare all the variables we've been using so freely.

\Y\B\4\X3:Global variables\X${}\mathrel+\E{}$\6
\&{long} \\{gl}${},\39\\{gu},\39\\{dl},\39\\{du}{}$;\C{ theoretical bounds }\6
\&{long} \\{pp};\C{ power of $p$ }\6
\&{long} \|s;\C{ accumulated sum }\6
\&{long} \|n;\C{ number of vertices }\6
\&{char} \\{bipartite};\C{ is the graph bipartite? }\par
\fi

\N{1}{12}Breadth-first search. The graphs produced by \PB{\\{raman}} are
symmetrical, in
the sense that there is an automorphism taking any vertex into any
other. Each vertex $V$ and each edge $P$ corresponds to a $2\times2$
matrix, and the path $P_1P_2\ldots P_k$ leading from vertex~$V$ to
vertex $VP_1P_2\ldots P_k$ has the same properties as the path leading
from vertex~$U$ to vertex $UP_1P_2\ldots P_k$. Therefore we can find
the girth and the diameter by starting at any vertex $v_0$.

We compute the number of points at distance $k$ from $v_0$ for
all $k$, by explicitly forming a linked list of all such points.
Utility field \PB{\\{link}} is used for the links. The lists
terminate with a non-null \PB{\\{sentinel}} value, so that we can also
use the condition \PB{$\\{link}\E\NULL$} to tell if a vertex has been
encountered before. Another utility field, \PB{\\{dist}}, contains the
distance from the starting point, and \PB{\\{back}} points to a
vertex one step closer.

\Y\B\4\D$\\{link}$ \5
$\|w.{}$\|V\C{ the field where we store links, initially \PB{$\NULL$} }\par
\B\4\D$\\{dist}$ \5
$\|v.{}$\|I\C{ the field where we store distances, initially 0 }\par
\B\4\D$\\{back}$ \5
$\|u.{}$\|V\C{ the field where we store backpointers, initially \PB{$\NULL$} }%
\par
\Y\B\4\X12:Compute and print the true girth and diameter of \PB{\|g}\X${}\E{}$\6
\\{printf}(\.{"Starting\ at\ any\ giv}\)\.{en\ vertex,\ there\ are}\)\.{\\n"});%
\6
${}\{{}$\5
\1\&{long} \|k;\C{ current distance being generated }\6
\&{long} \|c;\C{ how many we've seen so far at this distance }\6
\&{register} \&{Vertex} ${}{*}\|v{}$;\C{ current vertex in list at distance
$k-1$ }\6
\&{register} \&{Vertex} ${}{*}\|u{}$;\C{ head of list for distance $k$ }\6
\&{Vertex} ${}{*}\\{sentinel}\K\|g\MG\\{vertices}+\|n{}$;\C{ nonzero link at
end of lists }\6
\&{long} \\{girth}${}\K\T{999}{}$;\C{ length of smallest cycle found, initially
infinite }\7
${}\|k\K\T{0};{}$\6
${}\|u\K\|g\MG\\{vertices};{}$\6
${}\|u\MG\\{link}\K\\{sentinel};{}$\6
${}\|c\K\T{1};{}$\6
\&{while} (\|c)\5
${}\{{}$\1\6
\&{for} ${}(\|v\K\|u,\39\|u\K\\{sentinel},\39\|c\K\T{0},\39\|k\PP;{}$ ${}\|v\I%
\\{sentinel};{}$ ${}\|v\K\|v\MG\\{link}){}$\1\5
\X13:Place all vertices adjacent to \PB{\|v} onto list \PB{\|u}, unless they've
been encountered before, increasing \PB{\|c} whenever the list grows\X;\2\6
${}\\{printf}(\.{"\%8ld\ vertices\ at\ di}\)\.{stance\ \%ld\%s\\n"},\39\|c,\39%
\|k,\39\|c>\T{0}\?\.{","}:\.{"."});{}$\6
\4${}\}{}$\2\6
${}\\{printf}(\.{"So\ the\ diameter\ is\ }\)\.{\%ld,\ and\ the\ girth\ i}\)\.{s%
\ \%ld.\\n"},\39\|k-\T{1},\39\\{girth});{}$\6
\4${}\}{}$\2\par
\U2.\fi

\M{13}\B\X13:Place all vertices adjacent to \PB{\|v} onto list \PB{\|u}, unless
they've been encountered before, increasing \PB{\|c} whenever the list grows%
\X${}\E{}$\6
${}\{{}$\5
\1\&{register} \&{Arc} ${}{*}\|a;{}$\7
\&{for} ${}(\|a\K\|v\MG\\{arcs};{}$ \|a; ${}\|a\K\|a\MG\\{next}){}$\5
${}\{{}$\5
\1\&{register} \&{Vertex} ${}{*}\|w{}$;\C{ vertex adjacent to \PB{\|v} }\7
${}\|w\K\|a\MG\\{tip};{}$\6
\&{if} ${}(\|w\MG\\{link}\E\NULL){}$\5
${}\{{}$\1\6
${}\|w\MG\\{link}\K\|u;{}$\6
${}\|w\MG\\{dist}\K\|k;{}$\6
${}\|w\MG\\{back}\K\|v;{}$\6
${}\|u\K\|w;{}$\6
${}\|c\PP;{}$\6
\4${}\}{}$\5
\2\&{else} \&{if} ${}(\|w\MG\\{dist}+\|k<\\{girth}\W\|w\I\|v\MG\\{back}){}$\1\5
${}\\{girth}\K\|w\MG\\{dist}+\|k;{}$\2\6
\4${}\}{}$\2\6
\4${}\}{}$\2\par
\U12.\fi

\N{1}{14}Index. Finally, here's a list that shows where the identifiers of this
program are defined and used.

\fi


\inx
\fin
\con
