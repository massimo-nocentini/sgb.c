\input cwebmac
% This file is part of the Stanford GraphBase (c) Stanford University 1993
% This material goes at the beginning of all Stanford GraphBase CWEB files

\def\topofcontents{
  \leftline{\sc\today\ at \hours}\bigskip\bigskip
  \centerline{\titlefont\title}}

\font\ninett=cmtt9
\def\botofcontents{\vskip 0pt plus 1filll
    \ninerm\baselineskip10pt
    \noindent\copyright\ 1993 Stanford University
    \bigskip\noindent
    This file may be freely copied and distributed, provided that
    no changes whatsoever are made. All users are asked to help keep
    the Stanford GraphBase files consistent and ``uncorrupted,''
    identical everywhere in the world. Changes are permissible only
    if the modified file is given a new name, different from the names of
    existing files in the Stanford GraphBase, and only if the modified file is
    clearly identified as not being part of that GraphBase.
    (The {\ninett CWEB} system has a ``change file'' facility by
    which users can easily make minor alterations without modifying
    the master source files in any way. Everybody is supposed to use
    change files instead of changing the files.)
    The author has tried his best to produce correct and useful programs,
    in order to help promote computer science research,
    but no warranty of any kind should be assumed.
    \smallskip\noindent
    Preliminary work on the Stanford GraphBase project
    was supported in part by National Science
    Foundation grant CCR-86-10181.}

\def\prerequisite#1{\def\startsection{\noindent
    Important: Before reading {\sc\title},
    please read or at least skim the program for {\sc#1}.\bigskip
    \let\startsection=\stsec\stsec}}
\def\prerequisites#1#2{\def\startsection{\noindent
    Important: Before reading {\sc\title}, please read
    or at least skim the programs for {\sc#1} and {\sc#2}.\bigskip
    \let\startsection=\stsec\stsec}}



\def\title{LADDERS}

\prerequisites{GB\_WORDS}{GB\_\,DIJK}

\N{1}{1}Introduction. This demonstration program uses graphs
constructed by the {\sc GB\_WORDS} module to produce
an interactive program called \.{ladders}, which finds shortest paths
between two given five-letter words of English.

The program assumes that \UNIX/ conventions are being used. Some code in
sections listed under `\UNIX/ dependencies' in the index might need to change
if this program is ported to other operating systems.

\def\<#1>{$\langle${\rm#1}$\rangle$}
To run the program under \UNIX/, say `\.{ladders} \<options>', where \<options>
consists of zero or more of the following specifications in any order:

{\narrower
\def\\#1 {\smallskip\noindent
\hbox to 6em{\tt#1\hfill}\hangindent 8em\hangafter1 }
\\-v Verbosely print all words encountered during the shortest-path
computation, showing also their distances from the goal word.
\\-a Use alphabetic distance instead of considering adjacent words to be one
unit apart; for example, the alphabetic distance from `\.{words}' to
`\.{woods}' is~3, because `\.r' is three places from `\.o' in the
alphabet.
\\-f Use distance based on frequency (see below), instead of considering
adjacent words to be one unit apart. This option is ignored if either
\.{-a} or \.{-r} has been specified.
\\-h Use a lower-bound heuristic to shorten the search (see below). This option
is ignored if option \.{-f} has been selected.
\\-e Echo the input to the output (useful if input comes from a file instead
of from the terminal).
\\-n\<number> Limit the graph to the \PB{\|n} most common English words, where %
\PB{\|n}
is the given \<number>.
\\-r\<number> Limit the graph to \<number> randomly selected words. This option
is incompatible with~\.{-n}.
\\-s\<number> Use \<number> instead of 0 as the seed for random numbers, to get
different random samples or to explore words of equal frequency in
a different order.
\smallskip}
\noindent Option \.{-f} assigns a cost of 0 to the most common words and a
cost of 16 to the least common words; a cost between 0 and~16 is assigned to
words of intermediate frequency. The word ladders that are found will then have
minimum total cost by this criterion. Experience shows that the \.{-f} option
tends to give the ``friendliest,'' most intuitively appealing ladders.
\smallskip
Option \.{-h} attempts to focus the search by giving priority to words that
are near the goal. (More precisely, it modifies distances between adjacent
words by using a heuristic function $\\{hh}(v)$, which would be the shortest
possible distance between $v$ and the goal if every five-letter combination
happened to be an English word.) The {\sc GB\_\,DIJK} module explains more
about such heuristics; this option is most interesting to watch when used in
conjunction with \.{-v}.

\fi

\M{2}The program will prompt you for a starting word. If you simply type
\<return>, it exits; otherwise you should enter a five-letter word
(with no uppercase letters) before typing \<return>.

Then the program will prompt you for a goal word. If you simply type
\<return> at this point, it will go back and ask for a new starting word;
otherwise you should specify another five-letter word.

Then the program will find and display an optimal word ladder from the start
to the goal, if there is a path from one to the other
that changes only one letter at a time.

And then you have a chance to start all over again, with another starting word.

The start and goal words need not be present in the program's graph of
``known'' words. They are temporarily added to that graph, but removed
again whenever new start and goal words are given. (Thus you can go
from \.{sturm} to \.{drang} even though those words aren't English.)
If the \.{-f} option is being used, the cost of the goal word will be 20
when it is not in the program's dictionary.

\fi

\M{3}Here is the general layout of this program, as seen by the \CEE/ compiler:

\Y\B\8\#\&{include} \.{<ctype.h>}\C{ system file for character types }\6
\8\#\&{include} \.{"gb\_graph.h"}\C{ the standard GraphBase data structures }\6
\8\#\&{include} \.{"gb\_words.h"}\C{ routines for five-letter word graphs }\6
\8\#\&{include} \.{"gb\_dijk.h"}\C{ routines for shortest paths }\6
\ATH\7
\X4:Global variables\X\6
\X11:Subroutines\X\7
\1\1${}\\{main}(\\{argc},\39\\{argv}){}$\6
\&{int} \\{argc};\C{ the number of command-line arguments }\6
\&{char} ${}{*}\\{argv}[\,]{}$;\C{ an array of strings containing those
arguments }\2\2\6
${}\{{}$\1\6
\X5:Scan the command-line options\X;\6
\X6:Set up the graph of words\X;\6
\&{while} (\T{1})\5
${}\{{}$\1\6
\X26:Prompt for starting word and goal word; \PB{\&{break}} if none given\X;\6
\X13:Find a minimal ladder from \PB{\\{start}} to \PB{\\{goal}}, if one exists,
and print it\X;\6
\4${}\}{}$\2\6
\&{return} \T{0};\C{ normal exit }\6
\4${}\}{}$\2\par
\fi

\N{1}{4}Parsing the options. Let's get the \UNIX/ command-line junk out of the
way first, so that we can concentrate on meatier stuff. Our job in this part
of the program is to see if the default value zero of external variable
\PB{\\{verbose}} should change, and/or if the default values of any of the
following
internal variables should change:

\Y\B\4\X4:Global variables\X${}\E{}$\6
\&{char} \\{alph}${}\K\T{0}{}$;\C{ nonzero if the alphabetic distance option is
selected }\6
\&{char} \\{freq}${}\K\T{0}{}$;\C{ nonzero if the frequency-based distance
option is selected }\6
\&{char} \\{heur}${}\K\T{0}{}$;\C{ nonzero if the heuristic search option is
selected }\6
\&{char} \\{echo}${}\K\T{0}{}$;\C{ nonzero if the input-echo option is selected
}\6
\&{unsigned} \&{long} \|n${}\K\T{0}{}$;\C{ maximum number of words in the graph
(0 means infinity) }\6
\&{char} \\{randm}${}\K\T{0}{}$;\C{ nonzero if we will ignore the weight of
words }\6
\&{long} \\{seed}${}\K\T{0}{}$;\C{ seed for random number generator }\par
\As7, 12\ETs23.
\U3.\fi

\M{5}\B\X5:Scan the command-line options\X${}\E{}$\6
\&{while} ${}(\MM\\{argc}){}$\5
${}\{{}$\1\6
\&{if} ${}(\\{strcmp}(\\{argv}[\\{argc}],\39\.{"-v"})\E\T{0}){}$\1\5
${}\\{verbose}\K\T{1};{}$\2\6
\&{else} \&{if} ${}(\\{strcmp}(\\{argv}[\\{argc}],\39\.{"-a"})\E\T{0}){}$\1\5
${}\\{alph}\K\T{1};{}$\2\6
\&{else} \&{if} ${}(\\{strcmp}(\\{argv}[\\{argc}],\39\.{"-f"})\E\T{0}){}$\1\5
${}\\{freq}\K\T{1};{}$\2\6
\&{else} \&{if} ${}(\\{strcmp}(\\{argv}[\\{argc}],\39\.{"-h"})\E\T{0}){}$\1\5
${}\\{heur}\K\T{1};{}$\2\6
\&{else} \&{if} ${}(\\{strcmp}(\\{argv}[\\{argc}],\39\.{"-e"})\E\T{0}){}$\1\5
${}\\{echo}\K\T{1};{}$\2\6
\&{else} \&{if} ${}(\\{sscanf}(\\{argv}[\\{argc}],\39\.{"-n\%lu"},\39{\AND}\|n)%
\E\T{1}){}$\1\5
${}\\{randm}\K\T{0};{}$\2\6
\&{else} \&{if} ${}(\\{sscanf}(\\{argv}[\\{argc}],\39\.{"-r\%lu"},\39{\AND}\|n)%
\E\T{1}){}$\1\5
${}\\{randm}\K\T{1};{}$\2\6
\&{else} \&{if} ${}(\\{sscanf}(\\{argv}[\\{argc}],\39\.{"-s\%ld"},\39{\AND}%
\\{seed})\E\T{1}){}$\1\5
;\2\6
\&{else}\5
${}\{{}$\1\6
${}\\{fprintf}(\\{stderr},\39\.{"Usage:\ \%s\ [-v][-a][}\)%
\.{-f][-h][-e][-nN][-rN}\)\.{][-sN]\\n"},\39\\{argv}[\T{0}]);{}$\6
\&{return} ${}{-}\T{2};{}$\6
\4${}\}{}$\2\6
\4${}\}{}$\2\6
\&{if} ${}(\\{alph}\V\\{randm}){}$\1\5
${}\\{freq}\K\T{0};{}$\2\6
\&{if} (\\{freq})\1\5
${}\\{heur}\K\T{0}{}$;\2\par
\U3.\fi

\N{1}{6}Creating the graph. The GraphBase \PB{\\{words}} procedure will produce
the
five-letter words we want, organized in a graph structure.

\Y\B\4\D$\\{quit\_if}(\|x,\|c)$ \6
\&{if} (\|x)\5
${}\{{}$\1\6
${}\\{fprintf}(\\{stderr},\39\.{"Sorry,\ I\ couldn't\ b}\)\.{uild\ a\
dictionary\ (t}\)\.{rouble\ code\ \%ld)!\\n"},\39\|c);{}$\6
\&{return} \|c;\6
\4${}\}{}$\2\par
\Y\B\4\X6:Set up the graph of words\X${}\E{}$\6
$\|g\K\\{words}(\|n,\39(\\{randm}\?\\{zero\_vector}:\NULL),\39\T{0\$L},\39%
\\{seed});{}$\6
${}\\{quit\_if}(\|g\E\NULL,\39\\{panic\_code});{}$\6
\X8:Confirm the options selected\X;\6
\X9:Modify the edge lengths, if the \PB{\\{alph}} or \PB{\\{freq}} option was
selected\X;\6
\X10:Modify the priority queue algorithm, if unequal edge lengths are possible%
\X;\par
\U3.\fi

\M{7}\B\X4:Global variables\X${}\mathrel+\E{}$\6
\&{Graph} ${}{*}\|g{}$;\C{ graph created by \PB{\\{words}} }\6
\&{long} \\{zero\_vector}[\T{9}];\C{ weights to use when ignoring all frequency
information }\par
\fi

\M{8}The actual number of words might be decreased to the size of the GraphBase
dictionary, so we wait until the graph is generated before confirming
the user-selected options.

\Y\B\4\X8:Confirm the options selected\X${}\E{}$\6
\&{if} (\\{verbose})\5
${}\{{}$\1\6
\&{if} (\\{alph})\1\5
\\{printf}(\.{"(alphabetic\ distanc}\)\.{e\ selected)\\n"});\2\6
\&{if} (\\{freq})\1\5
\\{printf}(\.{"(frequency-based\ di}\)\.{stances\ selected)\\n"});\2\6
\&{if} (\\{heur})\1\5
\\{printf}(\.{"(lowerbound\ heurist}\)\.{ic\ will\ be\ used\ to\ f}\)\.{ocus\
the\ search)\\n"});\2\6
\&{if} (\\{randm})\1\5
${}\\{printf}(\.{"(random\ selection\ o}\)\.{f\ \%ld\ words\ with\ see}\)\.{d\ %
\%ld)\\n"},\39\|g\MG\|n,\39\\{seed});{}$\2\6
\&{else}\1\5
${}\\{printf}(\.{"(the\ graph\ has\ \%ld\ }\)\.{words)\\n"},\39\|g\MG\|n);{}$\2%
\6
\4${}\}{}$\2\par
\U6.\fi

\M{9}The edges in a \PB{\\{words}} graph normally have length 1, so we must
change them
if the user has selected \PB{\\{alph}} or \PB{\\{freq}}. The character position
in which
adjacent words differ is recorded in the \PB{\\{loc}} field of each arc. The
frequency of a word is stored in the \PB{\\{weight}} field of its vertex.

\Y\B\4\D$\\{a\_dist}(\|k)$ \5
$({*}(\|p+\|k)<{*}(\|q+\|k)\?{*}(\|q+\|k)-{*}(\|p+\|k):{*}(\|p+\|k)-{*}(\|q+%
\|k){}$)\par
\Y\B\4\X9:Modify the edge lengths, if the \PB{\\{alph}} or \PB{\\{freq}} option
was selected\X${}\E{}$\6
\&{if} (\\{alph})\5
${}\{{}$\5
\1\&{register} \&{Vertex} ${}{*}\|u;{}$\7
\&{for} ${}(\|u\K\|g\MG\\{vertices}+\|g\MG\|n-\T{1};{}$ ${}\|u\G\|g\MG%
\\{vertices};{}$ ${}\|u\MM){}$\5
${}\{{}$\5
\1\&{register} \&{Arc} ${}{*}\|a;{}$\6
\&{register} \&{char} ${}{*}\|p\K\|u\MG\\{name};{}$\7
\&{for} ${}(\|a\K\|u\MG\\{arcs};{}$ \|a; ${}\|a\K\|a\MG\\{next}){}$\5
${}\{{}$\5
\1\&{register} \&{char} ${}{*}\|q\K\|a\MG\\{tip}\MG\\{name};{}$\7
${}\|a\MG\\{len}\K\\{a\_dist}(\|a\MG\\{loc});{}$\6
\4${}\}{}$\2\6
\4${}\}{}$\2\6
\4${}\}{}$\5
\2\&{else} \&{if} (\\{freq})\5
${}\{{}$\5
\1\&{register} \&{Vertex} ${}{*}\|u;{}$\7
\&{for} ${}(\|u\K\|g\MG\\{vertices}+\|g\MG\|n-\T{1};{}$ ${}\|u\G\|g\MG%
\\{vertices};{}$ ${}\|u\MM){}$\5
${}\{{}$\5
\1\&{register} \&{Arc} ${}{*}\|a;{}$\7
\&{for} ${}(\|a\K\|u\MG\\{arcs};{}$ \|a; ${}\|a\K\|a\MG\\{next}){}$\1\5
${}\|a\MG\\{len}\K\\{freq\_cost}(\|a\MG\\{tip});{}$\2\6
\4${}\}{}$\2\6
\4${}\}{}$\2\par
\U6.\fi

\M{10}The default priority queue algorithm of \PB{\\{dijkstra}} is quite
efficient
when all edge lengths are~1. Otherwise we change it to the
alternative method that works best for edge lengths less than~128.

\Y\B\4\X10:Modify the priority queue algorithm, if unequal edge lengths are
possible\X${}\E{}$\6
\&{if} ${}(\\{alph}\V\\{freq}\V\\{heur}){}$\5
${}\{{}$\1\6
${}\\{init\_queue}\K\\{init\_128};{}$\6
${}\\{del\_min}\K\\{del\_128};{}$\6
${}\\{enqueue}\K\\{enq\_128};{}$\6
${}\\{requeue}\K\\{req\_128};{}$\6
\4${}\}{}$\2\par
\U6.\fi

\M{11}The frequency has been computed with the default weights explained in the
documentation of \PB{\\{words}}; it is usually less than $2^{16}$.
A word whose frequency is 0 costs~16; a word whose frequency is 1 costs~15;
a word whose frequency is 2 or 3 costs~14; and the costs keep decreasing
by~1 as the frequency doubles, until we reach a cost of~0.

\Y\B\4\X11:Subroutines\X${}\E{}$\6
\1\1\&{long} \\{freq\_cost}(\|v)\6
\&{Vertex} ${}{*}\|v;\2\2{}$\6
${}\{{}$\5
\1\&{register} \&{long} \\{acc}${}\K\|v\MG\\{weight}{}$;\C{ the frequency, to
be shifted right }\6
\&{register} \&{long} \|k${}\K\T{16};{}$\7
\&{while} (\\{acc})\1\5
${}\|k\MM,\39\\{acc}\MRL{{\GG}{\K}}\T{1};{}$\2\6
\&{return} ${}(\|k<\T{0}\?\T{0}:\|k);{}$\6
\4${}\}{}$\2\par
\As17, 18, 20, 22\ETs27.
\U3.\fi

\N{1}{12}Minimal ladders. The guts of this program is a routine to compute
shortest
paths between two given words, \PB{\\{start}} and \PB{\\{goal}}.

The \PB{\\{dijkstra}} procedure does this, in any graph with nonnegative arc
lengths.
The only complication we need to deal with here is that \PB{\\{start}} and \PB{%
\\{goal}}
might not themselves be present in the graph. In that case we want to insert
them, albeit temporarily.

The conventions of {\sc GB\_\,GRAPH} allow us to do the desired augmentation
by creating a new graph \PB{\\{gg}} whose vertices are borrowed from~\PB{\|g}.
The
graph~\PB{\|g} has space for two more vertices (actually for four), and any
new memory blocks allocated for the additional arcs present in~\PB{\\{gg}} will
be freed later by the operation \PB{\\{gb\_recycle}(\\{gg})} without confusion.

\Y\B\4\X4:Global variables\X${}\mathrel+\E{}$\6
\&{Graph} ${}{*}\\{gg}{}$;\C{ clone of \PB{\|g} with possible additional words
}\6
\&{char} \\{start}[\T{6}]${},\39\\{goal}[\T{6}]{}$;\C{ \.{words} dear to the
user's \.{heart}, plus \PB{\.{'\\0'}} }\6
\&{Vertex} ${}{*}\\{uu},\39{*}\\{vv}{}$;\C{ start and goal vertices in \PB{%
\\{gg}} }\par
\fi

\M{13}\B\X13:Find a minimal ladder from \PB{\\{start}} to \PB{\\{goal}}, if one
exists, and print it\X${}\E{}$\6
\X14:Build the amplified graph \PB{\\{gg}}\X;\6
\X21:Let \PB{\\{dijkstra}} do the hard work\X;\6
\X24:Print the answer\X;\6
\X25:Remove all traces of \PB{\\{gg}}\X;\par
\U3.\fi

\M{14}\B\X14:Build the amplified graph \PB{\\{gg}}\X${}\E{}$\6
$\\{gg}\K\\{gb\_new\_graph}(\T{0\$L});{}$\6
${}\\{quit\_if}(\\{gg}\E\NULL,\39\\{no\_room}+\T{5}){}$;\C{ out of memory }\6
${}\\{gg}\MG\\{vertices}\K\|g\MG\\{vertices};{}$\6
${}\\{gg}\MG\|n\K\|g\MG\|n;{}$\6
\X15:Put the \PB{\\{start}} word into \PB{\\{gg}}, and let \PB{\\{uu}} point to
it\X;\6
\X16:Put the \PB{\\{goal}} word into \PB{\\{gg}}, and let \PB{\\{vv}} point to
it\X;\6
\&{if} ${}(\\{gg}\MG\|n\E\|g\MG\|n+\T{2}){}$\1\5
\X19:Check if \PB{\\{start}} is adjacent to \PB{\\{goal}}\X;\2\6
${}\\{quit\_if}(\\{gb\_trouble\_code},\39\\{no\_room}+\T{6}){}$;\C{ out of
memory }\par
\U13.\fi

\M{15}The \PB{\\{find\_word}} procedure returns \PB{$\NULL$} if it can't find
the given word
in the graph just constructed by \PB{\\{words}}. In that case it has applied
its
second argument to every adjacent word. Hence the program logic here
does everything needed to add a new vertex to~\PB{\\{gg}} when necessary.

\Y\B\4\X15:Put the \PB{\\{start}} word into \PB{\\{gg}}, and let \PB{\\{uu}}
point to it\X${}\E{}$\6
$(\\{gg}\MG\\{vertices}+\\{gg}\MG\|n)\MG\\{name}\K\\{start}{}$;\C{ a tentative
new vertex }\6
${}\\{uu}\K\\{find\_word}(\\{start},\39\\{plant\_new\_edge});{}$\6
\&{if} ${}(\R\\{uu}){}$\1\5
${}\\{uu}\K\\{gg}\MG\\{vertices}+\\{gg}\MG\|n\PP{}$;\C{ recognize the new
vertex and refer to it }\2\par
\U14.\fi

\M{16}\B\X16:Put the \PB{\\{goal}} word into \PB{\\{gg}}, and let \PB{\\{vv}}
point to it\X${}\E{}$\6
\&{if} ${}(\\{strncmp}(\\{start},\39\\{goal},\39\T{5})\E\T{0}){}$\1\5
${}\\{vv}\K\\{uu}{}$;\C{ avoid inserting a word twice }\2\6
\&{else}\5
${}\{{}$\1\6
${}(\\{gg}\MG\\{vertices}+\\{gg}\MG\|n)\MG\\{name}\K\\{goal}{}$;\C{ a tentative
new vertex }\6
${}\\{vv}\K\\{find\_word}(\\{goal},\39\\{plant\_new\_edge});{}$\6
\&{if} ${}(\R\\{vv}){}$\1\5
${}\\{vv}\K\\{gg}\MG\\{vertices}+\\{gg}\MG\|n\PP{}$;\C{ recognize the new
vertex and refer to it }\2\6
\4${}\}{}$\2\par
\U14.\fi

\M{17}The \PB{\\{alph\_dist}} subroutine calculates the alphabetic distance
between
arbitrary five-letter words, whether they are adjacent or not.

\Y\B\4\X11:Subroutines\X${}\mathrel+\E{}$\6
\1\1\&{long} ${}\\{alph\_dist}(\|p,\39\|q){}$\6
\&{register} \&{char} ${}{*}\|p,\39{*}\|q;\2\2{}$\6
${}\{{}$\1\6
\&{return} \\{a\_dist}(\T{0})${}+\\{a\_dist}(\T{1})+\\{a\_dist}(\T{2})+\\{a%
\_dist}(\T{3})+\\{a\_dist}(\T{4});{}$\6
\4${}\}{}$\2\par
\fi

\M{18}\B\X11:Subroutines\X${}\mathrel+\E{}$\6
\1\1\&{void} \\{plant\_new\_edge}(\|v)\6
\&{Vertex} ${}{*}\|v;\2\2{}$\6
${}\{{}$\5
\1\&{Vertex} ${}{*}\|u\K\\{gg}\MG\\{vertices}+\\{gg}\MG\|n{}$;\C{ the new edge
runs from \PB{\|u} to \PB{\|v} }\7
${}\\{gb\_new\_edge}(\|u,\39\|v,\39\T{1\$L}){}$;\C{ we have $u>v$, hence \PB{$%
\|v\MG\\{arcs}\K\|u\MG\\{arcs}-\T{1}$} }\6
\&{if} (\\{alph})\1\5
${}\|u\MG\\{arcs}\MG\\{len}\K(\|u\MG\\{arcs}-\T{1})\MG\\{len}\K\\{alph\_dist}(%
\|u\MG\\{name},\39\|v\MG\\{name});{}$\2\6
\&{else} \&{if} (\\{freq})\5
${}\{{}$\1\6
${}\|u\MG\\{arcs}\MG\\{len}\K\\{freq\_cost}(\|v){}$;\C{ adjust the arc length
from \PB{\|u} to \PB{\|v} }\6
${}(\|u\MG\\{arcs}-\T{1})\MG\\{len}\K\T{20}{}$;\C{ adjust the arc length from %
\PB{\|v} to \PB{\|u} }\6
\4${}\}{}$\2\6
\4${}\}{}$\2\par
\fi

\M{19}There's a bug in the above logic that could be embarrassing,
although it will come up only when a user is trying to be clever: The
\PB{\\{find\_word}} routine knows only the words of~\PB{\|g}, so it will fail
to
make any direct connection between \PB{\\{start}} and \PB{\\{goal}} if they
happen
to be adjacent to each other yet not in the original graph. We had
better fix this, otherwise the computer will look stupid.

\Y\B\4\X19:Check if \PB{\\{start}} is adjacent to \PB{\\{goal}}\X${}\E{}$\6
\&{if} ${}(\\{hamm\_dist}(\\{start},\39\\{goal})\E\T{1}){}$\5
${}\{{}$\1\6
${}\\{gg}\MG\|n\MM{}$;\C{ temporarily pretend \PB{\\{vv}} hasn't been added yet
}\6
\\{plant\_new\_edge}(\\{uu});\C{ make \PB{\\{vv}} adjacent to \PB{\\{uu}} }\6
${}\\{gg}\MG\|n\PP{}$;\C{ and recognize it again }\6
\4${}\}{}$\2\par
\U14.\fi

\M{20}The Hamming distance between words is the number of character positions
in which they differ.

\Y\B\4\D$\\{h\_dist}(\|k)$ \5
$({*}(\|p+\|k)\E{*}(\|q+\|k)\?\T{0}:\T{1}{}$)\par
\Y\B\4\X11:Subroutines\X${}\mathrel+\E{}$\6
\1\1\&{long} ${}\\{hamm\_dist}(\|p,\39\|q){}$\6
\&{register} \&{char} ${}{*}\|p,\39{*}\|q;\2\2{}$\6
${}\{{}$\1\6
\&{return} \\{h\_dist}(\T{0})${}+\\{h\_dist}(\T{1})+\\{h\_dist}(\T{2})+\\{h%
\_dist}(\T{3})+\\{h\_dist}(\T{4});{}$\6
\4${}\}{}$\2\par
\fi

\M{21}OK, now we've got a graph in which \PB{\\{dijkstra}} can operate.

\Y\B\4\X21:Let \PB{\\{dijkstra}} do the hard work\X${}\E{}$\6
\&{if} ${}(\R\\{heur}){}$\1\5
${}\\{min\_dist}\K\\{dijkstra}(\\{uu},\39\\{vv},\39\\{gg},\39\NULL);{}$\2\6
\&{else} \&{if} (\\{alph})\1\5
${}\\{min\_dist}\K\\{dijkstra}(\\{uu},\39\\{vv},\39\\{gg},\39\\{alph%
\_heur});{}$\2\6
\&{else}\1\5
${}\\{min\_dist}\K\\{dijkstra}(\\{uu},\39\\{vv},\39\\{gg},\39\\{hamm%
\_heur}){}$;\2\par
\U13.\fi

\M{22}\B\X11:Subroutines\X${}\mathrel+\E{}$\6
\1\1\&{long} \\{alph\_heur}(\|v)\6
\&{Vertex} ${}{*}\|v;\2\2{}$\6
${}\{{}$\5
\1\&{return} \\{alph\_dist}${}(\|v\MG\\{name},\39\\{goal}){}$;\5
${}\}{}$\2\7
\1\1\&{long} \\{hamm\_heur}(\|v)\6
\&{Vertex} ${}{*}\|v;\2\2{}$\6
${}\{{}$\5
\1\&{return} \\{hamm\_dist}${}(\|v\MG\\{name},\39\\{goal}){}$;\5
${}\}{}$\2\par
\fi

\M{23}\B\X4:Global variables\X${}\mathrel+\E{}$\6
\&{long} \\{min\_dist};\C{ length of the shortest ladder }\par
\fi

\M{24}\B\X24:Print the answer\X${}\E{}$\6
\&{if} ${}(\\{min\_dist}<\T{0}){}$\1\5
${}\\{printf}(\.{"Sorry,\ there's\ no\ l}\)\.{adder\ from\ \%s\ to\ \%s.}\)\.{%
\\n"},\39\\{start},\39\\{goal});{}$\2\6
\&{else}\1\5
\\{print\_dijkstra\_result}(\\{vv});\2\par
\U13.\fi

\M{25}Finally, we have to clean up our tracks. It's easy to remove all arcs
from the new vertices of~\PB{\\{gg}} to the old vertices of~\PB{\|g}; it's a
bit
trickier to remove the arcs from old to new. The loop here will also
remove arcs properly between start and goal vertices, if they both
belong to \PB{\\{gg}} not~\PB{\|g}.

\Y\B\4\X25:Remove all traces of \PB{\\{gg}}\X${}\E{}$\6
\&{for} ${}(\\{uu}\K\|g\MG\\{vertices}+\\{gg}\MG\|n-\T{1};{}$ ${}\\{uu}\G\|g\MG%
\\{vertices}+\|g\MG\|n;{}$ ${}\\{uu}\MM){}$\5
${}\{{}$\5
\1\&{register} \&{Arc} ${}{*}\|a;{}$\7
\&{for} ${}(\|a\K\\{uu}\MG\\{arcs};{}$ \|a; ${}\|a\K\|a\MG\\{next}){}$\5
${}\{{}$\1\6
${}\\{vv}\K\|a\MG\\{tip}{}$;\C{ now \PB{$\\{vv}\MG\\{arcs}\E\|a-\T{1}$}, since
arcs for edges come in pairs }\6
${}\\{vv}\MG\\{arcs}\K\\{vv}\MG\\{arcs}\MG\\{next};{}$\6
\4${}\}{}$\2\6
${}\\{uu}\MG\\{arcs}\K\NULL{}$;\C{ we needn't clear \PB{$\\{uu}\MG\\{name}$} }\6
\4${}\}{}$\2\6
\\{gb\_recycle}(\\{gg});\C{ the \PB{$\\{gg}\MG\\{data}$} blocks disappear, but %
\PB{$\|g\MG\\{data}$} remains }\par
\U13.\fi

\N{1}{26}Terminal interaction. We've finished doing all the interesting things.
Only one minor part of the program still remains to be written.

\Y\B\4\X26:Prompt for starting word and goal word; \PB{\&{break}} if none given%
\X${}\E{}$\6
\\{putchar}(\.{'\\n'});\C{ make a blank line for visual punctuation }\6
\4\\{restart}:\C{ if we try to avoid this label,               the \PB{%
\&{break}} command will be broken }\6
\&{if} ${}(\\{prompt\_for\_five}(\.{"Starting"},\39\\{start})\I\T{0}){}$\1\5
\&{break};\2\6
\&{if} ${}(\\{prompt\_for\_five}(\.{"\ \ \ \ Goal"},\39\\{goal})\I\T{0}){}$\1\5
\&{goto} \\{restart};\2\par
\U3.\fi

\M{27}\B\X11:Subroutines\X${}\mathrel+\E{}$\6
\1\1\&{long} ${}\\{prompt\_for\_five}(\|s,\39\|p){}$\6
\&{char} ${}{*}\|s{}$;\C{ string used in prompt message }\6
\&{register} \&{char} ${}{*}\|p{}$;\C{ where to put a string typed by the user
}\2\2\6
${}\{{}$\5
\1\&{register} \&{char} ${}{*}\|q{}$;\C{ current position to store characters }%
\6
\&{register} \&{long} \|c;\C{ current character of input }\7
\&{while} (\T{1})\5
${}\{{}$\1\6
${}\\{printf}(\.{"\%s\ word:\ "},\39\|s);{}$\6
\\{fflush}(\\{stdout});\C{ make sure the user sees the prompt }\6
${}\|q\K\|p;{}$\6
\&{while} (\T{1})\5
${}\{{}$\1\6
${}\|c\K\\{getchar}(\,);{}$\6
\&{if} ${}(\|c\E\.{EOF}){}$\1\5
\&{return} ${}{-}\T{1}{}$;\C{ end-of-file }\2\6
\&{if} (\\{echo})\1\5
\\{putchar}(\|c);\2\6
\&{if} ${}(\|c\E\.{'\\n'}){}$\1\5
\&{break};\2\6
\&{if} ${}(\R\\{islower}(\|c)){}$\1\5
${}\|q\K\|p+\T{5};{}$\2\6
\&{else} \&{if} ${}(\|q<\|p+\T{5}){}$\1\5
${}{*}\|q\K\|c;{}$\2\6
${}\|q\PP;{}$\6
\4${}\}{}$\2\6
\&{if} ${}(\|q\E\|p+\T{5}){}$\1\5
\&{return} \T{0};\C{ got a good five-letter word }\2\6
\&{if} ${}(\|q\E\|p){}$\1\5
\&{return} \T{1};\C{ got just \<return> }\2\6
\\{printf}(\.{"(Please\ type\ five\ l}\)\.{owercase\ letters\ and}\)\.{\
RETURN.)\\n"});\6
\4${}\}{}$\2\6
\4${}\}{}$\2\par
\fi

\N{1}{28}Index. Finally, here's a list that shows where the identifiers of this
program are defined and used.

\fi


\inx
\fin
\con
