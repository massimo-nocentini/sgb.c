\input cwebmac
% This file is part of the Stanford GraphBase (c) Stanford University 1993
% This material goes at the beginning of all Stanford GraphBase CWEB files

\def\topofcontents{
  \leftline{\sc\today\ at \hours}\bigskip\bigskip
  \centerline{\titlefont\title}}

\font\ninett=cmtt9
\def\botofcontents{\vskip 0pt plus 1filll
    \ninerm\baselineskip10pt
    \noindent\copyright\ 1993 Stanford University
    \bigskip\noindent
    This file may be freely copied and distributed, provided that
    no changes whatsoever are made. All users are asked to help keep
    the Stanford GraphBase files consistent and ``uncorrupted,''
    identical everywhere in the world. Changes are permissible only
    if the modified file is given a new name, different from the names of
    existing files in the Stanford GraphBase, and only if the modified file is
    clearly identified as not being part of that GraphBase.
    (The {\ninett CWEB} system has a ``change file'' facility by
    which users can easily make minor alterations without modifying
    the master source files in any way. Everybody is supposed to use
    change files instead of changing the files.)
    The author has tried his best to produce correct and useful programs,
    in order to help promote computer science research,
    but no warranty of any kind should be assumed.
    \smallskip\noindent
    Preliminary work on the Stanford GraphBase project
    was supported in part by National Science
    Foundation grant CCR-86-10181.}

\def\prerequisite#1{\def\startsection{\noindent
    Important: Before reading {\sc\title},
    please read or at least skim the program for {\sc#1}.\bigskip
    \let\startsection=\stsec\stsec}}
\def\prerequisites#1#2{\def\startsection{\noindent
    Important: Before reading {\sc\title}, please read
    or at least skim the programs for {\sc#1} and {\sc#2}.\bigskip
    \let\startsection=\stsec\stsec}}



\def\title{MULTIPLY}

\prerequisite{GB\_\,GATES}

\N{1}{1}Introduction. This demonstration program uses graphs
constructed by the \PB{\\{prod}} procedure in the {\sc GB\_\,GATES} module to
produce
an interactive program called \.{multiply}, which multiplies and divides
small numbers the slow way---by simulating the behavior of
a logical circuit, one gate at a time.

The program assumes that \UNIX/ conventions are being used. Some code in
sections listed under `\UNIX/ dependencies' in the index might need to change
if this program is ported to other operating systems.

\def\<#1>{$\langle${\rm#1}$\rangle$}
To run the program under \UNIX/, say `\.{multiply} $m$ $n$ [\PB{\\{seed}}]',
where
$m$ and $n$ are the sizes of the numbers to be multiplied, in bits,
and where \PB{\\{seed}} is given if and only if you want the multiplier
to be a special-purpose circuit for multiplying a given $m$-bit
number by a randomly chosen $n$-bit constant.

The program will prompt you for two numbers (or for just one, if the
random constant option has been selected), and it will use the gate
network to compute their product. Then it will ask for more input, and so on.

\fi

\M{2}Here is the general layout of this program, as seen by the \CEE/ compiler:

\Y\B\8\#\&{include} \.{"gb\_graph.h"}\C{ the standard GraphBase data structures
}\6
\8\#\&{include} \.{"gb\_gates.h"}\C{ routines for gate graphs }\6
\ATH\7
\X4:Global variables\X\6
\X10:Handy subroutines\X\7
\1\1${}\\{main}(\\{argc},\39\\{argv}){}$\6
\&{int} \\{argc};\C{ the number of command-line arguments }\6
\&{char} ${}{*}\\{argv}[\,]{}$;\C{ an array of strings containing those
arguments }\2\2\6
${}\{{}$\1\6
\X5:Declare variables that ought to be in registers\X;\6
\X6:Obtain \PB{\|m}, \PB{\|n}, and optional \PB{\\{seed}} from the command line%
\X;\6
\X3:Make sure \PB{\|m} and \PB{\|n} are valid; generate the \PB{\\{prod}} graph
\PB{\|g}\X;\6
\&{if} ${}(\\{seed}<\T{0}{}$)\C{ no seed given }\1\6
${}\\{printf}(\.{"Here\ I\ am,\ ready\ to}\)\.{\ multiply\ \%ld-bit\ nu}\)%
\.{mbers\ by\ \%ld-bit\ num}\)\.{bers.\\n"},\39\|m,\39\|n);{}$\2\6
\&{else}\5
${}\{{}$\1\6
${}\|g\K\\{partial\_gates}(\|g,\39\|m,\39\T{0\$L},\39\\{seed},\39%
\\{buffer});{}$\6
\&{if} (\|g)\5
${}\{{}$\1\6
\X9:Set \PB{\|y} to the decimal value of the second input\X;\6
${}\\{printf}(\.{"OK,\ I'm\ ready\ to\ mu}\)\.{ltiply\ any\ \%ld-bit\ n}\)%
\.{umber\ by\ \%s.\\n"},\39\|m,\39\|y);{}$\6
\4${}\}{}$\5
\2\&{else}\5
${}\{{}$\C{ there was enough memory to make the original \PB{\|g}, but
       not enough to reduce it; this probably can't happen,                 but
who knows? }\1\6
${}\\{printf}(\.{"Sorry,\ I\ couldn't\ p}\)\.{rocess\ the\ graph\ (tr}\)%
\.{ouble\ code\ \%ld)!\\n"},\39\\{panic\_code});{}$\6
\&{return} ${}{-}\T{9};{}$\6
\4${}\}{}$\2\6
\4${}\}{}$\2\6
${}\\{printf}(\.{"(I'm\ simulating\ a\ l}\)\.{ogic\ circuit\ with\ \%l}\)\.{d\
gates,\ depth\ \%ld.)}\)\.{\\n"},\39\|g\MG\|n,\39\\{depth}(\|g));{}$\6
\&{while} (\T{1})\5
${}\{{}$\1\6
\X7:Prompt for one or two numbers; \PB{\&{break}} if unsuccessful\X;\6
\X11:Use the network to compute the product\X;\6
${}\\{printf}(\.{"\%sx\%s=\%s\%s.\\n"},\39\|x,\39\|y,\39(\\{strlen}(\|x)+%
\\{strlen}(\|y)>\T{35}\?\.{"\\n\ "}:\.{""}),\39\|z);{}$\6
\4${}\}{}$\2\6
\&{return} \T{0};\C{ normal exit }\6
\4${}\}{}$\2\par
\fi

\M{3}\B\X3:Make sure \PB{\|m} and \PB{\|n} are valid; generate the \PB{%
\\{prod}} graph \PB{\|g}\X${}\E{}$\6
\&{if} ${}(\|m<\T{2}){}$\1\5
${}\|m\K\T{2};{}$\2\6
\&{if} ${}(\|n<\T{2}){}$\1\5
${}\|n\K\T{2};{}$\2\6
\&{if} ${}(\|m>\T{999}\V\|n>\T{999}){}$\5
${}\{{}$\1\6
\\{printf}(\.{"Sorry,\ I'm\ set\ up\ o}\)\.{nly\ for\ precision\ le}\)\.{ss\
than\ 1000\ bits.\\n}\)\.{"});\6
\&{return} ${}{-}\T{1};{}$\6
\4${}\}{}$\2\6
\&{if} ${}((\|g\K\\{prod}(\|m,\39\|n))\E\NULL){}$\5
${}\{{}$\1\6
${}\\{printf}(\.{"Sorry,\ I\ couldn't\ g}\)\.{enerate\ the\ graph\ (n}\)\.{ot\
enough\ memory\ for}\)\.{\ \%s)!\\n"},\39\\{panic\_code}\E\\{no\_room}\?\.{"the%
\ gates"}:\\{panic\_code}\E\\{alloc\_fault}\?\.{"the\ wires"}:\.{"local\
optimization"});{}$\6
\&{return} ${}{-}\T{3};{}$\6
\4${}\}{}$\2\par
\U2.\fi

\M{4}To figure the maximum length of strings \PB{\|x} and \PB{\|y}, we note
that
$2^{999}\approx5.4\times10^{300}$.

\Y\B\4\X4:Global variables\X${}\E{}$\6
\&{Graph} ${}{*}\|g{}$;\C{ graph that defines a logical network for
multiplication }\6
\&{long} \|m${},\39\|n{}$;\C{ length of binary numbers to be multiplied }\6
\&{long} \\{seed};\C{ optional seed value, or $-1$ }\6
\&{char} \|x[\T{302}]${},\39\|y[\T{302}],\39\|z[\T{603}]{}$;\C{ input and
output numbers, as decimal strings }\6
\&{char} \\{buffer}[\T{2000}];\C{ workspace for communication between routines
}\par
\U2.\fi

\M{5}\B\X5:Declare variables that ought to be in registers\X${}\E{}$\6
\&{register} \&{char} ${}{*}\|p,\39{*}\|q,\39{*}\|r{}$;\C{ pointers for string
manipulation }\6
\&{register} \&{long} \|a${},\39\|b{}$;\C{ amounts being carried over while
doing radix conversion }\par
\U2.\fi

\M{6}\B\X6:Obtain \PB{\|m}, \PB{\|n}, and optional \PB{\\{seed}} from the
command line\X${}\E{}$\6
\&{if} ${}(\\{argc}<\T{3}\V\\{argc}>\T{4}\V\\{sscanf}(\\{argv}[\T{1}],\39\.{"%
\%ld"},\39{\AND}\|m)\I\T{1}\V\\{sscanf}(\\{argv}[\T{2}],\39\.{"\%ld"},\39{\AND}%
\|n)\I\T{1}){}$\5
${}\{{}$\1\6
${}\\{fprintf}(\\{stderr},\39\.{"Usage:\ \%s\ m\ n\ [seed}\)\.{]\\n"},\39%
\\{argv}[\T{0}]);{}$\6
\&{return} ${}{-}\T{2};{}$\6
\4${}\}{}$\2\6
\&{if} ${}(\|m<\T{0}){}$\1\5
${}\|m\K{-}\|m{}$;\C{ maybe the user attached \PB{\.{'-'}} to the argument }\2\6
\&{if} ${}(\|n<\T{0}){}$\1\5
${}\|n\K{-}\|n;{}$\2\6
${}\\{seed}\K{-}\T{1};{}$\6
\&{if} ${}(\\{argc}\E\T{4}\W\\{sscanf}(\\{argv}[\T{3}],\39\.{"\%ld"},\39{\AND}%
\\{seed})\E\T{1}\W\\{seed}<\T{0}){}$\1\5
${}\\{seed}\K{-}\\{seed}{}$;\2\par
\U2.\fi

\M{7}This program may not be user-friendly, but at least it is polite.

\Y\B\4\D$\\{prompt}(\|s)$ \6
${}\{{}$\5
\1\\{printf}(\|s);\5
\\{fflush}(\\{stdout});\C{ make sure the user sees the prompt }\6
\&{if} ${}(\\{fgets}(\\{buffer},\39\T{999},\39\\{stdin})\E\NULL){}$\1\5
\&{break};\5
\2${}\}{}$\2\par
\B\4\D$\\{retry}(\|s,\|t)$ \6
${}\{{}$\5
\1\\{printf}(\|s);\5
\&{goto} \|t;\5
${}\}{}$\2\par
\Y\B\4\X7:Prompt for one or two numbers; \PB{\&{break}} if unsuccessful\X${}%
\E{}$\6
\4\\{step1}:\5
\\{prompt}(\.{"\\nNumber,\ please?\ "});\6
\&{for} ${}(\|p\K\\{buffer};{}$ ${}{*}\|p\E\.{'0'};{}$ ${}\|p\PP){}$\1\5
;\C{ bypass leading zeroes }\2\6
\&{if} ${}({*}\|p\E\.{'\\n'}){}$\5
${}\{{}$\1\6
\&{if} ${}(\|p>\\{buffer}){}$\1\5
${}\|p\MM{}$;\C{ zero is acceptable }\2\6
\&{else}\1\5
\&{break};\C{ empty input terminates the run }\2\6
\4${}\}{}$\2\6
\&{for} ${}(\|q\K\|p;{}$ ${}{*}\|q\G\.{'0'}\W{*}\|q\Z\.{'9'};{}$ ${}\|q\PP){}$%
\1\5
;\C{ check for digits }\2\6
\&{if} ${}({*}\|q\I\.{'\\n'}){}$\1\5
${}\\{retry}(\.{"Excuse\ me...\ I'm\ lo}\)\.{oking\ for\ a\ nonnegat}\)\.{ive\
sequence\ of\ deci}\)\.{mal\ digits."},\39\\{step1});{}$\2\6
${}{*}\|q\K\T{0};{}$\6
\&{if} ${}(\\{strlen}(\|p)>\T{301}){}$\1\5
${}\\{retry}(\.{"Sorry,\ that's\ too\ b}\)\.{ig."},\39\\{step1});{}$\2\6
${}\\{strcpy}(\|x,\39\|p);{}$\6
\&{if} ${}(\\{seed}<\T{0}){}$\5
${}\{{}$\1\6
\X8:Do the same thing for \PB{\|y} instead of \PB{\|x}\X;\6
\4${}\}{}$\2\par
\U2.\fi

\M{8}\B\X8:Do the same thing for \PB{\|y} instead of \PB{\|x}\X${}\E{}$\6
\4\\{step2}:\5
\\{prompt}(\.{"Another?\ "});\6
\&{for} ${}(\|p\K\\{buffer};{}$ ${}{*}\|p\E\.{'0'};{}$ ${}\|p\PP){}$\1\5
;\C{ bypass leading zeroes }\2\6
\&{if} ${}({*}\|p\E\.{'\\n'}){}$\5
${}\{{}$\1\6
\&{if} ${}(\|p>\\{buffer}){}$\1\5
${}\|p\MM{}$;\C{ zero is acceptable }\2\6
\&{else}\1\5
\&{break};\C{ empty input terminates the run }\2\6
\4${}\}{}$\2\6
\&{for} ${}(\|q\K\|p;{}$ ${}{*}\|q\G\.{'0'}\W{*}\|q\Z\.{'9'};{}$ ${}\|q\PP){}$%
\1\5
;\C{ check for digits }\2\6
\&{if} ${}({*}\|q\I\.{'\\n'}){}$\1\5
${}\\{retry}(\.{"Excuse\ me...\ I'm\ lo}\)\.{oking\ for\ a\ nonnegat}\)\.{ive\
sequence\ of\ deci}\)\.{mal\ digits."},\39\\{step2});{}$\2\6
${}{*}\|q\K\T{0};{}$\6
\&{if} ${}(\\{strlen}(\|p)>\T{301}){}$\1\5
${}\\{retry}(\.{"Sorry,\ that's\ too\ b}\)\.{ig."},\39\\{step2});{}$\2\6
${}\\{strcpy}(\|y,\39\|p){}$;\par
\U7.\fi

\M{9}The binary value chosen at random by \PB{\\{partial\_gates}} appears as a
string of 0s and 1s in \PB{\\{buffer}}, in little-endian order. We compute
the corresponding decimal value by repeated doubling.

If the value turns out to be zero, the whole network will have collapsed.
Otherwise, however, the \PB{\|m} inputs from the first operand
will all remain present, because they all affect the output.

\Y\B\4\X9:Set \PB{\|y} to the decimal value of the second input\X${}\E{}$\6
${*}\|y\K\.{'0'}{}$;\5
${}{*}(\|y+\T{1})\K\T{0}{}$;\C{ now \PB{\|y} is \PB{\.{"0"}} }\6
\&{for} ${}(\|r\K\\{buffer}+\\{strlen}(\\{buffer})-\T{1};{}$ ${}\|r\G%
\\{buffer};{}$ ${}\|r\MM){}$\5
${}\{{}$\C{ we will set $y=2y+t$ where $t$ is the next bit, \PB{${*}\|r$} }\1\6
\&{if} ${}({*}\|y\G\.{'5'}){}$\1\5
${}\|a\K\T{0},\39\|p\K\|y;{}$\2\6
\&{else}\1\5
${}\|a\K{*}\|y-\.{'0'},\39\|p\K\|y+\T{1};{}$\2\6
\&{for} ${}(\|q\K\|y;{}$ ${}{*}\|p;{}$ ${}\|a\K\|b,\39\|p\PP,\39\|q\PP){}$\5
${}\{{}$\1\6
\&{if} ${}({*}\|p\G\.{'5'}){}$\5
${}\{{}$\1\6
${}\|b\K{*}\|p-\.{'5'};{}$\6
${}{*}\|q\K\T{2}*\|a+\.{'1'};{}$\6
\4${}\}{}$\5
\2\&{else}\5
${}\{{}$\1\6
${}\|b\K{*}\|p-\.{'0'};{}$\6
${}{*}\|q\K\T{2}*\|a+\.{'0'};{}$\6
\4${}\}{}$\2\6
\4${}\}{}$\2\6
\&{if} ${}({*}\|r\E\.{'1'}){}$\1\5
${}{*}\|q\K\T{2}*\|a+\.{'1'};{}$\2\6
\&{else}\1\5
${}{*}\|q\K\T{2}*\|a+\.{'0'};{}$\2\6
${}{*}\PP\|q\K\T{0}{}$;\C{ terminate the string }\6
\4${}\}{}$\2\6
\&{if} ${}(\\{strcmp}(\|y,\39\.{"0"})\E\T{0}){}$\5
${}\{{}$\1\6
${}\\{printf}(\.{"Please\ try\ another\ }\)\.{seed\ value;\ \%d\ makes}\)\.{\
the\ answer\ zero!\\n"},\39\\{seed});{}$\6
\&{return} ${}({-}\T{5});{}$\6
\4${}\}{}$\2\par
\U2.\fi

\N{1}{10}Using the network. The reader of the code in the previous section
will have noticed that we are representing high-precision decimal
numbers as strings. We might as well do that, since the only
operations we need to perform on them are input, output, doubling, and
halving. In fact, arithmetic on strings is kind of fun, if you like
that sort of thing.

Here is a subroutine that converts a decimal string to a binary string.
The decimal string is big-endian as usual, but the binary string is
little-endian. The decimal string is decimated in the process; it
should end up empty, unless the original value was too big.

\Y\B\4\X10:Handy subroutines\X${}\E{}$\6
\1\1${}\\{decimal\_to\_binary}(\|x,\39\|s,\39\|n){}$\6
\&{char} ${}{*}\|x{}$;\C{ decimal string }\6
\&{char} ${}{*}\|s{}$;\C{ binary string }\6
\&{long} \|n;\C{ length of \PB{\|s} }\2\2\6
${}\{{}$\5
\1\&{register} \&{long} \|k;\6
\&{register} \&{char} ${}{*}\|p,\39{*}\|q{}$;\C{ pointers for string
manipulation }\6
\&{register} \&{long} \|r;\C{ remainder }\7
\&{for} ${}(\|k\K\T{0};{}$ ${}\|k<\|n;{}$ ${}\|k\PP,\39\|s\PP){}$\5
${}\{{}$\1\6
\&{if} ${}({*}\|x\E\T{0}){}$\1\5
${}{*}\|s\K\.{'0'};{}$\2\6
\&{else}\5
${}\{{}$\C{ we will divide \PB{\|x} by 2 }\1\6
\&{if} ${}({*}\|x>\.{'1'}){}$\1\5
${}\|p\K\|x,\39\|r\K\T{0};{}$\2\6
\&{else}\1\5
${}\|p\K\|x+\T{1},\39\|r\K{*}\|x-\.{'0'};{}$\2\6
\&{for} ${}(\|q\K\|x;{}$ ${}{*}\|p;{}$ ${}\|p\PP,\39\|q\PP){}$\5
${}\{{}$\1\6
${}\|r\K\T{10}*\|r+{*}\|p-\.{'0'};{}$\6
${}{*}\|q\K(\|r\GG\T{1})+\.{'0'};{}$\6
${}\|r\K\|r\AND\T{1};{}$\6
\4${}\}{}$\2\6
${}{*}\|q\K\T{0}{}$;\C{ terminate string \PB{\|x} }\6
${}{*}\|s\K\.{'0'}+\|r;{}$\6
\4${}\}{}$\2\6
\4${}\}{}$\2\6
${}{*}\|s\K\T{0}{}$;\C{ terminate the output string }\6
\4${}\}{}$\2\par
\A13.
\U2.\fi

\M{11}\B\X11:Use the network to compute the product\X${}\E{}$\6
$\\{strcpy}(\|z,\39\|x);{}$\6
${}\\{decimal\_to\_binary}(\|z,\39\\{buffer},\39\|m);{}$\6
\&{if} ${}({*}\|z){}$\5
${}\{{}$\1\6
${}\\{printf}(\.{"(Sorry,\ \%s\ has\ more}\)\.{\ than\ \%ld\ bits.)\\n"},\39%
\|x,\39\|m);{}$\6
\&{continue};\6
\4${}\}{}$\2\6
\&{if} ${}(\\{seed}<\T{0}){}$\5
${}\{{}$\1\6
${}\\{strcpy}(\|z,\39\|y);{}$\6
${}\\{decimal\_to\_binary}(\|z,\39\\{buffer}+\|m,\39\|n);{}$\6
\&{if} ${}({*}\|z){}$\5
${}\{{}$\1\6
${}\\{printf}(\.{"(Sorry,\ \%s\ has\ more}\)\.{\ than\ \%ld\ bits.)\\n"},\39%
\|y,\39\|n);{}$\6
\&{continue};\6
\4${}\}{}$\2\6
\4${}\}{}$\2\6
\&{if} ${}(\\{gate\_eval}(\|g,\39\\{buffer},\39\\{buffer})<\T{0}){}$\5
${}\{{}$\1\6
\\{printf}(\.{"???\ An\ internal\ err}\)\.{or\ occurred!"});\6
\&{return} \T{666};\C{ this can't happen }\6
\4${}\}{}$\2\6
\X12:Convert the binary number in \PB{\\{buffer}} to the decimal string \PB{%
\|z}\X;\par
\U2.\fi

\M{12}The remaining task is almost identical to what we needed to do
when computing the value of \PB{\|y} after a random seed was specified.
But this time the binary number in \PB{\\{buffer}} is big-endian.

\Y\B\4\X12:Convert the binary number in \PB{\\{buffer}} to the decimal string %
\PB{\|z}\X${}\E{}$\6
${*}\|z\K\.{'0'}{}$;\5
${}{*}(\|z+\T{1})\K\T{0};{}$\6
\&{for} ${}(\|r\K\\{buffer};{}$ ${}{*}\|r;{}$ ${}\|r\PP){}$\5
${}\{{}$\C{ we'll set $z=2z+t$ where $t$ is the next bit, \PB{${*}\|r$} }\1\6
\&{if} ${}({*}\|z\G\.{'5'}){}$\1\5
${}\|a\K\T{0},\39\|p\K\|z;{}$\2\6
\&{else}\1\5
${}\|a\K{*}\|z-\.{'0'},\39\|p\K\|z+\T{1};{}$\2\6
\&{for} ${}(\|q\K\|z;{}$ ${}{*}\|p;{}$ ${}\|a\K\|b,\39\|p\PP,\39\|q\PP){}$\5
${}\{{}$\1\6
\&{if} ${}({*}\|p\G\.{'5'}){}$\5
${}\{{}$\1\6
${}\|b\K{*}\|p-\.{'5'};{}$\6
${}{*}\|q\K\T{2}*\|a+\.{'1'};{}$\6
\4${}\}{}$\5
\2\&{else}\5
${}\{{}$\1\6
${}\|b\K{*}\|p-\.{'0'};{}$\6
${}{*}\|q\K\T{2}*\|a+\.{'0'};{}$\6
\4${}\}{}$\2\6
\4${}\}{}$\2\6
\&{if} ${}({*}\|r\E\.{'1'}){}$\1\5
${}{*}\|q\K\T{2}*\|a+\.{'1'};{}$\2\6
\&{else}\1\5
${}{*}\|q\K\T{2}*\|a+\.{'0'};{}$\2\6
${}{*}\PP\|q\K\T{0}{}$;\C{ terminate the string }\6
\4${}\}{}$\2\par
\U11.\fi

\N{1}{13}Calculating the depth. The depth of a gate network produced by {\sc
GB\_\,GATES} is easily discovered by making one pass over the
vertices.  An input gate or a constant has depth~0; every other gate
has depth one greater than the maximum of its inputs.

This routine is more general than it needs to be for the circuits output
by \PB{\\{prod}}. The result of a latch is considered to have depth~0.

Utility field \PB{$\|u.\|I$} is set to the depth of each individual gate.

\Y\B\4\D$\\{dp}$ \5
$\|u.{}$\|I\par
\Y\B\4\X10:Handy subroutines\X${}\mathrel+\E{}$\6
\1\1\&{long} \\{depth}(\|g)\6
\&{Graph} ${}{*}\|g{}$;\C{ graph with gates as vertices }\2\2\6
${}\{{}$\5
\1\&{register} \&{Vertex} ${}{*}\|v{}$;\C{ the current vertex of interest }\6
\&{register} \&{Arc} ${}{*}\|a{}$;\C{ the current arc of interest }\6
\&{long} \|d;\C{ depth of current vertex }\7
\&{if} ${}(\R\|g){}$\1\5
\&{return} ${}{-}\T{1}{}$;\C{ no graph supplied! }\2\6
\&{for} ${}(\|v\K\|g\MG\\{vertices};{}$ ${}\|v<\|g\MG\\{vertices}+\|g\MG\|n;{}$
${}\|v\PP){}$\5
${}\{{}$\1\6
\&{switch} ${}(\|v\MG\\{typ}){}$\5
${}\{{}$\C{ branch on type of gate }\1\6
\4\&{case} \.{'I'}:\5
\&{case} \.{'L'}:\5
\&{case} \.{'C'}:\5
${}\|v\MG\\{dp}\K\T{0}{}$;\5
\&{break};\6
\4\&{default}:\5
\X14:Set \PB{\|d} to the maximum depth of an operand of \PB{\|v}\X;\6
${}\|v\MG\\{dp}\K\T{1}+\|d;{}$\6
\4${}\}{}$\2\6
\4${}\}{}$\2\6
\X15:Set \PB{\|d} to the maximum depth of an output of \PB{\|g}\X;\6
\&{return} \|d;\6
\4${}\}{}$\2\par
\fi

\M{14}\B\X14:Set \PB{\|d} to the maximum depth of an operand of \PB{\|v}\X${}%
\E{}$\6
$\|d\K\T{0};{}$\6
\&{for} ${}(\|a\K\|v\MG\\{arcs};{}$ \|a; ${}\|a\K\|a\MG\\{next}){}$\1\6
\&{if} ${}(\|a\MG\\{tip}\MG\\{dp}>\|d){}$\1\5
${}\|d\K\|a\MG\\{tip}\MG\\{dp}{}$;\2\2\par
\U13.\fi

\M{15}\B\X15:Set \PB{\|d} to the maximum depth of an output of \PB{\|g}\X${}%
\E{}$\6
$\|d\K\T{0};{}$\6
\&{for} ${}(\|a\K\|g\MG\\{outs};{}$ \|a; ${}\|a\K\|a\MG\\{next}){}$\1\6
\&{if} ${}(\R\\{is\_boolean}(\|a\MG\\{tip})\W\|a\MG\\{tip}\MG\\{dp}>\|d){}$\1\5
${}\|d\K\|a\MG\\{tip}\MG\\{dp}{}$;\2\2\par
\U13.\fi

\N{1}{16}Index. Finally, here's a list that shows where the identifiers of this
program are defined and used.
\fi

\inx
\fin
\con
