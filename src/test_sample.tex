\input cwebmac
% This file is part of the Stanford GraphBase (c) Stanford University 1993
% This material goes at the beginning of all Stanford GraphBase CWEB files

\def\topofcontents{
  \leftline{\sc\today\ at \hours}\bigskip\bigskip
  \centerline{\titlefont\title}}

\font\ninett=cmtt9
\def\botofcontents{\vskip 0pt plus 1filll
    \ninerm\baselineskip10pt
    \noindent\copyright\ 1993 Stanford University
    \bigskip\noindent
    This file may be freely copied and distributed, provided that
    no changes whatsoever are made. All users are asked to help keep
    the Stanford GraphBase files consistent and ``uncorrupted,''
    identical everywhere in the world. Changes are permissible only
    if the modified file is given a new name, different from the names of
    existing files in the Stanford GraphBase, and only if the modified file is
    clearly identified as not being part of that GraphBase.
    (The {\ninett CWEB} system has a ``change file'' facility by
    which users can easily make minor alterations without modifying
    the master source files in any way. Everybody is supposed to use
    change files instead of changing the files.)
    The author has tried his best to produce correct and useful programs,
    in order to help promote computer science research,
    but no warranty of any kind should be assumed.
    \smallskip\noindent
    Preliminary work on the Stanford GraphBase project
    was supported in part by National Science
    Foundation grant CCR-86-10181.}

\def\prerequisite#1{\def\startsection{\noindent
    Important: Before reading {\sc\title},
    please read or at least skim the program for {\sc#1}.\bigskip
    \let\startsection=\stsec\stsec}}
\def\prerequisites#1#2{\def\startsection{\noindent
    Important: Before reading {\sc\title}, please read
    or at least skim the programs for {\sc#1} and {\sc#2}.\bigskip
    \let\startsection=\stsec\stsec}}



\def\title{TEST\_\,SAMPLE}


\N{1}{1}Introduction. This GraphBase program is intended to be used only
when the Stanford GraphBase is being installed. It invokes the
most critical subroutines and creates a file that can be checked
against the correct output.
The testing is not exhaustive by any means, but it is designed to detect
errors of portability---cases where different results might occur
on different systems. Thus, if nothing goes wrong, one can assume that
the GraphBase routines are probably installed satisfactorily.

The basic idea of {\sc TEST\_\,SAMPLE} is quite simple: We generate a graph,
then print out a few of its salient characteristics. Then we recycle
the graph and generate another, etc. The test is passed if the output
file matches a ``correct'' output file generated at Stanford by the author.

Actually there are two output files. The main one, containing samples of
graph characteristics, is the standard output. The other, called \.{test.gb},
is a graph that has been saved in ASCII format with \PB{\\{save\_graph}}.

\Y\B\8\#\&{include} \.{"gb\_graph.h"}\C{ we use the {\sc GB\_\,GRAPH} data
structures }\6
\8\#\&{include} \.{"gb\_io.h"}\C{ and the GraphBase input/output routines }\6
\X2:Include headers for all of the GraphBase generation modules\X\7
\X7:Private variables\X\6
\X13:Procedures\X\7
\1\1\hbox{\4}\&{int} \\{main}(\,)\2\2\6
${}\{{}$\5
\1\&{Graph} ${}{*}\|g,\39{*}\\{gg}{}$;\5
\&{long} \|i;\5
\&{Vertex} ${}{*}\|v{}$;\C{ temporary registers }\7
\\{printf}(\.{"GraphBase\ samples\ g}\)\.{enerated\ by\ test\_sam}\)\.{ple:%
\\n"});\6
\X6:Save a graph to be restored later\X;\6
\X3:Print samples of generated graphs\X;\6
\&{return} \T{0};\C{ normal exit }\6
\4${}\}{}$\2\par
\fi

\M{2}\B\X2:Include headers for all of the GraphBase generation modules\X${}%
\E{}$\6
\8\#\&{include} \.{"gb\_basic.h"}\C{ we test the basic graph operations }\6
\8\#\&{include} \.{"gb\_books.h"}\C{ and the graphs based on literature }\6
\8\#\&{include} \.{"gb\_econ.h"}\C{ and the graphs based on economic data }\6
\8\#\&{include} \.{"gb\_games.h"}\C{ and the graphs based on football scores }\6
\8\#\&{include} \.{"gb\_gates.h"}\C{ and the graphs based on logic circuits }\6
\8\#\&{include} \.{"gb\_lisa.h"}\C{ and the graphs based on Mona Lisa }\6
\8\#\&{include} \.{"gb\_miles.h"}\C{ and the graphs based on mileage data }\6
\8\#\&{include} \.{"gb\_plane.h"}\C{ and the planar graphs }\6
\8\#\&{include} \.{"gb\_raman.h"}\C{ and the Ramanujan graphs }\6
\8\#\&{include} \.{"gb\_rand.h"}\C{ and the random graphs }\6
\8\#\&{include} \.{"gb\_roget.h"}\C{ and the graphs based on Roget's Thesaurus
}\6
\8\#\&{include} \.{"gb\_save.h"}\C{ and we save results in ASCII format }\6
\8\#\&{include} \.{"gb\_words.h"}\C{ and we also test five-letter-word graphs }%
\par
\U1.\fi

\M{3}The subroutine \PB{$\\{print\_sample}(\|g,\|n)$} will be specified later.
It prints global
characteristics of \PB{\|g} and local characteristics of the \PB{\|n}th vertex.

We begin the test cautiously by generating a graph that requires no input data
and no pseudo-random numbers. If this test fails, the fault must lie either in
{\sc GB\_\,GRAPH} or {\sc GB\_\,RAMAN}.

\Y\B\4\X3:Print samples of generated graphs\X${}\E{}$\6
$\\{print\_sample}(\\{raman}(\T{31\$L},\39\T{3\$L},\39\T{0\$L},\39\T{4\$L}),\39%
\T{4}){}$;\par
\As4, 5, 8, 9, 10\ETs11.
\U1.\fi

\M{4}Next we test part of {\sc GB\_\,BASIC} that relies on a particular
interpretation of the operation `\PB{$\|w\MRL{{\GG}{\K}}\T{1}$}'. If this part
of the test
fails, please look up `system dependencies' in the index to {\sc
GB\_\,BASIC}, and correct the problem on your system by making a change file
\.{gb\_basic.ch}. (See \.{queen\_wrap.ch} for an example of a change file.)

On the other hand, if {\sc TEST\_\,SAMPLE} fails only in this particular test
while passing all those that follow, chances are excellent that
you have a pretty good implementation of the GraphBase anyway,
because the bug detected here will rarely show up in practice. Ask
yourself: Can I live comfortably with such a bug?

\Y\B\4\X3:Print samples of generated graphs\X${}\mathrel+\E{}$\6
$\\{print\_sample}(\\{board}(\T{1\$L},\39\T{1\$L},\39\T{2\$L},\39{-}\T{33\$L},%
\39\T{1\$L},\39{-}\T{\^40000000\$L}-\T{\^40000000\$L},\39\T{1\$L}),\39%
\T{2000}){}$;\C{ coordinates 32 and 33 (only) should wrap around }\par
\fi

\M{5}Another system-dependent part of {\sc GB\_\,BASIC} is tested here,
this time involving character codes.

\Y\B\4\X3:Print samples of generated graphs\X${}\mathrel+\E{}$\6
$\\{print\_sample}(\\{subsets}(\T{32\$L},\39\T{18\$L},\39\T{16\$L},\39\T{0\$L},%
\39\T{999\$L},\39{-}\T{999\$L},\39\T{\^80000000\$L},\39\T{1\$L}),\39\T{1}){}$;%
\par
\fi

\M{6}If \.{test.gb} fails to match \.{test.correct}, the most likely culprit
is \PB{\\{vert\_offset}}, a ``pointer hack'' in {\sc GB\_\,BASIC}. That macro
absolutely must be made to work properly, because it is used heavily.
In particular, it is used in the \PB{\\{complement}} routine tested here,
and in the \PB{\\{gunion}} routine tested below.

\Y\B\4\X6:Save a graph to be restored later\X${}\E{}$\6
$\|g\K\\{random\_graph}(\T{3\$L},\39\T{10\$L},\39\T{1\$L},\39\T{1\$L},\39\T{0%
\$L},\39\NULL,\39\\{dst},\39\T{1\$L},\39\T{2\$L},\39\T{1\$L}){}$;\C{ a random
multigraph with 3 vertices, 10 edges }\6
${}\\{gg}\K\\{complement}(\|g,\39\T{1\$L},\39\T{1\$L},\39\T{0\$L}){}$;\C{ a
copy of \PB{\|g}, without multiple edges }\6
${}\|v\K\\{gb\_typed\_alloc}(\T{1},\39\&{Vertex},\39\\{gg}\MG\\{data}){}$;\C{
we create a stray vertex too }\6
${}\|v\MG\\{name}\K\\{gb\_save\_string}(\.{"Testing"});{}$\6
${}\\{gg}\MG\\{util\_types}[\T{10}]\K\.{'V'};{}$\6
${}\\{gg}\MG\\{ww}.\|V\K\|v{}$;\C{ the stray vertex is now part of \PB{\\{gg}}
}\6
${}\\{save\_graph}(\\{gg},\39\.{"test.gb"}){}$;\C{ so it will appear in %
\.{test.gb} (we hope) }\6
\\{gb\_recycle}(\|g);\5
\\{gb\_recycle}(\\{gg});\par
\U1.\fi

\M{7}\B\X7:Private variables\X${}\E{}$\6
\&{static} \&{long} \\{dst}[\,]${}\K\{\T{\^20000000},\39\T{\^10000000},\39\T{%
\^10000000}\}{}$;\C{ a probability distribution with frequencies 50\%, 25\%, 25%
\% }\par
\A12.
\U1.\fi

\M{8}Now we try to reconstruct the graph we saved before, and we also randomize
its lengths.

\Y\B\4\X3:Print samples of generated graphs\X${}\mathrel+\E{}$\6
$\|g\K\\{restore\_graph}(\.{"test.gb"});{}$\6
\&{if} ${}(\|i\K\\{random\_lengths}(\|g,\39\T{0\$L},\39\T{10\$L},\39\T{12\$L},%
\39\\{dst},\39\T{2\$L})){}$\1\5
${}\\{printf}(\.{"\\nFailure\ code\ \%ld\ }\)\.{returned\ by\ random\_l}\)%
\.{engths!\\n"},\39\|i);{}$\2\6
\&{else}\5
${}\{{}$\1\6
${}\\{gg}\K\\{random\_graph}(\T{3\$L},\39\T{10\$L},\39\T{1\$L},\39\T{1\$L},\39%
\T{0\$L},\39\NULL,\39\\{dst},\39\T{1\$L},\39\T{2\$L},\39\T{1\$L}){}$;\C{ same
as before }\6
${}\\{print\_sample}(\\{gunion}(\|g,\39\\{gg},\39\T{1\$L},\39\T{0\$L}),\39%
\T{2});{}$\6
\\{gb\_recycle}(\|g);\5
\\{gb\_recycle}(\\{gg});\6
\4${}\}{}$\2\par
\fi

\M{9}Partial evaluation of a RISC circuit involves fairly intricate pointer
manipulation, so this step should help to test the portability of the author's
favorite programming tricks.

\Y\B\4\X3:Print samples of generated graphs\X${}\mathrel+\E{}$\6
$\\{print\_sample}(\\{partial\_gates}(\\{risc}(\T{0\$L}),\39\T{1\$L},\39%
\T{43210\$L},\39\T{98765\$L},\39\NULL),\39\T{79}){}$;\par
\fi

\M{10}Now we're ready to test the mechanics of reading data files,
sorting with {\sc GB\_\,SORT}, and heavy randomization. Lots of computation
takes place in this section.

\Y\B\4\X3:Print samples of generated graphs\X${}\mathrel+\E{}$\6
$\\{print\_sample}(\\{book}(\.{"homer"},\39\T{500\$L},\39\T{400\$L},\39\T{2%
\$L},\39\T{12\$L},\39\T{10000\$L},\39{-}\T{123456\$L},\39\T{789\$L}),\39%
\T{81});{}$\6
${}\\{print\_sample}(\\{econ}(\T{40\$L},\39\T{0\$L},\39\T{400\$L},\39{-}\T{111%
\$L}),\39\T{11});{}$\6
${}\\{print\_sample}(\\{games}(\T{60\$L},\39\T{70\$L},\39\T{80\$L},\39{-}\T{90%
\$L},\39{-}\T{101\$L},\39\T{60\$L},\39\T{0\$L},\39\T{999999999\$L}),\39%
\T{14});{}$\6
${}\\{print\_sample}(\\{miles}(\T{50\$L},\39{-}\T{500\$L},\39\T{100\$L},\39\T{1%
\$L},\39\T{500\$L},\39\T{5\$L},\39\T{314159\$L}),\39\T{20});{}$\6
${}\\{print\_sample}(\\{plane\_lisa}(\T{100\$L},\39\T{100\$L},\39\T{50\$L},\39%
\T{1\$L},\39\T{300\$L},\39\T{1\$L},\39\T{200\$L},\39\T{50\$L}*\T{299\$L}*\T{199%
\$L},\39\T{200\$L}*\T{299\$L}*\T{199\$L}),\39\T{1294});{}$\6
${}\\{print\_sample}(\\{plane\_miles}(\T{50\$L},\39\T{500\$L},\39{-}\T{100\$L},%
\39\T{1\$L},\39\T{1\$L},\39\T{40000\$L},\39\T{271818\$L}),\39\T{14});{}$\6
${}\\{print\_sample}(\\{random\_bigraph}(\T{300\$L},\39\T{3\$L},\39\T{1000\$L},%
\39{-}\T{1\$L},\39\T{0\$L},\39\\{dst},\39{-}\T{500\$L},\39\T{500\$L},\39\T{666%
\$L}),\39\T{3});{}$\6
${}\\{print\_sample}(\\{roget}(\T{1000\$L},\39\T{3\$L},\39\T{1009\$L},\39%
\T{1009\$L}),\39\T{40}){}$;\par
\fi

\M{11}Finally, here's a picky, picky test that is supposed to fail the first
time,
succeed the second. (The weight vector just barely exceeds
the maximum weight threshold allowed by {\sc GB\_WORDS}. That test is
ultraconservative, but eminently reasonable nevertheless.)

\Y\B\4\X3:Print samples of generated graphs\X${}\mathrel+\E{}$\6
$\\{print\_sample}(\\{words}(\T{100\$L},\39\\{wt\_vector},\39\T{70000000\$L},%
\39\T{69\$L}),\39\T{5});{}$\6
${}\\{wt\_vector}[\T{1}]\PP;{}$\6
${}\\{print\_sample}(\\{words}(\T{100\$L},\39\\{wt\_vector},\39\T{70000000\$L},%
\39\T{69\$L}),\39\T{5});{}$\6
${}\\{print\_sample}(\\{words}(\T{0\$L},\39\NULL,\39\T{0\$L},\39\T{69\$L}),\39%
\T{5555}){}$;\par
\fi

\M{12}\B\X7:Private variables\X${}\mathrel+\E{}$\6
\&{static} \&{long} \\{wt\_vector}[\,]${}\K\{\T{100},\39{-}\T{80589},\39%
\T{50000},\39\T{18935},\39{-}\T{18935},\39\T{18935},\39\T{18935},\39\T{18935},%
\39\T{18935}\}{}$;\par
\fi

\N{1}{13}Printing the sample data. Given a graph \PB{\|g} in GraphBase format
and
an integer~\PB{\|n}, the subroutine \PB{$\\{print\_sample}(\|g,\|n)$} will
output
global characteristics of~\PB{\|g}, such as its name and size, together with
detailed information about its \PB{\|n}th vertex. Then \PB{\|g} will be
shredded
and recycled; the calling routine should not refer to it again.

\Y\B\4\X13:Procedures\X${}\E{}$\6
\&{static} \&{void} \\{pr\_vert}(\,);\C{ a subroutine for printing a vertex is
declared below }\6
\&{static} \&{void} \\{pr\_arc}(\,);\C{ likewise for arcs }\6
\&{static} \&{void} \\{pr\_util}(\,);\C{ and for utility fields in general }\7
\1\1\&{static} \&{void} ${}\\{print\_sample}(\|g,\39\|n){}$\6
\&{Graph} ${}{*}\|g{}$;\C{ graph to be sampled and destroyed }\6
\&{int} \|n;\C{ index to the sampled vertex }\2\2\6
${}\{{}$\1\6
\\{printf}(\.{"\\n"});\6
\&{if} ${}(\|g\E\NULL){}$\5
${}\{{}$\1\6
${}\\{printf}(\.{"Ooops,\ we\ just\ ran\ }\)\.{into\ panic\ code\ \%ld!}\)\.{%
\\n"},\39\\{panic\_code});{}$\6
\&{if} (\\{io\_errors})\1\5
${}\\{printf}(\.{"(The\ I/O\ error\ code}\)\.{\ is\ 0x\%lx)\\n"},\39{}$(%
\&{unsigned} \&{long})\,\\{io\_errors});\2\6
\4${}\}{}$\5
\2\&{else}\5
${}\{{}$\1\6
\X18:Print global characteristics of \PB{\|g}\X;\6
\X17:Print information about the \PB{\|n}th vertex\X;\6
\\{gb\_recycle}(\|g);\6
\4${}\}{}$\2\6
\4${}\}{}$\2\par
\As14, 15\ETs16.
\U1.\fi

\M{14}The graph's \PB{\\{util\_types}} are used to determine how much
information
should be printed. A level parameter also helps control the verbosity of
printout. In the most verbose mode, each utility field that points to a
vertex or arc, or contains integer or string data, will be printed.

\Y\B\4\X13:Procedures\X${}\mathrel+\E{}$\6
\1\1\&{static} \&{void} ${}\\{pr\_vert}(\|v,\39\|l,\39\|s){}$\6
\&{Vertex} ${}{*}\|v{}$;\C{ vertex to be printed }\6
\&{int} \|l;\C{ \PB{$\Z$ \T{0}} if the output should be terse }\6
\&{char} ${}{*}\|s{}$;\C{ format for graph utility fields }\2\2\6
${}\{{}$\1\6
\&{if} ${}(\|v\E\NULL){}$\1\5
\\{printf}(\.{"NULL"});\2\6
\&{else} \&{if} (\\{is\_boolean}(\|v))\1\5
\\{printf}(\.{"ONE"});\C{ see {\sc GB\_\,GATES} }\2\6
\&{else}\5
${}\{{}$\1\6
${}\\{printf}(\.{"\\"\%s\\""},\39\|v\MG\\{name});{}$\6
${}\\{pr\_util}(\|v\MG\|u,\39\|s[\T{0}],\39\|l-\T{1},\39\|s);{}$\6
${}\\{pr\_util}(\|v\MG\|v,\39\|s[\T{1}],\39\|l-\T{1},\39\|s);{}$\6
${}\\{pr\_util}(\|v\MG\|w,\39\|s[\T{2}],\39\|l-\T{1},\39\|s);{}$\6
${}\\{pr\_util}(\|v\MG\|x,\39\|s[\T{3}],\39\|l-\T{1},\39\|s);{}$\6
${}\\{pr\_util}(\|v\MG\|y,\39\|s[\T{4}],\39\|l-\T{1},\39\|s);{}$\6
${}\\{pr\_util}(\|v\MG\|z,\39\|s[\T{5}],\39\|l-\T{1},\39\|s);{}$\6
\&{if} ${}(\|l>\T{0}){}$\5
${}\{{}$\5
\1\&{register} \&{Arc} ${}{*}\|a;{}$\7
\&{for} ${}(\|a\K\|v\MG\\{arcs};{}$ \|a; ${}\|a\K\|a\MG\\{next}){}$\5
${}\{{}$\1\6
\\{printf}(\.{"\\n\ \ \ "});\6
${}\\{pr\_arc}(\|a,\39\T{1},\39\|s);{}$\6
\4${}\}{}$\2\6
\4${}\}{}$\2\6
\4${}\}{}$\2\6
\4${}\}{}$\2\par
\fi

\M{15}\B\X13:Procedures\X${}\mathrel+\E{}$\6
\1\1\&{static} \&{void} ${}\\{pr\_arc}(\|a,\39\|l,\39\|s){}$\6
\&{Arc} ${}{*}\|a{}$;\C{ non-null arc to be printed }\6
\&{int} \|l;\C{ \PB{$\Z$ \T{0}} if the output should be terse }\6
\&{char} ${}{*}\|s{}$;\C{ format for graph utility fields }\2\2\6
${}\{{}$\1\6
\\{printf}(\.{"->"});\6
${}\\{pr\_vert}(\|a\MG\\{tip},\39\T{0},\39\|s);{}$\6
\&{if} ${}(\|l>\T{0}){}$\5
${}\{{}$\1\6
${}\\{printf}(\.{",\ \%ld"},\39\|a\MG\\{len});{}$\6
${}\\{pr\_util}(\|a\MG\|a,\39\|s[\T{6}],\39\|l-\T{1},\39\|s);{}$\6
${}\\{pr\_util}(\|a\MG\|b,\39\|s[\T{7}],\39\|l-\T{1},\39\|s);{}$\6
\4${}\}{}$\2\6
\4${}\}{}$\2\par
\fi

\M{16}\B\X13:Procedures\X${}\mathrel+\E{}$\6
\1\1\&{static} \&{void} ${}\\{pr\_util}(\|u,\39\|c,\39\|l,\39\|s){}$\6
\&{util} \|u;\C{ a utility field to be printed }\6
\&{char} \|c;\C{ its type code }\6
\&{int} \|l;\C{ 0 if output should be terse, \PB{${-}\T{1}$} if pointers
omitted }\6
\&{char} ${}{*}\|s{}$;\C{ utility types for overall graph }\2\2\6
${}\{{}$\1\6
\&{switch} (\|c)\5
${}\{{}$\1\6
\4\&{case} \.{'I'}:\5
${}\\{printf}(\.{"[\%ld]"},\39\|u.\|I){}$;\5
\&{break};\6
\4\&{case} \.{'S'}:\5
${}\\{printf}(\.{"[\\"\%s\\"]"},\39\|u.\|S\?\|u.\|S:\.{"(null)"}){}$;\5
\&{break};\6
\4\&{case} \.{'A'}:\6
\&{if} ${}(\|l<\T{0}){}$\1\5
\&{break};\2\6
\\{printf}(\.{"["});\6
\&{if} ${}(\|u.\|A\E\NULL){}$\1\5
\\{printf}(\.{"NULL"});\2\6
\&{else}\1\5
${}\\{pr\_arc}(\|u.\|A,\39\|l,\39\|s);{}$\2\6
\\{printf}(\.{"]"});\6
\&{break};\6
\4\&{case} \.{'V'}:\6
\&{if} ${}(\|l<\T{0}){}$\1\5
\&{break};\C{ avoid infinite recursion }\2\6
\\{printf}(\.{"["});\6
${}\\{pr\_vert}(\|u.\|V,\39\|l,\39\|s);{}$\6
\\{printf}(\.{"]"});\6
\4\&{default}:\5
\&{break};\C{ case \PB{\.{'Z'}} does nothing, other cases won't occur }\6
\4${}\}{}$\2\6
\4${}\}{}$\2\par
\fi

\M{17}\B\X17:Print information about the \PB{\|n}th vertex\X${}\E{}$\6
$\\{printf}(\.{"V\%d:\ "},\39\|n);{}$\6
\&{if} ${}(\|n\G\|g\MG\|n\V\|n<\T{0}){}$\1\5
\\{printf}(\.{"index\ is\ out\ of\ ran}\)\.{ge!\\n"});\2\6
\&{else}\5
${}\{{}$\1\6
${}\\{pr\_vert}(\|g\MG\\{vertices}+\|n,\39\T{1},\39\|g\MG\\{util\_types});{}$\6
\\{printf}(\.{"\\n"});\6
\4${}\}{}$\2\par
\U13.\fi

\M{18}\B\X18:Print global characteristics of \PB{\|g}\X${}\E{}$\6
$\\{printf}(\.{"\\"\%s\\"\\n\%ld\ vertice}\)\.{s,\ \%ld\ arcs,\ util\_ty}\)%
\.{pes\ \%s"},\39\|g\MG\\{id},\39\|g\MG\|n,\39\|g\MG\|m,\39\|g\MG\\{util%
\_types});{}$\6
${}\\{pr\_util}(\|g\MG\\{uu},\39\|g\MG\\{util\_types}[\T{8}],\39\T{0},\39\|g\MG%
\\{util\_types});{}$\6
${}\\{pr\_util}(\|g\MG\\{vv},\39\|g\MG\\{util\_types}[\T{9}],\39\T{0},\39\|g\MG%
\\{util\_types});{}$\6
${}\\{pr\_util}(\|g\MG\\{ww},\39\|g\MG\\{util\_types}[\T{10}],\39\T{0},\39\|g%
\MG\\{util\_types});{}$\6
${}\\{pr\_util}(\|g\MG\\{xx},\39\|g\MG\\{util\_types}[\T{11}],\39\T{0},\39\|g%
\MG\\{util\_types});{}$\6
${}\\{pr\_util}(\|g\MG\\{yy},\39\|g\MG\\{util\_types}[\T{12}],\39\T{0},\39\|g%
\MG\\{util\_types});{}$\6
${}\\{pr\_util}(\|g\MG\\{zz},\39\|g\MG\\{util\_types}[\T{13}],\39\T{0},\39\|g%
\MG\\{util\_types});{}$\6
\\{printf}(\.{"\\n"});\par
\U13.\fi

\N{1}{19}Index. We end with the customary list of identifiers, showing where
they are used and where they are defined.
\fi

\inx
\fin
\con
