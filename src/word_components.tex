\input cwebmac
% This file is part of the Stanford GraphBase (c) Stanford University 1993
% This material goes at the beginning of all Stanford GraphBase CWEB files

\def\topofcontents{
  \leftline{\sc\today\ at \hours}\bigskip\bigskip
  \centerline{\titlefont\title}}

\font\ninett=cmtt9
\def\botofcontents{\vskip 0pt plus 1filll
    \ninerm\baselineskip10pt
    \noindent\copyright\ 1993 Stanford University
    \bigskip\noindent
    This file may be freely copied and distributed, provided that
    no changes whatsoever are made. All users are asked to help keep
    the Stanford GraphBase files consistent and ``uncorrupted,''
    identical everywhere in the world. Changes are permissible only
    if the modified file is given a new name, different from the names of
    existing files in the Stanford GraphBase, and only if the modified file is
    clearly identified as not being part of that GraphBase.
    (The {\ninett CWEB} system has a ``change file'' facility by
    which users can easily make minor alterations without modifying
    the master source files in any way. Everybody is supposed to use
    change files instead of changing the files.)
    The author has tried his best to produce correct and useful programs,
    in order to help promote computer science research,
    but no warranty of any kind should be assumed.
    \smallskip\noindent
    Preliminary work on the Stanford GraphBase project
    was supported in part by National Science
    Foundation grant CCR-86-10181.}

\def\prerequisite#1{\def\startsection{\noindent
    Important: Before reading {\sc\title},
    please read or at least skim the program for {\sc#1}.\bigskip
    \let\startsection=\stsec\stsec}}
\def\prerequisites#1#2{\def\startsection{\noindent
    Important: Before reading {\sc\title}, please read
    or at least skim the programs for {\sc#1} and {\sc#2}.\bigskip
    \let\startsection=\stsec\stsec}}



\def\title{WORD\_\,COMPONENTS}

\prerequisite{GB\_WORDS}

\N{1}{1}Components. \kern-.7pt
This simple demonstration program computes the connected
components of the GraphBase graph of five-letter words. It prints the
words in order of decreasing weight, showing the number of edges,
components, and isolated vertices present in the graph defined by the
first $n$ words for all~$n$.

\Y\B\8\#\&{include} \.{"gb\_graph.h"}\C{ the GraphBase data structures }\6
\8\#\&{include} \.{"gb\_words.h"}\C{ the \PB{\\{words}} routine }\6
\ATH\7
\1\1\\{main}(\,)\2\2\6
${}\{{}$\5
\1\&{Graph} ${}{*}\|g\K\\{words}(\T{0\$L},\39\T{0\$L},\39\T{0\$L},\39\T{0%
\$L}){}$;\C{ the graph we love }\6
\&{Vertex} ${}{*}\|v{}$;\C{ the current vertex being added to the component
structure }\6
\&{Arc} ${}{*}\|a{}$;\C{ the current arc of interest }\6
\&{long} \|n${}\K\T{0}{}$;\C{ the number of vertices in the component structure
}\6
\&{long} \\{isol}${}\K\T{0}{}$;\C{ the number of isolated vertices in the
component structure }\6
\&{long} \\{comp}${}\K\T{0}{}$;\C{ the current number of components }\6
\&{long} \|m${}\K\T{0}{}$;\C{ the current number of edges }\7
${}\\{printf}(\.{"Component\ analysis\ }\)\.{of\ \%s\\n"},\39\|g\MG\\{id});{}$\6
\&{for} ${}(\|v\K\|g\MG\\{vertices};{}$ ${}\|v<\|g\MG\\{vertices}+\|g\MG\|n;{}$
${}\|v\PP){}$\5
${}\{{}$\1\6
${}\|n\PP,\39\\{printf}(\.{"\%4ld:\ \%5ld\ \%s"},\39\|n,\39\|v\MG\\{weight},\39%
\|v\MG\\{name});{}$\6
\X2:Add vertex \PB{\|v} to the component structure, printing out any components
it joins\X;\6
${}\\{printf}(\.{";\ c=\%ld,i=\%ld,m=\%ld}\)\.{\\n"},\39\\{comp},\39\\{isol},%
\39\|m);{}$\6
\4${}\}{}$\2\6
\X5:Display all unusual components\X;\6
\&{return} \T{0};\C{ normal exit }\6
\4${}\}{}$\2\par
\fi

\M{2}The arcs from \PB{\|v} to previous vertices all appear on the list \PB{$%
\|v\MG\\{arcs}$}
after the arcs from \PB{\|v} to future vertices. In this program, we aren't
interested in the future, only the past; so we skip the initial arcs.

\Y\B\4\X2:Add vertex \PB{\|v} to the component structure, printing out any
components it joins\X${}\E{}$\6
\X3:Make \PB{\|v} a component all by itself\X;\6
${}\|a\K\|v\MG\\{arcs};{}$\6
\&{while} ${}(\|a\W\|a\MG\\{tip}>\|v){}$\1\5
${}\|a\K\|a\MG\\{next};{}$\2\6
\&{if} ${}(\R\|a){}$\1\5
\\{printf}(\.{"[1]"});\C{ indicate that this word is isolated }\2\6
\&{else}\5
${}\{{}$\5
\1\&{long} \|c${}\K\T{0}{}$;\C{ the number of merge steps performed because of %
\PB{\|v} }\7
\&{for} ( ; \|a; ${}\|a\K\|a\MG\\{next}){}$\5
${}\{{}$\5
\1\&{register} \&{Vertex} ${}{*}\|u\K\|a\MG\\{tip};{}$\7
${}\|m\PP;{}$\6
\X4:Merge the components of \PB{\|u} and \PB{\|v}, if they differ\X;\6
\4${}\}{}$\2\6
${}\\{printf}(\.{"\ in\ \%s[\%ld]"},\39\|v\MG\\{master}\MG\\{name},\39\|v\MG%
\\{master}\MG\\{size}){}$;\C{ show final component }\6
\4${}\}{}$\2\par
\U1.\fi

\M{3}We keep track of connected components by using circular lists, a
procedure that is known to take average time $O(n)$ on truly
random graphs [Knuth and Sch\"onhage, {\sl Theoretical Computer Science\/
\bf 6}  (1978), 281--315].

Namely, if \PB{\|v} is a vertex, all the vertices in its component will be
in the list
$$\hbox{\PB{\|v}, \ \PB{$\|v\MG\\{link}$}, \ \PB{$\|v\MG\\{link}\MG\\{link}$}, %
\ \dots,}$$
eventually returning to \PB{\|v} again. There is also a master vertex in
each component, \PB{$\|v\MG\\{master}$}; if \PB{\|v} is the master vertex, %
\PB{$\|v\MG\\{size}$} will
be the number of vertices in its component.

\Y\B\4\D$\\{link}$ \5
$\|z.{}$\|V\C{ link to next vertex in component (occupies utility field \PB{%
\|z}) }\par
\B\4\D$\\{master}$ \5
$\|y.{}$\|V\C{ pointer to master vertex in component }\par
\B\4\D$\\{size}$ \5
$\|x.{}$\|I\C{ size of component, kept up to date for master vertices only }\par
\Y\B\4\X3:Make \PB{\|v} a component all by itself\X${}\E{}$\6
$\|v\MG\\{link}\K\|v;{}$\6
${}\|v\MG\\{master}\K\|v;{}$\6
${}\|v\MG\\{size}\K\T{1};{}$\6
${}\\{isol}\PP;{}$\6
${}\\{comp}\PP{}$;\par
\U2.\fi

\M{4}When two components merge together, we change the identity of the master
vertex in the smaller component. The master vertex representing \PB{\|v} itself
will change if \PB{\|v} is adjacent to any prior vertex.

\Y\B\4\X4:Merge the components of \PB{\|u} and \PB{\|v}, if they differ\X${}%
\E{}$\6
$\|u\K\|u\MG\\{master};{}$\6
\&{if} ${}(\|u\I\|v\MG\\{master}){}$\5
${}\{{}$\5
\1\&{register} \&{Vertex} ${}{*}\|w\K\|v\MG\\{master},\39{*}\|t;{}$\7
\&{if} ${}(\|u\MG\\{size}<\|w\MG\\{size}){}$\5
${}\{{}$\1\6
\&{if} ${}(\|c\PP>\T{0}){}$\1\5
${}\\{printf}(\.{"\%s\ \%s[\%ld]"},\39(\|c\E\T{2}\?\.{"\ with"}:\.{","}),\39\|u%
\MG\\{name},\39\|u\MG\\{size});{}$\2\6
${}\|w\MG\\{size}\MRL{+{\K}}\|u\MG\\{size};{}$\6
\&{if} ${}(\|u\MG\\{size}\E\T{1}){}$\1\5
${}\\{isol}\MM;{}$\2\6
\&{for} ${}(\|t\K\|u\MG\\{link};{}$ ${}\|t\I\|u;{}$ ${}\|t\K\|t\MG\\{link}){}$%
\1\5
${}\|t\MG\\{master}\K\|w;{}$\2\6
${}\|u\MG\\{master}\K\|w;{}$\6
\4${}\}{}$\5
\2\&{else}\5
${}\{{}$\1\6
\&{if} ${}(\|c\PP>\T{0}){}$\1\5
${}\\{printf}(\.{"\%s\ \%s[\%ld]"},\39(\|c\E\T{2}\?\.{"\ with"}:\.{","}),\39\|w%
\MG\\{name},\39\|w\MG\\{size});{}$\2\6
\&{if} ${}(\|u\MG\\{size}\E\T{1}){}$\1\5
${}\\{isol}\MM;{}$\2\6
${}\|u\MG\\{size}\MRL{+{\K}}\|w\MG\\{size};{}$\6
\&{if} ${}(\|w\MG\\{size}\E\T{1}){}$\1\5
${}\\{isol}\MM;{}$\2\6
\&{for} ${}(\|t\K\|w\MG\\{link};{}$ ${}\|t\I\|w;{}$ ${}\|t\K\|t\MG\\{link}){}$%
\1\5
${}\|t\MG\\{master}\K\|u;{}$\2\6
${}\|w\MG\\{master}\K\|u;{}$\6
\4${}\}{}$\2\6
${}\|t\K\|u\MG\\{link};{}$\6
${}\|u\MG\\{link}\K\|w\MG\\{link};{}$\6
${}\|w\MG\\{link}\K\|t;{}$\6
${}\\{comp}\MM;{}$\6
\4${}\}{}$\2\par
\U2.\fi

\M{5}The \PB{\\{words}} graph has one giant component and lots of isolated
vertices.
We consider all other components unusual, so we print them out when the
other computation is done.

\Y\B\4\X5:Display all unusual components\X${}\E{}$\6
\\{printf}(\.{"\\nThe\ following\ non}\)\.{-isolated\ words\ didn}\)\.{'t\ join%
\ the\ giant\ co}\)\.{mponent:\\n"});\6
\&{for} ${}(\|v\K\|g\MG\\{vertices};{}$ ${}\|v<\|g\MG\\{vertices}+\|g\MG\|n;{}$
${}\|v\PP){}$\1\6
\&{if} ${}(\|v\MG\\{master}\E\|v\W\|v\MG\\{size}>\T{1}\W\|v\MG\\{size}+\|v\MG%
\\{size}<\|g\MG\|n){}$\5
${}\{{}$\5
\1\&{register} \&{Vertex} ${}{*}\|u;{}$\6
\&{long} \|c${}\K\T{1}{}$;\C{ count of number printed on current line }\7
${}\\{printf}(\.{"\%s"},\39\|v\MG\\{name});{}$\6
\&{for} ${}(\|u\K\|v\MG\\{link};{}$ ${}\|u\I\|v;{}$ ${}\|u\K\|u\MG\\{link}){}$\5
${}\{{}$\1\6
\&{if} ${}(\|c\PP\E\T{12}){}$\1\5
${}\\{putchar}(\.{'\\n'}),\39\|c\K\T{1};{}$\2\6
${}\\{printf}(\.{"\ \%s"},\39\|u\MG\\{name});{}$\6
\4${}\}{}$\2\6
\\{putchar}(\.{'\\n'});\6
\4${}\}{}$\2\2\par
\U1.\fi

\N{1}{6}Index. We close with a list that shows where the identifiers of this
program are defined and used.
\fi

\inx
\fin
\con
